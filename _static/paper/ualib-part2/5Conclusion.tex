% -*- TeX-master: "ualib-part2.tex" -*-
%%% Local Variables: 
%%% mode: latex
%%% TeX-engine: 'xetex
%%% End:
%%%%%%%%%%%%%%%%%%%%%%%%%%%%%%%%%%%%%%%%%%%%%%%%%%%%%%%%%%%%%%%%%%%%%%%%%
\section{Concluding Remarks}\label{sec:concluding-remarks}
We've reached the end of second installment in our three-part series describing
the \agdaualib.  The next part~\cite{DeMeo:2021-3} covers free algebras,
equational classes of algebras (i.e., varieties), and Birkhoff's HSP theorem. 

% We completed the first major milestone of this project in January 2021, which
% was the first formal, machine-checked proof of Birkhoff's HSP theorem.  Our
% next goal is to support current mathematical research by formalizing some
% recently proved theorems. For example, we intend to formalize theorems about
% the computational complexity of decidable properties of algebraic structures.
% Part of this effort will naturally involve further work on the library itself,
% and may lead to new observations or discoveries in dependent type theory or
% homotopy type theory. 

%% One natural question is whether there are any objects of our research that
%% cannot be represented constructively in type theory. % and, if so, whether
%% these have suitable constructive substitutes.
%% %Along these lines, %% we should revisit and perhaps refine our proof of
%% %Birkhoff's theorem. A 
%% As mentioned in \S~\ref{sec:contributions}, a constructive version of
%% Birkhoff's theorem was presented by Carlstr\"om in~\cite{Carlstrom:2008}. We
%% would like to know, for example, how the two new hypotheses required by
%% Carlstr\"om %% adds to the classical theorem in order to make it constructive  
%% compare with the assumptions we make in our Agda proof of this result.

We conclude by noting that one of our goals is to make computer formalization of
mathematics more accessible to mathematicians working in universal algebra and
model theory. We welcome feedback from the community and are happy to field
questions about the \ualib, how it is installed, and how it can be used to prove
theorems that are not yet part of the library.  Merge requests submitted to the
UALib's main gitlab repository are especially welcomed.  Please visit the
repository at \url{https://gitlab.com/ualib/ualib.gitlab.io/} and help us
improve it. 



