% -*- TeX-master: "ualib-part2.tex" -*-
%%% Local Variables: 
%%% mode: latex
%%% TeX-engine: 'xetex
%%% End:
%%%%%%%%%%%%%%%%%%%%%%%%%%%%%%%%%%%%%%%%%%%%%%%%%%%%%%%%%%%%%%%%%%%%%%%%%%%%%%%%%%%%%%%%
\section{Homomorphism Types}\label{sec:homomorphism-types}
In this section we define types that represent some of the most important concepts from general (universal) algebra.  Note that the Agda modules we describe here, and in succeeding sections depend on and import the modules that were presented in Part 1 of this series of three papers describing the \agdaualib (see~\cite{DeMeo:2021-1}). 

We begin in Subsection~\ref{sec:basic-definitions} with the basic definition of \emph{homomorphism}. In \S\ref{sec:homom-theor} we formalize the statement and proof of the first fundamental theorem about homomorphisms, which is sometimes referred to as the \emph{First Isomorphism Theorem}. This is followed by \S\ref{sec:isomorphisms}, in which we define the type of \emph{isomorphisms} between algebraic structures.  Finally, in \S\ref{sec:hom-images}, we define types that manifest the notion of \emph{homomorphic image}.

\subsection{Basic definitions}\label{sec:basic-definitions}
\firstsentence{Homomorphisms}{Basic}
% -*- TeX-master: "ualib-part2.tex" -*-
%%% Local Variables: 
%%% mode: latex
%%% TeX-engine: 'xetex
%%% End: 
Since this is the first module we introduce in the installment of our series of papers documenting the \agdaualib, we will begin by showing the start of the module in full.  In later modules, we will leave such details implicit.  Here is how the file \texttt{Homomorphisms/Basic.lagda} of the \ualib begins.
\ccpad
\begin{code}%
\>[0]\AgdaSymbol{\{-\#}\AgdaSpace{}%
\AgdaKeyword{OPTIONS}\AgdaSpace{}%
\AgdaPragma{--without-K}\AgdaSpace{}%
\AgdaPragma{--exact-split}\AgdaSpace{}%
\AgdaPragma{--safe}\AgdaSpace{}%
\AgdaSymbol{\#-\}}\<%
\\
%
\\[\AgdaEmptyExtraSkip]%
\>[0]\AgdaKeyword{open}\AgdaSpace{}%
\AgdaKeyword{import}\AgdaSpace{}%
\AgdaModule{Algebras.Signatures}\AgdaSpace{}%
\AgdaKeyword{using}\AgdaSpace{}%
\AgdaSymbol{(}\AgdaFunction{Signature}\AgdaSymbol{;}\AgdaSpace{}%
\AgdaGeneralizable{𝓞}\AgdaSymbol{;}\AgdaSpace{}%
\AgdaGeneralizable{𝓥}\AgdaSymbol{)}\<%
\\
\>[0]\AgdaKeyword{open}\AgdaSpace{}%
\AgdaKeyword{import}\AgdaSpace{}%
\AgdaModule{MGS-Subsingleton-Theorems}\AgdaSpace{}%
\AgdaKeyword{using}\AgdaSpace{}%
\AgdaSymbol{(}\AgdaFunction{global-dfunext}\AgdaSymbol{)}\<%
\\
%
\\[\AgdaEmptyExtraSkip]%
\>[0]\AgdaKeyword{module}\AgdaSpace{}%
\AgdaModule{Homomorphisms.Basic}\AgdaSpace{}%
\AgdaSymbol{\{}\AgdaBound{𝑆}\AgdaSpace{}%
\AgdaSymbol{:}\AgdaSpace{}%
\AgdaFunction{Signature}\AgdaSpace{}%
\AgdaGeneralizable{𝓞}\AgdaSpace{}%
\AgdaGeneralizable{𝓥}\AgdaSymbol{\}\{}\AgdaBound{gfe}\AgdaSpace{}%
\AgdaSymbol{:}\AgdaSpace{}%
\AgdaFunction{global-dfunext}\AgdaSymbol{\}}\AgdaSpace{}%
\AgdaKeyword{where}\<%
\\
%
\\[\AgdaEmptyExtraSkip]%
\>[0]\AgdaKeyword{open}\AgdaSpace{}%
\AgdaKeyword{import}\AgdaSpace{}%
\AgdaModule{Algebras.Congruences}\AgdaSymbol{\{}\AgdaArgument{𝑆}\AgdaSpace{}%
\AgdaSymbol{=}\AgdaSpace{}%
\AgdaBound{𝑆}\AgdaSymbol{\}}\AgdaSpace{}%
\AgdaKeyword{public}\<%
\\
\>[0]\AgdaKeyword{open}\AgdaSpace{}%
\AgdaKeyword{import}\AgdaSpace{}%
\AgdaModule{MGS-MLTT}\AgdaSpace{}%
\AgdaKeyword{using}\AgdaSpace{}%
\AgdaSymbol{(}\AgdaOperator{\AgdaFunction{\AgdaUnderscore{}≡⟨\AgdaUnderscore{}⟩\AgdaUnderscore{}}}\AgdaSymbol{;}\AgdaSpace{}%
\AgdaOperator{\AgdaFunction{\AgdaUnderscore{}∎}}\AgdaSymbol{)}\AgdaSpace{}%
\AgdaKeyword{public}\<%
\end{code}
\ccpad
The \AgdaKeyword{OPTIONS} \emph{pragma} sets some parameters that effect the type theoretic foundations that we assume.  These are discussed in~\cite[\S2.1]{DeMeo:2021-1}, but let's briefly review: \AgdaPragma{--without-K} disables \axiomk (see~\cite{agdaref-axiomk}); \AgdaPragma{--exact-split} makes Agda accept only definitions that use \emph{definitional} equalities (see~\cite{agdatools-patternmatching}); \AgdaPragma{--safe} ensures that nothing is postulated outright, so that every non-MLTT axiom has to be an explicit assumption (see~\cite{agdaref-safeagda} and~\cite{agdatools-patternmatching}).\footnote{As in~\cite{DeMeo:2021-1}, MLTT stands for Martin-L\"of Type Theory.}
















\subsubsection{Homomorphisms}\label{homomorphisms}

If \ab{𝑨} and \ab{𝑩} are \ab{𝑆}-algebras, then a \defn{homomorphism} is a function \ab{h} \as : \af ∣ \ab 𝑨 \af ∣ \as → \af ∣ \ab 𝑩 \af ∣ from the domain of \ab{𝑨} to the domain of \ab{𝑩} that is \defn{compatible} (or \defn{commutes}) with all of the basic operations of the signature; that is, for all operation symbols
\ab 𝑓 \as : \af ∣ \ab 𝑆 \af ∣ and all tuples \ab{𝑎} \as : \af{∥} \ab 𝑆 \af{∥} \ab 𝑓 \as → \af ∣ \ab 𝑨 \af ∣, the following equality holds:\footnote{%
Recall, \ab{h} \af ∘ \ab 𝑎 is the tuple whose \ab i-th component is \ab{h} (\ab 𝑎 \ab i).}\\[-4pt]

\ab{h} ((\ab 𝑓 \af ̂ \ab 𝑨) \ab 𝑎) ≡ (\ab 𝑓 \af ̂ \ab 𝑩) (\ab h \af ∘ \ab 𝑎).\\[4pt]
Instead of ``homomorphism,'' we sometimes use the nickname ``hom'' to refer to such a map.

To formalize the notion of homomorphism in type theory, we first define a type representing the assertion that a function \ab{h} \as : \af ∣ \ab 𝑨 \af ∣ \as → \af ∣ \ab 𝑩 \af ∣, from the domain of \ab{𝑨} to the domain of \ab{𝑩}, \emph{commutes} (or is \emph{compatible}) with an operation \ab 𝑓, interpreted in the algebras \ab{𝑨} and \ab{𝑩}. Pleasingly, the defining equation of the previous paragraph can be expressed in Agda unadulterated.\footnote{%
Here we put the definition inside an \emph{anonymous module}, which starts with the \AgdaKeyword{module} keyword followed by an underscore (instead of a module name). The purpose is simply to move the postulated typing judgments (the ``parameters'' of the module, e.g., \AgdaBound{𝓤}\AgdaSpace{}%
  \AgdaBound{𝓦}\AgdaSpace{}\AgdaSymbol{:}\AgdaSpace{}\AgdaFunction{Universe})
out of the way so they don't obfuscate the definitions inside the module. In descriptions of the \ualib, such as the present paper, we usually don't show the module declarations unless we wish to emphasize the typing judgments that are postulated in the module declaration.}
\ccpad
\begin{code}%
\>[0]\AgdaKeyword{module}\AgdaSpace{}%
\AgdaModule{\AgdaUnderscore{}}\AgdaSpace{}%
\AgdaSymbol{\{}\AgdaBound{𝓤}\AgdaSpace{}%
\AgdaBound{𝓦}\AgdaSpace{}%
\AgdaSymbol{:}\AgdaSpace{}%
\AgdaFunction{Universe}\AgdaSymbol{\}}\AgdaSpace{}%
\AgdaKeyword{where}\<%
\\
%
\\[\AgdaEmptyExtraSkip]%
\>[0][@{}l@{\AgdaIndent{0}}]%
\>[1]\AgdaFunction{compatible-op-map}\AgdaSpace{}%
\AgdaSymbol{:}%
\>[42I]\AgdaSymbol{(}\AgdaBound{𝑨}\AgdaSpace{}%
\AgdaSymbol{:}\AgdaSpace{}%
\AgdaFunction{Algebra}\AgdaSpace{}%
\AgdaBound{𝓤}\AgdaSpace{}%
\AgdaBound{𝑆}\AgdaSymbol{)(}\AgdaBound{𝑩}\AgdaSpace{}%
\AgdaSymbol{:}\AgdaSpace{}%
\AgdaFunction{Algebra}\AgdaSpace{}%
\AgdaBound{𝓦}\AgdaSpace{}%
\AgdaBound{𝑆}\AgdaSymbol{)}\<%
\\
\>[.][@{}l@{}]\<[42I]%
\>[21]\AgdaSymbol{(}\AgdaBound{𝑓}\AgdaSpace{}%
\AgdaSymbol{:}\AgdaSpace{}%
\AgdaOperator{\AgdaFunction{∣}}\AgdaSpace{}%
\AgdaBound{𝑆}\AgdaSpace{}%
\AgdaOperator{\AgdaFunction{∣}}\AgdaSymbol{)(}\AgdaBound{h}\AgdaSpace{}%
\AgdaSymbol{:}\AgdaSpace{}%
\AgdaOperator{\AgdaFunction{∣}}\AgdaSpace{}%
\AgdaBound{𝑨}\AgdaSpace{}%
\AgdaOperator{\AgdaFunction{∣}}%
\>[44]\AgdaSymbol{→}\AgdaSpace{}%
\AgdaOperator{\AgdaFunction{∣}}\AgdaSpace{}%
\AgdaBound{𝑩}\AgdaSpace{}%
\AgdaOperator{\AgdaFunction{∣}}\AgdaSymbol{)}\AgdaSpace{}%
\AgdaSymbol{→}\AgdaSpace{}%
\AgdaBound{𝓥}\AgdaSpace{}%
\AgdaOperator{\AgdaFunction{⊔}}\AgdaSpace{}%
\AgdaBound{𝓤}\AgdaSpace{}%
\AgdaOperator{\AgdaFunction{⊔}}\AgdaSpace{}%
\AgdaBound{𝓦}\AgdaSpace{}%
\AgdaOperator{\AgdaFunction{̇}}\<%
\\
%
\\[\AgdaEmptyExtraSkip]%
%
\>[1]\AgdaFunction{compatible-op-map}\AgdaSpace{}%
\AgdaBound{𝑨}\AgdaSpace{}%
\AgdaBound{𝑩}\AgdaSpace{}%
\AgdaBound{𝑓}\AgdaSpace{}%
\AgdaBound{h}\AgdaSpace{}%
\AgdaSymbol{=}\AgdaSpace{}%
\AgdaSymbol{∀}\AgdaSpace{}%
\AgdaBound{𝑎}\AgdaSpace{}%
\AgdaSymbol{→}\AgdaSpace{}%
\AgdaBound{h}\AgdaSpace{}%
\AgdaSymbol{((}\AgdaBound{𝑓}\AgdaSpace{}%
\AgdaOperator{\AgdaFunction{̂}}\AgdaSpace{}%
\AgdaBound{𝑨}\AgdaSymbol{)}\AgdaSpace{}%
\AgdaBound{𝑎}\AgdaSymbol{)}\AgdaSpace{}%
\AgdaOperator{\AgdaDatatype{≡}}\AgdaSpace{}%
\AgdaSymbol{(}\AgdaBound{𝑓}\AgdaSpace{}%
\AgdaOperator{\AgdaFunction{̂}}\AgdaSpace{}%
\AgdaBound{𝑩}\AgdaSymbol{)}\AgdaSpace{}%
\AgdaSymbol{(}\AgdaBound{h}\AgdaSpace{}%
\AgdaOperator{\AgdaFunction{∘}}\AgdaSpace{}%
\AgdaBound{𝑎}\AgdaSymbol{)}\<%
\end{code}
\ccpad
Note the appearance of the shorthand \as ∀ \ab 𝑎 in the definition of \af{compatible-op-map}. We can get away with this in place of (\ab 𝑎 \as : \af ∥ \ab 𝑆 \af ∥ \ab 𝑓 \as → \af ∣ \ab 𝑨 \af ∣) since Agda is able to infer that
the \ab{𝑎} here must be a tuple on \af{∣} \ab 𝑨 \af ∣ of ``length'' \af{∥} \ab 𝑆 \af ∥ \ab 𝑓 (the arity of \ab{𝑓}).

We now define the type \af{hom}~\ab 𝑨~\ab 𝑩 of homomorphisms from \ab{𝑨} to \ab{𝑩} by first defining the property
\af{is-homomorphism}.
\ccpad
\begin{code}%
\>[0][@{}l@{\AgdaIndent{1}}]%
\>[1]\AgdaFunction{is-homomorphism}\AgdaSpace{}%
\AgdaSymbol{:}\AgdaSpace{}%
\AgdaSymbol{(}\AgdaBound{𝑨}\AgdaSpace{}%
\AgdaSymbol{:}\AgdaSpace{}%
\AgdaFunction{Algebra}\AgdaSpace{}%
\AgdaBound{𝓤}\AgdaSpace{}%
\AgdaBound{𝑆}\AgdaSymbol{)(}\AgdaBound{𝑩}\AgdaSpace{}%
\AgdaSymbol{:}\AgdaSpace{}%
\AgdaFunction{Algebra}\AgdaSpace{}%
\AgdaBound{𝓦}\AgdaSpace{}%
\AgdaBound{𝑆}\AgdaSymbol{)}\AgdaSpace{}%
\AgdaSymbol{→}\AgdaSpace{}%
\AgdaSymbol{(}\AgdaOperator{\AgdaFunction{∣}}\AgdaSpace{}%
\AgdaBound{𝑨}\AgdaSpace{}%
\AgdaOperator{\AgdaFunction{∣}}\AgdaSpace{}%
\AgdaSymbol{→}\AgdaSpace{}%
\AgdaOperator{\AgdaFunction{∣}}\AgdaSpace{}%
\AgdaBound{𝑩}\AgdaSpace{}%
\AgdaOperator{\AgdaFunction{∣}}\AgdaSymbol{)}\AgdaSpace{}%
\AgdaSymbol{→}\AgdaSpace{}%
\AgdaBound{𝓞}\AgdaSpace{}%
\AgdaOperator{\AgdaFunction{⊔}}\AgdaSpace{}%
\AgdaBound{𝓥}\AgdaSpace{}%
\AgdaOperator{\AgdaFunction{⊔}}\AgdaSpace{}%
\AgdaBound{𝓤}\AgdaSpace{}%
\AgdaOperator{\AgdaFunction{⊔}}\AgdaSpace{}%
\AgdaBound{𝓦}\AgdaSpace{}%
\AgdaOperator{\AgdaFunction{̇}}\<%
\\
%
\>[1]\AgdaFunction{is-homomorphism}\AgdaSpace{}%
\AgdaBound{𝑨}\AgdaSpace{}%
\AgdaBound{𝑩}\AgdaSpace{}%
\AgdaBound{g}\AgdaSpace{}%
\AgdaSymbol{=}\AgdaSpace{}%
\AgdaSymbol{∀}\AgdaSpace{}%
\AgdaBound{𝑓}\AgdaSpace{}%
\AgdaSpace{}\AgdaSymbol{→}\AgdaSpace{}%
\AgdaSpace{}\AgdaFunction{compatible-op-map}\AgdaSpace{}%
\AgdaBound{𝑨}\AgdaSpace{}%
\AgdaBound{𝑩}\AgdaSpace{}%
\AgdaBound{𝑓}\AgdaSpace{}%
\AgdaBound{g}\<%
\\
%
\\[\AgdaEmptyExtraSkip]%
%
\>[1]\AgdaFunction{hom}\AgdaSpace{}%
\AgdaSymbol{:}\AgdaSpace{}%
\AgdaFunction{Algebra}\AgdaSpace{}%
\AgdaBound{𝓤}\AgdaSpace{}%
\AgdaBound{𝑆}\AgdaSpace{}%
\AgdaSymbol{→}\AgdaSpace{}%
\AgdaFunction{Algebra}\AgdaSpace{}%
\AgdaBound{𝓦}\AgdaSpace{}%
\AgdaBound{𝑆}%
\>[34]\AgdaSymbol{→}\AgdaSpace{}%
\AgdaBound{𝓞}\AgdaSpace{}%
\AgdaOperator{\AgdaFunction{⊔}}\AgdaSpace{}%
\AgdaBound{𝓥}\AgdaSpace{}%
\AgdaOperator{\AgdaFunction{⊔}}\AgdaSpace{}%
\AgdaBound{𝓤}\AgdaSpace{}%
\AgdaOperator{\AgdaFunction{⊔}}\AgdaSpace{}%
\AgdaBound{𝓦}\AgdaSpace{}%
\AgdaOperator{\AgdaFunction{̇}}\<%
\\
%
\>[1]\AgdaFunction{hom}\AgdaSpace{}%
\AgdaBound{𝑨}\AgdaSpace{}%
\AgdaBound{𝑩}\AgdaSpace{}%
\AgdaSymbol{=}\AgdaSpace{}%
\AgdaFunction{Σ}\AgdaSpace{}%
\AgdaBound{g}\AgdaSpace{}%
\AgdaFunction{꞉}\AgdaSpace{}%
\AgdaSymbol{(}\AgdaOperator{\AgdaFunction{∣}}\AgdaSpace{}%
\AgdaBound{𝑨}\AgdaSpace{}%
\AgdaOperator{\AgdaFunction{∣}}\AgdaSpace{}%
\AgdaSymbol{→}\AgdaSpace{}%
\AgdaOperator{\AgdaFunction{∣}}\AgdaSpace{}%
\AgdaBound{𝑩}\AgdaSpace{}%
\AgdaOperator{\AgdaFunction{∣}}\AgdaSpace{}%
\AgdaSymbol{)}\AgdaSpace{}%
\AgdaFunction{,}\AgdaSpace{}%
\AgdaFunction{is-homomorphism}\AgdaSpace{}%
\AgdaBound{𝑨}\AgdaSpace{}%
\AgdaBound{𝑩}\AgdaSpace{}%
\AgdaBound{g}\<%
\end{code}

\subsubsection{Examples}
Here we give a few very special examples of homomorphisms. In each case, the function in question commutes with the basic operations of \emph{all} algebras and so, no matter the algebras involved, is always a homomorphism (trivially).

The most obvious example of a homomorphism is the identity map, which is proved to be a homomorphism as follows.
\ccpad
\begin{code}%
\>[0]\AgdaFunction{𝒾𝒹}\AgdaSpace{}%
\AgdaSymbol{:}\AgdaSpace{}%
\AgdaSymbol{\{}\AgdaBound{𝓤}\AgdaSpace{}%
\AgdaSymbol{:}\AgdaSpace{}%
\AgdaFunction{Universe}\AgdaSymbol{\}}\AgdaSpace{}%
\AgdaSymbol{(}\AgdaBound{A}\AgdaSpace{}%
\AgdaSymbol{:}\AgdaSpace{}%
\AgdaFunction{Algebra}\AgdaSpace{}%
\AgdaBound{𝓤}\AgdaSpace{}%
\AgdaBound{𝑆}\AgdaSymbol{)}\AgdaSpace{}%
\AgdaSymbol{→}\AgdaSpace{}%
\AgdaFunction{hom}\AgdaSpace{}%
\AgdaBound{A}\AgdaSpace{}%
\AgdaBound{A}\<%
\\
\>[0]\AgdaFunction{𝒾𝒹}\AgdaSpace{}%
\AgdaSymbol{\AgdaUnderscore{}}\AgdaSpace{}%
\AgdaSymbol{=}\AgdaSpace{}%
\AgdaSymbol{(λ}\AgdaSpace{}%
\AgdaBound{x}\AgdaSpace{}%
\AgdaSymbol{→}\AgdaSpace{}%
\AgdaBound{x}\AgdaSymbol{)}\AgdaSpace{}%
\AgdaOperator{\AgdaInductiveConstructor{,}}\AgdaSpace{}%
\AgdaSymbol{λ}\AgdaSpace{}%
\AgdaBound{\AgdaUnderscore{}}\AgdaSpace{}%
\AgdaBound{\AgdaUnderscore{}}\AgdaSpace{}%
\AgdaSymbol{→}\AgdaSpace{}%
\AgdaInductiveConstructor{𝓇ℯ𝒻𝓁}\<%
\\
%
\\[\AgdaEmptyExtraSkip]%
\>[0]\AgdaFunction{id-is-hom}\AgdaSpace{}%
\AgdaSymbol{:}\AgdaSpace{}%
\AgdaSymbol{\{}\AgdaBound{𝓤}\AgdaSpace{}%
\AgdaSymbol{:}\AgdaSpace{}%
\AgdaFunction{Universe}\AgdaSymbol{\}\{}\AgdaBound{𝑨}\AgdaSpace{}%
\AgdaSymbol{:}\AgdaSpace{}%
\AgdaFunction{Algebra}\AgdaSpace{}%
\AgdaBound{𝓤}\AgdaSpace{}%
\AgdaBound{𝑆}\AgdaSymbol{\}}\AgdaSpace{}%
\AgdaSymbol{→}\AgdaSpace{}%
\AgdaFunction{is-homomorphism}\AgdaSpace{}%
\AgdaBound{𝑨}\AgdaSpace{}%
\AgdaBound{𝑨}\AgdaSpace{}%
\AgdaSymbol{(}\AgdaFunction{𝑖𝑑}\AgdaSpace{}%
\AgdaOperator{\AgdaFunction{∣}}\AgdaSpace{}%
\AgdaBound{𝑨}\AgdaSpace{}%
\AgdaOperator{\AgdaFunction{∣}}\AgdaSymbol{)}\<%
\\
\>[0]\AgdaFunction{id-is-hom}\AgdaSpace{}%
\AgdaSymbol{=}\AgdaSpace{}%
\AgdaSymbol{λ}\AgdaSpace{}%
\AgdaBound{\AgdaUnderscore{}}\AgdaSpace{}%
\AgdaBound{\AgdaUnderscore{}}\AgdaSpace{}%
\AgdaSymbol{→}\AgdaSpace{}%
\AgdaInductiveConstructor{𝓇ℯ𝒻𝓁}\<%
\end{code}
\ccpad
Next, \aic{lift} and \afld{lower}, defined in the \ualibLifts module, are (the maps of) homomorphisms. Again, the proof is trivial in each case.
\ccpad
\begin{code}
\>[1]\AgdaFunction{lift-is-hom}\AgdaSpace{}%
\AgdaSymbol{:}\AgdaSpace{}%
\AgdaSymbol{\{}\AgdaBound{𝑨}\AgdaSpace{}%
\AgdaSymbol{:}\AgdaSpace{}%
\AgdaFunction{Algebra}\AgdaSpace{}%
\AgdaBound{𝓤}\AgdaSpace{}%
\AgdaBound{𝑆}\AgdaSymbol{\}\{}\AgdaBound{𝓦}\AgdaSpace{}%
\AgdaSymbol{:}\AgdaSpace{}%
\AgdaFunction{Universe}\AgdaSymbol{\}}\AgdaSpace{}%
\AgdaSymbol{→}\AgdaSpace{}%
\AgdaFunction{is-homomorphism}\AgdaSpace{}%
\AgdaBound{𝑨}\AgdaSpace{}%
\AgdaSymbol{(}\AgdaFunction{lift-alg}\AgdaSpace{}%
\AgdaBound{𝑨}\AgdaSpace{}%
\AgdaBound{𝓦}\AgdaSymbol{)}\AgdaSpace{}%
\AgdaInductiveConstructor{lift}\<%
\\
%
\>[1]\AgdaFunction{lift-is-hom}\AgdaSpace{}%
\AgdaSymbol{\AgdaUnderscore{}}\AgdaSpace{}%
\AgdaSymbol{\AgdaUnderscore{}}\AgdaSpace{}%
\AgdaSymbol{=}\AgdaSpace{}%
\AgdaInductiveConstructor{𝓇ℯ𝒻𝓁}\<%
\\
%
\\[\AgdaEmptyExtraSkip]%
%
\>[1]\AgdaFunction{𝓁𝒾𝒻𝓉}\AgdaSpace{}%
\AgdaSymbol{:}\AgdaSpace{}%
\AgdaSymbol{\{}\AgdaBound{𝑨}\AgdaSpace{}%
\AgdaSymbol{:}\AgdaSpace{}%
\AgdaFunction{Algebra}\AgdaSpace{}%
\AgdaBound{𝓤}\AgdaSpace{}%
\AgdaBound{𝑆}\AgdaSymbol{\}\{}\AgdaBound{𝓦}\AgdaSpace{}%
\AgdaSymbol{:}\AgdaSpace{}%
\AgdaFunction{Universe}\AgdaSymbol{\}}\AgdaSpace{}%
\AgdaSymbol{→}\AgdaSpace{}%
\AgdaFunction{hom}\AgdaSpace{}%
\AgdaBound{𝑨}\AgdaSpace{}%
\AgdaSymbol{(}\AgdaFunction{lift-alg}\AgdaSpace{}%
\AgdaBound{𝑨}\AgdaSpace{}%
\AgdaBound{𝓦}\AgdaSymbol{)}\<%
\\
%
\>[1]\AgdaFunction{𝓁𝒾𝒻𝓉}\AgdaSpace{}%
\AgdaSymbol{=}\AgdaSpace{}%
\AgdaSymbol{(}\AgdaInductiveConstructor{lift}\AgdaSpace{}%
\AgdaOperator{\AgdaInductiveConstructor{,}}\AgdaSpace{}%
\AgdaFunction{lift-is-hom}\AgdaSymbol{)}\<%
\\
%
\\[\AgdaEmptyExtraSkip]%
%
\>[1]\AgdaFunction{lower-is-hom}\AgdaSpace{}%
\AgdaSymbol{:}\AgdaSpace{}%
\AgdaSymbol{\{}\AgdaBound{𝑨}\AgdaSpace{}%
\AgdaSymbol{:}\AgdaSpace{}%
\AgdaFunction{Algebra}\AgdaSpace{}%
\AgdaBound{𝓤}\AgdaSpace{}%
\AgdaBound{𝑆}\AgdaSymbol{\}\{}\AgdaBound{𝓦}\AgdaSpace{}%
\AgdaSymbol{:}\AgdaSpace{}%
\AgdaFunction{Universe}\AgdaSymbol{\}}\AgdaSpace{}%
\AgdaSymbol{→}\AgdaSpace{}%
\AgdaFunction{is-homomorphism}\AgdaSpace{}%
\AgdaSymbol{(}\AgdaFunction{lift-alg}\AgdaSpace{}%
\AgdaBound{𝑨}\AgdaSpace{}%
\AgdaBound{𝓦}\AgdaSymbol{)}\AgdaSpace{}%
\AgdaBound{𝑨}\AgdaSpace{}%
\AgdaField{lower}\<%
\\
%
\>[1]\AgdaFunction{lower-is-hom}\AgdaSpace{}%
\AgdaSymbol{\AgdaUnderscore{}}\AgdaSpace{}%
\AgdaSymbol{\AgdaUnderscore{}}\AgdaSpace{}%
\AgdaSymbol{=}\AgdaSpace{}%
\AgdaInductiveConstructor{𝓇ℯ𝒻𝓁}\<%
\\
%
\\[\AgdaEmptyExtraSkip]%
%
\>[1]\AgdaFunction{𝓁ℴ𝓌ℯ𝓇}\AgdaSpace{}%
\AgdaSymbol{:}\AgdaSpace{}%
\AgdaSymbol{(}\AgdaBound{𝑨}\AgdaSpace{}%
\AgdaSymbol{:}\AgdaSpace{}%
\AgdaFunction{Algebra}\AgdaSpace{}%
\AgdaBound{𝓤}\AgdaSpace{}%
\AgdaBound{𝑆}\AgdaSymbol{)\{}\AgdaBound{𝓦}\AgdaSpace{}%
\AgdaSymbol{:}\AgdaSpace{}%
\AgdaFunction{Universe}\AgdaSymbol{\}}\AgdaSpace{}%
\AgdaSymbol{→}\AgdaSpace{}%
\AgdaFunction{hom}\AgdaSpace{}%
\AgdaSymbol{(}\AgdaFunction{lift-alg}\AgdaSpace{}%
\AgdaBound{𝑨}\AgdaSpace{}%
\AgdaBound{𝓦}\AgdaSymbol{)}\AgdaSpace{}%
\AgdaBound{𝑨}\<%
\\
%
\>[1]\AgdaFunction{𝓁ℴ𝓌ℯ𝓇}\AgdaSpace{}%
\AgdaBound{𝑨}\AgdaSpace{}%
\AgdaSymbol{=}\AgdaSpace{}%
\AgdaSymbol{(}\AgdaField{lower}\AgdaSpace{}%
\AgdaOperator{\AgdaInductiveConstructor{,}}\AgdaSpace{}%
\AgdaFunction{lower-is-hom}\AgdaSymbol{\{}\AgdaBound{𝑨}\AgdaSymbol{\})}\<%
\end{code}

\subsubsection{Monomorphisms and epimorphisms}\label{sec:monom-epim}

We represent \defn{monomorphisms} (injective homomorphisms) and \defn{epimorphisms} (surjective homomorphisms) by the following types.
\ccpad
\begin{code}%
\>[1]\AgdaFunction{is-monomorphism}\AgdaSpace{}%
\AgdaSymbol{:}\AgdaSpace{}%
\AgdaSymbol{(}\AgdaBound{𝑨}\AgdaSpace{}%
\AgdaSymbol{:}\AgdaSpace{}%
\AgdaFunction{Algebra}\AgdaSpace{}%
\AgdaBound{𝓤}\AgdaSpace{}%
\AgdaBound{𝑆}\AgdaSymbol{)(}\AgdaBound{𝑩}\AgdaSpace{}%
\AgdaSymbol{:}\AgdaSpace{}%
\AgdaFunction{Algebra}\AgdaSpace{}%
\AgdaBound{𝓦}\AgdaSpace{}%
\AgdaBound{𝑆}\AgdaSymbol{)}\AgdaSpace{}%
\AgdaSymbol{→}\AgdaSpace{}%
\AgdaSymbol{(}\AgdaOperator{\AgdaFunction{∣}}\AgdaSpace{}%
\AgdaBound{𝑨}\AgdaSpace{}%
\AgdaOperator{\AgdaFunction{∣}}\AgdaSpace{}%
\AgdaSymbol{→}\AgdaSpace{}%
\AgdaOperator{\AgdaFunction{∣}}\AgdaSpace{}%
\AgdaBound{𝑩}\AgdaSpace{}%
\AgdaOperator{\AgdaFunction{∣}}\AgdaSymbol{)}\AgdaSpace{}%
\AgdaSymbol{→}\AgdaSpace{}%
\AgdaBound{𝓞}\AgdaSpace{}%
\AgdaOperator{\AgdaFunction{⊔}}\AgdaSpace{}%
\AgdaBound{𝓥}\AgdaSpace{}%
\AgdaOperator{\AgdaFunction{⊔}}\AgdaSpace{}%
\AgdaBound{𝓤}\AgdaSpace{}%
\AgdaOperator{\AgdaFunction{⊔}}\AgdaSpace{}%
\AgdaBound{𝓦}\AgdaSpace{}%
\AgdaOperator{\AgdaFunction{̇}}\<%
\\
%
\>[1]\AgdaFunction{is-monomorphism}\AgdaSpace{}%
\AgdaBound{𝑨}\AgdaSpace{}%
\AgdaBound{𝑩}\AgdaSpace{}%
\AgdaBound{g}\AgdaSpace{}%
\AgdaSymbol{=}\AgdaSpace{}%
\AgdaFunction{is-homomorphism}\AgdaSpace{}%
\AgdaBound{𝑨}\AgdaSpace{}%
\AgdaBound{𝑩}\AgdaSpace{}%
\AgdaBound{g}\AgdaSpace{}%
\AgdaOperator{\AgdaFunction{×}}\AgdaSpace{}%
\AgdaFunction{Monic}\AgdaSpace{}%
\AgdaBound{g}\<%
\\
%
\\[\AgdaEmptyExtraSkip]%
%
\>[1]\AgdaFunction{mon}\AgdaSpace{}%
\AgdaSymbol{:}\AgdaSpace{}%
\AgdaFunction{Algebra}\AgdaSpace{}%
\AgdaBound{𝓤}\AgdaSpace{}%
\AgdaBound{𝑆}\AgdaSpace{}%
\AgdaSymbol{→}\AgdaSpace{}%
\AgdaFunction{Algebra}\AgdaSpace{}%
\AgdaBound{𝓦}\AgdaSpace{}%
\AgdaBound{𝑆}%
\>[34]\AgdaSymbol{→}\AgdaSpace{}%
\AgdaBound{𝓞}\AgdaSpace{}%
\AgdaOperator{\AgdaFunction{⊔}}\AgdaSpace{}%
\AgdaBound{𝓥}\AgdaSpace{}%
\AgdaOperator{\AgdaFunction{⊔}}\AgdaSpace{}%
\AgdaBound{𝓤}\AgdaSpace{}%
\AgdaOperator{\AgdaFunction{⊔}}\AgdaSpace{}%
\AgdaBound{𝓦}\AgdaSpace{}%
\AgdaOperator{\AgdaFunction{̇}}\<%
\\
%
\>[1]\AgdaFunction{mon}\AgdaSpace{}%
\AgdaBound{𝑨}\AgdaSpace{}%
\AgdaBound{𝑩}\AgdaSpace{}%
\AgdaSymbol{=}\AgdaSpace{}%
\AgdaFunction{Σ}\AgdaSpace{}%
\AgdaBound{g}\AgdaSpace{}%
\AgdaFunction{꞉}\AgdaSpace{}%
\AgdaSymbol{(}\AgdaOperator{\AgdaFunction{∣}}\AgdaSpace{}%
\AgdaBound{𝑨}\AgdaSpace{}%
\AgdaOperator{\AgdaFunction{∣}}\AgdaSpace{}%
\AgdaSymbol{→}\AgdaSpace{}%
\AgdaOperator{\AgdaFunction{∣}}\AgdaSpace{}%
\AgdaBound{𝑩}\AgdaSpace{}%
\AgdaOperator{\AgdaFunction{∣}}\AgdaSymbol{)}\AgdaSpace{}%
\AgdaFunction{,}\AgdaSpace{}%
\AgdaFunction{is-monomorphism}\AgdaSpace{}%
\AgdaBound{𝑨}\AgdaSpace{}%
\AgdaBound{𝑩}\AgdaSpace{}%
\AgdaBound{g}\<%
\\
%
\\[\AgdaEmptyExtraSkip]%
%
\>[1]\AgdaFunction{is-epimorphism}\AgdaSpace{}%
\AgdaSymbol{:}\AgdaSpace{}%
\AgdaSymbol{(}\AgdaBound{𝑨}\AgdaSpace{}%
\AgdaSymbol{:}\AgdaSpace{}%
\AgdaFunction{Algebra}\AgdaSpace{}%
\AgdaBound{𝓤}\AgdaSpace{}%
\AgdaBound{𝑆}\AgdaSymbol{)(}\AgdaBound{𝑩}\AgdaSpace{}%
\AgdaSymbol{:}\AgdaSpace{}%
\AgdaFunction{Algebra}\AgdaSpace{}%
\AgdaBound{𝓦}\AgdaSpace{}%
\AgdaBound{𝑆}\AgdaSymbol{)}\AgdaSpace{}%
\AgdaSymbol{→}\AgdaSpace{}%
\AgdaSymbol{(}\AgdaOperator{\AgdaFunction{∣}}\AgdaSpace{}%
\AgdaBound{𝑨}\AgdaSpace{}%
\AgdaOperator{\AgdaFunction{∣}}\AgdaSpace{}%
\AgdaSymbol{→}\AgdaSpace{}%
\AgdaOperator{\AgdaFunction{∣}}\AgdaSpace{}%
\AgdaBound{𝑩}\AgdaSpace{}%
\AgdaOperator{\AgdaFunction{∣}}\AgdaSymbol{)}\AgdaSpace{}%
\AgdaSymbol{→}\AgdaSpace{}%
\AgdaBound{𝓞}\AgdaSpace{}%
\AgdaOperator{\AgdaFunction{⊔}}\AgdaSpace{}%
\AgdaBound{𝓥}\AgdaSpace{}%
\AgdaOperator{\AgdaFunction{⊔}}\AgdaSpace{}%
\AgdaBound{𝓤}\AgdaSpace{}%
\AgdaOperator{\AgdaFunction{⊔}}\AgdaSpace{}%
\AgdaBound{𝓦}\AgdaSpace{}%
\AgdaOperator{\AgdaFunction{̇}}\<%
\\
%
\>[1]\AgdaFunction{is-epimorphism}\AgdaSpace{}%
\AgdaBound{𝑨}\AgdaSpace{}%
\AgdaBound{𝑩}\AgdaSpace{}%
\AgdaBound{g}\AgdaSpace{}%
\AgdaSymbol{=}\AgdaSpace{}%
\AgdaFunction{is-homomorphism}\AgdaSpace{}%
\AgdaBound{𝑨}\AgdaSpace{}%
\AgdaBound{𝑩}\AgdaSpace{}%
\AgdaBound{g}\AgdaSpace{}%
\AgdaOperator{\AgdaFunction{×}}\AgdaSpace{}%
\AgdaFunction{Epic}\AgdaSpace{}%
\AgdaBound{g}\<%
\\
%
\\[\AgdaEmptyExtraSkip]%
%
\>[1]\AgdaFunction{epi}\AgdaSpace{}%
\AgdaSymbol{:}\AgdaSpace{}%
\AgdaFunction{Algebra}\AgdaSpace{}%
\AgdaBound{𝓤}\AgdaSpace{}%
\AgdaBound{𝑆}\AgdaSpace{}%
\AgdaSymbol{→}\AgdaSpace{}%
\AgdaFunction{Algebra}\AgdaSpace{}%
\AgdaBound{𝓦}\AgdaSpace{}%
\AgdaBound{𝑆}%
\>[34]\AgdaSymbol{→}\AgdaSpace{}%
\AgdaBound{𝓞}\AgdaSpace{}%
\AgdaOperator{\AgdaFunction{⊔}}\AgdaSpace{}%
\AgdaBound{𝓥}\AgdaSpace{}%
\AgdaOperator{\AgdaFunction{⊔}}\AgdaSpace{}%
\AgdaBound{𝓤}\AgdaSpace{}%
\AgdaOperator{\AgdaFunction{⊔}}\AgdaSpace{}%
\AgdaBound{𝓦}\AgdaSpace{}%
\AgdaOperator{\AgdaFunction{̇}}\<%
\\
%
\>[1]\AgdaFunction{epi}\AgdaSpace{}%
\AgdaBound{𝑨}\AgdaSpace{}%
\AgdaBound{𝑩}\AgdaSpace{}%
\AgdaSymbol{=}\AgdaSpace{}%
\AgdaFunction{Σ}\AgdaSpace{}%
\AgdaBound{g}\AgdaSpace{}%
\AgdaFunction{꞉}\AgdaSpace{}%
\AgdaSymbol{(}\AgdaOperator{\AgdaFunction{∣}}\AgdaSpace{}%
\AgdaBound{𝑨}\AgdaSpace{}%
\AgdaOperator{\AgdaFunction{∣}}\AgdaSpace{}%
\AgdaSymbol{→}\AgdaSpace{}%
\AgdaOperator{\AgdaFunction{∣}}\AgdaSpace{}%
\AgdaBound{𝑩}\AgdaSpace{}%
\AgdaOperator{\AgdaFunction{∣}}\AgdaSymbol{)}\AgdaSpace{}%
\AgdaFunction{,}\AgdaSpace{}%
\AgdaFunction{is-epimorphism}\AgdaSpace{}%
\AgdaBound{𝑨}\AgdaSpace{}%
\AgdaBound{𝑩}\AgdaSpace{}%
\AgdaBound{g}\<%
\end{code}
\ccpad
It will be convenient to have a function that takes an inhabitant of `mon` (or `epi`) and extracts the homomorphism part, or *hom reduct*, that is, the pair consisting of the map and a proof that the map is a homomorphism. 
\ccpad
\begin{code}%
\>[0][@{}l@{\AgdaIndent{1}}]%
\>[1]\AgdaFunction{mon-to-hom}\AgdaSpace{}%
\AgdaSymbol{:}\AgdaSpace{}%
\AgdaSymbol{(}\AgdaBound{𝑨}\AgdaSpace{}%
\AgdaSymbol{:}\AgdaSpace{}%
\AgdaFunction{Algebra}\AgdaSpace{}%
\AgdaBound{𝓤}\AgdaSpace{}%
\AgdaBound{𝑆}\AgdaSymbol{)\{}\AgdaBound{𝑩}\AgdaSpace{}%
\AgdaSymbol{:}\AgdaSpace{}%
\AgdaFunction{Algebra}\AgdaSpace{}%
\AgdaBound{𝓦}\AgdaSpace{}%
\AgdaBound{𝑆}\AgdaSymbol{\}}\AgdaSpace{}%
\AgdaSymbol{→}\AgdaSpace{}%
\AgdaFunction{mon}\AgdaSpace{}%
\AgdaBound{𝑨}\AgdaSpace{}%
\AgdaBound{𝑩}\AgdaSpace{}%
\AgdaSymbol{→}\AgdaSpace{}%
\AgdaFunction{hom}\AgdaSpace{}%
\AgdaBound{𝑨}\AgdaSpace{}%
\AgdaBound{𝑩}\<%
\\
%
\>[1]\AgdaFunction{mon-to-hom}\AgdaSpace{}%
\AgdaBound{𝑨}\AgdaSpace{}%
\AgdaBound{ϕ}\AgdaSpace{}%
\AgdaSymbol{=}\AgdaSpace{}%
\AgdaOperator{\AgdaFunction{∣}}\AgdaSpace{}%
\AgdaBound{ϕ}\AgdaSpace{}%
\AgdaOperator{\AgdaFunction{∣}}\AgdaSpace{}%
\AgdaOperator{\AgdaInductiveConstructor{,}}\AgdaSpace{}%
\AgdaFunction{fst}\AgdaSpace{}%
\AgdaOperator{\AgdaFunction{∥}}\AgdaSpace{}%
\AgdaBound{ϕ}\AgdaSpace{}%
\AgdaOperator{\AgdaFunction{∥}}\<%
\\
%
\\[\AgdaEmptyExtraSkip]%
%
\>[1]\AgdaFunction{epi-to-hom}\AgdaSpace{}%
\AgdaSymbol{:}\AgdaSpace{}%
\AgdaSymbol{\{}\AgdaBound{𝑨}\AgdaSpace{}%
\AgdaSymbol{:}\AgdaSpace{}%
\AgdaFunction{Algebra}\AgdaSpace{}%
\AgdaBound{𝓤}\AgdaSpace{}%
\AgdaBound{𝑆}\AgdaSymbol{\}(}\AgdaBound{𝑩}\AgdaSpace{}%
\AgdaSymbol{:}\AgdaSpace{}%
\AgdaFunction{Algebra}\AgdaSpace{}%
\AgdaBound{𝓦}\AgdaSpace{}%
\AgdaBound{𝑆}\AgdaSymbol{)}\AgdaSpace{}%
\AgdaSymbol{→}\AgdaSpace{}%
\AgdaFunction{epi}\AgdaSpace{}%
\AgdaBound{𝑨}\AgdaSpace{}%
\AgdaBound{𝑩}\AgdaSpace{}%
\AgdaSymbol{→}\AgdaSpace{}%
\AgdaFunction{hom}\AgdaSpace{}%
\AgdaBound{𝑨}\AgdaSpace{}%
\AgdaBound{𝑩}\<%
\\
%
\>[1]\AgdaFunction{epi-to-hom}\AgdaSpace{}%
\AgdaSymbol{\AgdaUnderscore{}}\AgdaSpace{}%
\AgdaBound{ϕ}\AgdaSpace{}%
\AgdaSymbol{=}\AgdaSpace{}%
\AgdaOperator{\AgdaFunction{∣}}\AgdaSpace{}%
\AgdaBound{ϕ}\AgdaSpace{}%
\AgdaOperator{\AgdaFunction{∣}}\AgdaSpace{}%
\AgdaOperator{\AgdaInductiveConstructor{,}}\AgdaSpace{}%
\AgdaFunction{fst}\AgdaSpace{}%
\AgdaOperator{\AgdaFunction{∥}}\AgdaSpace{}%
\AgdaBound{ϕ}\AgdaSpace{}%
\AgdaOperator{\AgdaFunction{∥}}\<%
\end{code}

\subsubsection{Equalizers in Agda}\label{equalizers-in-agda}

Recall, the equalizer of two functions (resp., homomorphisms) \ab{g h} \as : \ab A \as → \ab B is the subset of \ab{A} on which the values of the functions \ab{g} and \ab{h} agree. We define the equalizer of functions and homomorphisms in the \ualib as follows.
\ccpad
\begin{code}%
\>[1]\AgdaFunction{𝐸}\AgdaSpace{}%
\AgdaSymbol{:}\AgdaSpace{}%
\AgdaSymbol{\{}\AgdaBound{𝑩}\AgdaSpace{}%
\AgdaSymbol{:}\AgdaSpace{}%
\AgdaFunction{Algebra}\AgdaSpace{}%
\AgdaBound{𝓦}\AgdaSpace{}%
\AgdaBound{𝑆}\AgdaSymbol{\}(}\AgdaBound{g}\AgdaSpace{}%
\AgdaBound{h}\AgdaSpace{}%
\AgdaSymbol{:}\AgdaSpace{}%
\AgdaOperator{\AgdaFunction{∣}}\AgdaSpace{}%
\AgdaBound{𝑨}\AgdaSpace{}%
\AgdaOperator{\AgdaFunction{∣}}\AgdaSpace{}%
\AgdaSymbol{→}\AgdaSpace{}%
\AgdaOperator{\AgdaFunction{∣}}\AgdaSpace{}%
\AgdaBound{𝑩}\AgdaSpace{}%
\AgdaOperator{\AgdaFunction{∣}}\AgdaSymbol{)}\AgdaSpace{}%
\AgdaSymbol{→}\AgdaSpace{}%
\AgdaFunction{Pred}\AgdaSpace{}%
\AgdaOperator{\AgdaFunction{∣}}\AgdaSpace{}%
\AgdaBound{𝑨}\AgdaSpace{}%
\AgdaOperator{\AgdaFunction{∣}}\AgdaSpace{}%
\AgdaBound{𝓦}\<%
\\
%
\>[1]\AgdaFunction{𝐸}\AgdaSpace{}%
\AgdaBound{g}\AgdaSpace{}%
\AgdaBound{h}\AgdaSpace{}%
\AgdaBound{x}\AgdaSpace{}%
\AgdaSymbol{=}\AgdaSpace{}%
\AgdaBound{g}\AgdaSpace{}%
\AgdaBound{x}\AgdaSpace{}%
\AgdaOperator{\AgdaDatatype{≡}}\AgdaSpace{}%
\AgdaBound{h}\AgdaSpace{}%
\AgdaBound{x}\<%
\\
%
\\[\AgdaEmptyExtraSkip]%
%
\>[1]\AgdaFunction{𝐸hom}\AgdaSpace{}%
\AgdaSymbol{:}\AgdaSpace{}%
\AgdaSymbol{(}\AgdaBound{𝑩}\AgdaSpace{}%
\AgdaSymbol{:}\AgdaSpace{}%
\AgdaFunction{Algebra}\AgdaSpace{}%
\AgdaBound{𝓦}\AgdaSpace{}%
\AgdaBound{𝑆}\AgdaSymbol{)(}\AgdaBound{g}\AgdaSpace{}%
\AgdaBound{h}\AgdaSpace{}%
\AgdaSymbol{:}\AgdaSpace{}%
\AgdaFunction{hom}\AgdaSpace{}%
\AgdaBound{𝑨}\AgdaSpace{}%
\AgdaBound{𝑩}\AgdaSymbol{)}\AgdaSpace{}%
\AgdaSymbol{→}\AgdaSpace{}%
\AgdaFunction{Pred}\AgdaSpace{}%
\AgdaOperator{\AgdaFunction{∣}}\AgdaSpace{}%
\AgdaBound{𝑨}\AgdaSpace{}%
\AgdaOperator{\AgdaFunction{∣}}\AgdaSpace{}%
\AgdaBound{𝓦}\<%
\\
%
\>[1]\AgdaFunction{𝐸hom}\AgdaSpace{}%
\AgdaSymbol{\AgdaUnderscore{}}\AgdaSpace{}%
\AgdaBound{g}\AgdaSpace{}%
\AgdaBound{h}\AgdaSpace{}%
\AgdaBound{x}\AgdaSpace{}%
\AgdaSymbol{=}\AgdaSpace{}%
\AgdaOperator{\AgdaFunction{∣}}\AgdaSpace{}%
\AgdaBound{g}\AgdaSpace{}%
\AgdaOperator{\AgdaFunction{∣}}\AgdaSpace{}%
\AgdaBound{x}\AgdaSpace{}%
\AgdaOperator{\AgdaDatatype{≡}}\AgdaSpace{}%
\AgdaOperator{\AgdaFunction{∣}}\AgdaSpace{}%
\AgdaBound{h}\AgdaSpace{}%
\AgdaOperator{\AgdaFunction{∣}}\AgdaSpace{}%
\AgdaBound{x}\<%
\end{code}
\ccpad
We will define subuniverses in the \ualibSubuniverses module, but we note here that the equalizer of homomorphisms from \ab{𝑨} to \ab{𝑩} will turn out to be subuniverse of \ab{𝑨}. Indeed, this is easily proved as follows.
\ccpad
\begin{code}%
\>[1]\AgdaFunction{𝐸hom-closed}\AgdaSpace{}%
\AgdaSymbol{:}\AgdaSpace{}%
\>[494I]\AgdaSymbol{(}\AgdaBound{𝑩}\AgdaSpace{}%
\AgdaSymbol{:}\AgdaSpace{}%
\AgdaFunction{Algebra}\AgdaSpace{}%
\AgdaBound{𝓦}\AgdaSpace{}%
\AgdaBound{𝑆}\AgdaSymbol{)(}\AgdaBound{g}\AgdaSpace{}%
\AgdaBound{h}\AgdaSpace{}%
\AgdaSymbol{:}\AgdaSpace{}%
\AgdaFunction{hom}\AgdaSpace{}%
\AgdaBound{𝑨}\AgdaSpace{}%
\AgdaBound{𝑩}\AgdaSymbol{)}\<%
\\
\>[1][@{}l@{\AgdaIndent{0}}]%
\>[2]\AgdaSymbol{→}%
\>[.][@{}l@{}]\<[494I]%
\>[15]\AgdaSymbol{∀}\AgdaSpace{}%
\AgdaBound{𝑓}\AgdaSpace{}%
\AgdaBound{a}\AgdaSpace{}%
\AgdaSymbol{→}\AgdaSpace{}%
\AgdaSymbol{(∀}\AgdaSpace{}%
\AgdaBound{x}\AgdaSpace{}%
\AgdaSymbol{→}\AgdaSpace{}%
\AgdaBound{a}\AgdaSpace{}%
\AgdaBound{x}\AgdaSpace{}%
\AgdaOperator{\AgdaFunction{∈}}\AgdaSpace{}%
\AgdaFunction{𝐸hom}\AgdaSpace{}%
\AgdaBound{𝑩}\AgdaSpace{}%
\AgdaBound{g}\AgdaSpace{}%
\AgdaBound{h}\AgdaSymbol{)}\<%
\\
%
\>[15]\AgdaComment{------------------------------------------------------}\<%
\\
%
\>[2]\AgdaSymbol{→}%
\>[15]\AgdaOperator{\AgdaFunction{∣}}\AgdaSpace{}%
\AgdaBound{g}\AgdaSpace{}%
\AgdaOperator{\AgdaFunction{∣}}\AgdaSpace{}%
\AgdaSymbol{((}\AgdaBound{𝑓}\AgdaSpace{}%
\AgdaOperator{\AgdaFunction{̂}}\AgdaSpace{}%
\AgdaBound{𝑨}\AgdaSymbol{)}\AgdaSpace{}%
\AgdaBound{a}\AgdaSymbol{)}\AgdaSpace{}%
\AgdaOperator{\AgdaDatatype{≡}}\AgdaSpace{}%
\AgdaOperator{\AgdaFunction{∣}}\AgdaSpace{}%
\AgdaBound{h}\AgdaSpace{}%
\AgdaOperator{\AgdaFunction{∣}}\AgdaSpace{}%
\AgdaSymbol{((}\AgdaBound{𝑓}\AgdaSpace{}%
\AgdaOperator{\AgdaFunction{̂}}\AgdaSpace{}%
\AgdaBound{𝑨}\AgdaSymbol{)}\AgdaSpace{}%
\AgdaBound{a}\AgdaSymbol{)}\<%
\\
%
\\[\AgdaEmptyExtraSkip]%
%
\>[1]\AgdaFunction{𝐸hom-closed}\AgdaSpace{}%
\AgdaBound{𝑩}\AgdaSpace{}%
\AgdaBound{g}\AgdaSpace{}%
\AgdaBound{h}\AgdaSpace{}%
\AgdaBound{𝑓}\AgdaSpace{}%
\AgdaBound{a}\AgdaSpace{}%
\AgdaBound{p}\AgdaSpace{}%
\AgdaSymbol{=}%
\>[538I]\AgdaOperator{\AgdaFunction{∣}}\AgdaSpace{}%
\AgdaBound{g}\AgdaSpace{}%
\AgdaOperator{\AgdaFunction{∣}}\AgdaSpace{}%
\AgdaSymbol{((}\AgdaBound{𝑓}\AgdaSpace{}%
\AgdaOperator{\AgdaFunction{̂}}\AgdaSpace{}%
\AgdaBound{𝑨}\AgdaSymbol{)}\AgdaSpace{}%
\AgdaBound{a}\AgdaSymbol{)}%
\>[47]\AgdaOperator{\AgdaFunction{≡⟨}}\AgdaSpace{}%
\AgdaOperator{\AgdaFunction{∥}}\AgdaSpace{}%
\AgdaBound{g}\AgdaSpace{}%
\AgdaOperator{\AgdaFunction{∥}}\AgdaSpace{}%
\AgdaBound{𝑓}\AgdaSpace{}%
\AgdaBound{a}\AgdaSpace{}%
\AgdaOperator{\AgdaFunction{⟩}}\<%
\\
\>[.][@{}l@{}]\<[538I]%
\>[27]\AgdaSymbol{(}\AgdaBound{𝑓}\AgdaSpace{}%
\AgdaOperator{\AgdaFunction{̂}}\AgdaSpace{}%
\AgdaBound{𝑩}\AgdaSymbol{)(}\AgdaOperator{\AgdaFunction{∣}}\AgdaSpace{}%
\AgdaBound{g}\AgdaSpace{}%
\AgdaOperator{\AgdaFunction{∣}}\AgdaSpace{}%
\AgdaOperator{\AgdaFunction{∘}}\AgdaSpace{}%
\AgdaBound{a}\AgdaSymbol{)}%
\>[47]\AgdaOperator{\AgdaFunction{≡⟨}}\AgdaSpace{}%
\AgdaFunction{ap}\AgdaSpace{}%
\AgdaSymbol{(}\AgdaBound{𝑓}\AgdaSpace{}%
\AgdaOperator{\AgdaFunction{̂}}\AgdaSpace{}%
\AgdaBound{𝑩}\AgdaSymbol{)(}\AgdaBound{gfe}\AgdaSpace{}%
\AgdaBound{p}\AgdaSymbol{)}\AgdaSpace{}%
\AgdaOperator{\AgdaFunction{⟩}}\<%
\\
%
\>[27]\AgdaSymbol{(}\AgdaBound{𝑓}\AgdaSpace{}%
\AgdaOperator{\AgdaFunction{̂}}\AgdaSpace{}%
\AgdaBound{𝑩}\AgdaSymbol{)(}\AgdaOperator{\AgdaFunction{∣}}\AgdaSpace{}%
\AgdaBound{h}\AgdaSpace{}%
\AgdaOperator{\AgdaFunction{∣}}\AgdaSpace{}%
\AgdaOperator{\AgdaFunction{∘}}\AgdaSpace{}%
\AgdaBound{a}\AgdaSymbol{)}%
\>[47]\AgdaOperator{\AgdaFunction{≡⟨}}\AgdaSpace{}%
\AgdaSymbol{(}\AgdaOperator{\AgdaFunction{∥}}\AgdaSpace{}%
\AgdaBound{h}\AgdaSpace{}%
\AgdaOperator{\AgdaFunction{∥}}\AgdaSpace{}%
\AgdaBound{𝑓}\AgdaSpace{}%
\AgdaBound{a}\AgdaSymbol{)}\AgdaOperator{\AgdaFunction{⁻¹}}\AgdaSpace{}%
\AgdaOperator{\AgdaFunction{⟩}}\<%
\\
%
\>[27]\AgdaOperator{\AgdaFunction{∣}}\AgdaSpace{}%
\AgdaBound{h}\AgdaSpace{}%
\AgdaOperator{\AgdaFunction{∣}}\AgdaSpace{}%
\AgdaSymbol{((}\AgdaBound{𝑓}\AgdaSpace{}%
\AgdaOperator{\AgdaFunction{̂}}\AgdaSpace{}%
\AgdaBound{𝑨}\AgdaSymbol{)}\AgdaSpace{}%
\AgdaBound{a}\AgdaSymbol{)}%
\>[47]\AgdaOperator{\AgdaFunction{∎}}\<%
\end{code}

\subsubsection{Kernels of Homomorphisms}\label{kernels-of-homomorphisms}

The \defn{kernel} of a homomorphism is a congruence relation and conversely for every congruence relation \ab θ, there exists a homomorphism with kernel \ab θ (namely, that canonical projection onto the quotient modulo \ab θ).
\ccpad
\begin{code}%
\>[1]\AgdaFunction{homker-compatible}\AgdaSpace{}%
\AgdaSymbol{:}\AgdaSpace{}%
\>[99I]\AgdaSymbol{\{}\AgdaBound{𝑨}\AgdaSpace{}%
\AgdaSymbol{:}\AgdaSpace{}%
\AgdaFunction{Algebra}\AgdaSpace{}%
\AgdaBound{𝓤}\AgdaSpace{}%
\AgdaBound{𝑆}\AgdaSymbol{\}(}\AgdaBound{𝑩}\AgdaSpace{}%
\AgdaSymbol{:}\AgdaSpace{}%
\AgdaFunction{Algebra}\AgdaSpace{}%
\AgdaBound{𝓦}\AgdaSpace{}%
\AgdaBound{𝑆}\AgdaSymbol{)(}\AgdaBound{h}\AgdaSpace{}%
\AgdaSymbol{:}\AgdaSpace{}%
\AgdaFunction{hom}\AgdaSpace{}%
\AgdaBound{𝑨}\AgdaSpace{}%
\AgdaBound{𝑩}\AgdaSymbol{)}\<%
\\
\>[1][@{}l@{\AgdaIndent{0}}]%
\>[2]\AgdaSymbol{→}%
\>[.][@{}l@{}]\<[99I]%
\>[21]\AgdaFunction{compatible}\AgdaSpace{}%
\AgdaBound{𝑨}\AgdaSpace{}%
\AgdaSymbol{(}\AgdaFunction{KER-rel}\AgdaSpace{}%
\AgdaOperator{\AgdaFunction{∣}}\AgdaSpace{}%
\AgdaBound{h}\AgdaSpace{}%
\AgdaOperator{\AgdaFunction{∣}}\AgdaSymbol{)}\<%
\\
%
\\[\AgdaEmptyExtraSkip]%
%
\>[1]\AgdaFunction{homker-compatible}\AgdaSpace{}%
\AgdaSymbol{\{}\AgdaBound{𝑨}\AgdaSymbol{\}}\AgdaSpace{}%
\AgdaBound{𝑩}\AgdaSpace{}%
\AgdaBound{h}\AgdaSpace{}%
\AgdaBound{f}\AgdaSpace{}%
\AgdaSymbol{\{}\AgdaBound{u}\AgdaSymbol{\}\{}\AgdaBound{v}\AgdaSymbol{\}}\AgdaSpace{}%
\AgdaBound{Kerhab}\AgdaSpace{}%
\AgdaSymbol{=}\AgdaSpace{}%
\AgdaFunction{γ}\<%
\\
\>[1][@{}l@{\AgdaIndent{0}}]%
\>[2]\AgdaKeyword{where}\<%
\\
%
\>[2]\AgdaFunction{γ}\AgdaSpace{}%
\AgdaSymbol{:}\AgdaSpace{}%
\AgdaOperator{\AgdaFunction{∣}}\AgdaSpace{}%
\AgdaBound{h}\AgdaSpace{}%
\AgdaOperator{\AgdaFunction{∣}}\AgdaSpace{}%
\AgdaSymbol{((}\AgdaBound{f}\AgdaSpace{}%
\AgdaOperator{\AgdaFunction{̂}}\AgdaSpace{}%
\AgdaBound{𝑨}\AgdaSymbol{)}\AgdaSpace{}%
\AgdaBound{u}\AgdaSymbol{)}%
\>[25]\AgdaOperator{\AgdaDatatype{≡}}\AgdaSpace{}%
\AgdaOperator{\AgdaFunction{∣}}\AgdaSpace{}%
\AgdaBound{h}\AgdaSpace{}%
\AgdaOperator{\AgdaFunction{∣}}\AgdaSpace{}%
\AgdaSymbol{((}\AgdaBound{f}\AgdaSpace{}%
\AgdaOperator{\AgdaFunction{̂}}\AgdaSpace{}%
\AgdaBound{𝑨}\AgdaSymbol{)}\AgdaSpace{}%
\AgdaBound{v}\AgdaSymbol{)}\<%
\\
%
\>[2]\AgdaFunction{γ}\AgdaSpace{}%
\AgdaSymbol{=}%
\>[701I]\AgdaOperator{\AgdaFunction{∣}}\AgdaSpace{}%
\AgdaBound{h}\AgdaSpace{}%
\AgdaOperator{\AgdaFunction{∣}}\AgdaSpace{}%
\AgdaSymbol{((}\AgdaBound{f}\AgdaSpace{}%
\AgdaOperator{\AgdaFunction{̂}}\AgdaSpace{}%
\AgdaBound{𝑨}\AgdaSymbol{)}\AgdaSpace{}%
\AgdaBound{u}\AgdaSymbol{)}%
\>[25]\AgdaOperator{\AgdaFunction{≡⟨}}\AgdaSpace{}%
\AgdaOperator{\AgdaFunction{∥}}\AgdaSpace{}%
\AgdaBound{h}\AgdaSpace{}%
\AgdaOperator{\AgdaFunction{∥}}\AgdaSpace{}%
\AgdaBound{f}\AgdaSpace{}%
\AgdaBound{u}\AgdaSpace{}%
\AgdaOperator{\AgdaFunction{⟩}}\<%
\\
\>[.][@{}l@{}]\<[701I]%
\>[6]\AgdaSymbol{(}\AgdaBound{f}\AgdaSpace{}%
\AgdaOperator{\AgdaFunction{̂}}\AgdaSpace{}%
\AgdaBound{𝑩}\AgdaSymbol{)(}\AgdaOperator{\AgdaFunction{∣}}\AgdaSpace{}%
\AgdaBound{h}\AgdaSpace{}%
\AgdaOperator{\AgdaFunction{∣}}\AgdaSpace{}%
\AgdaOperator{\AgdaFunction{∘}}\AgdaSpace{}%
\AgdaBound{u}\AgdaSymbol{)}\AgdaSpace{}%
\AgdaOperator{\AgdaFunction{≡⟨}}\AgdaSpace{}%
\AgdaFunction{ap}\AgdaSpace{}%
\AgdaSymbol{(}\AgdaBound{f}\AgdaSpace{}%
\AgdaOperator{\AgdaFunction{̂}}\AgdaSpace{}%
\AgdaBound{𝑩}\AgdaSymbol{)(}\AgdaBound{gfe}\AgdaSpace{}%
\AgdaSymbol{λ}\AgdaSpace{}%
\AgdaBound{x}\AgdaSpace{}%
\AgdaSymbol{→}\AgdaSpace{}%
\AgdaBound{Kerhab}\AgdaSpace{}%
\AgdaBound{x}\AgdaSymbol{)}\AgdaSpace{}%
\AgdaOperator{\AgdaFunction{⟩}}\<%
\\
%
\>[6]\AgdaSymbol{(}\AgdaBound{f}\AgdaSpace{}%
\AgdaOperator{\AgdaFunction{̂}}\AgdaSpace{}%
\AgdaBound{𝑩}\AgdaSymbol{)(}\AgdaOperator{\AgdaFunction{∣}}\AgdaSpace{}%
\AgdaBound{h}\AgdaSpace{}%
\AgdaOperator{\AgdaFunction{∣}}\AgdaSpace{}%
\AgdaOperator{\AgdaFunction{∘}}\AgdaSpace{}%
\AgdaBound{v}\AgdaSymbol{)}\AgdaSpace{}%
\AgdaOperator{\AgdaFunction{≡⟨}}\AgdaSpace{}%
\AgdaSymbol{(}\AgdaOperator{\AgdaFunction{∥}}\AgdaSpace{}%
\AgdaBound{h}\AgdaSpace{}%
\AgdaOperator{\AgdaFunction{∥}}\AgdaSpace{}%
\AgdaBound{f}\AgdaSpace{}%
\AgdaBound{v}\AgdaSymbol{)}\AgdaOperator{\AgdaFunction{⁻¹}}\AgdaSpace{}%
\AgdaOperator{\AgdaFunction{⟩}}\<%
\\
%
\>[6]\AgdaOperator{\AgdaFunction{∣}}\AgdaSpace{}%
\AgdaBound{h}\AgdaSpace{}%
\AgdaOperator{\AgdaFunction{∣}}\AgdaSpace{}%
\AgdaSymbol{((}\AgdaBound{f}\AgdaSpace{}%
\AgdaOperator{\AgdaFunction{̂}}\AgdaSpace{}%
\AgdaBound{𝑨}\AgdaSymbol{)}\AgdaSpace{}%
\AgdaBound{v}\AgdaSymbol{)}%
\>[25]\AgdaOperator{\AgdaFunction{∎}}\<%
\end{code}
\scpad
\begin{code}
\>[1]\AgdaFunction{homker-equivalence}\AgdaSpace{}%
\AgdaSymbol{:}\AgdaSpace{}%
\>[98I]\AgdaSymbol{\{}\AgdaBound{𝑨}\AgdaSpace{}%
\AgdaSymbol{:}\AgdaSpace{}%
\AgdaFunction{Algebra}\AgdaSpace{}%
\AgdaBound{𝓤}\AgdaSpace{}%
\AgdaBound{𝑆}\AgdaSymbol{\}(}\AgdaBound{𝑩}\AgdaSpace{}%
\AgdaSymbol{:}\AgdaSpace{}%
\AgdaFunction{Algebra}\AgdaSpace{}%
\AgdaBound{𝓦}\AgdaSpace{}%
\AgdaBound{𝑆}\AgdaSymbol{)(}\AgdaBound{h}\AgdaSpace{}%
\AgdaSymbol{:}\AgdaSpace{}%
\AgdaFunction{hom}\AgdaSpace{}%
\AgdaBound{𝑨}\AgdaSpace{}%
\AgdaBound{𝑩}\AgdaSymbol{)}\<%
\\
\>[1][@{}l@{\AgdaIndent{0}}]%
\>[2]\AgdaSymbol{→}%
\>[.][@{}l@{}]\<[98I]%
\>[22]\AgdaRecord{IsEquivalence}\AgdaSpace{}%
\AgdaSymbol{(}\AgdaFunction{KER-rel}\AgdaSpace{}%
\AgdaOperator{\AgdaFunction{∣}}\AgdaSpace{}%
\AgdaBound{h}\AgdaSpace{}%
\AgdaOperator{\AgdaFunction{∣}}\AgdaSymbol{)}\<%
\\
%
\\[\AgdaEmptyExtraSkip]%
%
\>[1]\AgdaFunction{homker-equivalence}\AgdaSpace{}%
\AgdaBound{𝑨}\AgdaSpace{}%
\AgdaBound{h}\AgdaSpace{}%
\AgdaSymbol{=}\AgdaSpace{}%
\AgdaFunction{map-kernel-IsEquivalence}\AgdaSpace{}%
\AgdaOperator{\AgdaFunction{∣}}\AgdaSpace{}%
\AgdaBound{h}\AgdaSpace{}%
\AgdaOperator{\AgdaFunction{∣}}\<%
\end{code}
\ccpad
It is convenient to define a function that takes a homomorphism and constructs a congruence from its kernel. We call this function \af{kercon}.
\ccpad
\begin{code}%
\>[0][@{}l@{\AgdaIndent{1}}]%
\>[1]\AgdaFunction{kercon}\AgdaSpace{}%
\AgdaSymbol{:}\AgdaSpace{}%
\AgdaSymbol{\{}\AgdaBound{𝑨}\AgdaSpace{}%
\AgdaSymbol{:}\AgdaSpace{}%
\AgdaFunction{Algebra}\AgdaSpace{}%
\AgdaBound{𝓤}\AgdaSpace{}%
\AgdaBound{𝑆}\AgdaSymbol{\}(}\AgdaBound{𝑩}\AgdaSpace{}%
\AgdaSymbol{:}\AgdaSpace{}%
\AgdaFunction{Algebra}\AgdaSpace{}%
\AgdaBound{𝓦}\AgdaSpace{}%
\AgdaBound{𝑆}\AgdaSymbol{)(}\AgdaBound{h}\AgdaSpace{}%
\AgdaSymbol{:}\AgdaSpace{}%
\AgdaFunction{hom}\AgdaSpace{}%
\AgdaBound{𝑨}\AgdaSpace{}%
\AgdaBound{𝑩}\AgdaSymbol{)}\AgdaSpace{}%
\AgdaSymbol{→}\AgdaSpace{}%
\AgdaRecord{Congruence}\AgdaSpace{}%
\AgdaBound{𝑨}\<%
\\
%
\>[1]\AgdaFunction{kercon}\AgdaSpace{}%
\AgdaBound{𝑩}\AgdaSpace{}%
\AgdaBound{h}\AgdaSpace{}%
\AgdaSymbol{=}\AgdaSpace{}%
\AgdaInductiveConstructor{mkcon}\AgdaSpace{}%
\AgdaSymbol{(}\AgdaFunction{KER-rel}\AgdaSpace{}%
\AgdaOperator{\AgdaFunction{∣}}\AgdaSpace{}%
\AgdaBound{h}\AgdaSpace{}%
\AgdaOperator{\AgdaFunction{∣}}\AgdaSymbol{)(}\AgdaFunction{homker-compatible}\AgdaSpace{}%
\AgdaBound{𝑩}\AgdaSpace{}%
\AgdaBound{h}\AgdaSymbol{)(}\AgdaFunction{homker-equivalence}\AgdaSpace{}%
\AgdaBound{𝑩}\AgdaSpace{}%
\AgdaBound{h}\AgdaSymbol{)}\<%
\end{code}
\ccpad
With this congruence we define the corresponding quotient, along with some syntactic sugar to denote it.
\ccpad
\begin{code}%
\>[0][@{}l@{\AgdaIndent{1}}]%
\>[1]\AgdaFunction{kerquo}\AgdaSpace{}%
\AgdaSymbol{:}\AgdaSpace{}%
\AgdaSymbol{\{}\AgdaBound{𝑨}\AgdaSpace{}%
\AgdaSymbol{:}\AgdaSpace{}%
\AgdaFunction{Algebra}\AgdaSpace{}%
\AgdaBound{𝓤}\AgdaSpace{}%
\AgdaBound{𝑆}\AgdaSymbol{\}(}\AgdaBound{𝑩}\AgdaSpace{}%
\AgdaSymbol{:}\AgdaSpace{}%
\AgdaFunction{Algebra}\AgdaSpace{}%
\AgdaBound{𝓦}\AgdaSpace{}%
\AgdaBound{𝑆}\AgdaSymbol{)(}\AgdaBound{h}\AgdaSpace{}%
\AgdaSymbol{:}\AgdaSpace{}%
\AgdaFunction{hom}\AgdaSpace{}%
\AgdaBound{𝑨}\AgdaSpace{}%
\AgdaBound{𝑩}\AgdaSymbol{)}\AgdaSpace{}%
\AgdaSymbol{→}\AgdaSpace{}%
\AgdaFunction{Algebra}\AgdaSpace{}%
\AgdaSymbol{(}\AgdaBound{𝓤}\AgdaSpace{}%
\AgdaOperator{\AgdaFunction{⊔}}\AgdaSpace{}%
\AgdaBound{𝓦}\AgdaSpace{}%
\AgdaOperator{\AgdaFunction{⁺}}\AgdaSymbol{)}\AgdaSpace{}%
\AgdaBound{𝑆}\<%
\\
%
\>[1]\AgdaFunction{kerquo}\AgdaSpace{}%
\AgdaSymbol{\{}\AgdaBound{𝑨}\AgdaSymbol{\}}\AgdaSpace{}%
\AgdaBound{𝑩}\AgdaSpace{}%
\AgdaBound{h}\AgdaSpace{}%
\AgdaSymbol{=}\AgdaSpace{}%
\AgdaBound{𝑨}\AgdaSpace{}%
\AgdaOperator{\AgdaFunction{╱}}\AgdaSpace{}%
\AgdaSymbol{(}\AgdaFunction{kercon}\AgdaSpace{}%
\AgdaBound{𝑩}\AgdaSpace{}%
\AgdaBound{h}\AgdaSymbol{)}\<%
\\
%
\\[\AgdaEmptyExtraSkip]%
%
\>[1]\AgdaOperator{\AgdaFunction{\AgdaUnderscore{}[\AgdaUnderscore{}]/ker\AgdaUnderscore{}}}\AgdaSpace{}%
\AgdaSymbol{:}\AgdaSpace{}%
\AgdaSymbol{(}\AgdaBound{𝑨}\AgdaSpace{}%
\AgdaSymbol{:}\AgdaSpace{}%
\AgdaFunction{Algebra}\AgdaSpace{}%
\AgdaBound{𝓤}\AgdaSpace{}%
\AgdaBound{𝑆}\AgdaSymbol{)(}\AgdaBound{𝑩}\AgdaSpace{}%
\AgdaSymbol{:}\AgdaSpace{}%
\AgdaFunction{Algebra}\AgdaSpace{}%
\AgdaBound{𝓦}\AgdaSpace{}%
\AgdaBound{𝑆}\AgdaSymbol{)(}\AgdaBound{h}\AgdaSpace{}%
\AgdaSymbol{:}\AgdaSpace{}%
\AgdaFunction{hom}\AgdaSpace{}%
\AgdaBound{𝑨}\AgdaSpace{}%
\AgdaBound{𝑩}\AgdaSymbol{)}\AgdaSpace{}%
\AgdaSymbol{→}\AgdaSpace{}%
\AgdaFunction{Algebra}\AgdaSpace{}%
\AgdaSymbol{(}\AgdaBound{𝓤}\AgdaSpace{}%
\AgdaOperator{\AgdaFunction{⊔}}\AgdaSpace{}%
\AgdaBound{𝓦}\AgdaSpace{}%
\AgdaOperator{\AgdaFunction{⁺}}\AgdaSymbol{)}\AgdaSpace{}%
\AgdaBound{𝑆}\<%
\\
%
\>[1]\AgdaBound{𝑨}\AgdaSpace{}%
\AgdaOperator{\AgdaFunction{[}}\AgdaSpace{}%
\AgdaBound{𝑩}\AgdaSpace{}%
\AgdaOperator{\AgdaFunction{]/ker}}\AgdaSpace{}%
\AgdaBound{h}\AgdaSpace{}%
\AgdaSymbol{=}\AgdaSpace{}%
\AgdaFunction{kerquo}\AgdaSpace{}%
\AgdaSymbol{\{}\AgdaBound{𝑨}\AgdaSymbol{\}}\AgdaSpace{}%
\AgdaBound{𝑩}\AgdaSpace{}%
\AgdaBound{h}\<%
\end{code}
\ccpad
Thus, given \ab h \as : \af{hom} \ab 𝑨 \ab 𝑩, we can construct the quotient of \ab 𝑨 modulo the kernel of \ab h, and the \ualib syntax for this quotient is  \ab 𝑨 \af [ 𝑩 \af{]/ker} \ab h.

\subsubsection{The natural projection}\label{sec:natural-projection}
Given an algebra \ab{𝑨} and a congruence \ab{θ}, the \emph{natural} or \defn{canonical projection} is a map from \ab{𝑨} onto \ab{𝑨} \af ╱ \ab θ that is constructed, and proved epimorphic, as follows.
\ccpad
\begin{code}%
\>[0][@{}l@{\AgdaIndent{1}}]%
\>[1]\AgdaFunction{πepi}\AgdaSpace{}%
\AgdaSymbol{:}\AgdaSpace{}%
\AgdaSymbol{\{}\AgdaBound{𝑨}\AgdaSpace{}%
\AgdaSymbol{:}\AgdaSpace{}%
\AgdaFunction{Algebra}\AgdaSpace{}%
\AgdaBound{𝓤}\AgdaSpace{}%
\AgdaBound{𝑆}\AgdaSymbol{\}}\AgdaSpace{}%
\AgdaSymbol{(}\AgdaBound{θ}\AgdaSpace{}%
\AgdaSymbol{:}\AgdaSpace{}%
\AgdaRecord{Congruence}\AgdaSymbol{\{}\AgdaBound{𝓤}\AgdaSymbol{\}\{}\AgdaBound{𝓦}\AgdaSymbol{\}}\AgdaSpace{}%
\AgdaBound{𝑨}\AgdaSymbol{)}\AgdaSpace{}%
\AgdaSymbol{→}\AgdaSpace{}%
\AgdaFunction{epi}\AgdaSpace{}%
\AgdaBound{𝑨}\AgdaSpace{}%
\AgdaSymbol{(}\AgdaBound{𝑨}\AgdaSpace{}%
\AgdaOperator{\AgdaFunction{╱}}\AgdaSpace{}%
\AgdaBound{θ}\AgdaSymbol{)}\<%
\\
%
\>[1]\AgdaFunction{πepi}\AgdaSpace{}%
\AgdaSymbol{\{}\AgdaBound{𝑨}\AgdaSymbol{\}}\AgdaSpace{}%
\AgdaBound{θ}\AgdaSpace{}%
\AgdaSymbol{=}\AgdaSpace{}%
\AgdaFunction{cπ}\AgdaSpace{}%
\AgdaOperator{\AgdaInductiveConstructor{,}}\AgdaSpace{}%
\AgdaFunction{cπ-is-hom}\AgdaSpace{}%
\AgdaOperator{\AgdaInductiveConstructor{,}}\AgdaSpace{}%
\AgdaFunction{cπ-is-epic}\AgdaSpace{}%
\AgdaKeyword{where}\<%
\\
%
\\[\AgdaEmptyExtraSkip]%
\>[1][@{}l@{\AgdaIndent{0}}]%
\>[2]\AgdaFunction{cπ}\AgdaSpace{}%
\AgdaSymbol{:}\AgdaSpace{}%
\AgdaOperator{\AgdaFunction{∣}}\AgdaSpace{}%
\AgdaBound{𝑨}\AgdaSpace{}%
\AgdaOperator{\AgdaFunction{∣}}\AgdaSpace{}%
\AgdaSymbol{→}\AgdaSpace{}%
\AgdaOperator{\AgdaFunction{∣}}\AgdaSpace{}%
\AgdaBound{𝑨}\AgdaSpace{}%
\AgdaOperator{\AgdaFunction{╱}}\AgdaSpace{}%
\AgdaBound{θ}\AgdaSpace{}%
\AgdaOperator{\AgdaFunction{∣}}\<%
\\
%
\>[2]\AgdaFunction{cπ}\AgdaSpace{}%
\AgdaBound{a}\AgdaSpace{}%
\AgdaSymbol{=}\AgdaSpace{}%
\AgdaOperator{\AgdaFunction{⟦}}\AgdaSpace{}%
\AgdaBound{a}\AgdaSpace{}%
\AgdaOperator{\AgdaFunction{⟧}}\AgdaSymbol{\{}\AgdaOperator{\AgdaField{⟨}}\AgdaSpace{}%
\AgdaBound{θ}\AgdaSpace{}%
\AgdaOperator{\AgdaField{⟩}}\AgdaSymbol{\}}\<%
\\
%
\\[\AgdaEmptyExtraSkip]%
%
\>[2]\AgdaFunction{cπ-is-hom}\AgdaSpace{}%
\AgdaSymbol{:}\AgdaSpace{}%
\AgdaFunction{is-homomorphism}\AgdaSpace{}%
\AgdaBound{𝑨}\AgdaSpace{}%
\AgdaSymbol{(}\AgdaBound{𝑨}\AgdaSpace{}%
\AgdaOperator{\AgdaFunction{╱}}\AgdaSpace{}%
\AgdaBound{θ}\AgdaSymbol{)}\AgdaSpace{}%
\AgdaFunction{cπ}\<%
\\
%
\>[2]\AgdaFunction{cπ-is-hom}\AgdaSpace{}%
\AgdaSymbol{\AgdaUnderscore{}}\AgdaSpace{}%
\AgdaSymbol{\AgdaUnderscore{}}\AgdaSpace{}%
\AgdaSymbol{=}\AgdaSpace{}%
\AgdaInductiveConstructor{𝓇ℯ𝒻𝓁}\<%
\\
%
\\[\AgdaEmptyExtraSkip]%
%
\>[2]\AgdaFunction{cπ-is-epic}\AgdaSpace{}%
\AgdaSymbol{:}\AgdaSpace{}%
\AgdaFunction{Epic}\AgdaSpace{}%
\AgdaFunction{cπ}\<%
\\
%
\>[2]\AgdaFunction{cπ-is-epic}\AgdaSpace{}%
\AgdaSymbol{(}\AgdaDottedPattern{\AgdaSymbol{.(}}\AgdaDottedPattern{\AgdaOperator{\AgdaField{⟨}}}\AgdaSpace{}%
\AgdaDottedPattern{\AgdaBound{θ}}\AgdaSpace{}%
\AgdaDottedPattern{\AgdaOperator{\AgdaField{⟩}}}\AgdaSpace{}%
\AgdaDottedPattern{\AgdaBound{a}}\AgdaDottedPattern{\AgdaSymbol{)}}\AgdaSpace{}%
\AgdaOperator{\AgdaInductiveConstructor{,}}\AgdaSpace{}%
\AgdaBound{a}\AgdaSpace{}%
\AgdaOperator{\AgdaInductiveConstructor{,}}\AgdaSpace{}%
\AgdaInductiveConstructor{𝓇ℯ𝒻𝓁}\AgdaSymbol{)}\AgdaSpace{}%
\AgdaSymbol{=}\AgdaSpace{}%
\AgdaInductiveConstructor{Image\AgdaUnderscore{}∋\AgdaUnderscore{}.im}\AgdaSpace{}%
\AgdaBound{a}\<%
\end{code}
\ccpad
Sometimes we don't care about the surjectivity of \af{πepi} and we want to work with its \emph{homomorphic reduct}, which is obtained by applying \af{epi-to-hom}, like so.
\ccpad
\begin{code}%
\>[0][@{}l@{\AgdaIndent{1}}]%
\>[1]\AgdaFunction{πhom}\AgdaSpace{}%
\AgdaSymbol{:}\AgdaSpace{}%
\AgdaSymbol{\{}\AgdaBound{𝑨}\AgdaSpace{}%
\AgdaSymbol{:}\AgdaSpace{}%
\AgdaFunction{Algebra}\AgdaSpace{}%
\AgdaBound{𝓤}\AgdaSpace{}%
\AgdaBound{𝑆}\AgdaSymbol{\}(}\AgdaBound{θ}\AgdaSpace{}%
\AgdaSymbol{:}\AgdaSpace{}%
\AgdaRecord{Congruence}\AgdaSymbol{\{}\AgdaBound{𝓤}\AgdaSymbol{\}\{}\AgdaBound{𝓦}\AgdaSymbol{\}}\AgdaSpace{}%
\AgdaBound{𝑨}\AgdaSymbol{)}\AgdaSpace{}%
\AgdaSymbol{→}\AgdaSpace{}%
\AgdaFunction{hom}\AgdaSpace{}%
\AgdaBound{𝑨}\AgdaSpace{}%
\AgdaSymbol{(}\AgdaBound{𝑨}\AgdaSpace{}%
\AgdaOperator{\AgdaFunction{╱}}\AgdaSpace{}%
\AgdaBound{θ}\AgdaSymbol{)}\<%
\\
%
\>[1]\AgdaFunction{πhom}\AgdaSpace{}%
\AgdaSymbol{\{}\AgdaBound{𝑨}\AgdaSymbol{\}}\AgdaSpace{}%
\AgdaBound{θ}\AgdaSpace{}%
\AgdaSymbol{=}\AgdaSpace{}%
\AgdaFunction{epi-to-hom}\AgdaSpace{}%
\AgdaSymbol{(}\AgdaBound{𝑨}\AgdaSpace{}%
\AgdaOperator{\AgdaFunction{╱}}\AgdaSpace{}%
\AgdaBound{θ}\AgdaSymbol{)}\AgdaSpace{}%
\AgdaSymbol{(}\AgdaFunction{πepi}\AgdaSpace{}%
\AgdaBound{θ}\AgdaSymbol{)}\<%
\end{code}
\ccpad
We combine the foregoing to define a function that takes \ab 𝑆-algebras \ab{𝑨} and \ab{𝑩}, and a homomorphism \ab{h} \as : \af{hom}~\ab 𝑨~\ab 𝑩 and returns the canonical epimorphism from \ab{𝑨} onto the quotient algebra \ab{𝑨} \af{[} \ab 𝑩 \af{]}\af{/} \af{ker} \ab h (which, recall, is \ab{𝑨} modulo the kernel of \ab{h}).
\ccpad
\begin{code}%
\>[0][@{}l@{\AgdaIndent{1}}]%
\>[1]\AgdaFunction{πker}\AgdaSpace{}%
\AgdaSymbol{:}\AgdaSpace{}%
\AgdaSymbol{\{}\AgdaBound{𝑨}\AgdaSpace{}%
\AgdaSymbol{:}\AgdaSpace{}%
\AgdaFunction{Algebra}\AgdaSpace{}%
\AgdaBound{𝓤}\AgdaSpace{}%
\AgdaBound{𝑆}\AgdaSymbol{\}(}\AgdaBound{𝑩}\AgdaSpace{}%
\AgdaSymbol{:}\AgdaSpace{}%
\AgdaFunction{Algebra}\AgdaSpace{}%
\AgdaBound{𝓦}\AgdaSpace{}%
\AgdaBound{𝑆}\AgdaSymbol{)(}\AgdaBound{h}\AgdaSpace{}%
\AgdaSymbol{:}\AgdaSpace{}%
\AgdaFunction{hom}\AgdaSpace{}%
\AgdaBound{𝑨}\AgdaSpace{}%
\AgdaBound{𝑩}\AgdaSymbol{)}%
\>[57]\AgdaSymbol{→}%
\>[60]\AgdaFunction{epi}\AgdaSpace{}%
\AgdaBound{𝑨}\AgdaSpace{}%
\AgdaSymbol{(}\AgdaBound{𝑨}\AgdaSpace{}%
\AgdaOperator{\AgdaFunction{[}}\AgdaSpace{}%
\AgdaBound{𝑩}\AgdaSpace{}%
\AgdaOperator{\AgdaFunction{]/ker}}\AgdaSpace{}%
\AgdaBound{h}\AgdaSymbol{)}\<%
\\
%
\>[1]\AgdaFunction{πker}\AgdaSpace{}%
\AgdaSymbol{\{}\AgdaBound{𝑨}\AgdaSymbol{\}}\AgdaSpace{}%
\AgdaBound{𝑩}\AgdaSpace{}%
\AgdaBound{h}\AgdaSpace{}%
\AgdaSymbol{=}\AgdaSpace{}%
\AgdaFunction{πepi}\AgdaSpace{}%
\AgdaSymbol{(}\AgdaFunction{kercon}\AgdaSpace{}%
\AgdaBound{𝑩}\AgdaSpace{}%
\AgdaBound{h}\AgdaSymbol{)}\<%
\end{code}
\ccpad
The kernel of the canonical projection of \ab{𝑨} onto \ab{𝑨} \af / \ab{θ} is equal to \ab{θ}, but since equality of
inhabitants of certain types (like \ar{Congruence} or \af{Rel}) can be a tricky business, we settle for proving the containment \ab{𝑨} \af / \ab θ \af ⊆ \ab θ. Of the two containments, this is the easier one
to prove; luckily it is also the one we need later.
\ccpad
\begin{code}%
\>[1]\AgdaFunction{ker-in-con}\AgdaSpace{}%
\AgdaSymbol{:}\AgdaSpace{}%
\>[111I]\AgdaSymbol{(}\AgdaBound{𝑨}\AgdaSpace{}%
\AgdaSymbol{:}\AgdaSpace{}%
\AgdaFunction{Algebra}\AgdaSpace{}%
\AgdaBound{𝓤}\AgdaSpace{}%
\AgdaBound{𝑆}\AgdaSymbol{)(}\AgdaBound{θ}\AgdaSpace{}%
\AgdaSymbol{:}\AgdaSpace{}%
\AgdaRecord{Congruence}\AgdaSymbol{\{}\AgdaBound{𝓤}\AgdaSymbol{\}\{}\AgdaBound{𝓦}\AgdaSymbol{\}}\AgdaSpace{}%
\AgdaBound{𝑨}\AgdaSymbol{)(}\AgdaBound{x}\AgdaSpace{}%
\AgdaBound{y}\AgdaSpace{}%
\AgdaSymbol{:}\AgdaSpace{}%
\AgdaOperator{\AgdaFunction{∣}}\AgdaSpace{}%
\AgdaBound{𝑨}\AgdaSpace{}%
\AgdaOperator{\AgdaFunction{∣}}\AgdaSymbol{)}\<%
\\
\>[1][@{}l@{\AgdaIndent{0}}]%
\>[2]\AgdaSymbol{→}%
\>[.][@{}l@{}]\<[111I]%
\>[14]\AgdaOperator{\AgdaField{⟨}}\AgdaSpace{}%
\AgdaFunction{kercon}\AgdaSpace{}%
\AgdaSymbol{(}\AgdaBound{𝑨}\AgdaSpace{}%
\AgdaOperator{\AgdaFunction{╱}}\AgdaSpace{}%
\AgdaBound{θ}\AgdaSymbol{)}\AgdaSpace{}%
\AgdaSymbol{(}\AgdaFunction{πhom}\AgdaSpace{}%
\AgdaBound{θ}\AgdaSymbol{)}\AgdaSpace{}%
\AgdaOperator{\AgdaField{⟩}}\AgdaSpace{}%
\AgdaBound{x}\AgdaSpace{}%
\AgdaBound{y}%
\>[47]\AgdaSymbol{→}%
\>[50]\AgdaOperator{\AgdaField{⟨}}\AgdaSpace{}%
\AgdaBound{θ}\AgdaSpace{}%
\AgdaOperator{\AgdaField{⟩}}\AgdaSpace{}%
\AgdaBound{x}\AgdaSpace{}%
\AgdaBound{y}\<%
\\
%
\\[\AgdaEmptyExtraSkip]%
%
\>[1]\AgdaFunction{ker-in-con}\AgdaSpace{}%
\AgdaBound{𝑨}\AgdaSpace{}%
\AgdaBound{θ}\AgdaSpace{}%
\AgdaBound{x}\AgdaSpace{}%
\AgdaBound{y}\AgdaSpace{}%
\AgdaBound{hyp}\AgdaSpace{}%
\AgdaSymbol{=}\AgdaSpace{}%
\AgdaFunction{╱-refl}\AgdaSpace{}%
\AgdaBound{θ}\AgdaSpace{}%
\AgdaBound{hyp}\<%
\end{code}

\subsubsection{Product homomorphisms}\label{product-homomorphisms}
Suppose we have an algebra \ab 𝑨 \as : \af{Algebra} \ab 𝓤 \ab 𝑆, a type \abt{I}{𝓘}\af ̇, and a family \ab ℬ \as : \ab I \as → \af{Algebra} \ab 𝓦 \ab 𝑆 of algebras. We sometimes refer to the inhabitants of \ab I as \defn{indices}, and call \ab ℬ an \emph{indexed family of algebras}. If in addition we have a family \ab 𝒽 \as : (\ab i \as : \ab I) \as → \af{hom} \ab 𝑨 (\ab ℬ \ab i) of homomorphisms, then we can construct a homomorphism from \ab 𝑨 to the product \af ⨅ \ab ℬ in the natural way.
\ccpad
\begin{code}%
\>[1]\AgdaFunction{⨅-hom-co}\AgdaSpace{}%
\AgdaSymbol{:}\AgdaSpace{}%
\>[222I]
\AgdaSymbol{\{}\AgdaBound{𝑨}\AgdaSpace{}%
\AgdaSymbol{:}\AgdaSpace{}%
\AgdaFunction{Algebra}\AgdaSpace{}%
\AgdaBound{𝓤}\AgdaSpace{}%
\AgdaBound{𝑆}\AgdaSymbol{\}\{}\AgdaBound{I}\AgdaSpace{}%
\AgdaSymbol{:}\AgdaSpace{}%
\AgdaBound{𝓘}\AgdaSpace{}%
\AgdaOperator{\AgdaFunction{̇}}\AgdaSymbol{\}(}\AgdaBound{ℬ}\AgdaSpace{}%
\AgdaSymbol{:}\AgdaSpace{}%
\AgdaBound{I}\AgdaSpace{}%
\AgdaSymbol{→}\AgdaSpace{}%
\AgdaFunction{Algebra}\AgdaSpace{}%
\AgdaBound{𝓦}\AgdaSpace{}%
\AgdaBound{𝑆}\AgdaSymbol{)}\<%
\\
\>[1][@{}l@{\AgdaIndent{0}}]%
\>[2]\AgdaSymbol{→}%
\>[.][@{}l@{}]\<[222I]%
\>[12]\AgdaFunction{Π}\AgdaSpace{}%
\AgdaBound{i}\AgdaSpace{}%
\AgdaFunction{꞉}\AgdaSpace{}%
\AgdaBound{I}\AgdaSpace{}%
\AgdaFunction{,}\AgdaSpace{}%
\AgdaFunction{hom}\AgdaSpace{}%
\AgdaBound{𝑨}\AgdaSpace{}%
\AgdaSymbol{(}\AgdaBound{ℬ}\AgdaSpace{}%
\AgdaBound{i}\AgdaSymbol{)}%
\>[35]\AgdaSymbol{→}%
\>[38]\AgdaFunction{hom}\AgdaSpace{}%
\AgdaBound{𝑨}\AgdaSpace{}%
\AgdaSymbol{(}\AgdaFunction{⨅}\AgdaSpace{}%
\AgdaBound{ℬ}\AgdaSymbol{)}\<%
\\
%
\\[\AgdaEmptyExtraSkip]%
%
\>[1]\AgdaFunction{⨅-hom-co}\AgdaSpace{}%
\AgdaSymbol{\{}\AgdaBound{𝑨}\AgdaSymbol{\}}\AgdaSpace{}%
\AgdaBound{ℬ}\AgdaSpace{}%
\AgdaBound{𝒽}\AgdaSpace{}%
\AgdaSymbol{=}\AgdaSpace{}%
\AgdaFunction{ϕ}\AgdaSpace{}%
\AgdaOperator{\AgdaInductiveConstructor{,}}\AgdaSpace{}%
\AgdaFunction{ϕhom}\<%
\\
\>[1][@{}l@{\AgdaIndent{0}}]%
\>[2]\AgdaKeyword{where}\<%
\\
%
\>[2]\AgdaFunction{ϕ}\AgdaSpace{}%
\AgdaSymbol{:}\AgdaSpace{}%
\AgdaOperator{\AgdaFunction{∣}}\AgdaSpace{}%
\AgdaBound{𝑨}\AgdaSpace{}%
\AgdaOperator{\AgdaFunction{∣}}\AgdaSpace{}%
\AgdaSymbol{→}\AgdaSpace{}%
\AgdaOperator{\AgdaFunction{∣}}\AgdaSpace{}%
\AgdaFunction{⨅}\AgdaSpace{}%
\AgdaBound{ℬ}\AgdaSpace{}%
\AgdaOperator{\AgdaFunction{∣}}\<%
\\
%
\>[2]\AgdaFunction{ϕ}\AgdaSpace{}%
\AgdaBound{a}\AgdaSpace{}%
\AgdaSymbol{=}\AgdaSpace{}%
\AgdaSymbol{λ}\AgdaSpace{}%
\AgdaBound{i}\AgdaSpace{}%
\AgdaSymbol{→}\AgdaSpace{}%
\AgdaOperator{\AgdaFunction{∣}}\AgdaSpace{}%
\AgdaBound{𝒽}\AgdaSpace{}%
\AgdaBound{i}\AgdaSpace{}%
\AgdaOperator{\AgdaFunction{∣}}\AgdaSpace{}%
\AgdaBound{a}\<%
\\
%
\\[\AgdaEmptyExtraSkip]%
%
\>[2]\AgdaFunction{ϕhom}\AgdaSpace{}%
\AgdaSymbol{:}\AgdaSpace{}%
\AgdaFunction{is-homomorphism}\AgdaSpace{}%
\AgdaBound{𝑨}\AgdaSpace{}%
\AgdaSymbol{(}\AgdaFunction{⨅}\AgdaSpace{}%
\AgdaBound{ℬ}\AgdaSymbol{)}\AgdaSpace{}%
\AgdaFunction{ϕ}\<%
\\
%
\>[2]\AgdaFunction{ϕhom}\AgdaSpace{}%
\AgdaBound{𝑓}\AgdaSpace{}%
\AgdaBound{𝒶}\AgdaSpace{}%
\AgdaSymbol{=}\AgdaSpace{}%
\AgdaBound{gfe}\AgdaSpace{}%
\AgdaSymbol{λ}\AgdaSpace{}%
\AgdaBound{i}\AgdaSpace{}%
\AgdaSymbol{→}\AgdaSpace{}%
\AgdaOperator{\AgdaFunction{∥}}\AgdaSpace{}%
\AgdaBound{𝒽}\AgdaSpace{}%
\AgdaBound{i}\AgdaSpace{}%
\AgdaOperator{\AgdaFunction{∥}}\AgdaSpace{}%
\AgdaBound{𝑓}\AgdaSpace{}%
\AgdaBound{𝒶}\<%
\end{code}
\ccpad
The family \ab 𝒽 of homomorphisms inhabits the dependent type \af Π \ab i \af ꞉ \ab I , \af{hom} \ab 𝑨 (\ab ℬ \ab i).  The syntax we use to represent this type is available to us because of the way \af{-Π} is defined in the \typetopology library.  We like this syntax because it is very close to the notation one finds in the standard type theory literature.  However, we could equally well have used one of the following alternatives, which may be closer to ``standard Agda'' syntax:\\[-5pt]

 \af Π \as λ \ab i \as → \af{hom} \ab 𝑨 (\ab ℬ \ab i) \hskip2mm or \hskip2mm (\ab i \as : \ab I) \as → \af{hom} \ab 𝑨 (\ab ℬ \ab i) \hskip2mm or \hskip2mm \as ∀ \ab i → \af{hom} \ab 𝑨 (\ab ℬ \ab i).\\[-4pt]

The foregoing generalizes easily to the case in which the domain is also a product of a family of algebras.  That is, if we are given \ab 𝒜 \as : \ab I \as → \af{Algebra} \ab 𝓤 \ab 𝑆 and \ab ℬ \as : \ab I \as → \af{Algebra} \ab 𝓦 \ab 𝑆 (two families of \ab 𝑆-algebras), and \AgdaBound{𝒽} \as : \AgdaFunction{Π} \ab i \af ꞉ \ab I \af ,
\af{hom} (\AgdaBound{𝒜} \ab i) (\ab ℬ \ab i) (a family of homomorphisms), then we can construct a homomorphism from \af ⨅ \ab 𝒜 to \af ⨅ \ab ℬ in the following natural way.
\ccpad
\begin{code}
\>[1]\AgdaFunction{⨅-hom}\AgdaSpace{}%
\AgdaSymbol{:}\AgdaSpace{}%
\>[223I]\AgdaSymbol{\{}\AgdaBound{I}\AgdaSpace{}%
\AgdaSymbol{:}\AgdaSpace{}%
\AgdaBound{𝓘}\AgdaSpace{}%
\AgdaOperator{\AgdaFunction{̇}}\AgdaSymbol{\}(}\AgdaBound{𝒜}\AgdaSpace{}%
\AgdaSymbol{:}\AgdaSpace{}%
\AgdaBound{I}\AgdaSpace{}%
\AgdaSymbol{→}\AgdaSpace{}%
\AgdaFunction{Algebra}\AgdaSpace{}%
\AgdaBound{𝓤}\AgdaSpace{}%
\AgdaBound{𝑆}\AgdaSymbol{)(}\AgdaBound{ℬ}\AgdaSpace{}%
\AgdaSymbol{:}\AgdaSpace{}%
\AgdaBound{I}\AgdaSpace{}%
\AgdaSymbol{→}\AgdaSpace{}%
\AgdaFunction{Algebra}\AgdaSpace{}%
\AgdaBound{𝓦}\AgdaSpace{}%
\AgdaBound{𝑆}\AgdaSymbol{)}\<%
\\
\>[1][@{}l@{\AgdaIndent{0}}]%
\>[2]\AgdaSymbol{→}%
\>[.][@{}l@{}]\<[223I]%
\>[9]\AgdaFunction{Π}\AgdaSpace{}%
\AgdaBound{i}\AgdaSpace{}%
\AgdaFunction{꞉}\AgdaSpace{}%
\AgdaBound{I}\AgdaSpace{}%
\AgdaFunction{,}\AgdaSpace{}%
\AgdaFunction{hom}\AgdaSpace{}%
\AgdaSymbol{(}\AgdaBound{𝒜}\AgdaSpace{}%
\AgdaBound{i}\AgdaSymbol{)(}\AgdaBound{ℬ}\AgdaSpace{}%
\AgdaBound{i}\AgdaSymbol{)}%
\>[35]\AgdaSymbol{→}%
\>[38]\AgdaFunction{hom}\AgdaSpace{}%
\AgdaSymbol{(}\AgdaFunction{⨅}\AgdaSpace{}%
\AgdaBound{𝒜}\AgdaSymbol{)(}\AgdaFunction{⨅}\AgdaSpace{}%
\AgdaBound{ℬ}\AgdaSymbol{)}\<%
\\
%
\\[\AgdaEmptyExtraSkip]%
%
\>[1]\AgdaFunction{⨅-hom}\AgdaSpace{}%
\AgdaBound{𝒜}\AgdaSpace{}%
\AgdaBound{ℬ}\AgdaSpace{}%
\AgdaBound{𝒽}\AgdaSpace{}%
\AgdaSymbol{=}\AgdaSpace{}%
\AgdaFunction{ϕ}\AgdaSpace{}%
\AgdaOperator{\AgdaInductiveConstructor{,}}\AgdaSpace{}%
\AgdaFunction{ϕhom}\<%
\\
\>[1][@{}l@{\AgdaIndent{0}}]%
\>[2]\AgdaKeyword{where}\<%
\\
%
\>[2]\AgdaFunction{ϕ}\AgdaSpace{}%
\AgdaSymbol{:}\AgdaSpace{}%
\AgdaOperator{\AgdaFunction{∣}}\AgdaSpace{}%
\AgdaFunction{⨅}\AgdaSpace{}%
\AgdaBound{𝒜}\AgdaSpace{}%
\AgdaOperator{\AgdaFunction{∣}}\AgdaSpace{}%
\AgdaSymbol{→}\AgdaSpace{}%
\AgdaOperator{\AgdaFunction{∣}}\AgdaSpace{}%
\AgdaFunction{⨅}\AgdaSpace{}%
\AgdaBound{ℬ}\AgdaSpace{}%
\AgdaOperator{\AgdaFunction{∣}}\<%
\\
%
\>[2]\AgdaFunction{ϕ}\AgdaSpace{}%
\AgdaSymbol{=}\AgdaSpace{}%
\AgdaSymbol{λ}\AgdaSpace{}%
\AgdaBound{x}\AgdaSpace{}%
\AgdaBound{i}\AgdaSpace{}%
\AgdaSymbol{→}\AgdaSpace{}%
\AgdaOperator{\AgdaFunction{∣}}\AgdaSpace{}%
\AgdaBound{𝒽}\AgdaSpace{}%
\AgdaBound{i}\AgdaSpace{}%
\AgdaOperator{\AgdaFunction{∣}}\AgdaSpace{}%
\AgdaSymbol{(}\AgdaBound{x}\AgdaSpace{}%
\AgdaBound{i}\AgdaSymbol{)}\<%
\\
%
\\[\AgdaEmptyExtraSkip]%
%
\>[2]\AgdaFunction{ϕhom}\AgdaSpace{}%
\AgdaSymbol{:}\AgdaSpace{}%
\AgdaFunction{is-homomorphism}\AgdaSpace{}%
\AgdaSymbol{(}\AgdaFunction{⨅}\AgdaSpace{}%
\AgdaBound{𝒜}\AgdaSymbol{)}\AgdaSpace{}%
\AgdaSymbol{(}\AgdaFunction{⨅}\AgdaSpace{}%
\AgdaBound{ℬ}\AgdaSymbol{)}\AgdaSpace{}%
\AgdaFunction{ϕ}\<%
\\
%
\>[2]\AgdaFunction{ϕhom}\AgdaSpace{}%
\AgdaBound{𝑓}\AgdaSpace{}%
\AgdaBound{𝒶}\AgdaSpace{}%
\AgdaSymbol{=}\AgdaSpace{}%
\AgdaBound{gfe}\AgdaSpace{}%
\AgdaSymbol{λ}\AgdaSpace{}%
\AgdaBound{i}\AgdaSpace{}%
\AgdaSymbol{→}\AgdaSpace{}%
\AgdaOperator{\AgdaFunction{∥}}\AgdaSpace{}%
\AgdaBound{𝒽}\AgdaSpace{}%
\AgdaBound{i}\AgdaSpace{}%
\AgdaOperator{\AgdaFunction{∥}}\AgdaSpace{}%
\AgdaBound{𝑓}\AgdaSpace{}%
\AgdaSymbol{(λ}\AgdaSpace{}%
\AgdaBound{x}\AgdaSpace{}%
\AgdaSymbol{→}\AgdaSpace{}%
\AgdaBound{𝒶}\AgdaSpace{}%
\AgdaBound{x}\AgdaSpace{}%
\AgdaBound{i}\AgdaSymbol{)}\<%
\end{code}

\subsubsection{Projection homomorphisms}\label{projection-homomorphisms}

Later we will need a proof of the fact that the natural projection out of a product algebra onto one of its factors is a homomorphism.
\ccpad
\begin{code}%
\>[1]\AgdaFunction{⨅-projection-hom}\AgdaSpace{}%
\AgdaSymbol{:}\AgdaSpace{}%
\AgdaSymbol{\{}\AgdaBound{I}\AgdaSpace{}%
\AgdaSymbol{:}\AgdaSpace{}%
\AgdaBound{𝓘}\AgdaSpace{}%
\AgdaOperator{\AgdaFunction{̇}}\AgdaSymbol{\}(}\AgdaBound{ℬ}\AgdaSpace{}%
\AgdaSymbol{:}\AgdaSpace{}%
\AgdaBound{I}\AgdaSpace{}%
\AgdaSymbol{→}\AgdaSpace{}%
\AgdaFunction{Algebra}\AgdaSpace{}%
\AgdaBound{𝓦}\AgdaSpace{}%
\AgdaBound{𝑆}\AgdaSymbol{)}\AgdaSpace{}%
\AgdaSymbol{→}\AgdaSpace{}%
\AgdaFunction{Π}\AgdaSpace{}%
\AgdaBound{i}\AgdaSpace{}%
\AgdaFunction{꞉}\AgdaSpace{}%
\AgdaBound{I}\AgdaSpace{}%
\AgdaFunction{,}\AgdaSpace{}%
\AgdaFunction{hom}\AgdaSpace{}%
\AgdaSymbol{(}\AgdaFunction{⨅}\AgdaSpace{}%
\AgdaBound{ℬ}\AgdaSymbol{)}\AgdaSpace{}%
\AgdaSymbol{(}\AgdaBound{ℬ}\AgdaSpace{}%
\AgdaBound{i}\AgdaSymbol{)}\<%
\\
%
\\[\AgdaEmptyExtraSkip]%
%
\>[1]\AgdaFunction{⨅-projection-hom}\AgdaSpace{}%
\AgdaBound{ℬ}\AgdaSpace{}%
\AgdaSymbol{=}\AgdaSpace{}%
\AgdaSymbol{λ}\AgdaSpace{}%
\AgdaBound{i}\AgdaSpace{}%
\AgdaSymbol{→}\AgdaSpace{}%
\AgdaFunction{𝒽}\AgdaSpace{}%
\AgdaBound{i}\AgdaSpace{}%
\AgdaOperator{\AgdaInductiveConstructor{,}}\AgdaSpace{}%
\AgdaFunction{𝒽hom}\AgdaSpace{}%
\AgdaBound{i}\<%
\\
\>[1][@{}l@{\AgdaIndent{0}}]%
\>[2]\AgdaKeyword{where}\<%
\\
%
\>[2]\AgdaFunction{𝒽}\AgdaSpace{}%
\AgdaSymbol{:}\AgdaSpace{}%
\AgdaSymbol{∀}\AgdaSpace{}%
\AgdaBound{i}\AgdaSpace{}%
\AgdaSymbol{→}\AgdaSpace{}%
\AgdaOperator{\AgdaFunction{∣}}\AgdaSpace{}%
\AgdaFunction{⨅}\AgdaSpace{}%
\AgdaBound{ℬ}\AgdaSpace{}%
\AgdaOperator{\AgdaFunction{∣}}\AgdaSpace{}%
\AgdaSymbol{→}\AgdaSpace{}%
\AgdaOperator{\AgdaFunction{∣}}\AgdaSpace{}%
\AgdaBound{ℬ}\AgdaSpace{}%
\AgdaBound{i}\AgdaSpace{}%
\AgdaOperator{\AgdaFunction{∣}}\<%
\\
%
\>[2]\AgdaFunction{𝒽}\AgdaSpace{}%
\AgdaBound{i}\AgdaSpace{}%
\AgdaSymbol{=}\AgdaSpace{}%
\AgdaSymbol{λ}\AgdaSpace{}%
\AgdaBound{x}\AgdaSpace{}%
\AgdaSymbol{→}\AgdaSpace{}%
\AgdaBound{x}\AgdaSpace{}%
\AgdaBound{i}\<%
\\
%
\\[\AgdaEmptyExtraSkip]%
%
\>[2]\AgdaFunction{𝒽hom}\AgdaSpace{}%
\AgdaSymbol{:}\AgdaSpace{}%
\AgdaSymbol{∀}\AgdaSpace{}%
\AgdaBound{i}\AgdaSpace{}%
\AgdaSymbol{→}\AgdaSpace{}%
\AgdaFunction{is-homomorphism}\AgdaSpace{}%
\AgdaSymbol{(}\AgdaFunction{⨅}\AgdaSpace{}%
\AgdaBound{ℬ}\AgdaSymbol{)}\AgdaSpace{}%
\AgdaSymbol{(}\AgdaBound{ℬ}\AgdaSpace{}%
\AgdaBound{i}\AgdaSymbol{)}\AgdaSpace{}%
\AgdaSymbol{(}\AgdaFunction{𝒽}\AgdaSpace{}%
\AgdaBound{i}\AgdaSymbol{)}\<%
\\
%
\>[2]\AgdaFunction{𝒽hom}\AgdaSpace{}%
\AgdaSymbol{\AgdaUnderscore{}}\AgdaSpace{}%
\AgdaSymbol{\AgdaUnderscore{}}\AgdaSpace{}%
\AgdaSymbol{\AgdaUnderscore{}}\AgdaSpace{}%
\AgdaSymbol{=}\AgdaSpace{}%
\AgdaInductiveConstructor{𝓇ℯ𝒻𝓁}\<%
\end{code}
\ccpad
Of course, we could prove a more general result involving projections onto multiple factors, but so far the single-factor result has sufficed.


\subsection{Homomorphism Theorems}\label{sec:homom-theor}
\firstsentence{Homomorphisms}{Noether}
% -*- TeX-master: "ualib-part2.tex" -*-
%%% Local Variables: 
%%% mode: latex
%%% TeX-engine: 'xetex
%%% End: 
\subsubsection{The First Homomorphism Theorem}\label{the-first-homomorphism-theorem}

Here is a version of the so-called \emph{First Homomorphism theorem}, sometimes called Noether's First Homomorphism theorem, after Emmy Noether who was among the first proponents of the abstract approach to algebra that we now refer to as ``modern algebra.''
\ccpad
\begin{code}%
\>[0]\AgdaKeyword{open}\AgdaSpace{}%
\AgdaModule{Congruence}\<%
\\
%
\>[0]\AgdaKeyword{module}\AgdaSpace{}%
\AgdaModule{\AgdaUnderscore{}}%
\>[31I]\AgdaSymbol{\{}\AgdaBound{𝓤}\AgdaSpace{}%
\AgdaBound{𝓦}\AgdaSpace{}%
\AgdaSymbol{:}\AgdaSpace{}%
\AgdaFunction{Universe}\AgdaSymbol{\}}\<%
\\
\>[.][@{}l@{}]\<[31I]%
\>[9]\AgdaComment{-- extensionality assumptions --}\<%
\\
\>[9][@{}l@{\AgdaIndent{0}}]%
\>[12]\AgdaSymbol{(}\AgdaBound{fe}\AgdaSpace{}%
\AgdaSymbol{:}\AgdaSpace{}%
\AgdaFunction{dfunext}\AgdaSpace{}%
\AgdaBound{𝓥}\AgdaSpace{}%
\AgdaBound{𝓦}\AgdaSymbol{)}\AgdaSpace{}%
\AgdaSymbol{(}\AgdaBound{pe}\AgdaSpace{}%
\AgdaSymbol{:}\AgdaSpace{}%
\AgdaFunction{prop-ext}\AgdaSpace{}%
\AgdaBound{𝓤}\AgdaSpace{}%
\AgdaBound{𝓦}\AgdaSymbol{)}\<%
\\
%
\\[\AgdaEmptyExtraSkip]%
%
\>[9]\AgdaSymbol{(}\AgdaBound{𝑨}\AgdaSpace{}%
\AgdaSymbol{:}\AgdaSpace{}%
\AgdaFunction{Algebra}\AgdaSpace{}%
\AgdaBound{𝓤}\AgdaSpace{}%
\AgdaBound{𝑆}\AgdaSymbol{)(}\AgdaBound{𝑩}\AgdaSpace{}%
\AgdaSymbol{:}\AgdaSpace{}%
\AgdaFunction{Algebra}\AgdaSpace{}%
\AgdaBound{𝓦}\AgdaSpace{}%
\AgdaBound{𝑆}\AgdaSymbol{)(}\AgdaBound{h}\AgdaSpace{}%
\AgdaSymbol{:}\AgdaSpace{}%
\AgdaFunction{hom}\AgdaSpace{}%
\AgdaBound{𝑨}\AgdaSpace{}%
\AgdaBound{𝑩}\AgdaSymbol{)}\<%
\\
%
\\[\AgdaEmptyExtraSkip]%
%
\>[9]\AgdaComment{-- truncation assumptions --}\<%
\\
\>[9][@{}l@{\AgdaIndent{0}}]%
\>[12]\AgdaSymbol{(}\AgdaBound{Bset}\AgdaSpace{}%
\AgdaSymbol{:}\AgdaSpace{}%
\AgdaFunction{is-set}\AgdaSpace{}%
\AgdaOperator{\AgdaFunction{∣}}\AgdaSpace{}%
\AgdaBound{𝑩}\AgdaSpace{}%
\AgdaOperator{\AgdaFunction{∣}}\AgdaSymbol{)}\<%
\\
%
\>[12]\AgdaSymbol{(}\AgdaBound{ssR}\AgdaSpace{}%
\AgdaSymbol{:}\AgdaSpace{}%
\AgdaSymbol{∀}\AgdaSpace{}%
\AgdaBound{a}\AgdaSpace{}%
\AgdaBound{x}\AgdaSpace{}%
\AgdaSymbol{→}\AgdaSpace{}%
\AgdaFunction{is-subsingleton}\AgdaSpace{}%
\AgdaSymbol{(}\AgdaOperator{\AgdaField{⟨}}\AgdaSpace{}%
\AgdaFunction{kercon}\AgdaSpace{}%
\AgdaBound{𝑩}\AgdaSpace{}%
\AgdaBound{h}\AgdaSpace{}%
\AgdaOperator{\AgdaField{⟩}}\AgdaSpace{}%
\AgdaBound{a}\AgdaSpace{}%
\AgdaBound{x}\AgdaSymbol{))}\<%
\\
%
\>[12]\AgdaSymbol{(}\AgdaBound{ssA}\AgdaSpace{}%
\AgdaSymbol{:}\AgdaSpace{}%
\AgdaSymbol{∀}\AgdaSpace{}%
\AgdaBound{C}\AgdaSpace{}%
\AgdaSymbol{→}\AgdaSpace{}%
\AgdaFunction{is-subsingleton}\AgdaSpace{}%
\AgdaSymbol{(}\AgdaFunction{𝒞}\AgdaSymbol{\{}\AgdaArgument{A}\AgdaSpace{}%
\AgdaSymbol{=}\AgdaSpace{}%
\AgdaOperator{\AgdaFunction{∣}}\AgdaSpace{}%
\AgdaBound{𝑨}\AgdaSpace{}%
\AgdaOperator{\AgdaFunction{∣}}\AgdaSymbol{\}\{}\AgdaOperator{\AgdaField{⟨}}\AgdaSpace{}%
\AgdaFunction{kercon}\AgdaSpace{}%
\AgdaBound{𝑩}\AgdaSpace{}%
\AgdaBound{h}\AgdaSpace{}%
\AgdaOperator{\AgdaField{⟩}}\AgdaSymbol{\}}\AgdaSpace{}%
\AgdaBound{C}\AgdaSymbol{))}\<%
\\
%
\\[\AgdaEmptyExtraSkip]%
\>[0][@{}l@{\AgdaIndent{0}}]%
\>[1]\AgdaKeyword{where}\<%
\\
%
\\[\AgdaEmptyExtraSkip]%
%
\>[1]\AgdaFunction{FirstHomomorphismTheorem}\AgdaSpace{}%
\AgdaSymbol{:}\<%
\\
%
\\[\AgdaEmptyExtraSkip]%
\>[1][@{}l@{\AgdaIndent{0}}]%
\>[2]\AgdaFunction{Σ}\AgdaSpace{}%
\AgdaBound{ϕ}%
\>[90I]\AgdaFunction{꞉}\AgdaSpace{}%
\AgdaFunction{hom}\AgdaSpace{}%
\AgdaSymbol{(}\AgdaBound{𝑨}\AgdaSpace{}%
\AgdaOperator{\AgdaFunction{[}}\AgdaSpace{}%
\AgdaBound{𝑩}\AgdaSpace{}%
\AgdaOperator{\AgdaFunction{]/ker}}\AgdaSpace{}%
\AgdaBound{h}\AgdaSymbol{)}\AgdaSpace{}%
\AgdaBound{𝑩}\AgdaSpace{}%
\AgdaFunction{,}\<%
\\
\>[90I][@{}l@{\AgdaIndent{0}}]%
\>[7]\AgdaSymbol{(}\AgdaOperator{\AgdaFunction{∣}}\AgdaSpace{}%
\AgdaBound{h}\AgdaSpace{}%
\AgdaOperator{\AgdaFunction{∣}}\AgdaSpace{}%
\AgdaOperator{\AgdaDatatype{≡}}\AgdaSpace{}%
\AgdaOperator{\AgdaFunction{∣}}\AgdaSpace{}%
\AgdaBound{ϕ}\AgdaSpace{}%
\AgdaOperator{\AgdaFunction{∣}}\AgdaSpace{}%
\AgdaOperator{\AgdaFunction{∘}}\AgdaSpace{}%
\AgdaOperator{\AgdaFunction{∣}}\AgdaSpace{}%
\AgdaFunction{πker}\AgdaSpace{}%
\AgdaBound{𝑩}\AgdaSpace{}%
\AgdaBound{h}\AgdaSpace{}%
\AgdaOperator{\AgdaFunction{∣}}\AgdaSymbol{)}\AgdaSpace{}%
\AgdaOperator{\AgdaFunction{×}}\AgdaSpace{}%
\AgdaFunction{Monic}\AgdaSpace{}%
\AgdaOperator{\AgdaFunction{∣}}\AgdaSpace{}%
\AgdaBound{ϕ}\AgdaSpace{}%
\AgdaOperator{\AgdaFunction{∣}}\AgdaSpace{}%
\AgdaOperator{\AgdaFunction{×}}\AgdaSpace{}%
\AgdaFunction{is-embedding}\AgdaSpace{}%
\AgdaOperator{\AgdaFunction{∣}}\AgdaSpace{}%
\AgdaBound{ϕ}\AgdaSpace{}%
\AgdaOperator{\AgdaFunction{∣}}\<%
\\
%
\\[\AgdaEmptyExtraSkip]%
%
\>[1]\AgdaFunction{FirstHomomorphismTheorem}\AgdaSpace{}%
\AgdaSymbol{=}\AgdaSpace{}%
\AgdaSymbol{(}\AgdaFunction{ϕ}\AgdaSpace{}%
\AgdaOperator{\AgdaInductiveConstructor{,}}\AgdaSpace{}%
\AgdaFunction{ϕhom}\AgdaSymbol{)}\AgdaSpace{}%
\AgdaOperator{\AgdaInductiveConstructor{,}}\AgdaSpace{}%
\AgdaFunction{ϕcom}\AgdaSpace{}%
\AgdaOperator{\AgdaInductiveConstructor{,}}\AgdaSpace{}%
\AgdaFunction{ϕmon}\AgdaSpace{}%
\AgdaOperator{\AgdaInductiveConstructor{,}}\AgdaSpace{}%
\AgdaFunction{ϕemb}\<%
\\
\>[1][@{}l@{\AgdaIndent{0}}]%
\>[2]\AgdaKeyword{where}\<%
\\
%
\>[2]\AgdaFunction{θ}\AgdaSpace{}%
\AgdaSymbol{:}\AgdaSpace{}%
\AgdaRecord{Congruence}\AgdaSpace{}%
\AgdaBound{𝑨}\<%
\\
%
\>[2]\AgdaFunction{θ}\AgdaSpace{}%
\AgdaSymbol{=}\AgdaSpace{}%
\AgdaFunction{kercon}\AgdaSpace{}%
\AgdaBound{𝑩}\AgdaSpace{}%
\AgdaBound{h}\<%
\\
%
\\[\AgdaEmptyExtraSkip]%
%
\>[2]\AgdaFunction{ϕ}\AgdaSpace{}%
\AgdaSymbol{:}\AgdaSpace{}%
\AgdaOperator{\AgdaFunction{∣}}\AgdaSpace{}%
\AgdaBound{𝑨}\AgdaSpace{}%
\AgdaOperator{\AgdaFunction{[}}\AgdaSpace{}%
\AgdaBound{𝑩}\AgdaSpace{}%
\AgdaOperator{\AgdaFunction{]/ker}}\AgdaSpace{}%
\AgdaBound{h}\AgdaSpace{}%
\AgdaOperator{\AgdaFunction{∣}}\AgdaSpace{}%
\AgdaSymbol{→}\AgdaSpace{}%
\AgdaOperator{\AgdaFunction{∣}}\AgdaSpace{}%
\AgdaBound{𝑩}\AgdaSpace{}%
\AgdaOperator{\AgdaFunction{∣}}\<%
\\
%
\>[2]\AgdaFunction{ϕ}\AgdaSpace{}%
\AgdaBound{a}\AgdaSpace{}%
\AgdaSymbol{=}\AgdaSpace{}%
\AgdaOperator{\AgdaFunction{∣}}\AgdaSpace{}%
\AgdaBound{h}\AgdaSpace{}%
\AgdaOperator{\AgdaFunction{∣}}\AgdaSpace{}%
\AgdaOperator{\AgdaFunction{⌜}}\AgdaSpace{}%
\AgdaBound{a}\AgdaSpace{}%
\AgdaOperator{\AgdaFunction{⌝}}\<%
\\
%
\\[\AgdaEmptyExtraSkip]%
%
\>[2]\AgdaFunction{𝑹}\AgdaSpace{}%
\AgdaSymbol{:}\AgdaSpace{}%
\AgdaFunction{Pred₂}\AgdaSpace{}%
\AgdaOperator{\AgdaFunction{∣}}\AgdaSpace{}%
\AgdaBound{𝑨}\AgdaSpace{}%
\AgdaOperator{\AgdaFunction{∣}}\AgdaSpace{}%
\AgdaBound{𝓦}\<%
\\
%
\>[2]\AgdaFunction{𝑹}\AgdaSpace{}%
\AgdaSymbol{=}\AgdaSpace{}%
\AgdaOperator{\AgdaField{⟨}}\AgdaSpace{}%
\AgdaFunction{kercon}\AgdaSpace{}%
\AgdaBound{𝑩}\AgdaSpace{}%
\AgdaBound{h}\AgdaSpace{}%
\AgdaOperator{\AgdaField{⟩}}\AgdaSpace{}%
\AgdaOperator{\AgdaInductiveConstructor{,}}\AgdaSpace{}%
\AgdaBound{ssR}\<%
\\
%
\\[\AgdaEmptyExtraSkip]%
%
\>[2]\AgdaFunction{ϕhom}\AgdaSpace{}%
\AgdaSymbol{:}\AgdaSpace{}%
\AgdaFunction{is-homomorphism}\AgdaSpace{}%
\AgdaSymbol{(}\AgdaBound{𝑨}\AgdaSpace{}%
\AgdaOperator{\AgdaFunction{[}}\AgdaSpace{}%
\AgdaBound{𝑩}\AgdaSpace{}%
\AgdaOperator{\AgdaFunction{]/ker}}\AgdaSpace{}%
\AgdaBound{h}\AgdaSymbol{)}\AgdaSpace{}%
\AgdaBound{𝑩}\AgdaSpace{}%
\AgdaFunction{ϕ}\<%
\\
%
\>[2]\AgdaFunction{ϕhom}\AgdaSpace{}%
\AgdaBound{𝑓}\AgdaSpace{}%
\AgdaBound{𝒂}%
\>[183I]\AgdaSymbol{=}%
\>[14]\AgdaOperator{\AgdaFunction{∣}}\AgdaSpace{}%
\AgdaBound{h}\AgdaSpace{}%
\AgdaOperator{\AgdaFunction{∣}}\AgdaSpace{}%
\AgdaSymbol{(}\AgdaSpace{}%
\AgdaSymbol{(}\AgdaBound{𝑓}\AgdaSpace{}%
\AgdaOperator{\AgdaFunction{̂}}\AgdaSpace{}%
\AgdaBound{𝑨}\AgdaSymbol{)}\AgdaSpace{}%
\AgdaSymbol{(λ}\AgdaSpace{}%
\AgdaBound{x}\AgdaSpace{}%
\AgdaSymbol{→}\AgdaSpace{}%
\AgdaOperator{\AgdaFunction{⌜}}\AgdaSpace{}%
\AgdaBound{𝒂}\AgdaSpace{}%
\AgdaBound{x}\AgdaSpace{}%
\AgdaOperator{\AgdaFunction{⌝}}\AgdaSymbol{)}\AgdaSpace{}%
\AgdaSymbol{)}\AgdaSpace{}%
\AgdaOperator{\AgdaFunction{≡⟨}}\AgdaSpace{}%
\AgdaOperator{\AgdaFunction{∥}}\AgdaSpace{}%
\AgdaBound{h}\AgdaSpace{}%
\AgdaOperator{\AgdaFunction{∥}}\AgdaSpace{}%
\AgdaBound{𝑓}\AgdaSpace{}%
\AgdaSymbol{(λ}\AgdaSpace{}%
\AgdaBound{x}\AgdaSpace{}%
\AgdaSymbol{→}\AgdaSpace{}%
\AgdaOperator{\AgdaFunction{⌜}}\AgdaSpace{}%
\AgdaBound{𝒂}\AgdaSpace{}%
\AgdaBound{x}\AgdaSpace{}%
\AgdaOperator{\AgdaFunction{⌝}}\AgdaSymbol{)}%
\>[76]\AgdaOperator{\AgdaFunction{⟩}}\<%
\\
\>[183I][@{}l@{\AgdaIndent{0}}]%
\>[13]\AgdaSymbol{(}\AgdaBound{𝑓}\AgdaSpace{}%
\AgdaOperator{\AgdaFunction{̂}}\AgdaSpace{}%
\AgdaBound{𝑩}\AgdaSymbol{)}\AgdaSpace{}%
\AgdaSymbol{(}\AgdaOperator{\AgdaFunction{∣}}\AgdaSpace{}%
\AgdaBound{h}\AgdaSpace{}%
\AgdaOperator{\AgdaFunction{∣}}\AgdaSpace{}%
\AgdaOperator{\AgdaFunction{∘}}\AgdaSpace{}%
\AgdaSymbol{(λ}\AgdaSpace{}%
\AgdaBound{x}\AgdaSpace{}%
\AgdaSymbol{→}\AgdaSpace{}%
\AgdaOperator{\AgdaFunction{⌜}}\AgdaSpace{}%
\AgdaBound{𝒂}\AgdaSpace{}%
\AgdaBound{x}\AgdaSpace{}%
\AgdaOperator{\AgdaFunction{⌝}}\AgdaSymbol{))}\AgdaSpace{}%
\AgdaOperator{\AgdaFunction{≡⟨}}\AgdaSpace{}%
\AgdaFunction{ap}\AgdaSpace{}%
\AgdaSymbol{(}\AgdaBound{𝑓}\AgdaSpace{}%
\AgdaOperator{\AgdaFunction{̂}}\AgdaSpace{}%
\AgdaBound{𝑩}\AgdaSymbol{)}\AgdaSpace{}%
\AgdaSymbol{(}\AgdaBound{fe}\AgdaSpace{}%
\AgdaSymbol{λ}\AgdaSpace{}%
\AgdaBound{x}\AgdaSpace{}%
\AgdaSymbol{→}\AgdaSpace{}%
\AgdaInductiveConstructor{𝓇ℯ𝒻𝓁}\AgdaSymbol{)}\AgdaSpace{}%
\AgdaOperator{\AgdaFunction{⟩}}\<%
\\
%
\>[13]\AgdaSymbol{(}\AgdaBound{𝑓}\AgdaSpace{}%
\AgdaOperator{\AgdaFunction{̂}}\AgdaSpace{}%
\AgdaBound{𝑩}\AgdaSymbol{)}\AgdaSpace{}%
\AgdaSymbol{(λ}\AgdaSpace{}%
\AgdaBound{x}\AgdaSpace{}%
\AgdaSymbol{→}\AgdaSpace{}%
\AgdaFunction{ϕ}\AgdaSpace{}%
\AgdaSymbol{(}\AgdaBound{𝒂}\AgdaSpace{}%
\AgdaBound{x}\AgdaSymbol{))}%
\>[49]\AgdaOperator{\AgdaFunction{∎}}\<%
\\
%
\\[\AgdaEmptyExtraSkip]%
%
\>[2]\AgdaFunction{ϕmon}\AgdaSpace{}%
\AgdaSymbol{:}\AgdaSpace{}%
\AgdaFunction{Monic}\AgdaSpace{}%
\AgdaFunction{ϕ}\<%
\\
%
\>[2]\AgdaFunction{ϕmon}\AgdaSpace{}%
\AgdaSymbol{(}\AgdaDottedPattern{\AgdaSymbol{.(}}\AgdaDottedPattern{\AgdaOperator{\AgdaField{⟨}}}\AgdaSpace{}%
\AgdaDottedPattern{\AgdaFunction{θ}}\AgdaSpace{}%
\AgdaDottedPattern{\AgdaOperator{\AgdaField{⟩}}}\AgdaSpace{}%
\AgdaDottedPattern{\AgdaBound{u}}\AgdaDottedPattern{\AgdaSymbol{)}}\AgdaSpace{}%
\AgdaOperator{\AgdaInductiveConstructor{,}}\AgdaSpace{}%
\AgdaBound{u}\AgdaSpace{}%
\AgdaOperator{\AgdaInductiveConstructor{,}}\AgdaSpace{}%
\AgdaInductiveConstructor{refl}\AgdaSpace{}%
\AgdaSymbol{\AgdaUnderscore{})}\AgdaSpace{}%
\AgdaSymbol{(}\AgdaDottedPattern{\AgdaSymbol{.(}}\AgdaDottedPattern{\AgdaOperator{\AgdaField{⟨}}}\AgdaSpace{}%
\AgdaDottedPattern{\AgdaFunction{θ}}\AgdaSpace{}%
\AgdaDottedPattern{\AgdaOperator{\AgdaField{⟩}}}\AgdaSpace{}%
\AgdaDottedPattern{\AgdaBound{v}}\AgdaDottedPattern{\AgdaSymbol{)}}\AgdaSpace{}%
\AgdaOperator{\AgdaInductiveConstructor{,}}\AgdaSpace{}%
\AgdaBound{v}\AgdaSpace{}%
\AgdaOperator{\AgdaInductiveConstructor{,}}\AgdaSpace{}%
\AgdaInductiveConstructor{refl}\AgdaSpace{}%
\AgdaSymbol{\AgdaUnderscore{})}\AgdaSpace{}%
\AgdaBound{ϕuv}\AgdaSpace{}%
\AgdaSymbol{=}\<%
\\
\>[2][@{}l@{\AgdaIndent{0}}]%
\>[3]\AgdaFunction{class-extensionality'}\AgdaSpace{}%
\AgdaSymbol{\{}\AgdaArgument{𝑹}\AgdaSpace{}%
\AgdaSymbol{=}\AgdaSpace{}%
\AgdaFunction{𝑹}\AgdaSymbol{\}}\AgdaSpace{}%
\AgdaBound{pe}\AgdaSpace{}%
\AgdaBound{ssA}\AgdaSpace{}%
\AgdaSymbol{(}\AgdaField{IsEquiv}\AgdaSpace{}%
\AgdaFunction{θ}\AgdaSymbol{)}\AgdaSpace{}%
\AgdaBound{ϕuv}\<%
\\
%
\\[\AgdaEmptyExtraSkip]%
%
\>[2]\AgdaFunction{ϕcom}\AgdaSpace{}%
\AgdaSymbol{:}\AgdaSpace{}%
\AgdaOperator{\AgdaFunction{∣}}\AgdaSpace{}%
\AgdaBound{h}\AgdaSpace{}%
\AgdaOperator{\AgdaFunction{∣}}\AgdaSpace{}%
\AgdaOperator{\AgdaDatatype{≡}}\AgdaSpace{}%
\AgdaFunction{ϕ}\AgdaSpace{}%
\AgdaOperator{\AgdaFunction{∘}}\AgdaSpace{}%
\AgdaOperator{\AgdaFunction{∣}}\AgdaSpace{}%
\AgdaFunction{πker}\AgdaSpace{}%
\AgdaBound{𝑩}\AgdaSpace{}%
\AgdaBound{h}\AgdaSpace{}%
\AgdaOperator{\AgdaFunction{∣}}\<%
\\
%
\>[2]\AgdaFunction{ϕcom}\AgdaSpace{}%
\AgdaSymbol{=}\AgdaSpace{}%
\AgdaInductiveConstructor{𝓇ℯ𝒻𝓁}\<%
\\
%
\\[\AgdaEmptyExtraSkip]%
%
\>[2]\AgdaFunction{ϕemb}\AgdaSpace{}%
\AgdaSymbol{:}\AgdaSpace{}%
\AgdaFunction{is-embedding}\AgdaSpace{}%
\AgdaFunction{ϕ}\<%
\\
%
\>[2]\AgdaFunction{ϕemb}\AgdaSpace{}%
\AgdaSymbol{=}\AgdaSpace{}%
\AgdaFunction{monic-is-embedding|sets}\AgdaSpace{}%
\AgdaFunction{ϕ}\AgdaSpace{}%
\AgdaBound{Bset}\AgdaSpace{}%
\AgdaFunction{ϕmon}\<%
\end{code}
\ccpad
Next we show that the homomorphism \ab{ϕ}, whose existence we just proved, is unique.
\ccpad
\begin{code}%
\>[0][@{}l@{\AgdaIndent{1}}]%
\>[1]\AgdaFunction{NoetherHomUnique}\AgdaSpace{}%
\AgdaSymbol{:}\AgdaSpace{}%
\>[234I]\AgdaSymbol{(}\AgdaBound{f}\AgdaSpace{}%
\AgdaBound{g}\AgdaSpace{}%
\AgdaSymbol{:}\AgdaSpace{}%
\AgdaFunction{hom}\AgdaSpace{}%
\AgdaSymbol{(}\AgdaBound{𝑨}\AgdaSpace{}%
\AgdaOperator{\AgdaFunction{[}}\AgdaSpace{}%
\AgdaBound{𝑩}\AgdaSpace{}%
\AgdaOperator{\AgdaFunction{]/ker}}\AgdaSpace{}%
\AgdaBound{h}\AgdaSymbol{)}\AgdaSpace{}%
\AgdaBound{𝑩}\AgdaSymbol{)}\<%
\\
\>[1][@{}l@{\AgdaIndent{0}}]%
\>[2]\AgdaSymbol{→}%
\>[.][@{}l@{}]\<[234I]%
\>[20]\AgdaOperator{\AgdaFunction{∣}}\AgdaSpace{}%
\AgdaBound{h}\AgdaSpace{}%
\AgdaOperator{\AgdaFunction{∣}}\AgdaSpace{}%
\AgdaOperator{\AgdaDatatype{≡}}\AgdaSpace{}%
\AgdaOperator{\AgdaFunction{∣}}\AgdaSpace{}%
\AgdaBound{f}\AgdaSpace{}%
\AgdaOperator{\AgdaFunction{∣}}\AgdaSpace{}%
\AgdaOperator{\AgdaFunction{∘}}\AgdaSpace{}%
\AgdaOperator{\AgdaFunction{∣}}\AgdaSpace{}%
\AgdaFunction{πker}\AgdaSpace{}%
\AgdaBound{𝑩}\AgdaSpace{}%
\AgdaBound{h}\AgdaSpace{}%
\AgdaOperator{\AgdaFunction{∣}}\AgdaSpace{}%
\AgdaSymbol{→}\AgdaSpace{}%
\AgdaOperator{\AgdaFunction{∣}}\AgdaSpace{}%
\AgdaBound{h}\AgdaSpace{}%
\AgdaOperator{\AgdaFunction{∣}}\AgdaSpace{}%
\AgdaOperator{\AgdaDatatype{≡}}\AgdaSpace{}%
\AgdaOperator{\AgdaFunction{∣}}\AgdaSpace{}%
\AgdaBound{g}\AgdaSpace{}%
\AgdaOperator{\AgdaFunction{∣}}\AgdaSpace{}%
\AgdaOperator{\AgdaFunction{∘}}\AgdaSpace{}%
\AgdaOperator{\AgdaFunction{∣}}\AgdaSpace{}%
\AgdaFunction{πker}\AgdaSpace{}%
\AgdaBound{𝑩}\AgdaSpace{}%
\AgdaBound{h}\AgdaSpace{}%
\AgdaOperator{\AgdaFunction{∣}}\<%
\\
%
\>[20]\AgdaComment{--------------------------------------------------------------------}\<%
\\
%
\>[2]\AgdaSymbol{→}%
\>[20]\AgdaSymbol{∀}\AgdaSpace{}%
\AgdaBound{a}%
\>[25]\AgdaSymbol{→}%
\>[28]\AgdaOperator{\AgdaFunction{∣}}\AgdaSpace{}%
\AgdaBound{f}\AgdaSpace{}%
\AgdaOperator{\AgdaFunction{∣}}\AgdaSpace{}%
\AgdaBound{a}\AgdaSpace{}%
\AgdaOperator{\AgdaDatatype{≡}}\AgdaSpace{}%
\AgdaOperator{\AgdaFunction{∣}}\AgdaSpace{}%
\AgdaBound{g}\AgdaSpace{}%
\AgdaOperator{\AgdaFunction{∣}}\AgdaSpace{}%
\AgdaBound{a}\<%
\\
%
\\[\AgdaEmptyExtraSkip]%
%
\>[1]\AgdaFunction{NoetherHomUnique}\AgdaSpace{}%
\AgdaBound{f}\AgdaSpace{}%
\AgdaBound{g}\AgdaSpace{}%
\AgdaBound{hfk}\AgdaSpace{}%
\AgdaBound{hgk}\AgdaSpace{}%
\AgdaSymbol{(}\AgdaDottedPattern{\AgdaSymbol{.(}}\AgdaDottedPattern{\AgdaOperator{\AgdaField{⟨}}}\AgdaSpace{}%
\AgdaDottedPattern{\AgdaFunction{kercon}}\AgdaSpace{}%
\AgdaDottedPattern{\AgdaBound{𝑩}}\AgdaSpace{}%
\AgdaDottedPattern{\AgdaBound{h}}\AgdaSpace{}%
\AgdaDottedPattern{\AgdaOperator{\AgdaField{⟩}}}\AgdaSpace{}%
\AgdaDottedPattern{\AgdaBound{a}}\AgdaDottedPattern{\AgdaSymbol{)}}\AgdaSpace{}%
\AgdaOperator{\AgdaInductiveConstructor{,}}\AgdaSpace{}%
\AgdaBound{a}\AgdaSpace{}%
\AgdaOperator{\AgdaInductiveConstructor{,}}\AgdaSpace{}%
\AgdaInductiveConstructor{𝓇ℯ𝒻𝓁}\AgdaSymbol{)}\AgdaSpace{}%
\AgdaSymbol{=}\<%
\\
%
\\[\AgdaEmptyExtraSkip]%
\>[1][@{}l@{\AgdaIndent{0}}]%
\>[2]\AgdaKeyword{let}\AgdaSpace{}%
\AgdaBound{θ}\AgdaSpace{}%
\AgdaSymbol{=}\AgdaSpace{}%
\AgdaSymbol{(}\AgdaOperator{\AgdaField{⟨}}\AgdaSpace{}%
\AgdaFunction{kercon}\AgdaSpace{}%
\AgdaBound{𝑩}\AgdaSpace{}%
\AgdaBound{h}\AgdaSpace{}%
\AgdaOperator{\AgdaField{⟩}}\AgdaSpace{}%
\AgdaBound{a}\AgdaSpace{}%
\AgdaOperator{\AgdaInductiveConstructor{,}}\AgdaSpace{}%
\AgdaBound{a}\AgdaSpace{}%
\AgdaOperator{\AgdaInductiveConstructor{,}}\AgdaSpace{}%
\AgdaInductiveConstructor{𝓇ℯ𝒻𝓁}\AgdaSymbol{)}\AgdaSpace{}%
\AgdaKeyword{in}\<%
\\
%
\\[\AgdaEmptyExtraSkip]%
\>[2][@{}l@{\AgdaIndent{0}}]%
\>[3]\AgdaOperator{\AgdaFunction{∣}}\AgdaSpace{}%
\AgdaBound{f}\AgdaSpace{}%
\AgdaOperator{\AgdaFunction{∣}}\AgdaSpace{}%
\AgdaBound{θ}%
\>[13]\AgdaOperator{\AgdaFunction{≡⟨}}\AgdaSpace{}%
\AgdaFunction{cong-app}\AgdaSpace{}%
\AgdaSymbol{(}\AgdaBound{hfk}\AgdaSpace{}%
\AgdaOperator{\AgdaFunction{⁻¹}}\AgdaSymbol{)}\AgdaSpace{}%
\AgdaBound{a}\AgdaSpace{}%
\AgdaOperator{\AgdaFunction{⟩}}%
\>[37]\AgdaOperator{\AgdaFunction{∣}}\AgdaSpace{}%
\AgdaBound{h}\AgdaSpace{}%
\AgdaOperator{\AgdaFunction{∣}}\AgdaSpace{}%
\AgdaBound{a}%
\>[47]\AgdaOperator{\AgdaFunction{≡⟨}}\AgdaSpace{}%
\AgdaFunction{cong-app}\AgdaSpace{}%
\AgdaSymbol{(}\AgdaBound{hgk}\AgdaSymbol{)}\AgdaSpace{}%
\AgdaBound{a}\AgdaSpace{}%
\AgdaOperator{\AgdaFunction{⟩}}%
\>[69]\AgdaOperator{\AgdaFunction{∣}}\AgdaSpace{}%
\AgdaBound{g}\AgdaSpace{}%
\AgdaOperator{\AgdaFunction{∣}}\AgdaSpace{}%
\AgdaBound{θ}%
\>[79]\AgdaOperator{\AgdaFunction{∎}}\<%
\end{code}
\ccpad
If we postulate function extensionality, then we have\footnote{%
We already assumed \emph{global} function extensionality in this module, so we could just appeal to that in this case.  However, we make a local function extensionality assumption explicit here merely to highlight where and how the principle is applied.}
\ccpad
\begin{code}%
\>[0][@{}l@{\AgdaIndent{1}}]%
\>[1]\AgdaFunction{fe-NoetherHomUnique}\AgdaSpace{}%
\AgdaSymbol{:}\AgdaSpace{}%
\>[123I]\AgdaFunction{funext}\AgdaSpace{}%
\AgdaSymbol{(}\AgdaBound{𝓤}\AgdaSpace{}%
\AgdaOperator{\AgdaFunction{⊔}}\AgdaSpace{}%
\AgdaBound{𝓦}\AgdaSpace{}%
\AgdaOperator{\AgdaFunction{⁺}}\AgdaSymbol{)}\AgdaSpace{}%
\AgdaBound{𝓦}\AgdaSpace{}%
\AgdaSymbol{→}\AgdaSpace{}%
\AgdaSymbol{(}\AgdaBound{f}\AgdaSpace{}%
\AgdaBound{g}\AgdaSpace{}%
\AgdaSymbol{:}\AgdaSpace{}%
\AgdaFunction{hom}\AgdaSpace{}%
\AgdaSymbol{(}\AgdaBound{𝑨}\AgdaSpace{}%
\AgdaOperator{\AgdaFunction{[}}\AgdaSpace{}%
\AgdaBound{𝑩}\AgdaSpace{}%
\AgdaOperator{\AgdaFunction{]/ker}}\AgdaSpace{}%
\AgdaBound{h}\AgdaSymbol{)}\AgdaSpace{}%
\AgdaBound{𝑩}\AgdaSymbol{)}\<%
\\
\>[1][@{}l@{\AgdaIndent{0}}]%
\>[2]\AgdaSymbol{→}%
\>[.][@{}l@{}]\<[123I]%
\>[23]\AgdaOperator{\AgdaFunction{∣}}\AgdaSpace{}%
\AgdaBound{h}\AgdaSpace{}%
\AgdaOperator{\AgdaFunction{∣}}\AgdaSpace{}%
\AgdaOperator{\AgdaDatatype{≡}}\AgdaSpace{}%
\AgdaOperator{\AgdaFunction{∣}}\AgdaSpace{}%
\AgdaBound{f}\AgdaSpace{}%
\AgdaOperator{\AgdaFunction{∣}}\AgdaSpace{}%
\AgdaOperator{\AgdaFunction{∘}}\AgdaSpace{}%
\AgdaOperator{\AgdaFunction{∣}}\AgdaSpace{}%
\AgdaFunction{πker}\AgdaSpace{}%
\AgdaBound{𝑩}\AgdaSpace{}%
\AgdaBound{h}\AgdaSpace{}%
\AgdaOperator{\AgdaFunction{∣}}\AgdaSpace{}%
\AgdaSymbol{→}\AgdaSpace{}%
\AgdaOperator{\AgdaFunction{∣}}\AgdaSpace{}%
\AgdaBound{h}\AgdaSpace{}%
\AgdaOperator{\AgdaFunction{∣}}\AgdaSpace{}%
\AgdaOperator{\AgdaDatatype{≡}}\AgdaSpace{}%
\AgdaOperator{\AgdaFunction{∣}}\AgdaSpace{}%
\AgdaBound{g}\AgdaSpace{}%
\AgdaOperator{\AgdaFunction{∣}}\AgdaSpace{}%
\AgdaOperator{\AgdaFunction{∘}}\AgdaSpace{}%
\AgdaOperator{\AgdaFunction{∣}}\AgdaSpace{}%
\AgdaFunction{πker}\AgdaSpace{}%
\AgdaBound{𝑩}\AgdaSpace{}%
\AgdaBound{h}\AgdaSpace{}%
\AgdaOperator{\AgdaFunction{∣}}\<%
\\
%
\>[23]\AgdaComment{---------------------------------------------------------------------}\<%
\\
%
\>[2]\AgdaSymbol{→}%
\>[23]\AgdaOperator{\AgdaFunction{∣}}\AgdaSpace{}%
\AgdaBound{f}\AgdaSpace{}%
\AgdaOperator{\AgdaFunction{∣}}\AgdaSpace{}%
\AgdaOperator{\AgdaDatatype{≡}}\AgdaSpace{}%
\AgdaOperator{\AgdaFunction{∣}}\AgdaSpace{}%
\AgdaBound{g}\AgdaSpace{}%
\AgdaOperator{\AgdaFunction{∣}}\<%
\\
%
\\[\AgdaEmptyExtraSkip]%
%
\>[1]\AgdaFunction{fe-NoetherHomUnique}\AgdaSpace{}%
\AgdaBound{fe}\AgdaSpace{}%
\AgdaBound{f}\AgdaSpace{}%
\AgdaBound{g}\AgdaSpace{}%
\AgdaBound{hfk}\AgdaSpace{}%
\AgdaBound{hgk}\AgdaSpace{}%
\AgdaSymbol{=}\AgdaSpace{}%
\AgdaBound{fe}\AgdaSpace{}%
\AgdaSymbol{(}\AgdaFunction{NoetherHomUnique}\AgdaSpace{}%
\AgdaBound{f}\AgdaSpace{}%
\AgdaBound{g}\AgdaSpace{}%
\AgdaBound{hfk}\AgdaSpace{}%
\AgdaBound{hgk}\AgdaSymbol{)}\<%
\end{code}
\ccpad
Assuming the hypotheses of the First Homomorphism theorem, if we add the assumption that \ab{h} is epic, then we get the so-called \defn{First Isomorphism theorem}.
\ccpad
\begin{code}%
\>[0][@{}l@{\AgdaIndent{1}}]%
\>[1]\AgdaFunction{FirstIsomorphismTheorem}\AgdaSpace{}%
\AgdaSymbol{:}\<%
\\
%
\\[\AgdaEmptyExtraSkip]%
\>[1][@{}l@{\AgdaIndent{0}}]%
\>[2]\AgdaFunction{Epic}\AgdaSpace{}%
\AgdaOperator{\AgdaFunction{∣}}\AgdaSpace{}%
\AgdaBound{h}\AgdaSpace{}%
\AgdaOperator{\AgdaFunction{∣}}\AgdaSpace{}%
\AgdaSymbol{→}\AgdaSpace{}%
\AgdaFunction{Σ}\AgdaSpace{}%
\AgdaBound{f}\AgdaSpace{}%
\AgdaFunction{꞉}\AgdaSpace{}%
\AgdaSymbol{(}\AgdaFunction{epi}\AgdaSpace{}%
\AgdaSymbol{(}\AgdaBound{𝑨}\AgdaSpace{}%
\AgdaOperator{\AgdaFunction{[}}\AgdaSpace{}%
\AgdaBound{𝑩}\AgdaSpace{}%
\AgdaOperator{\AgdaFunction{]/ker}}\AgdaSpace{}%
\AgdaBound{h}\AgdaSymbol{)}\AgdaSpace{}%
\AgdaBound{𝑩}\AgdaSymbol{)}\AgdaSpace{}%
\AgdaFunction{,}\AgdaSpace{}%
\AgdaSymbol{(}\AgdaOperator{\AgdaFunction{∣}}\AgdaSpace{}%
\AgdaBound{h}\AgdaSpace{}%
\AgdaOperator{\AgdaFunction{∣}}\AgdaSpace{}%
\AgdaOperator{\AgdaDatatype{≡}}\AgdaSpace{}%
\AgdaOperator{\AgdaFunction{∣}}\AgdaSpace{}%
\AgdaBound{f}\AgdaSpace{}%
\AgdaOperator{\AgdaFunction{∣}}\AgdaSpace{}%
\AgdaOperator{\AgdaFunction{∘}}\AgdaSpace{}%
\AgdaOperator{\AgdaFunction{∣}}\AgdaSpace{}%
\AgdaFunction{πker}\AgdaSpace{}%
\AgdaBound{𝑩}\AgdaSpace{}%
\AgdaBound{h}\AgdaSpace{}%
\AgdaOperator{\AgdaFunction{∣}}\AgdaSymbol{)}\AgdaSpace{}%
\AgdaOperator{\AgdaFunction{×}}\AgdaSpace{}%
\AgdaFunction{is-embedding}\AgdaSpace{}%
\AgdaOperator{\AgdaFunction{∣}}\AgdaSpace{}%
\AgdaBound{f}\AgdaSpace{}%
\AgdaOperator{\AgdaFunction{∣}}\<%
\\
%
\\[\AgdaEmptyExtraSkip]%
%
\>[1]\AgdaFunction{FirstIsomorphismTheorem}\AgdaSpace{}%
\AgdaBound{hE}\AgdaSpace{}%
\AgdaSymbol{=}\AgdaSpace{}%
\AgdaSymbol{(}\AgdaFunction{fmap}\AgdaSpace{}%
\AgdaOperator{\AgdaInductiveConstructor{,}}\AgdaSpace{}%
\AgdaFunction{fhom}\AgdaSpace{}%
\AgdaOperator{\AgdaInductiveConstructor{,}}\AgdaSpace{}%
\AgdaFunction{fepic}\AgdaSymbol{)}\AgdaSpace{}%
\AgdaOperator{\AgdaInductiveConstructor{,}}\AgdaSpace{}%
\AgdaInductiveConstructor{𝓇ℯ𝒻𝓁}\AgdaSpace{}%
\AgdaOperator{\AgdaInductiveConstructor{,}}\AgdaSpace{}%
\AgdaFunction{femb}\<%
\\
\>[1][@{}l@{\AgdaIndent{0}}]%
\>[2]\AgdaKeyword{where}\<%
\\
%
\>[2]\AgdaFunction{θ}\AgdaSpace{}%
\AgdaSymbol{:}\AgdaSpace{}%
\AgdaRecord{Congruence}\AgdaSpace{}%
\AgdaBound{𝑨}\<%
\\
%
\>[2]\AgdaFunction{θ}\AgdaSpace{}%
\AgdaSymbol{=}\AgdaSpace{}%
\AgdaFunction{kercon}\AgdaSpace{}%
\AgdaBound{𝑩}\AgdaSpace{}%
\AgdaBound{h}\<%
\\
%
\\[\AgdaEmptyExtraSkip]%
%
\>[2]\AgdaFunction{fmap}\AgdaSpace{}%
\AgdaSymbol{:}\AgdaSpace{}%
\AgdaOperator{\AgdaFunction{∣}}\AgdaSpace{}%
\AgdaBound{𝑨}\AgdaSpace{}%
\AgdaOperator{\AgdaFunction{[}}\AgdaSpace{}%
\AgdaBound{𝑩}\AgdaSpace{}%
\AgdaOperator{\AgdaFunction{]/ker}}\AgdaSpace{}%
\AgdaBound{h}\AgdaSpace{}%
\AgdaOperator{\AgdaFunction{∣}}\AgdaSpace{}%
\AgdaSymbol{→}\AgdaSpace{}%
\AgdaOperator{\AgdaFunction{∣}}\AgdaSpace{}%
\AgdaBound{𝑩}\AgdaSpace{}%
\AgdaOperator{\AgdaFunction{∣}}\<%
\\
%
\>[2]\AgdaFunction{fmap}\AgdaSpace{}%
\AgdaBound{⟦a⟧}\AgdaSpace{}%
\AgdaSymbol{=}\AgdaSpace{}%
\AgdaOperator{\AgdaFunction{∣}}\AgdaSpace{}%
\AgdaBound{h}\AgdaSpace{}%
\AgdaOperator{\AgdaFunction{∣}}\AgdaSpace{}%
\AgdaOperator{\AgdaFunction{⌜}}\AgdaSpace{}%
\AgdaBound{⟦a⟧}\AgdaSpace{}%
\AgdaOperator{\AgdaFunction{⌝}}\<%
\\
%
\\[\AgdaEmptyExtraSkip]%
%
\>[2]\AgdaFunction{fhom}\AgdaSpace{}%
\AgdaSymbol{:}\AgdaSpace{}%
\AgdaFunction{is-homomorphism}\AgdaSpace{}%
\AgdaSymbol{(}\AgdaBound{𝑨}\AgdaSpace{}%
\AgdaOperator{\AgdaFunction{[}}\AgdaSpace{}%
\AgdaBound{𝑩}\AgdaSpace{}%
\AgdaOperator{\AgdaFunction{]/ker}}\AgdaSpace{}%
\AgdaBound{h}\AgdaSymbol{)}\AgdaSpace{}%
\AgdaBound{𝑩}\AgdaSpace{}%
\AgdaFunction{fmap}\<%
\\
%
\>[2]\AgdaFunction{fhom}\AgdaSpace{}%
\AgdaBound{𝑓}\AgdaSpace{}%
\AgdaBound{a}\AgdaSpace{}%
\AgdaSymbol{=}%
\>[14]\AgdaOperator{\AgdaFunction{∣}}\AgdaSpace{}%
\AgdaBound{h}\AgdaSpace{}%
\AgdaOperator{\AgdaFunction{∣}}\AgdaSymbol{((}\AgdaBound{𝑓}\AgdaSpace{}%
\AgdaOperator{\AgdaFunction{̂}}\AgdaSpace{}%
\AgdaBound{𝑨}\AgdaSymbol{)}\AgdaSpace{}%
\AgdaSymbol{λ}\AgdaSpace{}%
\AgdaBound{x}\AgdaSpace{}%
\AgdaSymbol{→}\AgdaSpace{}%
\AgdaOperator{\AgdaFunction{⌜}}\AgdaSpace{}%
\AgdaBound{a}\AgdaSpace{}%
\AgdaBound{x}\AgdaSpace{}%
\AgdaOperator{\AgdaFunction{⌝}}\AgdaSymbol{)}%
\>[45]\AgdaOperator{\AgdaFunction{≡⟨}}\AgdaSpace{}%
\AgdaOperator{\AgdaFunction{∥}}\AgdaSpace{}%
\AgdaBound{h}\AgdaSpace{}%
\AgdaOperator{\AgdaFunction{∥}}\AgdaSpace{}%
\AgdaBound{𝑓}\AgdaSpace{}%
\AgdaSymbol{(λ}\AgdaSpace{}%
\AgdaBound{x}\AgdaSpace{}%
\AgdaSymbol{→}\AgdaSpace{}%
\AgdaOperator{\AgdaFunction{⌜}}\AgdaSpace{}%
\AgdaBound{a}\AgdaSpace{}%
\AgdaBound{x}\AgdaSpace{}%
\AgdaOperator{\AgdaFunction{⌝}}\AgdaSymbol{)}%
\>[73]\AgdaOperator{\AgdaFunction{⟩}}\<%
\\
%
\>[14]\AgdaSymbol{(}\AgdaBound{𝑓}\AgdaSpace{}%
\AgdaOperator{\AgdaFunction{̂}}\AgdaSpace{}%
\AgdaBound{𝑩}\AgdaSymbol{)(}\AgdaOperator{\AgdaFunction{∣}}\AgdaSpace{}%
\AgdaBound{h}\AgdaSpace{}%
\AgdaOperator{\AgdaFunction{∣}}\AgdaSpace{}%
\AgdaOperator{\AgdaFunction{∘}}\AgdaSpace{}%
\AgdaSymbol{λ}\AgdaSpace{}%
\AgdaBound{x}\AgdaSpace{}%
\AgdaSymbol{→}\AgdaSpace{}%
\AgdaOperator{\AgdaFunction{⌜}}\AgdaSpace{}%
\AgdaBound{a}\AgdaSpace{}%
\AgdaBound{x}\AgdaSpace{}%
\AgdaOperator{\AgdaFunction{⌝}}\AgdaSymbol{)}\AgdaSpace{}%
\AgdaOperator{\AgdaFunction{≡⟨}}\AgdaSpace{}%
\AgdaFunction{ap}\AgdaSymbol{(}\AgdaBound{𝑓}\AgdaSpace{}%
\AgdaOperator{\AgdaFunction{̂}}\AgdaSpace{}%
\AgdaBound{𝑩}\AgdaSymbol{)(}\AgdaBound{gfe}\AgdaSpace{}%
\AgdaSymbol{λ}\AgdaSpace{}%
\AgdaBound{\AgdaUnderscore{}}\AgdaSpace{}%
\AgdaSymbol{→}\AgdaSpace{}%
\AgdaInductiveConstructor{𝓇ℯ𝒻𝓁}\AgdaSymbol{)}\AgdaOperator{\AgdaFunction{⟩}}\<%
\\
%
\>[14]\AgdaSymbol{(}\AgdaBound{𝑓}\AgdaSpace{}%
\AgdaOperator{\AgdaFunction{̂}}\AgdaSpace{}%
\AgdaBound{𝑩}\AgdaSymbol{)}\AgdaSpace{}%
\AgdaSymbol{(}\AgdaFunction{fmap}\AgdaSpace{}%
\AgdaOperator{\AgdaFunction{∘}}\AgdaSpace{}%
\AgdaBound{a}\AgdaSymbol{)}%
\>[46]\AgdaOperator{\AgdaFunction{∎}}\<%
\\
%
\\[\AgdaEmptyExtraSkip]%
%
\>[2]\AgdaFunction{fepic}\AgdaSpace{}%
\AgdaSymbol{:}\AgdaSpace{}%
\AgdaFunction{Epic}\AgdaSpace{}%
\AgdaFunction{fmap}\<%
\\
%
\>[2]\AgdaFunction{fepic}\AgdaSpace{}%
\AgdaBound{b}\AgdaSpace{}%
\AgdaSymbol{=}\AgdaSpace{}%
\AgdaFunction{γ}\AgdaSpace{}%
\AgdaKeyword{where}\<%
\\
\>[2][@{}l@{\AgdaIndent{0}}]%
\>[3]\AgdaFunction{a}\AgdaSpace{}%
\AgdaSymbol{:}\AgdaSpace{}%
\AgdaOperator{\AgdaFunction{∣}}\AgdaSpace{}%
\AgdaBound{𝑨}\AgdaSpace{}%
\AgdaOperator{\AgdaFunction{∣}}\<%
\\
%
\>[3]\AgdaFunction{a}\AgdaSpace{}%
\AgdaSymbol{=}\AgdaSpace{}%
\AgdaFunction{EpicInv}\AgdaSpace{}%
\AgdaOperator{\AgdaFunction{∣}}\AgdaSpace{}%
\AgdaBound{h}\AgdaSpace{}%
\AgdaOperator{\AgdaFunction{∣}}\AgdaSpace{}%
\AgdaBound{hE}\AgdaSpace{}%
\AgdaBound{b}\<%
\\
%
\\[\AgdaEmptyExtraSkip]%
%
\>[3]\AgdaFunction{bfa}\AgdaSpace{}%
\AgdaSymbol{:}\AgdaSpace{}%
\AgdaBound{b}\AgdaSpace{}%
\AgdaOperator{\AgdaDatatype{≡}}\AgdaSpace{}%
\AgdaFunction{fmap}\AgdaSpace{}%
\AgdaOperator{\AgdaFunction{⟦}}\AgdaSpace{}%
\AgdaFunction{a}\AgdaSpace{}%
\AgdaOperator{\AgdaFunction{⟧}}\<%
\\
%
\>[3]\AgdaFunction{bfa}\AgdaSpace{}%
\AgdaSymbol{=}\AgdaSpace{}%
\AgdaSymbol{(}\AgdaFunction{cong-app}\AgdaSpace{}%
\AgdaSymbol{(}\AgdaFunction{EpicInvIsRightInv}\AgdaSpace{}%
\AgdaBound{gfe}\AgdaSpace{}%
\AgdaOperator{\AgdaFunction{∣}}\AgdaSpace{}%
\AgdaBound{h}\AgdaSpace{}%
\AgdaOperator{\AgdaFunction{∣}}\AgdaSpace{}%
\AgdaBound{hE}\AgdaSymbol{)}\AgdaSpace{}%
\AgdaBound{b}\AgdaSymbol{)}\AgdaOperator{\AgdaFunction{⁻¹}}\<%
\\
%
\\[\AgdaEmptyExtraSkip]%
%
\>[3]\AgdaFunction{γ}\AgdaSpace{}%
\AgdaSymbol{:}\AgdaSpace{}%
\AgdaOperator{\AgdaDatatype{Image}}\AgdaSpace{}%
\AgdaFunction{fmap}\AgdaSpace{}%
\AgdaOperator{\AgdaDatatype{∋}}\AgdaSpace{}%
\AgdaBound{b}\<%
\\
%
\>[3]\AgdaFunction{γ}\AgdaSpace{}%
\AgdaSymbol{=}\AgdaSpace{}%
\AgdaInductiveConstructor{Image\AgdaUnderscore{}∋\AgdaUnderscore{}.eq}\AgdaSpace{}%
\AgdaBound{b}\AgdaSpace{}%
\AgdaOperator{\AgdaFunction{⟦}}\AgdaSpace{}%
\AgdaFunction{a}\AgdaSpace{}%
\AgdaOperator{\AgdaFunction{⟧}}\AgdaSpace{}%
\AgdaFunction{bfa}\<%
\\
%
\\[\AgdaEmptyExtraSkip]%
%
\>[2]\AgdaFunction{fmon}\AgdaSpace{}%
\AgdaSymbol{:}\AgdaSpace{}%
\AgdaFunction{Monic}\AgdaSpace{}%
\AgdaFunction{fmap}\<%
\\
%
\>[2]\AgdaFunction{fmon}\AgdaSpace{}%
\AgdaSymbol{(}\AgdaDottedPattern{\AgdaSymbol{.(}}\AgdaDottedPattern{\AgdaOperator{\AgdaField{⟨}}}\AgdaSpace{}%
\AgdaDottedPattern{\AgdaFunction{θ}}\AgdaSpace{}%
\AgdaDottedPattern{\AgdaOperator{\AgdaField{⟩}}}\AgdaSpace{}%
\AgdaDottedPattern{\AgdaBound{a}}\AgdaDottedPattern{\AgdaSymbol{)}}\AgdaSpace{}%
\AgdaOperator{\AgdaInductiveConstructor{,}}\AgdaSpace{}%
\AgdaBound{a}\AgdaSpace{}%
\AgdaOperator{\AgdaInductiveConstructor{,}}\AgdaSpace{}%
\AgdaInductiveConstructor{𝓇ℯ𝒻𝓁}\AgdaSymbol{)}\AgdaSpace{}%
\AgdaSymbol{(}\AgdaDottedPattern{\AgdaSymbol{.(}}\AgdaDottedPattern{\AgdaOperator{\AgdaField{⟨}}}\AgdaSpace{}%
\AgdaDottedPattern{\AgdaFunction{θ}}\AgdaSpace{}%
\AgdaDottedPattern{\AgdaOperator{\AgdaField{⟩}}}\AgdaSpace{}%
\AgdaDottedPattern{\AgdaBound{a'}}\AgdaDottedPattern{\AgdaSymbol{)}}\AgdaSpace{}%
\AgdaOperator{\AgdaInductiveConstructor{,}}\AgdaSpace{}%
\AgdaBound{a'}\AgdaSpace{}%
\AgdaOperator{\AgdaInductiveConstructor{,}}\AgdaSpace{}%
\AgdaInductiveConstructor{𝓇ℯ𝒻𝓁}\AgdaSymbol{)}\AgdaSpace{}%
\AgdaBound{faa'}\AgdaSpace{}%
\AgdaSymbol{=}\<%
\\
\>[2][@{}l@{\AgdaIndent{0}}]%
\>[3]\AgdaFunction{class-extensionality'}\AgdaSpace{}%
\AgdaSymbol{\{}\AgdaArgument{𝑹}\AgdaSpace{}%
\AgdaSymbol{=}\AgdaSpace{}%
\AgdaOperator{\AgdaField{⟨}}\AgdaSpace{}%
\AgdaFunction{kercon}\AgdaSpace{}%
\AgdaBound{𝑩}\AgdaSpace{}%
\AgdaBound{h}\AgdaSpace{}%
\AgdaOperator{\AgdaField{⟩}}\AgdaSpace{}%
\AgdaOperator{\AgdaInductiveConstructor{,}}\AgdaSpace{}%
\AgdaBound{ssR}\AgdaSymbol{\}}\AgdaSpace{}%
\AgdaBound{pe}\AgdaSpace{}%
\AgdaBound{ssA}\AgdaSpace{}%
\AgdaSymbol{(}\AgdaField{IsEquiv}\AgdaSpace{}%
\AgdaFunction{θ}\AgdaSymbol{)}\AgdaSpace{}%
\AgdaBound{faa'}\<%
\\
%
\\[\AgdaEmptyExtraSkip]%
%
\>[2]\AgdaFunction{femb}\AgdaSpace{}%
\AgdaSymbol{:}\AgdaSpace{}%
\AgdaFunction{is-embedding}\AgdaSpace{}%
\AgdaFunction{fmap}\<%
\\
%
\>[2]\AgdaFunction{femb}\AgdaSpace{}%
\AgdaSymbol{=}\AgdaSpace{}%
\AgdaFunction{monic-is-embedding|sets}\AgdaSpace{}%
\AgdaFunction{fmap}\AgdaSpace{}%
\AgdaBound{Bset}\AgdaSpace{}%
\AgdaFunction{fmon}\<%
\end{code}
\ccpad
The argument used above to prove \af{NoetherHomUnique} can also be used to prove uniqueness of the epimorphism \ab f found in the isomorphism theorem.
\ccpad
\begin{code}%
\>[0][@{}l@{\AgdaIndent{1}}]%
\>[1]\AgdaFunction{NoetherIsoUnique}\AgdaSpace{}%
\AgdaSymbol{:}\AgdaSpace{}%
\>[334I]\AgdaSymbol{(}\AgdaBound{f}\AgdaSpace{}%
\AgdaBound{g}\AgdaSpace{}%
\AgdaSymbol{:}\AgdaSpace{}%
\AgdaFunction{epi}\AgdaSpace{}%
\AgdaSymbol{(}\AgdaBound{𝑨}\AgdaSpace{}%
\AgdaOperator{\AgdaFunction{[}}\AgdaSpace{}%
\AgdaBound{𝑩}\AgdaSpace{}%
\AgdaOperator{\AgdaFunction{]/ker}}\AgdaSpace{}%
\AgdaBound{h}\AgdaSymbol{)}\AgdaSpace{}%
\AgdaBound{𝑩}\AgdaSymbol{)}\AgdaSpace{}%
\AgdaSymbol{→}\AgdaSpace{}%
\AgdaOperator{\AgdaFunction{∣}}\AgdaSpace{}%
\AgdaBound{h}\AgdaSpace{}%
\AgdaOperator{\AgdaFunction{∣}}\AgdaSpace{}%
\AgdaOperator{\AgdaDatatype{≡}}\AgdaSpace{}%
\AgdaOperator{\AgdaFunction{∣}}\AgdaSpace{}%
\AgdaBound{f}\AgdaSpace{}%
\AgdaOperator{\AgdaFunction{∣}}\AgdaSpace{}%
\AgdaOperator{\AgdaFunction{∘}}\AgdaSpace{}%
\AgdaOperator{\AgdaFunction{∣}}\AgdaSpace{}%
\AgdaFunction{πker}\AgdaSpace{}%
\AgdaBound{𝑩}\AgdaSpace{}%
\AgdaBound{h}\AgdaSpace{}%
\AgdaOperator{\AgdaFunction{∣}}\<%
\\
\>[1][@{}l@{\AgdaIndent{0}}]%
\>[2]\AgdaSymbol{→}%
\>[.][@{}l@{}]\<[334I]%
\>[20]\AgdaOperator{\AgdaFunction{∣}}\AgdaSpace{}%
\AgdaBound{h}\AgdaSpace{}%
\AgdaOperator{\AgdaFunction{∣}}\AgdaSpace{}%
\AgdaOperator{\AgdaDatatype{≡}}\AgdaSpace{}%
\AgdaOperator{\AgdaFunction{∣}}\AgdaSpace{}%
\AgdaBound{g}\AgdaSpace{}%
\AgdaOperator{\AgdaFunction{∣}}\AgdaSpace{}%
\AgdaOperator{\AgdaFunction{∘}}\AgdaSpace{}%
\AgdaOperator{\AgdaFunction{∣}}\AgdaSpace{}%
\AgdaFunction{πker}\AgdaSpace{}%
\AgdaBound{𝑩}\AgdaSpace{}%
\AgdaBound{h}\AgdaSpace{}%
\AgdaOperator{\AgdaFunction{∣}}\AgdaSpace{}%
\AgdaSymbol{→}\AgdaSpace{}%
\AgdaSymbol{∀}\AgdaSpace{}%
\AgdaBound{a}\AgdaSpace{}%
\AgdaSymbol{→}\AgdaSpace{}%
\AgdaOperator{\AgdaFunction{∣}}\AgdaSpace{}%
\AgdaBound{f}\AgdaSpace{}%
\AgdaOperator{\AgdaFunction{∣}}\AgdaSpace{}%
\AgdaBound{a}\AgdaSpace{}%
\AgdaOperator{\AgdaDatatype{≡}}\AgdaSpace{}%
\AgdaOperator{\AgdaFunction{∣}}\AgdaSpace{}%
\AgdaBound{g}\AgdaSpace{}%
\AgdaOperator{\AgdaFunction{∣}}\AgdaSpace{}%
\AgdaBound{a}\<%
\\
%
\\[\AgdaEmptyExtraSkip]%
%
\>[1]\AgdaFunction{NoetherIsoUnique}\AgdaSpace{}%
\AgdaBound{f}\AgdaSpace{}%
\AgdaBound{g}\AgdaSpace{}%
\AgdaBound{hfk}\AgdaSpace{}%
\AgdaBound{hgk}\AgdaSpace{}%
\AgdaSymbol{(}\AgdaDottedPattern{\AgdaSymbol{.(}}\AgdaDottedPattern{\AgdaOperator{\AgdaField{⟨}}}\AgdaSpace{}%
\AgdaDottedPattern{\AgdaFunction{kercon}}\AgdaSpace{}%
\AgdaDottedPattern{\AgdaBound{𝑩}}\AgdaSpace{}%
\AgdaDottedPattern{\AgdaBound{h}}\AgdaSpace{}%
\AgdaDottedPattern{\AgdaOperator{\AgdaField{⟩}}}\AgdaSpace{}%
\AgdaDottedPattern{\AgdaBound{a}}\AgdaDottedPattern{\AgdaSymbol{)}}\AgdaSpace{}%
\AgdaOperator{\AgdaInductiveConstructor{,}}\AgdaSpace{}%
\AgdaBound{a}\AgdaSpace{}%
\AgdaOperator{\AgdaInductiveConstructor{,}}\AgdaSpace{}%
\AgdaInductiveConstructor{𝓇ℯ𝒻𝓁}\AgdaSymbol{)}\AgdaSpace{}%
\AgdaSymbol{=}\<%
\\
%
\\[\AgdaEmptyExtraSkip]%
\>[1][@{}l@{\AgdaIndent{0}}]%
\>[2]\AgdaKeyword{let}\AgdaSpace{}%
\AgdaBound{θ}\AgdaSpace{}%
\AgdaSymbol{=}\AgdaSpace{}%
\AgdaSymbol{(}\AgdaOperator{\AgdaField{⟨}}\AgdaSpace{}%
\AgdaFunction{kercon}\AgdaSpace{}%
\AgdaBound{𝑩}\AgdaSpace{}%
\AgdaBound{h}\AgdaSpace{}%
\AgdaOperator{\AgdaField{⟩}}\AgdaSpace{}%
\AgdaBound{a}\AgdaSpace{}%
\AgdaOperator{\AgdaInductiveConstructor{,}}\AgdaSpace{}%
\AgdaBound{a}\AgdaSpace{}%
\AgdaOperator{\AgdaInductiveConstructor{,}}\AgdaSpace{}%
\AgdaInductiveConstructor{𝓇ℯ𝒻𝓁}\AgdaSymbol{)}\AgdaSpace{}%
\AgdaKeyword{in}\<%
\\
%
\\[\AgdaEmptyExtraSkip]%
\>[2][@{}l@{\AgdaIndent{0}}]%
\>[3]\AgdaOperator{\AgdaFunction{∣}}\AgdaSpace{}%
\AgdaBound{f}\AgdaSpace{}%
\AgdaOperator{\AgdaFunction{∣}}\AgdaSpace{}%
\AgdaBound{θ}%
\>[13]\AgdaOperator{\AgdaFunction{≡⟨}}\AgdaSpace{}%
\AgdaFunction{cong-app}\AgdaSymbol{(}\AgdaBound{hfk}\AgdaSpace{}%
\AgdaOperator{\AgdaFunction{⁻¹}}\AgdaSymbol{)}\AgdaBound{a}\AgdaSpace{}%
\AgdaOperator{\AgdaFunction{⟩}}%
\>[37]\AgdaOperator{\AgdaFunction{∣}}\AgdaSpace{}%
\AgdaBound{h}\AgdaSpace{}%
\AgdaOperator{\AgdaFunction{∣}}\AgdaSpace{}%
\AgdaBound{a}%
\>[47]\AgdaOperator{\AgdaFunction{≡⟨}}\AgdaSpace{}%
\AgdaFunction{cong-app}\AgdaSymbol{(}\AgdaBound{hgk}\AgdaSymbol{)}\AgdaBound{a}\AgdaSpace{}%
\AgdaOperator{\AgdaFunction{⟩}}%
\>[69]\AgdaOperator{\AgdaFunction{∣}}\AgdaSpace{}%
\AgdaBound{g}\AgdaSpace{}%
\AgdaOperator{\AgdaFunction{∣}}\AgdaSpace{}%
\AgdaBound{θ}%
\>[79]\AgdaOperator{\AgdaFunction{∎}}\<%
\end{code}

\subsubsection{Composition of homomorphisms}\label{homomorphism-composition}

If we compose homomorphisms the result is again a homomorphism. We can formalize this in type theory and \agda in a number of alternative ways.
\ccpad
\begin{code}%
\>[1]\AgdaFunction{HCompClosed}\AgdaSpace{}%
\AgdaSymbol{:}\AgdaSpace{}%
\>[444I]\AgdaSymbol{(}\AgdaBound{𝑨}\AgdaSpace{}%
\AgdaSymbol{:}\AgdaSpace{}%
\AgdaFunction{Algebra}\AgdaSpace{}%
\AgdaBound{𝓧}\AgdaSpace{}%
\AgdaBound{𝑆}\AgdaSymbol{)(}\AgdaBound{𝑩}\AgdaSpace{}%
\AgdaSymbol{:}\AgdaSpace{}%
\AgdaFunction{Algebra}\AgdaSpace{}%
\AgdaBound{𝓨}\AgdaSpace{}%
\AgdaBound{𝑆}\AgdaSymbol{)(}\AgdaBound{𝑪}\AgdaSpace{}%
\AgdaSymbol{:}\AgdaSpace{}%
\AgdaFunction{Algebra}\AgdaSpace{}%
\AgdaBound{𝓩}\AgdaSpace{}%
\AgdaBound{𝑆}\AgdaSymbol{)}\<%
\\
\>[1][@{}l@{\AgdaIndent{0}}]%
\>[2]\AgdaSymbol{→}%
\>[.][@{}l@{}]\<[444I]%
\>[15]\AgdaFunction{hom}\AgdaSpace{}%
\AgdaBound{𝑨}\AgdaSpace{}%
\AgdaBound{𝑩}%
\>[24]\AgdaSymbol{→}%
\>[27]\AgdaFunction{hom}\AgdaSpace{}%
\AgdaBound{𝑩}\AgdaSpace{}%
\AgdaBound{𝑪}\AgdaSpace{}%
\AgdaSymbol{→}\AgdaSpace{}%
\AgdaFunction{hom}\AgdaSpace{}%
\AgdaBound{𝑨}\AgdaSpace{}%
\AgdaBound{𝑪}\<%
\\
%
\\[\AgdaEmptyExtraSkip]%
%
\>[1]\AgdaFunction{HCompClosed}\AgdaSpace{}%
\AgdaBound{𝑨}\AgdaSpace{}%
\AgdaBound{𝑩}\AgdaSpace{}%
\AgdaBound{𝑪}\AgdaSpace{}%
\AgdaSymbol{(}\AgdaBound{g}\AgdaSpace{}%
\AgdaOperator{\AgdaInductiveConstructor{,}}\AgdaSpace{}%
\AgdaBound{ghom}\AgdaSymbol{)}\AgdaSpace{}%
\AgdaSymbol{(}\AgdaBound{h}\AgdaSpace{}%
\AgdaOperator{\AgdaInductiveConstructor{,}}\AgdaSpace{}%
\AgdaBound{hhom}\AgdaSymbol{)}\AgdaSpace{}%
\AgdaSymbol{=}\AgdaSpace{}%
\AgdaBound{h}\AgdaSpace{}%
\AgdaOperator{\AgdaFunction{∘}}\AgdaSpace{}%
\AgdaBound{g}\AgdaSpace{}%
\AgdaOperator{\AgdaInductiveConstructor{,}}\AgdaSpace{}%
\AgdaFunction{γ}\AgdaSpace{}%
\AgdaKeyword{where}\<%
\\
%
\\[\AgdaEmptyExtraSkip]%
\>[1][@{}l@{\AgdaIndent{0}}]%
\>[2]\AgdaFunction{γ}\AgdaSpace{}%
\AgdaSymbol{:}\AgdaSpace{}%
\AgdaSymbol{(}\AgdaBound{𝑓}\AgdaSpace{}%
\AgdaSymbol{:}\AgdaSpace{}%
\AgdaOperator{\AgdaFunction{∣}}\AgdaSpace{}%
\AgdaBound{𝑆}\AgdaSpace{}%
\AgdaOperator{\AgdaFunction{∣}}\AgdaSymbol{)(}\AgdaBound{a}\AgdaSpace{}%
\AgdaSymbol{:}\AgdaSpace{}%
\AgdaOperator{\AgdaFunction{∥}}\AgdaSpace{}%
\AgdaBound{𝑆}\AgdaSpace{}%
\AgdaOperator{\AgdaFunction{∥}}\AgdaSpace{}%
\AgdaBound{𝑓}%
\>[31]\AgdaSymbol{→}%
\>[34]\AgdaOperator{\AgdaFunction{∣}}\AgdaSpace{}%
\AgdaBound{𝑨}\AgdaSpace{}%
\AgdaOperator{\AgdaFunction{∣}}\AgdaSymbol{)}\AgdaSpace{}%
\AgdaSymbol{→}\AgdaSpace{}%
\AgdaSymbol{(}\AgdaBound{h}\AgdaSpace{}%
\AgdaOperator{\AgdaFunction{∘}}\AgdaSpace{}%
\AgdaBound{g}\AgdaSymbol{)((}\AgdaBound{𝑓}\AgdaSpace{}%
\AgdaOperator{\AgdaFunction{̂}}\AgdaSpace{}%
\AgdaBound{𝑨}\AgdaSymbol{)}\AgdaSpace{}%
\AgdaBound{a}\AgdaSymbol{)}\AgdaSpace{}%
\AgdaOperator{\AgdaDatatype{≡}}\AgdaSpace{}%
\AgdaSymbol{(}\AgdaBound{𝑓}\AgdaSpace{}%
\AgdaOperator{\AgdaFunction{̂}}\AgdaSpace{}%
\AgdaBound{𝑪}\AgdaSymbol{)(}\AgdaBound{h}\AgdaSpace{}%
\AgdaOperator{\AgdaFunction{∘}}\AgdaSpace{}%
\AgdaBound{g}\AgdaSpace{}%
\AgdaOperator{\AgdaFunction{∘}}\AgdaSpace{}%
\AgdaBound{a}\AgdaSymbol{)}\<%
\\
%
\\[\AgdaEmptyExtraSkip]%
%
\>[2]\AgdaFunction{γ}\AgdaSpace{}%
\AgdaBound{𝑓}\AgdaSpace{}%
\AgdaBound{a}\AgdaSpace{}%
\AgdaSymbol{=}%
\>[825I]\AgdaSymbol{(}\AgdaBound{h}\AgdaSpace{}%
\AgdaOperator{\AgdaFunction{∘}}\AgdaSpace{}%
\AgdaBound{g}\AgdaSymbol{)}\AgdaSpace{}%
\AgdaSymbol{((}\AgdaBound{𝑓}\AgdaSpace{}%
\AgdaOperator{\AgdaFunction{̂}}\AgdaSpace{}%
\AgdaBound{𝑨}\AgdaSymbol{)}\AgdaSpace{}%
\AgdaBound{a}\AgdaSymbol{)}\AgdaSpace{}%
\AgdaOperator{\AgdaFunction{≡⟨}}\AgdaSpace{}%
\AgdaFunction{ap}\AgdaSpace{}%
\AgdaBound{h}\AgdaSpace{}%
\AgdaSymbol{(}\AgdaSpace{}%
\AgdaBound{ghom}\AgdaSpace{}%
\AgdaBound{𝑓}\AgdaSpace{}%
\AgdaBound{a}\AgdaSpace{}%
\AgdaSymbol{)}\AgdaSpace{}%
\AgdaOperator{\AgdaFunction{⟩}}\<%
\\
\>[.][@{}l@{}]\<[825I]%
\>[10]\AgdaBound{h}\AgdaSpace{}%
\AgdaSymbol{((}\AgdaBound{𝑓}\AgdaSpace{}%
\AgdaOperator{\AgdaFunction{̂}}\AgdaSpace{}%
\AgdaBound{𝑩}\AgdaSymbol{)}\AgdaSpace{}%
\AgdaSymbol{(}\AgdaBound{g}\AgdaSpace{}%
\AgdaOperator{\AgdaFunction{∘}}\AgdaSpace{}%
\AgdaBound{a}\AgdaSymbol{))}\AgdaSpace{}%
\AgdaOperator{\AgdaFunction{≡⟨}}\AgdaSpace{}%
\AgdaBound{hhom}\AgdaSpace{}%
\AgdaBound{𝑓}\AgdaSpace{}%
\AgdaSymbol{(}\AgdaSpace{}%
\AgdaBound{g}\AgdaSpace{}%
\AgdaOperator{\AgdaFunction{∘}}\AgdaSpace{}%
\AgdaBound{a}\AgdaSpace{}%
\AgdaSymbol{)}\AgdaSpace{}%
\AgdaOperator{\AgdaFunction{⟩}}\<%
\\
%
\>[10]\AgdaSymbol{(}\AgdaBound{𝑓}\AgdaSpace{}%
\AgdaOperator{\AgdaFunction{̂}}\AgdaSpace{}%
\AgdaBound{𝑪}\AgdaSymbol{)}\AgdaSpace{}%
\AgdaSymbol{(}\AgdaBound{h}\AgdaSpace{}%
\AgdaOperator{\AgdaFunction{∘}}\AgdaSpace{}%
\AgdaBound{g}\AgdaSpace{}%
\AgdaOperator{\AgdaFunction{∘}}\AgdaSpace{}%
\AgdaBound{a}\AgdaSymbol{)}\AgdaSpace{}%
\AgdaOperator{\AgdaFunction{∎}}\<%
\\
%
\\[\AgdaEmptyExtraSkip]%
%
\\[\AgdaEmptyExtraSkip]%
%
\>[1]\AgdaFunction{HomComp}\AgdaSpace{}%
\AgdaSymbol{:}\AgdaSpace{}%
\>[331I]\AgdaSymbol{(}\AgdaBound{𝑨}\AgdaSpace{}%
\AgdaSymbol{:}\AgdaSpace{}%
\AgdaFunction{Algebra}\AgdaSpace{}%
\AgdaBound{𝓧}\AgdaSpace{}%
\AgdaBound{𝑆}\AgdaSymbol{)\{}\AgdaBound{𝑩}\AgdaSpace{}%
\AgdaSymbol{:}\AgdaSpace{}%
\AgdaFunction{Algebra}\AgdaSpace{}%
\AgdaBound{𝓨}\AgdaSpace{}%
\AgdaBound{𝑆}\AgdaSymbol{\}(}\AgdaBound{𝑪}\AgdaSpace{}%
\AgdaSymbol{:}\AgdaSpace{}%
\AgdaFunction{Algebra}\AgdaSpace{}%
\AgdaBound{𝓩}\AgdaSpace{}%
\AgdaBound{𝑆}\AgdaSymbol{)}\AgdaSpace%
\\
\>[1][@{}l@{\AgdaIndent{0}}]%
\>[2]\AgdaSymbol{→}%
\>[.][@{}l@{}]\<[331I]%
\>[11]\AgdaFunction{hom}\AgdaSpace{}%
\AgdaBound{𝑨}\AgdaSpace{}%
\AgdaBound{𝑩}%
\>[20]\AgdaSymbol{→}%
\>[23]\AgdaFunction{hom}\AgdaSpace{}%
\AgdaBound{𝑩}\AgdaSpace{}%
\AgdaBound{𝑪}\AgdaSpace{}%
\AgdaSymbol{→}\AgdaSpace{}%
\AgdaFunction{hom}\AgdaSpace{}%
\AgdaBound{𝑨}\AgdaSpace{}%
\AgdaBound{𝑪}\<%
\\
%
\\[\AgdaEmptyExtraSkip]%
%
\>[1]\AgdaFunction{HomComp}\AgdaSpace{}%
\AgdaBound{𝑨}\AgdaSpace{}%
\AgdaSymbol{\{}\AgdaBound{𝑩}\AgdaSymbol{\}}\AgdaSpace{}%
\AgdaBound{𝑪}\AgdaSpace{}%
\AgdaBound{f}\AgdaSpace{}%
\AgdaBound{g}\AgdaSpace{}%
\AgdaSymbol{=}\AgdaSpace{}%
\AgdaFunction{HCompClosed}\AgdaSpace{}%
\AgdaBound{𝑨}\AgdaSpace{}%
\AgdaBound{𝑩}\AgdaSpace{}%
\AgdaBound{𝑪}\AgdaSpace{}%
\AgdaBound{f}\AgdaSpace{}%
\AgdaBound{g}\<%
\\
%
\\[\AgdaEmptyExtraSkip]%
%
\\[\AgdaEmptyExtraSkip]%
%
\>[1]\AgdaFunction{∘-hom}\AgdaSpace{}%
\AgdaSymbol{:}%
\>[897I]\AgdaSymbol{(}\AgdaBound{𝑨}\AgdaSpace{}%
\AgdaSymbol{:}\AgdaSpace{}%
\AgdaFunction{Algebra}\AgdaSpace{}%
\AgdaBound{𝓧}\AgdaSpace{}%
\AgdaBound{𝑆}\AgdaSymbol{)(}\AgdaBound{𝑩}\AgdaSpace{}%
\AgdaSymbol{:}\AgdaSpace{}%
\AgdaFunction{Algebra}\AgdaSpace{}%
\AgdaBound{𝓨}\AgdaSpace{}%
\AgdaBound{𝑆}\AgdaSymbol{)(}\AgdaBound{𝑪}\AgdaSpace{}%
\AgdaSymbol{:}\AgdaSpace{}%
\AgdaFunction{Algebra}\AgdaSpace{}%
\AgdaBound{𝓩}\AgdaSpace{}%
\AgdaBound{𝑆}\AgdaSymbol{)}\<%
\\
\>[.][@{}l@{}]\<[897I]%
\>[9]\AgdaSymbol{\{}\AgdaBound{f}\AgdaSpace{}%
\AgdaSymbol{:}\AgdaSpace{}%
\AgdaOperator{\AgdaFunction{∣}}\AgdaSpace{}%
\AgdaBound{𝑨}\AgdaSpace{}%
\AgdaOperator{\AgdaFunction{∣}}\AgdaSpace{}%
\AgdaSymbol{→}\AgdaSpace{}%
\AgdaOperator{\AgdaFunction{∣}}\AgdaSpace{}%
\AgdaBound{𝑩}\AgdaSpace{}%
\AgdaOperator{\AgdaFunction{∣}}\AgdaSymbol{\}}\AgdaSpace{}%
\AgdaSymbol{\{}\AgdaBound{g}\AgdaSpace{}%
\AgdaSymbol{:}\AgdaSpace{}%
\AgdaOperator{\AgdaFunction{∣}}\AgdaSpace{}%
\AgdaBound{𝑩}\AgdaSpace{}%
\AgdaOperator{\AgdaFunction{∣}}\AgdaSpace{}%
\AgdaSymbol{→}\AgdaSpace{}%
\AgdaOperator{\AgdaFunction{∣}}\AgdaSpace{}%
\AgdaBound{𝑪}\AgdaSpace{}%
\AgdaOperator{\AgdaFunction{∣}}\AgdaSymbol{\}}\<%
\\
\>[1][@{}l@{\AgdaIndent{0}}]%
\>[2]\AgdaSymbol{→}%
\>[9]\AgdaFunction{is-homomorphism}\AgdaSpace{}%
\AgdaBound{𝑨}\AgdaSpace{}%
\AgdaBound{𝑩}\AgdaSpace{}%
\AgdaBound{f}\AgdaSpace{}%
\AgdaSymbol{→}\AgdaSpace{}%
\AgdaFunction{is-homomorphism}\AgdaSpace{}%
\AgdaBound{𝑩}\AgdaSpace{}%
\AgdaBound{𝑪}\AgdaSpace{}%
\AgdaBound{g}\<%
\\
%
\>[9]\AgdaComment{----------------------------------------------------------------}\<%
\\
%
\>[2]\AgdaSymbol{→}%
\>[9]\AgdaFunction{is-homomorphism}\AgdaSpace{}%
\AgdaBound{𝑨}\AgdaSpace{}%
\AgdaBound{𝑪}\AgdaSpace{}%
\AgdaSymbol{(}\AgdaBound{g}\AgdaSpace{}%
\AgdaOperator{\AgdaFunction{∘}}\AgdaSpace{}%
\AgdaBound{f}\AgdaSymbol{)}\<%
\\
%
\\[\AgdaEmptyExtraSkip]%
%
\>[1]\AgdaFunction{∘-hom}\AgdaSpace{}%
\AgdaBound{𝑨}\AgdaSpace{}%
\AgdaBound{𝑩}\AgdaSpace{}%
\AgdaBound{𝑪}\AgdaSpace{}%
\AgdaSymbol{\{}\AgdaBound{f}\AgdaSymbol{\}}\AgdaSpace{}%
\AgdaSymbol{\{}\AgdaBound{g}\AgdaSymbol{\}}\AgdaSpace{}%
\AgdaBound{fhom}\AgdaSpace{}%
\AgdaBound{ghom}\AgdaSpace{}%
\AgdaSymbol{=}\AgdaSpace{}%
\AgdaOperator{\AgdaFunction{∥}}\AgdaSpace{}%
\AgdaFunction{HCompClosed}\AgdaSpace{}%
\AgdaBound{𝑨}\AgdaSpace{}%
\AgdaBound{𝑩}\AgdaSpace{}%
\AgdaBound{𝑪}\AgdaSpace{}%
\AgdaSymbol{(}\AgdaBound{f}\AgdaSpace{}%
\AgdaOperator{\AgdaInductiveConstructor{,}}\AgdaSpace{}%
\AgdaBound{fhom}\AgdaSymbol{)}\AgdaSpace{}%
\AgdaSymbol{(}\AgdaBound{g}\AgdaSpace{}%
\AgdaOperator{\AgdaInductiveConstructor{,}}\AgdaSpace{}%
\AgdaBound{ghom}\AgdaSymbol{)}\AgdaSpace{}%
\AgdaOperator{\AgdaFunction{∥}}\<%
\\
%
\\[\AgdaEmptyExtraSkip]%
%
\\[\AgdaEmptyExtraSkip]%
%
\>[1]\AgdaFunction{∘-Hom}\AgdaSpace{}%
\AgdaSymbol{:}%
\>[961I]\AgdaSymbol{(}\AgdaBound{𝑨}\AgdaSpace{}%
\AgdaSymbol{:}\AgdaSpace{}%
\AgdaFunction{Algebra}\AgdaSpace{}%
\AgdaBound{𝓧}\AgdaSpace{}%
\AgdaBound{𝑆}\AgdaSymbol{)\{}\AgdaBound{𝑩}\AgdaSpace{}%
\AgdaSymbol{:}\AgdaSpace{}%
\AgdaFunction{Algebra}\AgdaSpace{}%
\AgdaBound{𝓨}\AgdaSpace{}%
\AgdaBound{𝑆}\AgdaSymbol{\}(}\AgdaBound{𝑪}\AgdaSpace{}%
\AgdaSymbol{:}\AgdaSpace{}%
\AgdaFunction{Algebra}\AgdaSpace{}%
\AgdaBound{𝓩}\AgdaSpace{}%
\AgdaBound{𝑆}\AgdaSymbol{)}\<%
\\
\>[.][@{}l@{}]\<[961I]%
\>[9]\AgdaSymbol{\{}\AgdaBound{f}\AgdaSpace{}%
\AgdaSymbol{:}\AgdaSpace{}%
\AgdaOperator{\AgdaFunction{∣}}\AgdaSpace{}%
\AgdaBound{𝑨}\AgdaSpace{}%
\AgdaOperator{\AgdaFunction{∣}}\AgdaSpace{}%
\AgdaSymbol{→}\AgdaSpace{}%
\AgdaOperator{\AgdaFunction{∣}}\AgdaSpace{}%
\AgdaBound{𝑩}\AgdaSpace{}%
\AgdaOperator{\AgdaFunction{∣}}\AgdaSymbol{\}}\AgdaSpace{}%
\AgdaSymbol{\{}\AgdaBound{g}\AgdaSpace{}%
\AgdaSymbol{:}\AgdaSpace{}%
\AgdaOperator{\AgdaFunction{∣}}\AgdaSpace{}%
\AgdaBound{𝑩}\AgdaSpace{}%
\AgdaOperator{\AgdaFunction{∣}}\AgdaSpace{}%
\AgdaSymbol{→}\AgdaSpace{}%
\AgdaOperator{\AgdaFunction{∣}}\AgdaSpace{}%
\AgdaBound{𝑪}\AgdaSpace{}%
\AgdaOperator{\AgdaFunction{∣}}\AgdaSymbol{\}}\<%
\\
\>[1][@{}l@{\AgdaIndent{0}}]%
\>[2]\AgdaSymbol{→}%
\>[9]\AgdaFunction{is-homomorphism}\AgdaSpace{}%
\AgdaBound{𝑨}\AgdaSpace{}%
\AgdaBound{𝑩}\AgdaSpace{}%
\AgdaBound{f}%
\>[32]\AgdaSymbol{→}%
\>[35]\AgdaFunction{is-homomorphism}\AgdaSpace{}%
\AgdaBound{𝑩}\AgdaSpace{}%
\AgdaBound{𝑪}\AgdaSpace{}%
\AgdaBound{g}\<%
\\
%
\>[9]\AgdaComment{----------------------------------------------------------------}\<%
\\
%
\>[2]\AgdaSymbol{→}%
\>[9]\AgdaFunction{is-homomorphism}\AgdaSpace{}%
\AgdaBound{𝑨}\AgdaSpace{}%
\AgdaBound{𝑪}\AgdaSpace{}%
\AgdaSymbol{(}\AgdaBound{g}\AgdaSpace{}%
\AgdaOperator{\AgdaFunction{∘}}\AgdaSpace{}%
\AgdaBound{f}\AgdaSymbol{)}\<%
\\
%
\\[\AgdaEmptyExtraSkip]%
%
\>[1]\AgdaFunction{∘-Hom}\AgdaSpace{}%
\AgdaBound{𝑨}\AgdaSpace{}%
\AgdaSymbol{\{}\AgdaBound{𝑩}\AgdaSymbol{\}}\AgdaSpace{}%
\AgdaBound{𝑪}\AgdaSpace{}%
\AgdaSymbol{\{}\AgdaBound{f}\AgdaSymbol{\}}\AgdaSpace{}%
\AgdaSymbol{\{}\AgdaBound{g}\AgdaSymbol{\}}\AgdaSpace{}%
\AgdaSymbol{=}\AgdaSpace{}%
\AgdaFunction{∘-hom}\AgdaSpace{}%
\AgdaBound{𝑨}\AgdaSpace{}%
\AgdaBound{𝑩}\AgdaSpace{}%
\AgdaBound{𝑪}\AgdaSpace{}%
\AgdaSymbol{\{}\AgdaBound{f}\AgdaSymbol{\}}\AgdaSpace{}%
\AgdaSymbol{\{}\AgdaBound{g}\AgdaSymbol{\}}\<%
\\
%
\\[\AgdaEmptyExtraSkip]%
%
\\[\AgdaEmptyExtraSkip]%
%
\>[1]\AgdaFunction{trans-hom}\AgdaSpace{}%
\AgdaSymbol{:}%
\>[1015I]\AgdaSymbol{(}\AgdaBound{𝑨}\AgdaSpace{}%
\AgdaSymbol{:}\AgdaSpace{}%
\AgdaFunction{Algebra}\AgdaSpace{}%
\AgdaBound{𝓧}\AgdaSpace{}%
\AgdaBound{𝑆}\AgdaSymbol{)(}\AgdaBound{𝑩}\AgdaSpace{}%
\AgdaSymbol{:}\AgdaSpace{}%
\AgdaFunction{Algebra}\AgdaSpace{}%
\AgdaBound{𝓨}\AgdaSpace{}%
\AgdaBound{𝑆}\AgdaSymbol{)(}\AgdaBound{𝑪}\AgdaSpace{}%
\AgdaSymbol{:}\AgdaSpace{}%
\AgdaFunction{Algebra}\AgdaSpace{}%
\AgdaBound{𝓩}\AgdaSpace{}%
\AgdaBound{𝑆}\AgdaSymbol{)}\<%
\\
\>[.][@{}l@{}]\<[1015I]%
\>[13]\AgdaSymbol{(}\AgdaBound{f}\AgdaSpace{}%
\AgdaSymbol{:}\AgdaSpace{}%
\AgdaOperator{\AgdaFunction{∣}}\AgdaSpace{}%
\AgdaBound{𝑨}\AgdaSpace{}%
\AgdaOperator{\AgdaFunction{∣}}\AgdaSpace{}%
\AgdaSymbol{→}\AgdaSpace{}%
\AgdaOperator{\AgdaFunction{∣}}\AgdaSpace{}%
\AgdaBound{𝑩}\AgdaSpace{}%
\AgdaOperator{\AgdaFunction{∣}}\AgdaSpace{}%
\AgdaSymbol{)(}\AgdaBound{g}\AgdaSpace{}%
\AgdaSymbol{:}\AgdaSpace{}%
\AgdaOperator{\AgdaFunction{∣}}\AgdaSpace{}%
\AgdaBound{𝑩}\AgdaSpace{}%
\AgdaOperator{\AgdaFunction{∣}}\AgdaSpace{}%
\AgdaSymbol{→}\AgdaSpace{}%
\AgdaOperator{\AgdaFunction{∣}}\AgdaSpace{}%
\AgdaBound{𝑪}\AgdaSpace{}%
\AgdaOperator{\AgdaFunction{∣}}\AgdaSpace{}%
\AgdaSymbol{)}\<%
\\
\>[1][@{}l@{\AgdaIndent{0}}]%
\>[2]\AgdaSymbol{→}%
\>[13]\AgdaFunction{is-homomorphism}\AgdaSpace{}%
\AgdaBound{𝑨}\AgdaSpace{}%
\AgdaBound{𝑩}\AgdaSpace{}%
\AgdaBound{f}%
\>[36]\AgdaSymbol{→}%
\>[39]\AgdaFunction{is-homomorphism}\AgdaSpace{}%
\AgdaBound{𝑩}\AgdaSpace{}%
\AgdaBound{𝑪}\AgdaSpace{}%
\AgdaBound{g}\<%
\\
%
\>[13]\AgdaComment{------------------------------------------------------}\<%
\\
%
\>[2]\AgdaSymbol{→}%
\>[13]\AgdaFunction{is-homomorphism}\AgdaSpace{}%
\AgdaBound{𝑨}\AgdaSpace{}%
\AgdaBound{𝑪}\AgdaSpace{}%
\AgdaSymbol{(}\AgdaBound{g}\AgdaSpace{}%
\AgdaOperator{\AgdaFunction{∘}}\AgdaSpace{}%
\AgdaBound{f}\AgdaSymbol{)}\<%
\\
%
\\[\AgdaEmptyExtraSkip]%
%
\>[1]\AgdaFunction{trans-hom}\AgdaSpace{}%
\AgdaBound{𝑨}\AgdaSpace{}%
\AgdaBound{𝑩}\AgdaSpace{}%
\AgdaBound{𝑪}\AgdaSpace{}%
\AgdaBound{f}\AgdaSpace{}%
\AgdaBound{g}\AgdaSpace{}%
\AgdaSymbol{=}\AgdaSpace{}%
\AgdaFunction{∘-hom}\AgdaSpace{}%
\AgdaBound{𝑨}\AgdaSpace{}%
\AgdaBound{𝑩}\AgdaSpace{}%
\AgdaBound{𝑪}\AgdaSpace{}%
\AgdaSymbol{\{}\AgdaBound{f}\AgdaSymbol{\}\{}\AgdaBound{g}\AgdaSymbol{\}}\<%
\end{code}

\subsubsection{Homomorphism decomposition}\label{homomorphism-decomposition}

If \ab{g} \as : \af{hom} \ab 𝑨 \ab 𝑩, \ab{h} \as : \af{hom} \ab 𝑨 \ab 𝑪, \ab{h} is surjective, and \af{ker} \ab h \af ⊆ \af{ker} \ab g, then there exists \ab{ϕ} \as : \af{hom} \ab 𝑪 \ab 𝑩 such that \ab{g} \as = \ab ϕ \af ∘ \ab h, that is, such that the following diagram commutes;

\begin{verbatim}
𝑨---- h -->>𝑪
 \         .
  \       .
   g     ∃ϕ
    \   .
     \ .
      V
      𝑩
\end{verbatim}

This, or some variation of it, is sometimes referred to as the \defn{Second Isomorphism Theorem}. We formalize its statement and proof as follows. (Notice that the proof is constructive.)
\ccpad
\begin{code}%
\>[0]\AgdaFunction{homFactor}\AgdaSpace{}%
\AgdaSymbol{:}%
\>[1069I]\AgdaSymbol{\{}\AgdaBound{𝓤}\AgdaSpace{}%
\AgdaSymbol{:}\AgdaSpace{}%
\AgdaFunction{Universe}\AgdaSymbol{\}}\AgdaSpace{}%
\AgdaSymbol{→}\AgdaSpace{}%
\AgdaFunction{funext}\AgdaSpace{}%
\AgdaBound{𝓤}\AgdaSpace{}%
\AgdaBound{𝓤}\AgdaSpace{}%
\AgdaSymbol{→}\AgdaSpace{}%
\AgdaSymbol{\{}\AgdaBound{𝑨}\AgdaSpace{}%
\AgdaBound{𝑩}\AgdaSpace{}%
\AgdaBound{𝑪}\AgdaSpace{}%
\AgdaSymbol{:}\AgdaSpace{}%
\AgdaFunction{Algebra}\AgdaSpace{}%
\AgdaBound{𝓤}\AgdaSpace{}%
\AgdaBound{𝑆}\AgdaSymbol{\}}\<%
\\
\>[.][@{}l@{}]\<[1069I]%
\>[12]\AgdaSymbol{(}\AgdaBound{g}\AgdaSpace{}%
\AgdaSymbol{:}\AgdaSpace{}%
\AgdaFunction{hom}\AgdaSpace{}%
\AgdaBound{𝑨}\AgdaSpace{}%
\AgdaBound{𝑩}\AgdaSymbol{)}\AgdaSpace{}%
\AgdaSymbol{(}\AgdaBound{h}\AgdaSpace{}%
\AgdaSymbol{:}\AgdaSpace{}%
\AgdaFunction{hom}\AgdaSpace{}%
\AgdaBound{𝑨}\AgdaSpace{}%
\AgdaBound{𝑪}\AgdaSymbol{)}\<%
\\
\>[0][@{}l@{\AgdaIndent{0}}]%
\>[1]\AgdaSymbol{→}%
\>[12]\AgdaFunction{ker-pred}\AgdaSpace{}%
\AgdaOperator{\AgdaFunction{∣}}\AgdaSpace{}%
\AgdaBound{h}\AgdaSpace{}%
\AgdaOperator{\AgdaFunction{∣}}\AgdaSpace{}%
\AgdaOperator{\AgdaFunction{⊆}}\AgdaSpace{}%
\AgdaFunction{ker-pred}\AgdaSpace{}%
\AgdaOperator{\AgdaFunction{∣}}\AgdaSpace{}%
\AgdaBound{g}\AgdaSpace{}%
\AgdaOperator{\AgdaFunction{∣}}%
\>[45]\AgdaSymbol{→}%
\>[49]\AgdaFunction{Epic}\AgdaSpace{}%
\AgdaOperator{\AgdaFunction{∣}}\AgdaSpace{}%
\AgdaBound{h}\AgdaSpace{}%
\AgdaOperator{\AgdaFunction{∣}}\<%
\\
%
\>[12]\AgdaComment{-------------------------------------------------------}\<%
\\
%
\>[1]\AgdaSymbol{→}%
\>[12]\AgdaFunction{Σ}\AgdaSpace{}%
\AgdaBound{ϕ}\AgdaSpace{}%
\AgdaFunction{꞉}\AgdaSpace{}%
\AgdaSymbol{(}\AgdaFunction{hom}\AgdaSpace{}%
\AgdaBound{𝑪}\AgdaSpace{}%
\AgdaBound{𝑩}\AgdaSymbol{)}\AgdaSpace{}%
\AgdaFunction{,}\AgdaSpace{}%
\AgdaOperator{\AgdaFunction{∣}}\AgdaSpace{}%
\AgdaBound{g}\AgdaSpace{}%
\AgdaOperator{\AgdaFunction{∣}}\AgdaSpace{}%
\AgdaOperator{\AgdaDatatype{≡}}\AgdaSpace{}%
\AgdaOperator{\AgdaFunction{∣}}\AgdaSpace{}%
\AgdaBound{ϕ}\AgdaSpace{}%
\AgdaOperator{\AgdaFunction{∣}}\AgdaSpace{}%
\AgdaOperator{\AgdaFunction{∘}}\AgdaSpace{}%
\AgdaOperator{\AgdaFunction{∣}}\AgdaSpace{}%
\AgdaBound{h}\AgdaSpace{}%
\AgdaOperator{\AgdaFunction{∣}}\<%
\\
%
\\[\AgdaEmptyExtraSkip]%
\>[0]\AgdaFunction{homFactor}\AgdaSpace{}%
\AgdaBound{fe}\AgdaSymbol{\{}\AgdaBound{𝑨}\AgdaSymbol{\}\{}\AgdaBound{𝑩}\AgdaSymbol{\}\{}\AgdaBound{𝑪}\AgdaSymbol{\}(}\AgdaBound{g}\AgdaSpace{}%
\AgdaOperator{\AgdaInductiveConstructor{,}}\AgdaSpace{}%
\AgdaBound{ghom}\AgdaSymbol{)(}\AgdaBound{h}\AgdaSpace{}%
\AgdaOperator{\AgdaInductiveConstructor{,}}\AgdaSpace{}%
\AgdaBound{hhom}\AgdaSymbol{)}\AgdaSpace{}%
\AgdaBound{Kh⊆Kg}\AgdaSpace{}%
\AgdaBound{hEpi}\AgdaSpace{}%
\AgdaSymbol{=}\AgdaSpace{}%
\AgdaSymbol{(}\AgdaFunction{ϕ}\AgdaSpace{}%
\AgdaOperator{\AgdaInductiveConstructor{,}}\AgdaSpace{}%
\AgdaFunction{ϕIsHomCB}\AgdaSymbol{)}\AgdaSpace{}%
\AgdaOperator{\AgdaInductiveConstructor{,}}\AgdaSpace{}%
\AgdaFunction{g≡ϕ∘h}\<%
\\
\>[0][@{}l@{\AgdaIndent{0}}]%
\>[1]\AgdaKeyword{where}\<%
\\
%
\>[1]\AgdaFunction{hInv}\AgdaSpace{}%
\AgdaSymbol{:}\AgdaSpace{}%
\AgdaOperator{\AgdaFunction{∣}}\AgdaSpace{}%
\AgdaBound{𝑪}\AgdaSpace{}%
\AgdaOperator{\AgdaFunction{∣}}\AgdaSpace{}%
\AgdaSymbol{→}\AgdaSpace{}%
\AgdaOperator{\AgdaFunction{∣}}\AgdaSpace{}%
\AgdaBound{𝑨}\AgdaSpace{}%
\AgdaOperator{\AgdaFunction{∣}}\<%
\\
%
\>[1]\AgdaFunction{hInv}\AgdaSpace{}%
\AgdaSymbol{=}\AgdaSpace{}%
\AgdaSymbol{λ}\AgdaSpace{}%
\AgdaBound{c}\AgdaSpace{}%
\AgdaSymbol{→}\AgdaSpace{}%
\AgdaSymbol{(}\AgdaFunction{EpicInv}\AgdaSpace{}%
\AgdaBound{h}\AgdaSpace{}%
\AgdaBound{hEpi}\AgdaSymbol{)}\AgdaSpace{}%
\AgdaBound{c}\<%
\\
%
\\[\AgdaEmptyExtraSkip]%
%
\>[1]\AgdaFunction{ϕ}\AgdaSpace{}%
\AgdaSymbol{:}\AgdaSpace{}%
\AgdaOperator{\AgdaFunction{∣}}\AgdaSpace{}%
\AgdaBound{𝑪}\AgdaSpace{}%
\AgdaOperator{\AgdaFunction{∣}}\AgdaSpace{}%
\AgdaSymbol{→}\AgdaSpace{}%
\AgdaOperator{\AgdaFunction{∣}}\AgdaSpace{}%
\AgdaBound{𝑩}\AgdaSpace{}%
\AgdaOperator{\AgdaFunction{∣}}\<%
\\
%
\>[1]\AgdaFunction{ϕ}\AgdaSpace{}%
\AgdaSymbol{=}\AgdaSpace{}%
\AgdaSymbol{λ}\AgdaSpace{}%
\AgdaBound{c}\AgdaSpace{}%
\AgdaSymbol{→}\AgdaSpace{}%
\AgdaBound{g}\AgdaSpace{}%
\AgdaSymbol{(}\AgdaSpace{}%
\AgdaFunction{hInv}\AgdaSpace{}%
\AgdaBound{c}\AgdaSpace{}%
\AgdaSymbol{)}\<%
\\
%
\\[\AgdaEmptyExtraSkip]%
%
\>[1]\AgdaFunction{ξ}\AgdaSpace{}%
\AgdaSymbol{:}\AgdaSpace{}%
\AgdaSymbol{∀}\AgdaSpace{}%
\AgdaBound{x}\AgdaSpace{}%
\AgdaSymbol{→}\AgdaSpace{}%
\AgdaFunction{ker-pred}\AgdaSpace{}%
\AgdaBound{h}\AgdaSpace{}%
\AgdaSymbol{(}\AgdaBound{x}\AgdaSpace{}%
\AgdaOperator{\AgdaInductiveConstructor{,}}\AgdaSpace{}%
\AgdaFunction{hInv}\AgdaSpace{}%
\AgdaSymbol{(}\AgdaBound{h}\AgdaSpace{}%
\AgdaBound{x}\AgdaSymbol{))}\<%
\\
%
\>[1]\AgdaFunction{ξ}\AgdaSpace{}%
\AgdaBound{x}\AgdaSpace{}%
\AgdaSymbol{=}\AgdaSpace{}%
\AgdaSymbol{(}\AgdaFunction{cong-app}\AgdaSpace{}%
\AgdaSymbol{(}\AgdaFunction{EpicInvIsRightInv}\AgdaSpace{}%
\AgdaBound{fe}\AgdaSpace{}%
\AgdaBound{h}\AgdaSpace{}%
\AgdaBound{hEpi}\AgdaSymbol{)}\AgdaSpace{}%
\AgdaSymbol{(}\AgdaBound{h}\AgdaSpace{}%
\AgdaBound{x}\AgdaSymbol{))}\AgdaOperator{\AgdaFunction{⁻¹}}\<%
\\
%
\\[\AgdaEmptyExtraSkip]%
%
\>[1]\AgdaFunction{g≡ϕ∘h}\AgdaSpace{}%
\AgdaSymbol{:}\AgdaSpace{}%
\AgdaBound{g}\AgdaSpace{}%
\AgdaOperator{\AgdaDatatype{≡}}\AgdaSpace{}%
\AgdaFunction{ϕ}\AgdaSpace{}%
\AgdaOperator{\AgdaFunction{∘}}\AgdaSpace{}%
\AgdaBound{h}\<%
\\
%
\>[1]\AgdaFunction{g≡ϕ∘h}\AgdaSpace{}%
\AgdaSymbol{=}\AgdaSpace{}%
\AgdaBound{fe}%
\>[13]\AgdaSymbol{λ}\AgdaSpace{}%
\AgdaBound{x}\AgdaSpace{}%
\AgdaSymbol{→}\AgdaSpace{}%
\AgdaBound{Kh⊆Kg}\AgdaSpace{}%
\AgdaSymbol{(}\AgdaFunction{ξ}\AgdaSpace{}%
\AgdaBound{x}\AgdaSymbol{)}\<%
\\
%
\\[\AgdaEmptyExtraSkip]%
%
\>[1]\AgdaFunction{ζ}\AgdaSpace{}%
\AgdaSymbol{:}\AgdaSpace{}%
\AgdaSymbol{(}\AgdaBound{𝑓}\AgdaSpace{}%
\AgdaSymbol{:}\AgdaSpace{}%
\AgdaOperator{\AgdaFunction{∣}}\AgdaSpace{}%
\AgdaBound{𝑆}\AgdaSpace{}%
\AgdaOperator{\AgdaFunction{∣}}\AgdaSymbol{)(}\AgdaBound{𝒄}\AgdaSpace{}%
\AgdaSymbol{:}\AgdaSpace{}%
\AgdaOperator{\AgdaFunction{∥}}\AgdaSpace{}%
\AgdaBound{𝑆}\AgdaSpace{}%
\AgdaOperator{\AgdaFunction{∥}}\AgdaSpace{}%
\AgdaBound{𝑓}\AgdaSpace{}%
\AgdaSymbol{→}\AgdaSpace{}%
\AgdaOperator{\AgdaFunction{∣}}\AgdaSpace{}%
\AgdaBound{𝑪}\AgdaSpace{}%
\AgdaOperator{\AgdaFunction{∣}}\AgdaSymbol{)(}\AgdaBound{x}\AgdaSpace{}%
\AgdaSymbol{:}\AgdaSpace{}%
\AgdaOperator{\AgdaFunction{∥}}\AgdaSpace{}%
\AgdaBound{𝑆}\AgdaSpace{}%
\AgdaOperator{\AgdaFunction{∥}}\AgdaSpace{}%
\AgdaBound{𝑓}\AgdaSymbol{)}\AgdaSpace{}%
\AgdaSymbol{→}%
\>[54]\AgdaBound{𝒄}\AgdaSpace{}%
\AgdaBound{x}\AgdaSpace{}%
\AgdaOperator{\AgdaDatatype{≡}}\AgdaSpace{}%
\AgdaSymbol{(}\AgdaBound{h}\AgdaSpace{}%
\AgdaOperator{\AgdaFunction{∘}}\AgdaSpace{}%
\AgdaFunction{hInv}\AgdaSymbol{)(}\AgdaBound{𝒄}\AgdaSpace{}%
\AgdaBound{x}\AgdaSymbol{)}\<%
\\
%
\>[1]\AgdaFunction{ζ}%
\>[4]\AgdaBound{𝑓}\AgdaSpace{}%
\AgdaBound{𝒄}\AgdaSpace{}%
\AgdaBound{x}\AgdaSpace{}%
\AgdaSymbol{=}\AgdaSpace{}%
\AgdaSymbol{(}\AgdaFunction{cong-app}\AgdaSpace{}%
\AgdaSymbol{(}\AgdaFunction{EpicInvIsRightInv}\AgdaSpace{}%
\AgdaBound{fe}\AgdaSpace{}%
\AgdaBound{h}\AgdaSpace{}%
\AgdaBound{hEpi}\AgdaSymbol{)}\AgdaSpace{}%
\AgdaSymbol{(}\AgdaBound{𝒄}\AgdaSpace{}%
\AgdaBound{x}\AgdaSymbol{))}\AgdaOperator{\AgdaFunction{⁻¹}}\<%
\\
%
\\[\AgdaEmptyExtraSkip]%
%
\>[1]\AgdaFunction{ι}\AgdaSpace{}%
\AgdaSymbol{:}\AgdaSpace{}%
\AgdaSymbol{(}\AgdaBound{𝑓}\AgdaSpace{}%
\AgdaSymbol{:}\AgdaSpace{}%
\AgdaOperator{\AgdaFunction{∣}}\AgdaSpace{}%
\AgdaBound{𝑆}\AgdaSpace{}%
\AgdaOperator{\AgdaFunction{∣}}\AgdaSymbol{)(}\AgdaBound{𝒄}\AgdaSpace{}%
\AgdaSymbol{:}\AgdaSpace{}%
\AgdaOperator{\AgdaFunction{∥}}\AgdaSpace{}%
\AgdaBound{𝑆}\AgdaSpace{}%
\AgdaOperator{\AgdaFunction{∥}}\AgdaSpace{}%
\AgdaBound{𝑓}\AgdaSpace{}%
\AgdaSymbol{→}\AgdaSpace{}%
\AgdaOperator{\AgdaFunction{∣}}\AgdaSpace{}%
\AgdaBound{𝑪}\AgdaSpace{}%
\AgdaOperator{\AgdaFunction{∣}}\AgdaSymbol{)}\AgdaSpace{}%
\AgdaSymbol{→}%
\>[41]\AgdaBound{𝒄}\AgdaSpace{}%
\AgdaOperator{\AgdaDatatype{≡}}\AgdaSpace{}%
\AgdaBound{h}\AgdaSpace{}%
\AgdaOperator{\AgdaFunction{∘}}\AgdaSpace{}%
\AgdaSymbol{(}\AgdaFunction{hInv}\AgdaSpace{}%
\AgdaOperator{\AgdaFunction{∘}}\AgdaSpace{}%
\AgdaBound{𝒄}\AgdaSymbol{)}\<%
\\
%
\>[1]\AgdaFunction{ι}\AgdaSpace{}%
\AgdaBound{𝑓}\AgdaSpace{}%
\AgdaBound{𝒄}\AgdaSpace{}%
\AgdaSymbol{=}\AgdaSpace{}%
\AgdaFunction{ap}\AgdaSpace{}%
\AgdaSymbol{(λ}\AgdaSpace{}%
\AgdaBound{-}\AgdaSpace{}%
\AgdaSymbol{→}\AgdaSpace{}%
\AgdaBound{-}\AgdaSpace{}%
\AgdaOperator{\AgdaFunction{∘}}\AgdaSpace{}%
\AgdaBound{𝒄}\AgdaSymbol{)(}\AgdaFunction{EpicInvIsRightInv}\AgdaSpace{}%
\AgdaBound{fe}\AgdaSpace{}%
\AgdaBound{h}\AgdaSpace{}%
\AgdaBound{hEpi}\AgdaSymbol{)}\AgdaOperator{\AgdaFunction{⁻¹}}\<%
\\
%
\\[\AgdaEmptyExtraSkip]%
%
\>[1]\AgdaFunction{useker}\AgdaSpace{}%
\AgdaSymbol{:}\AgdaSpace{}%
\AgdaSymbol{∀}\AgdaSpace{}%
\AgdaBound{𝑓}\AgdaSpace{}%
\AgdaBound{𝒄}\AgdaSpace{}%
\AgdaSymbol{→}\AgdaSpace{}%
\AgdaBound{g}\AgdaSymbol{(}\AgdaFunction{hInv}\AgdaSpace{}%
\AgdaSymbol{(}\AgdaBound{h}\AgdaSymbol{((}\AgdaBound{𝑓}\AgdaSpace{}%
\AgdaOperator{\AgdaFunction{̂}}\AgdaSpace{}%
\AgdaBound{𝑨}\AgdaSymbol{)(}\AgdaFunction{hInv}\AgdaSpace{}%
\AgdaOperator{\AgdaFunction{∘}}\AgdaSpace{}%
\AgdaBound{𝒄}\AgdaSymbol{))))}\AgdaSpace{}%
\AgdaOperator{\AgdaDatatype{≡}}\AgdaSpace{}%
\AgdaBound{g}\AgdaSymbol{((}\AgdaBound{𝑓}\AgdaSpace{}%
\AgdaOperator{\AgdaFunction{̂}}\AgdaSpace{}%
\AgdaBound{𝑨}\AgdaSymbol{)(}\AgdaFunction{hInv}\AgdaSpace{}%
\AgdaOperator{\AgdaFunction{∘}}\AgdaSpace{}%
\AgdaBound{𝒄}\AgdaSymbol{))}\<%
\\
%
\>[1]\AgdaFunction{useker}\AgdaSpace{}%
\AgdaBound{𝑓}\AgdaSpace{}%
\AgdaBound{c}\AgdaSpace{}%
\AgdaSymbol{=}\AgdaSpace{}%
\AgdaBound{Kh⊆Kg}\AgdaSpace{}%
\AgdaSymbol{(}\AgdaFunction{cong-app}\AgdaSymbol{(}\AgdaFunction{EpicInvIsRightInv}\AgdaSpace{}%
\AgdaBound{fe}\AgdaSpace{}%
\AgdaBound{h}\AgdaSpace{}%
\AgdaBound{hEpi}\AgdaSymbol{)(}\AgdaBound{h}\AgdaSpace{}%
\AgdaSymbol{((}\AgdaBound{𝑓}\AgdaSpace{}%
\AgdaOperator{\AgdaFunction{̂}}\AgdaSpace{}%
\AgdaBound{𝑨}\AgdaSymbol{)(}\AgdaFunction{hInv}\AgdaSpace{}%
\AgdaOperator{\AgdaFunction{∘}}\AgdaSpace{}%
\AgdaBound{c}\AgdaSymbol{))))}\<%
\\
%
\\[\AgdaEmptyExtraSkip]%
%
\>[1]\AgdaFunction{ϕIsHomCB}\AgdaSpace{}%
\AgdaSymbol{:}\AgdaSpace{}%
\AgdaSymbol{(}\AgdaBound{𝑓}\AgdaSpace{}%
\AgdaSymbol{:}\AgdaSpace{}%
\AgdaOperator{\AgdaFunction{∣}}\AgdaSpace{}%
\AgdaBound{𝑆}\AgdaSpace{}%
\AgdaOperator{\AgdaFunction{∣}}\AgdaSymbol{)(}\AgdaBound{𝒄}\AgdaSpace{}%
\AgdaSymbol{:}\AgdaSpace{}%
\AgdaOperator{\AgdaFunction{∥}}\AgdaSpace{}%
\AgdaBound{𝑆}\AgdaSpace{}%
\AgdaOperator{\AgdaFunction{∥}}\AgdaSpace{}%
\AgdaBound{𝑓}\AgdaSpace{}%
\AgdaSymbol{→}\AgdaSpace{}%
\AgdaOperator{\AgdaFunction{∣}}\AgdaSpace{}%
\AgdaBound{𝑪}\AgdaSpace{}%
\AgdaOperator{\AgdaFunction{∣}}\AgdaSymbol{)}\AgdaSpace{}%
\AgdaSymbol{→}\AgdaSpace{}%
\AgdaFunction{ϕ}\AgdaSymbol{((}\AgdaBound{𝑓}\AgdaSpace{}%
\AgdaOperator{\AgdaFunction{̂}}\AgdaSpace{}%
\AgdaBound{𝑪}\AgdaSymbol{)}\AgdaSpace{}%
\AgdaBound{𝒄}\AgdaSymbol{)}\AgdaSpace{}%
\>[666I]\AgdaOperator{\AgdaDatatype{≡}}\AgdaSpace{}%
\AgdaSymbol{(}\AgdaBound{𝑓}\AgdaSpace{}%
\AgdaOperator{\AgdaFunction{̂}}\AgdaSpace{}%
\AgdaBound{𝑩}\AgdaSymbol{)(}\AgdaFunction{ϕ}\AgdaSpace{}%
\AgdaOperator{\AgdaFunction{∘}}\AgdaSpace{}%
\AgdaBound{𝒄}\AgdaSymbol{)}\<%
\\
%
\\[\AgdaEmptyExtraSkip]%
%
\>[1]\AgdaFunction{ϕIsHomCB}\AgdaSpace{}%
\AgdaBound{𝑓}\AgdaSpace{}%
\AgdaBound{𝒄}%
\>[1330I]\AgdaSymbol{=}%
\>[17]\AgdaBound{g}\AgdaSpace{}%
\AgdaSymbol{(}\AgdaFunction{hInv}\AgdaSpace{}%
\AgdaSymbol{((}\AgdaBound{𝑓}\AgdaSpace{}%
\AgdaOperator{\AgdaFunction{̂}}\AgdaSpace{}%
\AgdaBound{𝑪}\AgdaSymbol{)}\AgdaSpace{}%
\AgdaBound{𝒄}\AgdaSymbol{))}%
\>[666I][@{}l@{\AgdaIndent{0}}]%
\>[52]\AgdaOperator{\AgdaFunction{≡⟨}}\AgdaSpace{}%
\AgdaFunction{i}%
\>[60]\AgdaOperator{\AgdaFunction{⟩}}\<%
\\
\>[1330I][@{}l@{\AgdaIndent{0}}]%
\>[16]\AgdaBound{g}\AgdaSpace{}%
\AgdaSymbol{(}\AgdaFunction{hInv}\AgdaSpace{}%
\AgdaSymbol{((}\AgdaBound{𝑓}\AgdaSpace{}%
\AgdaOperator{\AgdaFunction{̂}}\AgdaSpace{}%
\AgdaBound{𝑪}\AgdaSymbol{)(}\AgdaBound{h}\AgdaSpace{}%
\AgdaOperator{\AgdaFunction{∘}}\AgdaSpace{}%
\AgdaSymbol{(}\AgdaFunction{hInv}\AgdaSpace{}%
\AgdaOperator{\AgdaFunction{∘}}\AgdaSpace{}%
\AgdaBound{𝒄}\AgdaSymbol{))))}\AgdaSpace{}%
\>[52]\AgdaOperator{\AgdaFunction{≡⟨}}\AgdaSpace{}%
\AgdaFunction{ii}%
\>[60]\AgdaOperator{\AgdaFunction{⟩}}\<%
\\
%
\>[16]\AgdaBound{g}\AgdaSpace{}%
\AgdaSymbol{(}\AgdaFunction{hInv}\AgdaSpace{}%
\AgdaSymbol{(}\AgdaBound{h}\AgdaSpace{}%
\AgdaSymbol{((}\AgdaBound{𝑓}\AgdaSpace{}%
\AgdaOperator{\AgdaFunction{̂}}\AgdaSpace{}%
\AgdaBound{𝑨}\AgdaSymbol{)(}\AgdaFunction{hInv}\AgdaSpace{}%
\AgdaOperator{\AgdaFunction{∘}}\AgdaSpace{}%
\AgdaBound{𝒄}\AgdaSymbol{))))}%
\>[52]\AgdaOperator{\AgdaFunction{≡⟨}}\AgdaSpace{}%
\AgdaFunction{iii}\AgdaSpace{}%
\>[60]\AgdaOperator{\AgdaFunction{⟩}}\<%
\\
%
\>[16]\AgdaBound{g}\AgdaSpace{}%
\AgdaSymbol{((}\AgdaBound{𝑓}\AgdaSpace{}%
\AgdaOperator{\AgdaFunction{̂}}\AgdaSpace{}%
\AgdaBound{𝑨}\AgdaSymbol{)(}\AgdaFunction{hInv}\AgdaSpace{}%
\AgdaOperator{\AgdaFunction{∘}}\AgdaSpace{}%
\AgdaBound{𝒄}\AgdaSymbol{))}%
\>[52]\AgdaOperator{\AgdaFunction{≡⟨}}\AgdaSpace{}%
\AgdaFunction{iv}%
\>[60]\AgdaOperator{\AgdaFunction{⟩}}\<%
\\
%
\>[16]\AgdaSymbol{(}\AgdaBound{𝑓}\AgdaSpace{}%
\AgdaOperator{\AgdaFunction{̂}}\AgdaSpace{}%
\AgdaBound{𝑩}\AgdaSymbol{)(λ}\AgdaSpace{}%
\AgdaBound{x}\AgdaSpace{}%
\AgdaSymbol{→}\AgdaSpace{}%
\AgdaBound{g}\AgdaSpace{}%
\AgdaSymbol{(}\AgdaFunction{hInv}\AgdaSpace{}%
\AgdaSymbol{(}\AgdaBound{𝒄}\AgdaSpace{}%
\AgdaBound{x}\AgdaSymbol{)))}%
\>[52]\AgdaOperator{\AgdaFunction{∎}}\<%
\\
\>[1][@{}l@{\AgdaIndent{0}}]%
\>[2]\AgdaKeyword{where}\<%
\\
%
\>[2]\AgdaFunction{i}%
\>[6]\AgdaSymbol{=}\AgdaSpace{}%
\AgdaFunction{ap}\AgdaSpace{}%
\AgdaSymbol{(}\AgdaBound{g}\AgdaSpace{}%
\AgdaOperator{\AgdaFunction{∘}}\AgdaSpace{}%
\AgdaFunction{hInv}\AgdaSymbol{)}\AgdaSpace{}%
\AgdaSymbol{(}\AgdaFunction{ap}\AgdaSpace{}%
\AgdaSymbol{(}\AgdaBound{𝑓}\AgdaSpace{}%
\AgdaOperator{\AgdaFunction{̂}}\AgdaSpace{}%
\AgdaBound{𝑪}\AgdaSymbol{)}\AgdaSpace{}%
\AgdaSymbol{(}\AgdaFunction{ι}\AgdaSpace{}%
\AgdaBound{𝑓}\AgdaSpace{}%
\AgdaBound{𝒄}\AgdaSymbol{))}\<%
\\
%
\>[2]\AgdaFunction{ii}%
\>[6]\AgdaSymbol{=}\AgdaSpace{}%
\AgdaFunction{ap}\AgdaSpace{}%
\AgdaSymbol{(}\AgdaBound{g}\AgdaSpace{}%
\AgdaOperator{\AgdaFunction{∘}}\AgdaSpace{}%
\AgdaFunction{hInv}\AgdaSymbol{)}\AgdaSpace{}%
\AgdaSymbol{(}\AgdaBound{hhom}\AgdaSpace{}%
\AgdaBound{𝑓}\AgdaSpace{}%
\AgdaSymbol{(}\AgdaFunction{hInv}\AgdaSpace{}%
\AgdaOperator{\AgdaFunction{∘}}\AgdaSpace{}%
\AgdaBound{𝒄}\AgdaSymbol{))}\AgdaOperator{\AgdaFunction{⁻¹}}\<%
\\
%
\>[2]\AgdaFunction{iii}\AgdaSpace{}%
\AgdaSymbol{=}\AgdaSpace{}%
\AgdaFunction{useker}\AgdaSpace{}%
\AgdaBound{𝑓}\AgdaSpace{}%
\AgdaBound{𝒄}\<%
\\
%
\>[2]\AgdaFunction{iv}%
\>[6]\AgdaSymbol{=}\AgdaSpace{}%
\AgdaBound{ghom}\AgdaSpace{}%
\AgdaBound{𝑓}\AgdaSpace{}%
\AgdaSymbol{(}\AgdaFunction{hInv}\AgdaSpace{}%
\AgdaOperator{\AgdaFunction{∘}}\AgdaSpace{}%
\AgdaBound{𝒄}\AgdaSymbol{)}\<%
\end{code}
\ccpad
Here's a more general version.
\ccpad
\begin{code}%
\>[1]\AgdaFunction{HomFactor}\AgdaSpace{}%
\AgdaSymbol{:}%
\>[1407I]\AgdaSymbol{(}\AgdaBound{𝑨}\AgdaSpace{}%
\AgdaSymbol{:}\AgdaSpace{}%
\AgdaFunction{Algebra}\AgdaSpace{}%
\AgdaBound{𝓧}\AgdaSpace{}%
\AgdaBound{𝑆}\AgdaSymbol{)\{}\AgdaBound{𝑩}\AgdaSpace{}%
\AgdaSymbol{:}\AgdaSpace{}%
\AgdaFunction{Algebra}\AgdaSpace{}%
\AgdaBound{𝓨}\AgdaSpace{}%
\AgdaBound{𝑆}\AgdaSymbol{\}\{}\AgdaBound{𝑪}\AgdaSpace{}%
\AgdaSymbol{:}\AgdaSpace{}%
\AgdaFunction{Algebra}\AgdaSpace{}%
\AgdaBound{𝓩}\AgdaSpace{}%
\AgdaBound{𝑆}\AgdaSymbol{\}}\<%
\\
\>[.][@{}l@{}]\<[1407I]%
\>[13]\AgdaSymbol{(}\AgdaBound{β}\AgdaSpace{}%
\AgdaSymbol{:}\AgdaSpace{}%
\AgdaFunction{hom}\AgdaSpace{}%
\AgdaBound{𝑨}\AgdaSpace{}%
\AgdaBound{𝑩}\AgdaSymbol{)}\AgdaSpace{}%
\AgdaSymbol{(}\AgdaBound{γ}\AgdaSpace{}%
\AgdaSymbol{:}\AgdaSpace{}%
\AgdaFunction{hom}\AgdaSpace{}%
\AgdaBound{𝑨}\AgdaSpace{}%
\AgdaBound{𝑪}\AgdaSymbol{)}\<%
\\
\>[1][@{}l@{\AgdaIndent{0}}]%
\>[2]\AgdaSymbol{→}%
\>[13]\AgdaFunction{Epic}\AgdaSpace{}%
\AgdaOperator{\AgdaFunction{∣}}\AgdaSpace{}%
\AgdaBound{γ}\AgdaSpace{}%
\AgdaOperator{\AgdaFunction{∣}}\AgdaSpace{}%
\AgdaSymbol{→}\AgdaSpace{}%
\AgdaSymbol{(}\AgdaFunction{KER-pred}\AgdaSpace{}%
\AgdaOperator{\AgdaFunction{∣}}\AgdaSpace{}%
\AgdaBound{γ}\AgdaSpace{}%
\AgdaOperator{\AgdaFunction{∣}}\AgdaSymbol{)}\AgdaSpace{}%
\AgdaOperator{\AgdaFunction{⊆}}\AgdaSpace{}%
\AgdaSymbol{(}\AgdaFunction{KER-pred}\AgdaSpace{}%
\AgdaOperator{\AgdaFunction{∣}}\AgdaSpace{}%
\AgdaBound{β}\AgdaSpace{}%
\AgdaOperator{\AgdaFunction{∣}}\AgdaSymbol{)}\<%
\\
%
\>[13]\AgdaComment{-----------------------------------------------------------}\<%
\\
%
\>[2]\AgdaSymbol{→}%
\>[13]\AgdaFunction{Σ}\AgdaSpace{}%
\AgdaBound{ϕ}\AgdaSpace{}%
\AgdaFunction{꞉}\AgdaSpace{}%
\AgdaSymbol{(}\AgdaFunction{hom}\AgdaSpace{}%
\AgdaBound{𝑪}\AgdaSpace{}%
\AgdaBound{𝑩}\AgdaSymbol{)}\AgdaSpace{}%
\AgdaFunction{,}\AgdaSpace{}%
\AgdaOperator{\AgdaFunction{∣}}\AgdaSpace{}%
\AgdaBound{β}\AgdaSpace{}%
\AgdaOperator{\AgdaFunction{∣}}\AgdaSpace{}%
\AgdaOperator{\AgdaDatatype{≡}}\AgdaSpace{}%
\AgdaOperator{\AgdaFunction{∣}}\AgdaSpace{}%
\AgdaBound{ϕ}\AgdaSpace{}%
\AgdaOperator{\AgdaFunction{∣}}\AgdaSpace{}%
\AgdaOperator{\AgdaFunction{∘}}\AgdaSpace{}%
\AgdaOperator{\AgdaFunction{∣}}\AgdaSpace{}%
\AgdaBound{γ}\AgdaSpace{}%
\AgdaOperator{\AgdaFunction{∣}}\<%
\\
%
\\[\AgdaEmptyExtraSkip]%
%
\>[1]\AgdaFunction{HomFactor}\AgdaSpace{}%
\AgdaBound{𝑨}\AgdaSpace{}%
\AgdaSymbol{\{}\AgdaBound{𝑩}\AgdaSymbol{\}\{}\AgdaBound{𝑪}\AgdaSymbol{\}}\AgdaSpace{}%
\AgdaBound{β}\AgdaSpace{}%
\AgdaBound{γ}\AgdaSpace{}%
\AgdaBound{γE}\AgdaSpace{}%
\AgdaBound{Kγβ}\AgdaSpace{}%
\AgdaSymbol{=}\AgdaSpace{}%
\AgdaSymbol{(}\AgdaFunction{ϕ}\AgdaSpace{}%
\AgdaOperator{\AgdaInductiveConstructor{,}}\AgdaSpace{}%
\AgdaFunction{ϕIsHomCB}\AgdaSymbol{)}\AgdaSpace{}%
\AgdaOperator{\AgdaInductiveConstructor{,}}\AgdaSpace{}%
\AgdaFunction{βϕγ}\<%
\\
\>[1][@{}l@{\AgdaIndent{0}}]%
\>[2]\AgdaKeyword{where}\<%
\\
%
\>[2]\AgdaFunction{γInv}\AgdaSpace{}%
\AgdaSymbol{:}\AgdaSpace{}%
\AgdaOperator{\AgdaFunction{∣}}\AgdaSpace{}%
\AgdaBound{𝑪}\AgdaSpace{}%
\AgdaOperator{\AgdaFunction{∣}}\AgdaSpace{}%
\AgdaSymbol{→}\AgdaSpace{}%
\AgdaOperator{\AgdaFunction{∣}}\AgdaSpace{}%
\AgdaBound{𝑨}\AgdaSpace{}%
\AgdaOperator{\AgdaFunction{∣}}\<%
\\
%
\>[2]\AgdaFunction{γInv}\AgdaSpace{}%
\AgdaSymbol{=}\AgdaSpace{}%
\AgdaSymbol{λ}\AgdaSpace{}%
\AgdaBound{y}\AgdaSpace{}%
\AgdaSymbol{→}\AgdaSpace{}%
\AgdaSymbol{(}\AgdaFunction{EpicInv}\AgdaSpace{}%
\AgdaOperator{\AgdaFunction{∣}}\AgdaSpace{}%
\AgdaBound{γ}\AgdaSpace{}%
\AgdaOperator{\AgdaFunction{∣}}\AgdaSpace{}%
\AgdaBound{γE}\AgdaSymbol{)}\AgdaSpace{}%
\AgdaBound{y}\<%
\\
%
\\[\AgdaEmptyExtraSkip]%
%
\>[2]\AgdaFunction{ϕ}\AgdaSpace{}%
\AgdaSymbol{:}\AgdaSpace{}%
\AgdaOperator{\AgdaFunction{∣}}\AgdaSpace{}%
\AgdaBound{𝑪}\AgdaSpace{}%
\AgdaOperator{\AgdaFunction{∣}}\AgdaSpace{}%
\AgdaSymbol{→}\AgdaSpace{}%
\AgdaOperator{\AgdaFunction{∣}}\AgdaSpace{}%
\AgdaBound{𝑩}\AgdaSpace{}%
\AgdaOperator{\AgdaFunction{∣}}\<%
\\
%
\>[2]\AgdaFunction{ϕ}\AgdaSpace{}%
\AgdaSymbol{=}\AgdaSpace{}%
\AgdaSymbol{λ}\AgdaSpace{}%
\AgdaBound{y}\AgdaSpace{}%
\AgdaSymbol{→}\AgdaSpace{}%
\AgdaOperator{\AgdaFunction{∣}}\AgdaSpace{}%
\AgdaBound{β}\AgdaSpace{}%
\AgdaOperator{\AgdaFunction{∣}}\AgdaSpace{}%
\AgdaSymbol{(}\AgdaSpace{}%
\AgdaFunction{γInv}\AgdaSpace{}%
\AgdaBound{y}\AgdaSpace{}%
\AgdaSymbol{)}\<%
\\
%
\\[\AgdaEmptyExtraSkip]%
%
\>[2]\AgdaFunction{ξ}\AgdaSpace{}%
\AgdaSymbol{:}\AgdaSpace{}%
\AgdaSymbol{(}\AgdaBound{x}\AgdaSpace{}%
\AgdaSymbol{:}\AgdaSpace{}%
\AgdaOperator{\AgdaFunction{∣}}\AgdaSpace{}%
\AgdaBound{𝑨}\AgdaSpace{}%
\AgdaOperator{\AgdaFunction{∣}}\AgdaSymbol{)}\AgdaSpace{}%
\AgdaSymbol{→}\AgdaSpace{}%
\AgdaFunction{KER-pred}\AgdaSpace{}%
\AgdaOperator{\AgdaFunction{∣}}\AgdaSpace{}%
\AgdaBound{γ}\AgdaSpace{}%
\AgdaOperator{\AgdaFunction{∣}}\AgdaSpace{}%
\AgdaSymbol{(}\AgdaBound{x}\AgdaSpace{}%
\AgdaOperator{\AgdaInductiveConstructor{,}}\AgdaSpace{}%
\AgdaFunction{γInv}\AgdaSpace{}%
\AgdaSymbol{(}\AgdaOperator{\AgdaFunction{∣}}\AgdaSpace{}%
\AgdaBound{γ}\AgdaSpace{}%
\AgdaOperator{\AgdaFunction{∣}}\AgdaSpace{}%
\AgdaBound{x}\AgdaSymbol{))}\<%
\\
%
\>[2]\AgdaFunction{ξ}\AgdaSpace{}%
\AgdaBound{x}\AgdaSpace{}%
\AgdaSymbol{=}%
\>[9]\AgdaSymbol{(}\AgdaSpace{}%
\AgdaFunction{cong-app}\AgdaSpace{}%
\AgdaSymbol{(}\AgdaFunction{EpicInvIsRightInv}\AgdaSpace{}%
\AgdaBound{gfe}\AgdaSpace{}%
\AgdaOperator{\AgdaFunction{∣}}\AgdaSpace{}%
\AgdaBound{γ}\AgdaSpace{}%
\AgdaOperator{\AgdaFunction{∣}}\AgdaSpace{}%
\AgdaBound{γE}\AgdaSymbol{)}\AgdaSpace{}%
\AgdaSymbol{(}\AgdaSpace{}%
\AgdaOperator{\AgdaFunction{∣}}\AgdaSpace{}%
\AgdaBound{γ}\AgdaSpace{}%
\AgdaOperator{\AgdaFunction{∣}}\AgdaSpace{}%
\AgdaBound{x}\AgdaSpace{}%
\AgdaSymbol{)}\AgdaSpace{}%
\AgdaSymbol{)}\AgdaOperator{\AgdaFunction{⁻¹}}\<%
\\
%
\\[\AgdaEmptyExtraSkip]%
%
\>[2]\AgdaFunction{βϕγ}\AgdaSpace{}%
\AgdaSymbol{:}\AgdaSpace{}%
\AgdaOperator{\AgdaFunction{∣}}\AgdaSpace{}%
\AgdaBound{β}\AgdaSpace{}%
\AgdaOperator{\AgdaFunction{∣}}\AgdaSpace{}%
\AgdaOperator{\AgdaDatatype{≡}}\AgdaSpace{}%
\AgdaFunction{ϕ}\AgdaSpace{}%
\AgdaOperator{\AgdaFunction{∘}}\AgdaSpace{}%
\AgdaOperator{\AgdaFunction{∣}}\AgdaSpace{}%
\AgdaBound{γ}\AgdaSpace{}%
\AgdaOperator{\AgdaFunction{∣}}\<%
\\
%
\>[2]\AgdaFunction{βϕγ}\AgdaSpace{}%
\AgdaSymbol{=}\AgdaSpace{}%
\AgdaBound{gfe}\AgdaSpace{}%
\AgdaSymbol{λ}\AgdaSpace{}%
\AgdaBound{x}\AgdaSpace{}%
\AgdaSymbol{→}\AgdaSpace{}%
\AgdaBound{Kγβ}\AgdaSpace{}%
\AgdaSymbol{(}\AgdaFunction{ξ}\AgdaSpace{}%
\AgdaBound{x}\AgdaSymbol{)}\<%
\\
%
\\[\AgdaEmptyExtraSkip]%
%
\>[2]\AgdaFunction{ι}\AgdaSpace{}%
\AgdaSymbol{:}\AgdaSpace{}%
\AgdaSymbol{(}\AgdaBound{𝑓}\AgdaSpace{}%
\AgdaSymbol{:}\AgdaSpace{}%
\AgdaOperator{\AgdaFunction{∣}}\AgdaSpace{}%
\AgdaBound{𝑆}\AgdaSpace{}%
\AgdaOperator{\AgdaFunction{∣}}\AgdaSymbol{)(}\AgdaBound{𝒄}\AgdaSpace{}%
\AgdaSymbol{:}\AgdaSpace{}%
\AgdaOperator{\AgdaFunction{∥}}\AgdaSpace{}%
\AgdaBound{𝑆}\AgdaSpace{}%
\AgdaOperator{\AgdaFunction{∥}}\AgdaSpace{}%
\AgdaBound{𝑓}\AgdaSpace{}%
\AgdaSymbol{→}\AgdaSpace{}%
\AgdaOperator{\AgdaFunction{∣}}\AgdaSpace{}%
\AgdaBound{𝑪}\AgdaSpace{}%
\AgdaOperator{\AgdaFunction{∣}}\AgdaSymbol{)}\AgdaSpace{}%
\AgdaSymbol{→}\AgdaSpace{}%
\AgdaBound{𝒄}\AgdaSpace{}%
\AgdaOperator{\AgdaDatatype{≡}}%
\>[46]\AgdaOperator{\AgdaFunction{∣}}\AgdaSpace{}%
\AgdaBound{γ}\AgdaSpace{}%
\AgdaOperator{\AgdaFunction{∣}}\AgdaSpace{}%
\AgdaOperator{\AgdaFunction{∘}}\AgdaSpace{}%
\AgdaSymbol{(}\AgdaFunction{γInv}\AgdaSpace{}%
\AgdaOperator{\AgdaFunction{∘}}\AgdaSpace{}%
\AgdaBound{𝒄}\AgdaSymbol{)}\<%
\\
%
\>[2]\AgdaFunction{ι}\AgdaSpace{}%
\AgdaBound{𝑓}\AgdaSpace{}%
\AgdaBound{𝒄}\AgdaSpace{}%
\AgdaSymbol{=}\AgdaSpace{}%
\AgdaFunction{ap}\AgdaSpace{}%
\AgdaSymbol{(λ}\AgdaSpace{}%
\AgdaBound{-}\AgdaSpace{}%
\AgdaSymbol{→}\AgdaSpace{}%
\AgdaBound{-}\AgdaSpace{}%
\AgdaOperator{\AgdaFunction{∘}}\AgdaSpace{}%
\AgdaBound{𝒄}\AgdaSymbol{)(}\AgdaFunction{EpicInvIsRightInv}\AgdaSpace{}%
\AgdaBound{gfe}\AgdaSpace{}%
\AgdaOperator{\AgdaFunction{∣}}\AgdaSpace{}%
\AgdaBound{γ}\AgdaSpace{}%
\AgdaOperator{\AgdaFunction{∣}}\AgdaSpace{}%
\AgdaBound{γE}\AgdaSymbol{)}\AgdaOperator{\AgdaFunction{⁻¹}}\<%
\\
%
\\[\AgdaEmptyExtraSkip]%
%
\>[2]\AgdaFunction{useker}\AgdaSpace{}%
\AgdaSymbol{:}\AgdaSpace{}%
\AgdaSymbol{∀}\AgdaSpace{}%
\AgdaBound{𝑓}\AgdaSpace{}%
\AgdaBound{𝒄}\AgdaSpace{}%
\AgdaSymbol{→}\AgdaSpace{}%
\AgdaOperator{\AgdaFunction{∣}}\AgdaSpace{}%
\AgdaBound{β}\AgdaSpace{}%
\AgdaOperator{\AgdaFunction{∣}}\AgdaSpace{}%
\AgdaSymbol{(}\AgdaFunction{γInv}\AgdaSpace{}%
\AgdaSymbol{(}\AgdaOperator{\AgdaFunction{∣}}\AgdaSpace{}%
\AgdaBound{γ}\AgdaSpace{}%
\AgdaOperator{\AgdaFunction{∣}}\AgdaSpace{}%
\AgdaSymbol{((}\AgdaBound{𝑓}\AgdaSpace{}%
\AgdaOperator{\AgdaFunction{̂}}\AgdaSpace{}%
\AgdaBound{𝑨}\AgdaSymbol{)}\AgdaSpace{}%
\AgdaSymbol{(}\AgdaFunction{γInv}\AgdaSpace{}%
\AgdaOperator{\AgdaFunction{∘}}\AgdaSpace{}%
\AgdaBound{𝒄}\AgdaSymbol{))))}\AgdaSpace{}%
\AgdaOperator{\AgdaDatatype{≡}}\AgdaSpace{}%
\AgdaOperator{\AgdaFunction{∣}}\AgdaSpace{}%
\>[667I]\AgdaBound{β}\AgdaSpace{}%
\AgdaOperator{\AgdaFunction{∣}}\AgdaSymbol{((}\AgdaBound{𝑓}\AgdaSpace{}%
\AgdaOperator{\AgdaFunction{̂}}\AgdaSpace{}%
\AgdaBound{𝑨}\AgdaSymbol{)}\AgdaSpace{}%
\AgdaSymbol{(}\AgdaFunction{γInv}\AgdaSpace{}%
\AgdaOperator{\AgdaFunction{∘}}\AgdaSpace{}%
\AgdaBound{𝒄}\AgdaSymbol{))}\<%
\\
%
\>[2]\AgdaFunction{useker}\AgdaSpace{}%
\AgdaBound{𝑓}\AgdaSpace{}%
\AgdaBound{𝒄}\AgdaSpace{}%
\AgdaSymbol{=}\AgdaSpace{}%
\AgdaBound{Kγβ}\AgdaSpace{}%
\AgdaSymbol{(}\AgdaFunction{cong-app}\AgdaSpace{}%
\AgdaSymbol{(}\AgdaFunction{EpicInvIsRightInv}\AgdaSpace{}%
\AgdaBound{gfe}\AgdaSpace{}%
\AgdaOperator{\AgdaFunction{∣}}\AgdaSpace{}%
\AgdaBound{γ}\AgdaSpace{}%
\AgdaOperator{\AgdaFunction{∣}}\AgdaSpace{}%
\AgdaBound{γE}\AgdaSymbol{)(}\AgdaOperator{\AgdaFunction{∣}}\AgdaSpace{}%
\AgdaBound{γ}\AgdaSpace{}%
\AgdaOperator{\AgdaFunction{∣}}\AgdaSpace{}%
\AgdaSymbol{((}\AgdaBound{𝑓}\AgdaSpace{}%
\AgdaOperator{\AgdaFunction{̂}}\AgdaSpace{}%
\AgdaBound{𝑨}\AgdaSymbol{)(}\AgdaFunction{γInv}\AgdaSpace{}%
\AgdaOperator{\AgdaFunction{∘}}\AgdaSpace{}%
\AgdaBound{𝒄}\AgdaSymbol{))))}\<%
\\
%
\\[\AgdaEmptyExtraSkip]%
%
\>[2]\AgdaFunction{ϕIsHomCB}\AgdaSpace{}%
\AgdaSymbol{:}\AgdaSpace{}%
\AgdaSymbol{∀}\AgdaSpace{}%
\AgdaBound{𝑓}\AgdaSpace{}%
\AgdaBound{𝒄}\AgdaSpace{}%
\AgdaSymbol{→}\AgdaSpace{}%
\AgdaFunction{ϕ}\AgdaSpace{}%
\AgdaSymbol{((}\AgdaBound{𝑓}\AgdaSpace{}%
\AgdaOperator{\AgdaFunction{̂}}\AgdaSpace{}%
\AgdaBound{𝑪}\AgdaSymbol{)}\AgdaSpace{}%
\AgdaBound{𝒄}\AgdaSymbol{)}\AgdaSpace{}%
\AgdaOperator{\AgdaDatatype{≡}}\AgdaSpace{}%
\AgdaSymbol{((}\AgdaBound{𝑓}\AgdaSpace{}%
\AgdaOperator{\AgdaFunction{̂}}\AgdaSpace{}%
\AgdaBound{𝑩}\AgdaSymbol{)(}\AgdaFunction{ϕ}\AgdaSpace{}%
\AgdaOperator{\AgdaFunction{∘}}\AgdaSpace{}%
\AgdaBound{𝒄}\AgdaSymbol{))}\<%
\\
%
\\[\AgdaEmptyExtraSkip]%
%
\>[2]\AgdaFunction{ϕIsHomCB}\AgdaSpace{}%
\AgdaBound{𝑓}\AgdaSpace{}%
\AgdaBound{𝒄}%
\>[1662I]\AgdaSymbol{=}\AgdaSpace{}%
\AgdaOperator{\AgdaFunction{∣}}\AgdaSpace{}%
\AgdaBound{β}\AgdaSpace{}%
\AgdaOperator{\AgdaFunction{∣}}\AgdaSpace{}%
\AgdaSymbol{(}\AgdaFunction{γInv}\AgdaSpace{}%
\AgdaSymbol{((}\AgdaBound{𝑓}\AgdaSpace{}%
\AgdaOperator{\AgdaFunction{̂}}\AgdaSpace{}%
\AgdaBound{𝑪}\AgdaSymbol{)}\AgdaSpace{}%
\AgdaBound{𝒄}\AgdaSymbol{))}%
\>[667I][@{}l@{\AgdaIndent{0}}]%
\>[60]\AgdaOperator{\AgdaFunction{≡⟨}}\AgdaSpace{}%
\AgdaFunction{i}%
\>[67]\AgdaOperator{\AgdaFunction{⟩}}\<%
\\
\>[1662I][@{}l@{\AgdaIndent{0}}]%
\>[16]\AgdaOperator{\AgdaFunction{∣}}\AgdaSpace{}%
\AgdaBound{β}\AgdaSpace{}%
\AgdaOperator{\AgdaFunction{∣}}\AgdaSpace{}%
\AgdaSymbol{(}\AgdaFunction{γInv}\AgdaSpace{}%
\AgdaSymbol{((}\AgdaBound{𝑓}\AgdaSpace{}%
\AgdaOperator{\AgdaFunction{̂}}\AgdaSpace{}%
\AgdaBound{𝑪}\AgdaSymbol{)(}\AgdaOperator{\AgdaFunction{∣}}\AgdaSpace{}%
\AgdaBound{γ}\AgdaSpace{}%
\AgdaOperator{\AgdaFunction{∣}}\AgdaSpace{}%
\AgdaOperator{\AgdaFunction{∘}}\AgdaSpace{}%
\AgdaSymbol{(}\AgdaFunction{γInv}\AgdaSpace{}%
\AgdaOperator{\AgdaFunction{∘}}\AgdaSpace{}%
\AgdaBound{𝒄}\AgdaSymbol{))))}\AgdaSpace{}%
\>[60]\AgdaOperator{\AgdaFunction{≡⟨}}\AgdaSpace{}%
\AgdaFunction{ii}%
\>[67]\AgdaOperator{\AgdaFunction{⟩}}\<%
\\
%
\>[16]\AgdaOperator{\AgdaFunction{∣}}\AgdaSpace{}%
\AgdaBound{β}\AgdaSpace{}%
\AgdaOperator{\AgdaFunction{∣}}\AgdaSpace{}%
\AgdaSymbol{(}\AgdaFunction{γInv}\AgdaSpace{}%
\AgdaSymbol{(}\AgdaOperator{\AgdaFunction{∣}}\AgdaSpace{}%
\AgdaBound{γ}\AgdaSpace{}%
\AgdaOperator{\AgdaFunction{∣}}\AgdaSpace{}%
\AgdaSymbol{((}\AgdaBound{𝑓}\AgdaSpace{}%
\AgdaOperator{\AgdaFunction{̂}}\AgdaSpace{}%
\AgdaBound{𝑨}\AgdaSymbol{)(}\AgdaFunction{γInv}\AgdaSpace{}%
\AgdaOperator{\AgdaFunction{∘}}\AgdaSpace{}%
\AgdaBound{𝒄}\AgdaSymbol{))))}%
\>[60]\AgdaOperator{\AgdaFunction{≡⟨}}\AgdaSpace{}%
\AgdaFunction{iii}\AgdaSpace{}%
\>[67]\AgdaOperator{\AgdaFunction{⟩}}\<%
\\
%
\>[16]\AgdaOperator{\AgdaFunction{∣}}\AgdaSpace{}%
\AgdaBound{β}\AgdaSpace{}%
\AgdaOperator{\AgdaFunction{∣}}\AgdaSpace{}%
\AgdaSymbol{((}\AgdaBound{𝑓}\AgdaSpace{}%
\AgdaOperator{\AgdaFunction{̂}}\AgdaSpace{}%
\AgdaBound{𝑨}\AgdaSymbol{)(}\AgdaFunction{γInv}\AgdaSpace{}%
\AgdaOperator{\AgdaFunction{∘}}\AgdaSpace{}%
\AgdaBound{𝒄}\AgdaSymbol{))}%
\>[60]\AgdaOperator{\AgdaFunction{≡⟨}}\AgdaSpace{}%
\AgdaFunction{iv}%
\>[67]\AgdaOperator{\AgdaFunction{⟩}}\<%
\\
%
\>[16]\AgdaSymbol{((}\AgdaBound{𝑓}\AgdaSpace{}%
\AgdaOperator{\AgdaFunction{̂}}\AgdaSpace{}%
\AgdaBound{𝑩}\AgdaSymbol{)(λ}\AgdaSpace{}%
\AgdaBound{x}\AgdaSpace{}%
\AgdaSymbol{→}\AgdaSpace{}%
\AgdaOperator{\AgdaFunction{∣}}\AgdaSpace{}%
\AgdaBound{β}\AgdaSpace{}%
\AgdaOperator{\AgdaFunction{∣}}\AgdaSpace{}%
\AgdaSymbol{(}\AgdaFunction{γInv}\AgdaSpace{}%
\AgdaSymbol{(}\AgdaBound{𝒄}\AgdaSpace{}%
\AgdaBound{x}\AgdaSymbol{))))}%
\>[60]\AgdaOperator{\AgdaFunction{∎}}\<%
\\
\>[2][@{}l@{\AgdaIndent{0}}]%
\>[3]\AgdaKeyword{where}\<%
\\
%
\>[3]\AgdaFunction{i}%
\>[7]\AgdaSymbol{=}\AgdaSpace{}%
\AgdaFunction{ap}\AgdaSpace{}%
\AgdaSymbol{(}\AgdaOperator{\AgdaFunction{∣}}\AgdaSpace{}%
\AgdaBound{β}\AgdaSpace{}%
\AgdaOperator{\AgdaFunction{∣}}\AgdaSpace{}%
\AgdaOperator{\AgdaFunction{∘}}\AgdaSpace{}%
\AgdaFunction{γInv}\AgdaSymbol{)}\AgdaSpace{}%
\AgdaSymbol{(}\AgdaFunction{ap}\AgdaSpace{}%
\AgdaSymbol{(}\AgdaBound{𝑓}\AgdaSpace{}%
\AgdaOperator{\AgdaFunction{̂}}\AgdaSpace{}%
\AgdaBound{𝑪}\AgdaSymbol{)}\AgdaSpace{}%
\AgdaSymbol{(}\AgdaFunction{ι}\AgdaSpace{}%
\AgdaBound{𝑓}\AgdaSpace{}%
\AgdaBound{𝒄}\AgdaSymbol{))}\<%
\\
%
\>[3]\AgdaFunction{ii}%
\>[7]\AgdaSymbol{=}\AgdaSpace{}%
\AgdaFunction{ap}\AgdaSpace{}%
\AgdaSymbol{(}\AgdaOperator{\AgdaFunction{∣}}\AgdaSpace{}%
\AgdaBound{β}\AgdaSpace{}%
\AgdaOperator{\AgdaFunction{∣}}\AgdaSpace{}%
\AgdaOperator{\AgdaFunction{∘}}\AgdaSpace{}%
\AgdaFunction{γInv}\AgdaSymbol{)}\AgdaSpace{}%
\AgdaSymbol{(}\AgdaOperator{\AgdaFunction{∥}}\AgdaSpace{}%
\AgdaBound{γ}\AgdaSpace{}%
\AgdaOperator{\AgdaFunction{∥}}\AgdaSpace{}%
\AgdaBound{𝑓}\AgdaSpace{}%
\AgdaSymbol{(}\AgdaFunction{γInv}\AgdaSpace{}%
\AgdaOperator{\AgdaFunction{∘}}\AgdaSpace{}%
\AgdaBound{𝒄}\AgdaSymbol{))}\AgdaOperator{\AgdaFunction{⁻¹}}\<%
\\
%
\>[3]\AgdaFunction{iii}\AgdaSpace{}%
\AgdaSymbol{=}\AgdaSpace{}%
\AgdaFunction{useker}\AgdaSpace{}%
\AgdaBound{𝑓}\AgdaSpace{}%
\AgdaBound{𝒄}\<%
\\
%
\>[3]\AgdaFunction{iv}%
\>[7]\AgdaSymbol{=}\AgdaSpace{}%
\AgdaOperator{\AgdaFunction{∥}}\AgdaSpace{}%
\AgdaBound{β}\AgdaSpace{}%
\AgdaOperator{\AgdaFunction{∥}}\AgdaSpace{}%
\AgdaBound{𝑓}\AgdaSpace{}%
\AgdaSymbol{(}\AgdaFunction{γInv}\AgdaSpace{}%
\AgdaOperator{\AgdaFunction{∘}}\AgdaSpace{}%
\AgdaBound{𝒄}\AgdaSymbol{)}\<%
\end{code}
\ccpad
If, in addition, both \ab β and \ab γ are epic, then so is \ab ϕ.
\ccpad
\begin{code}%
\>[0][@{}l@{\AgdaIndent{1}}]%
\>[1]\AgdaFunction{HomFactorEpi}\AgdaSpace{}%
\AgdaSymbol{:}%
\>[1755I]\AgdaSymbol{(}\AgdaBound{𝑨}\AgdaSpace{}%
\AgdaSymbol{:}\AgdaSpace{}%
\AgdaFunction{Algebra}\AgdaSpace{}%
\AgdaBound{𝓧}\AgdaSpace{}%
\AgdaBound{𝑆}\AgdaSymbol{)\{}\AgdaBound{𝑩}\AgdaSpace{}%
\AgdaSymbol{:}\AgdaSpace{}%
\AgdaFunction{Algebra}\AgdaSpace{}%
\AgdaBound{𝓨}\AgdaSpace{}%
\AgdaBound{𝑆}\AgdaSymbol{\}\{}\AgdaBound{𝑪}\AgdaSpace{}%
\AgdaSymbol{:}\AgdaSpace{}%
\AgdaFunction{Algebra}\AgdaSpace{}%
\AgdaBound{𝓩}\AgdaSpace{}%
\AgdaBound{𝑆}\AgdaSymbol{\}}\<%
\\
\>[.][@{}l@{}]\<[1755I]%
\>[16]\AgdaSymbol{(}\AgdaBound{β}\AgdaSpace{}%
\AgdaSymbol{:}\AgdaSpace{}%
\AgdaFunction{hom}\AgdaSpace{}%
\AgdaBound{𝑨}\AgdaSpace{}%
\AgdaBound{𝑩}\AgdaSymbol{)}\AgdaSpace{}%
\AgdaSymbol{(}\AgdaBound{βe}\AgdaSpace{}%
\AgdaSymbol{:}\AgdaSpace{}%
\AgdaFunction{Epic}\AgdaSpace{}%
\AgdaOperator{\AgdaFunction{∣}}\AgdaSpace{}%
\AgdaBound{β}\AgdaSpace{}%
\AgdaOperator{\AgdaFunction{∣}}\AgdaSymbol{)}\<%
\\
%
\>[16]\AgdaSymbol{(}\AgdaBound{ξ}\AgdaSpace{}%
\AgdaSymbol{:}\AgdaSpace{}%
\AgdaFunction{hom}\AgdaSpace{}%
\AgdaBound{𝑨}\AgdaSpace{}%
\AgdaBound{𝑪}\AgdaSymbol{)}\AgdaSpace{}%
\AgdaSymbol{(}\AgdaBound{ξe}\AgdaSpace{}%
\AgdaSymbol{:}\AgdaSpace{}%
\AgdaFunction{Epic}\AgdaSpace{}%
\AgdaOperator{\AgdaFunction{∣}}\AgdaSpace{}%
\AgdaBound{ξ}\AgdaSpace{}%
\AgdaOperator{\AgdaFunction{∣}}\AgdaSymbol{)}\<%
\\
\>[1][@{}l@{\AgdaIndent{0}}]%
\>[2]\AgdaSymbol{→}%
\>[16]\AgdaSymbol{(}\AgdaFunction{KER-pred}\AgdaSpace{}%
\AgdaOperator{\AgdaFunction{∣}}\AgdaSpace{}%
\AgdaBound{ξ}\AgdaSpace{}%
\AgdaOperator{\AgdaFunction{∣}}\AgdaSymbol{)}\AgdaSpace{}%
\AgdaOperator{\AgdaFunction{⊆}}\AgdaSpace{}%
\AgdaSymbol{(}\AgdaFunction{KER-pred}\AgdaSpace{}%
\AgdaOperator{\AgdaFunction{∣}}\AgdaSpace{}%
\AgdaBound{β}\AgdaSpace{}%
\AgdaOperator{\AgdaFunction{∣}}\AgdaSymbol{)}\<%
\\
%
\>[16]\AgdaComment{----------------------------------}\<%
\\
%
\>[2]\AgdaSymbol{→}%
\>[16]\AgdaFunction{Σ}\AgdaSpace{}%
\AgdaBound{ϕ}\AgdaSpace{}%
\AgdaFunction{꞉}\AgdaSpace{}%
\AgdaSymbol{(}\AgdaFunction{epi}\AgdaSpace{}%
\AgdaBound{𝑪}\AgdaSpace{}%
\AgdaBound{𝑩}\AgdaSymbol{)}\AgdaSpace{}%
\AgdaFunction{,}\AgdaSpace{}%
\AgdaOperator{\AgdaFunction{∣}}\AgdaSpace{}%
\AgdaBound{β}\AgdaSpace{}%
\AgdaOperator{\AgdaFunction{∣}}\AgdaSpace{}%
\AgdaOperator{\AgdaDatatype{≡}}\AgdaSpace{}%
\AgdaOperator{\AgdaFunction{∣}}\AgdaSpace{}%
\AgdaBound{ϕ}\AgdaSpace{}%
\AgdaOperator{\AgdaFunction{∣}}\AgdaSpace{}%
\AgdaOperator{\AgdaFunction{∘}}\AgdaSpace{}%
\AgdaOperator{\AgdaFunction{∣}}\AgdaSpace{}%
\AgdaBound{ξ}\AgdaSpace{}%
\AgdaOperator{\AgdaFunction{∣}}\<%
\\
%
\\[\AgdaEmptyExtraSkip]%
%
\>[1]\AgdaFunction{HomFactorEpi}\AgdaSpace{}%
\AgdaBound{𝑨}\AgdaSpace{}%
\AgdaSymbol{\{}\AgdaBound{𝑩}\AgdaSymbol{\}\{}\AgdaBound{𝑪}\AgdaSymbol{\}}\AgdaSpace{}%
\AgdaBound{β}\AgdaSpace{}%
\AgdaBound{βe}\AgdaSpace{}%
\AgdaBound{ξ}\AgdaSpace{}%
\AgdaBound{ξe}\AgdaSpace{}%
\AgdaBound{kerincl}\AgdaSpace{}%
\AgdaSymbol{=}\AgdaSpace{}%
\AgdaSymbol{(}\AgdaFunction{fst}\AgdaSpace{}%
\AgdaOperator{\AgdaFunction{∣}}\AgdaSpace{}%
\AgdaFunction{ϕF}\AgdaSpace{}%
\AgdaOperator{\AgdaFunction{∣}}\AgdaSpace{}%
\AgdaOperator{\AgdaInductiveConstructor{,}}\AgdaSpace{}%
\AgdaSymbol{(}\AgdaFunction{snd}\AgdaSpace{}%
\AgdaOperator{\AgdaFunction{∣}}\AgdaSpace{}%
\AgdaFunction{ϕF}\AgdaSpace{}%
\AgdaOperator{\AgdaFunction{∣}}\AgdaSpace{}%
\AgdaOperator{\AgdaInductiveConstructor{,}}\AgdaSpace{}%
\AgdaFunction{ϕE}\AgdaSymbol{))}\AgdaSpace{}%
\AgdaOperator{\AgdaInductiveConstructor{,}}\AgdaSpace{}%
\AgdaOperator{\AgdaFunction{∥}}\AgdaSpace{}%
\AgdaFunction{ϕF}\AgdaSpace{}%
\AgdaOperator{\AgdaFunction{∥}}\<%
\\
\>[1][@{}l@{\AgdaIndent{0}}]%
\>[2]\AgdaKeyword{where}\<%
\\
%
\>[2]\AgdaFunction{ϕF}\AgdaSpace{}%
\AgdaSymbol{:}\AgdaSpace{}%
\AgdaFunction{Σ}\AgdaSpace{}%
\AgdaBound{ϕ}\AgdaSpace{}%
\AgdaFunction{꞉}\AgdaSpace{}%
\AgdaSymbol{(}\AgdaFunction{hom}\AgdaSpace{}%
\AgdaBound{𝑪}\AgdaSpace{}%
\AgdaBound{𝑩}\AgdaSymbol{)}\AgdaSpace{}%
\AgdaFunction{,}\AgdaSpace{}%
\AgdaOperator{\AgdaFunction{∣}}\AgdaSpace{}%
\AgdaBound{β}\AgdaSpace{}%
\AgdaOperator{\AgdaFunction{∣}}\AgdaSpace{}%
\AgdaOperator{\AgdaDatatype{≡}}\AgdaSpace{}%
\AgdaOperator{\AgdaFunction{∣}}\AgdaSpace{}%
\AgdaBound{ϕ}\AgdaSpace{}%
\AgdaOperator{\AgdaFunction{∣}}\AgdaSpace{}%
\AgdaOperator{\AgdaFunction{∘}}\AgdaSpace{}%
\AgdaOperator{\AgdaFunction{∣}}\AgdaSpace{}%
\AgdaBound{ξ}\AgdaSpace{}%
\AgdaOperator{\AgdaFunction{∣}}\<%
\\
%
\>[2]\AgdaFunction{ϕF}\AgdaSpace{}%
\AgdaSymbol{=}\AgdaSpace{}%
\AgdaFunction{HomFactor}%
\>[18]\AgdaBound{𝑨}\AgdaSpace{}%
\AgdaSymbol{\{}\AgdaBound{𝑩}\AgdaSymbol{\}\{}\AgdaBound{𝑪}\AgdaSymbol{\}}\AgdaSpace{}%
\AgdaBound{β}\AgdaSpace{}%
\AgdaBound{ξ}\AgdaSpace{}%
\AgdaBound{ξe}\AgdaSpace{}%
\AgdaBound{kerincl}\<%
\\
%
\\[\AgdaEmptyExtraSkip]%
%
\>[2]\AgdaFunction{ξinv}\AgdaSpace{}%
\AgdaSymbol{:}\AgdaSpace{}%
\AgdaOperator{\AgdaFunction{∣}}\AgdaSpace{}%
\AgdaBound{𝑪}\AgdaSpace{}%
\AgdaOperator{\AgdaFunction{∣}}\AgdaSpace{}%
\AgdaSymbol{→}\AgdaSpace{}%
\AgdaOperator{\AgdaFunction{∣}}\AgdaSpace{}%
\AgdaBound{𝑨}\AgdaSpace{}%
\AgdaOperator{\AgdaFunction{∣}}\<%
\\
%
\>[2]\AgdaFunction{ξinv}\AgdaSpace{}%
\AgdaSymbol{=}\AgdaSpace{}%
\AgdaSymbol{λ}\AgdaSpace{}%
\AgdaBound{c}\AgdaSpace{}%
\AgdaSymbol{→}\AgdaSpace{}%
\AgdaSymbol{(}\AgdaFunction{EpicInv}\AgdaSpace{}%
\AgdaOperator{\AgdaFunction{∣}}\AgdaSpace{}%
\AgdaBound{ξ}\AgdaSpace{}%
\AgdaOperator{\AgdaFunction{∣}}\AgdaSpace{}%
\AgdaBound{ξe}\AgdaSymbol{)}\AgdaSpace{}%
\AgdaBound{c}\<%
\\
%
\\[\AgdaEmptyExtraSkip]%
%
\>[2]\AgdaFunction{βinv}\AgdaSpace{}%
\AgdaSymbol{:}\AgdaSpace{}%
\AgdaOperator{\AgdaFunction{∣}}\AgdaSpace{}%
\AgdaBound{𝑩}\AgdaSpace{}%
\AgdaOperator{\AgdaFunction{∣}}\AgdaSpace{}%
\AgdaSymbol{→}\AgdaSpace{}%
\AgdaOperator{\AgdaFunction{∣}}\AgdaSpace{}%
\AgdaBound{𝑨}\AgdaSpace{}%
\AgdaOperator{\AgdaFunction{∣}}\<%
\\
%
\>[2]\AgdaFunction{βinv}\AgdaSpace{}%
\AgdaSymbol{=}\AgdaSpace{}%
\AgdaSymbol{λ}\AgdaSpace{}%
\AgdaBound{b}\AgdaSpace{}%
\AgdaSymbol{→}\AgdaSpace{}%
\AgdaSymbol{(}\AgdaFunction{EpicInv}\AgdaSpace{}%
\AgdaOperator{\AgdaFunction{∣}}\AgdaSpace{}%
\AgdaBound{β}\AgdaSpace{}%
\AgdaOperator{\AgdaFunction{∣}}\AgdaSpace{}%
\AgdaBound{βe}\AgdaSymbol{)}\AgdaSpace{}%
\AgdaBound{b}\<%
\\
%
\\[\AgdaEmptyExtraSkip]%
%
\>[2]\AgdaFunction{ϕ}\AgdaSpace{}%
\AgdaSymbol{:}\AgdaSpace{}%
\AgdaOperator{\AgdaFunction{∣}}\AgdaSpace{}%
\AgdaBound{𝑪}\AgdaSpace{}%
\AgdaOperator{\AgdaFunction{∣}}\AgdaSpace{}%
\AgdaSymbol{→}\AgdaSpace{}%
\AgdaOperator{\AgdaFunction{∣}}\AgdaSpace{}%
\AgdaBound{𝑩}\AgdaSpace{}%
\AgdaOperator{\AgdaFunction{∣}}\<%
\\
%
\>[2]\AgdaFunction{ϕ}\AgdaSpace{}%
\AgdaSymbol{=}\AgdaSpace{}%
\AgdaSymbol{λ}\AgdaSpace{}%
\AgdaBound{c}\AgdaSpace{}%
\AgdaSymbol{→}\AgdaSpace{}%
\AgdaOperator{\AgdaFunction{∣}}\AgdaSpace{}%
\AgdaBound{β}\AgdaSpace{}%
\AgdaOperator{\AgdaFunction{∣}}\AgdaSpace{}%
\AgdaSymbol{(}\AgdaSpace{}%
\AgdaFunction{ξinv}\AgdaSpace{}%
\AgdaBound{c}\AgdaSpace{}%
\AgdaSymbol{)}\<%
\\
%
\\[\AgdaEmptyExtraSkip]%
%
\>[2]\AgdaFunction{ϕE}\AgdaSpace{}%
\AgdaSymbol{:}\AgdaSpace{}%
\AgdaFunction{Epic}\AgdaSpace{}%
\AgdaFunction{ϕ}\<%
\\
%
\>[2]\AgdaFunction{ϕE}\AgdaSpace{}%
\AgdaSymbol{=}\AgdaSpace{}%
\AgdaFunction{epic-factor}\AgdaSpace{}%
\AgdaBound{gfe}\AgdaSpace{}%
\AgdaOperator{\AgdaFunction{∣}}\AgdaSpace{}%
\AgdaBound{β}\AgdaSpace{}%
\AgdaOperator{\AgdaFunction{∣}}\AgdaSpace{}%
\AgdaOperator{\AgdaFunction{∣}}\AgdaSpace{}%
\AgdaBound{ξ}\AgdaSpace{}%
\AgdaOperator{\AgdaFunction{∣}}\AgdaSpace{}%
\AgdaFunction{ϕ}\AgdaSpace{}%
\AgdaOperator{\AgdaFunction{∥}}\AgdaSpace{}%
\AgdaFunction{ϕF}\AgdaSpace{}%
\AgdaOperator{\AgdaFunction{∥}}\AgdaSpace{}%
\AgdaBound{βe}\<%
\end{code}


\subsection{Isomorphisms}\label{sec:isomorphisms}
\firstsentence{Homomorphisms}{Isomorphisms}
Here we formalize the notion of \emph{isomorphism} between algebraic structures.
% \ccpad
% \begin{code}%
% \>[0]\AgdaSymbol{\{-\#}\AgdaSpace{}%
% \AgdaKeyword{OPTIONS}\AgdaSpace{}%
% \AgdaPragma{--without-K}\AgdaSpace{}%
% \AgdaPragma{--exact-split}\AgdaSpace{}%
% \AgdaPragma{--safe}\AgdaSpace{}%
% \AgdaSymbol{\#-\}}\<%
% \\
% %
% \\[\AgdaEmptyExtraSkip]%
% \>[0]\AgdaKeyword{open}\AgdaSpace{}%
% \AgdaKeyword{import}\AgdaSpace{}%
% \AgdaModule{Algebras.Signatures}\AgdaSpace{}%
% \AgdaKeyword{using}\AgdaSpace{}%
% \AgdaSymbol{(}\AgdaFunction{Signature}\AgdaSymbol{;}\AgdaSpace{}%
% \AgdaGeneralizable{𝓞}\AgdaSymbol{;}\AgdaSpace{}%
% \AgdaGeneralizable{𝓥}\AgdaSymbol{)}\<%
% \\
% \>[0]\AgdaKeyword{open}\AgdaSpace{}%
% \AgdaKeyword{import}\AgdaSpace{}%
% \AgdaModule{MGS-Subsingleton-Theorems}\AgdaSpace{}%
% \AgdaKeyword{using}\AgdaSpace{}%
% \AgdaSymbol{(}\AgdaFunction{global-dfunext}\AgdaSymbol{)}\<%
% \\
% %
% \\[\AgdaEmptyExtraSkip]%
% \>[0]\AgdaKeyword{module}\AgdaSpace{}%
% \AgdaModule{Homomorphisms.Isomorphisms}\AgdaSpace{}%
% \AgdaSymbol{\{}\AgdaBound{𝑆}\AgdaSpace{}%
% \AgdaSymbol{:}\AgdaSpace{}%
% \AgdaFunction{Signature}\AgdaSpace{}%
% \AgdaGeneralizable{𝓞}\AgdaSpace{}%
% \AgdaGeneralizable{𝓥}\AgdaSymbol{\}\{}\AgdaBound{gfe}\AgdaSpace{}%
% \AgdaSymbol{:}\AgdaSpace{}%
% \AgdaFunction{global-dfunext}\AgdaSymbol{\}}\AgdaSpace{}%
% \AgdaKeyword{where}\<%
% \\
% %
% \\[\AgdaEmptyExtraSkip]%
% \>[0]\AgdaKeyword{open}\AgdaSpace{}%
% \AgdaKeyword{import}\AgdaSpace{}%
% \AgdaModule{Homomorphisms.Noether}\AgdaSymbol{\{}\AgdaArgument{𝑆}\AgdaSpace{}%
% \AgdaSymbol{=}\AgdaSpace{}%
% \AgdaBound{𝑆}\AgdaSymbol{\}\{}\AgdaBound{gfe}\AgdaSymbol{\}}\AgdaSpace{}%
% \AgdaKeyword{public}\<%
% \\
% \>[0]\AgdaKeyword{open}\AgdaSpace{}%
% \AgdaKeyword{import}\AgdaSpace{}%
% \AgdaModule{MGS-Embeddings}\AgdaSpace{}%
% \AgdaKeyword{using}\AgdaSpace{}%
% \AgdaSymbol{(}\AgdaFunction{Nat}\AgdaSymbol{;}\AgdaSpace{}%
% \AgdaFunction{NatΠ}\AgdaSymbol{;}\AgdaSpace{}%
% \AgdaFunction{NatΠ-is-embedding}\AgdaSymbol{)}\AgdaSpace{}%
% \AgdaKeyword{public}\<%
% \end{code}

\subsubsection{Definition of isomorphism}\label{definition-of-isomorphism}

Recall, \ab{f} \af ~ \ab g means \ab f and \ab g are   %\textasciitilde{}
\emph{extensionally} (or \emph{point-wise}) \emph{equal}; i.e., \as ∀ \ab x, \ab f \ab x \aod ≡ \ab g \ab x. We use this notion of equality of functions in the following definition of \defn{isomorphism}.
\ccpad
\begin{code}%
\>[0]\AgdaOperator{\AgdaFunction{\AgdaUnderscore{}≅\AgdaUnderscore{}}}\AgdaSpace{}%
\AgdaSymbol{:}\AgdaSpace{}%
\AgdaSymbol{\{}\AgdaBound{𝓤}\AgdaSpace{}%
\AgdaBound{𝓦}\AgdaSpace{}%
\AgdaSymbol{:}\AgdaSpace{}%
\AgdaFunction{Universe}\AgdaSymbol{\}(}\AgdaBound{𝑨}\AgdaSpace{}%
\AgdaSymbol{:}\AgdaSpace{}%
\AgdaFunction{Algebra}\AgdaSpace{}%
\AgdaBound{𝓤}\AgdaSpace{}%
\AgdaBound{𝑆}\AgdaSymbol{)(}\AgdaBound{𝑩}\AgdaSpace{}%
\AgdaSymbol{:}\AgdaSpace{}%
\AgdaFunction{Algebra}\AgdaSpace{}%
\AgdaBound{𝓦}\AgdaSpace{}%
\AgdaBound{𝑆}\AgdaSymbol{)}\AgdaSpace{}%
\AgdaSymbol{→}\AgdaSpace{}%
\AgdaBound{𝓞}\AgdaSpace{}%
\AgdaOperator{\AgdaFunction{⊔}}\AgdaSpace{}%
\AgdaBound{𝓥}\AgdaSpace{}%
\AgdaOperator{\AgdaFunction{⊔}}\AgdaSpace{}%
\AgdaBound{𝓤}\AgdaSpace{}%
\AgdaOperator{\AgdaFunction{⊔}}\AgdaSpace{}%
\AgdaBound{𝓦}\AgdaSpace{}%
\AgdaOperator{\AgdaFunction{̇}}\<%
\\
\>[0]\AgdaBound{𝑨}\AgdaSpace{}%
\AgdaOperator{\AgdaFunction{≅}}\AgdaSpace{}%
\AgdaBound{𝑩}\AgdaSpace{}%
\AgdaSymbol{=}%
\>[9]\AgdaFunction{Σ}\AgdaSpace{}%
\AgdaBound{f}\AgdaSpace{}%
\AgdaFunction{꞉}\AgdaSpace{}%
\AgdaSymbol{(}\AgdaFunction{hom}\AgdaSpace{}%
\AgdaBound{𝑨}\AgdaSpace{}%
\AgdaBound{𝑩}\AgdaSymbol{)}\AgdaSpace{}%
\AgdaFunction{,}\AgdaSpace{}%
\AgdaFunction{Σ}\AgdaSpace{}%
\AgdaBound{g}\AgdaSpace{}%
\AgdaFunction{꞉}\AgdaSpace{}%
\AgdaSymbol{(}\AgdaFunction{hom}\AgdaSpace{}%
\AgdaBound{𝑩}\AgdaSpace{}%
\AgdaBound{𝑨}\AgdaSymbol{)}%
\>[73I]\AgdaFunction{,}\AgdaSpace{}%
\AgdaSymbol{(}\AgdaOperator{\AgdaFunction{∣}}\AgdaSpace{}%
\AgdaBound{f}\AgdaSpace{}%
\AgdaOperator{\AgdaFunction{∣}}\AgdaSpace{}%
\AgdaOperator{\AgdaFunction{∘}}\AgdaSpace{}%
\AgdaOperator{\AgdaFunction{∣}}\AgdaSpace{}%
\AgdaBound{g}\AgdaSpace{}%
\AgdaOperator{\AgdaFunction{∣}}\AgdaSpace{}%
\AgdaOperator{\AgdaFunction{∼}}\AgdaSpace{}%
\AgdaOperator{\AgdaFunction{∣}}\AgdaSpace{}%
\AgdaFunction{𝒾𝒹}\AgdaSpace{}%
\AgdaBound{𝑩}\AgdaSpace{}%
\AgdaOperator{\AgdaFunction{∣}}\AgdaSymbol{)}\<%
\\
\>[.][@{}l@{}]\<[73I]%
\>[43]\AgdaOperator{\AgdaFunction{×}}\AgdaSpace{}%
\AgdaSymbol{(}\AgdaOperator{\AgdaFunction{∣}}\AgdaSpace{}%
\AgdaBound{g}\AgdaSpace{}%
\AgdaOperator{\AgdaFunction{∣}}\AgdaSpace{}%
\AgdaOperator{\AgdaFunction{∘}}\AgdaSpace{}%
\AgdaOperator{\AgdaFunction{∣}}\AgdaSpace{}%
\AgdaBound{f}\AgdaSpace{}%
\AgdaOperator{\AgdaFunction{∣}}\AgdaSpace{}%
\AgdaOperator{\AgdaFunction{∼}}\AgdaSpace{}%
\AgdaOperator{\AgdaFunction{∣}}\AgdaSpace{}%
\AgdaFunction{𝒾𝒹}\AgdaSpace{}%
\AgdaBound{𝑨}\AgdaSpace{}%
\AgdaOperator{\AgdaFunction{∣}}\AgdaSymbol{)}\<%
\end{code}
\ccpad
That is, two structures are \defn{isomorphic} provided there are homomorphisms going back and forth between them which compose to the identity map.

\subsubsection{Isomorphism toolbox}\label{isomorphism-toolbox}
Here we collect some tools that will come in handy later on. The reader is invited to skip over this section and return to it as needed.
\ccpad
\begin{code}%
\>[0]\AgdaKeyword{module}\AgdaSpace{}%
\AgdaModule{\AgdaUnderscore{}}\AgdaSpace{}%
\AgdaSymbol{\{}\AgdaBound{𝓤}\AgdaSpace{}%
\AgdaBound{𝓦}\AgdaSpace{}%
\AgdaSymbol{:}\AgdaSpace{}%
\AgdaFunction{Universe}\AgdaSymbol{\}\{}\AgdaBound{𝑨}\AgdaSpace{}%
\AgdaSymbol{:}\AgdaSpace{}%
\AgdaFunction{Algebra}\AgdaSpace{}%
\AgdaBound{𝓤}\AgdaSpace{}%
\AgdaBound{𝑆}\AgdaSymbol{\}\{}\AgdaBound{𝑩}\AgdaSpace{}%
\AgdaSymbol{:}\AgdaSpace{}%
\AgdaFunction{Algebra}\AgdaSpace{}%
\AgdaBound{𝓦}\AgdaSpace{}%
\AgdaBound{𝑆}\AgdaSymbol{\}}\AgdaSpace{}%
\AgdaKeyword{where}\<%
\\
%
\\[\AgdaEmptyExtraSkip]%
\>[0][@{}l@{\AgdaIndent{0}}]%
\>[1]\AgdaFunction{≅-hom}\AgdaSpace{}%
\AgdaSymbol{:}\AgdaSpace{}%
\AgdaSymbol{(}\AgdaBound{ϕ}\AgdaSpace{}%
\AgdaSymbol{:}\AgdaSpace{}%
\AgdaBound{𝑨}\AgdaSpace{}%
\AgdaOperator{\AgdaFunction{≅}}\AgdaSpace{}%
\AgdaBound{𝑩}\AgdaSymbol{)}\AgdaSpace{}%
\AgdaSymbol{→}\AgdaSpace{}%
\AgdaFunction{hom}\AgdaSpace{}%
\AgdaBound{𝑨}\AgdaSpace{}%
\AgdaBound{𝑩}\<%
\\
%
\>[1]\AgdaFunction{≅-hom}\AgdaSpace{}%
\AgdaBound{ϕ}\AgdaSpace{}%
\AgdaSymbol{=}\AgdaSpace{}%
\AgdaOperator{\AgdaFunction{∣}}\AgdaSpace{}%
\AgdaBound{ϕ}\AgdaSpace{}%
\AgdaOperator{\AgdaFunction{∣}}\<%
\\
%
\\[\AgdaEmptyExtraSkip]%
%
\>[1]\AgdaFunction{≅-inv-hom}\AgdaSpace{}%
\AgdaSymbol{:}\AgdaSpace{}%
\AgdaSymbol{(}\AgdaBound{ϕ}\AgdaSpace{}%
\AgdaSymbol{:}\AgdaSpace{}%
\AgdaBound{𝑨}\AgdaSpace{}%
\AgdaOperator{\AgdaFunction{≅}}\AgdaSpace{}%
\AgdaBound{𝑩}\AgdaSymbol{)}\AgdaSpace{}%
\AgdaSymbol{→}\AgdaSpace{}%
\AgdaFunction{hom}\AgdaSpace{}%
\AgdaBound{𝑩}\AgdaSpace{}%
\AgdaBound{𝑨}\<%
\\
%
\>[1]\AgdaFunction{≅-inv-hom}\AgdaSpace{}%
\AgdaBound{ϕ}\AgdaSpace{}%
\AgdaSymbol{=}\AgdaSpace{}%
\AgdaFunction{fst}\AgdaSpace{}%
\AgdaOperator{\AgdaFunction{∥}}\AgdaSpace{}%
\AgdaBound{ϕ}\AgdaSpace{}%
\AgdaOperator{\AgdaFunction{∥}}\<%
\\
%
\\[\AgdaEmptyExtraSkip]%
%
\>[1]\AgdaFunction{≅-map}\AgdaSpace{}%
\AgdaSymbol{:}\AgdaSpace{}%
\AgdaSymbol{(}\AgdaBound{ϕ}\AgdaSpace{}%
\AgdaSymbol{:}\AgdaSpace{}%
\AgdaBound{𝑨}\AgdaSpace{}%
\AgdaOperator{\AgdaFunction{≅}}\AgdaSpace{}%
\AgdaBound{𝑩}\AgdaSymbol{)}\AgdaSpace{}%
\AgdaSymbol{→}\AgdaSpace{}%
\AgdaOperator{\AgdaFunction{∣}}\AgdaSpace{}%
\AgdaBound{𝑨}\AgdaSpace{}%
\AgdaOperator{\AgdaFunction{∣}}\AgdaSpace{}%
\AgdaSymbol{→}\AgdaSpace{}%
\AgdaOperator{\AgdaFunction{∣}}\AgdaSpace{}%
\AgdaBound{𝑩}\AgdaSpace{}%
\AgdaOperator{\AgdaFunction{∣}}\<%
\\
%
\>[1]\AgdaFunction{≅-map}\AgdaSpace{}%
\AgdaBound{ϕ}\AgdaSpace{}%
\AgdaSymbol{=}\AgdaSpace{}%
\AgdaOperator{\AgdaFunction{∣}}\AgdaSpace{}%
\AgdaFunction{≅-hom}\AgdaSpace{}%
\AgdaBound{ϕ}\AgdaSpace{}%
\AgdaOperator{\AgdaFunction{∣}}\<%
\\
%
\\[\AgdaEmptyExtraSkip]%
%
\>[1]\AgdaFunction{≅-map-is-homomorphism}\AgdaSpace{}%
\AgdaSymbol{:}\AgdaSpace{}%
\AgdaSymbol{(}\AgdaBound{ϕ}\AgdaSpace{}%
\AgdaSymbol{:}\AgdaSpace{}%
\AgdaBound{𝑨}\AgdaSpace{}%
\AgdaOperator{\AgdaFunction{≅}}\AgdaSpace{}%
\AgdaBound{𝑩}\AgdaSymbol{)}\AgdaSpace{}%
\AgdaSymbol{→}\AgdaSpace{}%
\AgdaFunction{is-homomorphism}\AgdaSpace{}%
\AgdaBound{𝑨}\AgdaSpace{}%
\AgdaBound{𝑩}\AgdaSpace{}%
\AgdaSymbol{(}\AgdaFunction{≅-map}\AgdaSpace{}%
\AgdaBound{ϕ}\AgdaSymbol{)}\<%
\\
%
\>[1]\AgdaFunction{≅-map-is-homomorphism}\AgdaSpace{}%
\AgdaBound{ϕ}\AgdaSpace{}%
\AgdaSymbol{=}\AgdaSpace{}%
\AgdaOperator{\AgdaFunction{∥}}\AgdaSpace{}%
\AgdaFunction{≅-hom}\AgdaSpace{}%
\AgdaBound{ϕ}\AgdaSpace{}%
\AgdaOperator{\AgdaFunction{∥}}\<%
\\
%
\\[\AgdaEmptyExtraSkip]%
%
\>[1]\AgdaFunction{≅-inv-map}\AgdaSpace{}%
\AgdaSymbol{:}\AgdaSpace{}%
\AgdaSymbol{(}\AgdaBound{ϕ}\AgdaSpace{}%
\AgdaSymbol{:}\AgdaSpace{}%
\AgdaBound{𝑨}\AgdaSpace{}%
\AgdaOperator{\AgdaFunction{≅}}\AgdaSpace{}%
\AgdaBound{𝑩}\AgdaSymbol{)}\AgdaSpace{}%
\AgdaSymbol{→}\AgdaSpace{}%
\AgdaOperator{\AgdaFunction{∣}}\AgdaSpace{}%
\AgdaBound{𝑩}\AgdaSpace{}%
\AgdaOperator{\AgdaFunction{∣}}\AgdaSpace{}%
\AgdaSymbol{→}\AgdaSpace{}%
\AgdaOperator{\AgdaFunction{∣}}\AgdaSpace{}%
\AgdaBound{𝑨}\AgdaSpace{}%
\AgdaOperator{\AgdaFunction{∣}}\<%
\\
%
\>[1]\AgdaFunction{≅-inv-map}\AgdaSpace{}%
\AgdaBound{ϕ}\AgdaSpace{}%
\AgdaSymbol{=}\AgdaSpace{}%
\AgdaOperator{\AgdaFunction{∣}}\AgdaSpace{}%
\AgdaFunction{≅-inv-hom}\AgdaSpace{}%
\AgdaBound{ϕ}\AgdaSpace{}%
\AgdaOperator{\AgdaFunction{∣}}\<%
\\
%
\\[\AgdaEmptyExtraSkip]%
%
\>[1]\AgdaFunction{≅-inv-map-is-hom}\AgdaSpace{}%
\AgdaSymbol{:}\AgdaSpace{}%
\AgdaSymbol{(}\AgdaBound{ϕ}\AgdaSpace{}%
\AgdaSymbol{:}\AgdaSpace{}%
\AgdaBound{𝑨}\AgdaSpace{}%
\AgdaOperator{\AgdaFunction{≅}}\AgdaSpace{}%
\AgdaBound{𝑩}\AgdaSymbol{)}\AgdaSpace{}%
\AgdaSymbol{→}\AgdaSpace{}%
\AgdaFunction{is-homomorphism}\AgdaSpace{}%
\AgdaBound{𝑩}\AgdaSpace{}%
\AgdaBound{𝑨}\AgdaSpace{}%
\AgdaSymbol{(}\AgdaFunction{≅-inv-map}\AgdaSpace{}%
\AgdaBound{ϕ}\AgdaSymbol{)}\<%
\\
%
\>[1]\AgdaFunction{≅-inv-map-is-hom}\AgdaSpace{}%
\AgdaBound{ϕ}\AgdaSpace{}%
\AgdaSymbol{=}\AgdaSpace{}%
\AgdaOperator{\AgdaFunction{∥}}\AgdaSpace{}%
\AgdaFunction{≅-inv-hom}\AgdaSpace{}%
\AgdaBound{ϕ}\AgdaSpace{}%
\AgdaOperator{\AgdaFunction{∥}}\<%
\\
%
\\[\AgdaEmptyExtraSkip]%
%
\>[1]\AgdaFunction{≅-map-invertible}\AgdaSpace{}%
\AgdaSymbol{:}\AgdaSpace{}%
\AgdaSymbol{(}\AgdaBound{ϕ}\AgdaSpace{}%
\AgdaSymbol{:}\AgdaSpace{}%
\AgdaBound{𝑨}\AgdaSpace{}%
\AgdaOperator{\AgdaFunction{≅}}\AgdaSpace{}%
\AgdaBound{𝑩}\AgdaSymbol{)}\AgdaSpace{}%
\AgdaSymbol{→}\AgdaSpace{}%
\AgdaFunction{invertible}\AgdaSpace{}%
\AgdaSymbol{(}\AgdaFunction{≅-map}\AgdaSpace{}%
\AgdaBound{ϕ}\AgdaSymbol{)}\<%
\\
%
\>[1]\AgdaFunction{≅-map-invertible}\AgdaSpace{}%
\AgdaBound{ϕ}\AgdaSpace{}%
\AgdaSymbol{=}\AgdaSpace{}%
\AgdaSymbol{(}\AgdaFunction{≅-inv-map}\AgdaSpace{}%
\AgdaBound{ϕ}\AgdaSymbol{)}\AgdaSpace{}%
\AgdaOperator{\AgdaInductiveConstructor{,}}\AgdaSpace{}%
\AgdaSymbol{(}\AgdaOperator{\AgdaFunction{∥}}\AgdaSpace{}%
\AgdaFunction{snd}\AgdaSpace{}%
\AgdaOperator{\AgdaFunction{∥}}\AgdaSpace{}%
\AgdaBound{ϕ}\AgdaSpace{}%
\AgdaOperator{\AgdaFunction{∥}}\AgdaSpace{}%
\AgdaOperator{\AgdaFunction{∥}}\AgdaSpace{}%
\AgdaOperator{\AgdaInductiveConstructor{,}}\AgdaSpace{}%
\AgdaOperator{\AgdaFunction{∣}}\AgdaSpace{}%
\AgdaFunction{snd}\AgdaSpace{}%
\AgdaOperator{\AgdaFunction{∥}}\AgdaSpace{}%
\AgdaBound{ϕ}\AgdaSpace{}%
\AgdaOperator{\AgdaFunction{∥}}\AgdaSpace{}%
\AgdaOperator{\AgdaFunction{∣}}\AgdaSymbol{)}\<%
\\
%
\\[\AgdaEmptyExtraSkip]%
%
\>[1]\AgdaFunction{≅-map-is-equiv}\AgdaSpace{}%
\AgdaSymbol{:}\AgdaSpace{}%
\AgdaSymbol{(}\AgdaBound{ϕ}\AgdaSpace{}%
\AgdaSymbol{:}\AgdaSpace{}%
\AgdaBound{𝑨}\AgdaSpace{}%
\AgdaOperator{\AgdaFunction{≅}}\AgdaSpace{}%
\AgdaBound{𝑩}\AgdaSymbol{)}\AgdaSpace{}%
\AgdaSymbol{→}\AgdaSpace{}%
\AgdaFunction{is-equiv}\AgdaSpace{}%
\AgdaSymbol{(}\AgdaFunction{≅-map}\AgdaSpace{}%
\AgdaBound{ϕ}\AgdaSymbol{)}\<%
\\
%
\>[1]\AgdaFunction{≅-map-is-equiv}\AgdaSpace{}%
\AgdaBound{ϕ}\AgdaSpace{}%
\AgdaSymbol{=}\AgdaSpace{}%
\AgdaFunction{invertibles-are-equivs}\AgdaSpace{}%
\AgdaSymbol{(}\AgdaFunction{≅-map}\AgdaSpace{}%
\AgdaBound{ϕ}\AgdaSymbol{)}\AgdaSpace{}%
\AgdaSymbol{(}\AgdaFunction{≅-map-invertible}\AgdaSpace{}%
\AgdaBound{ϕ}\AgdaSymbol{)}\<%
\\
%
\\[\AgdaEmptyExtraSkip]%
%
\>[1]\AgdaFunction{≅-map-is-embedding}\AgdaSpace{}%
\AgdaSymbol{:}\AgdaSpace{}%
\AgdaSymbol{(}\AgdaBound{ϕ}\AgdaSpace{}%
\AgdaSymbol{:}\AgdaSpace{}%
\AgdaBound{𝑨}\AgdaSpace{}%
\AgdaOperator{\AgdaFunction{≅}}\AgdaSpace{}%
\AgdaBound{𝑩}\AgdaSymbol{)}\AgdaSpace{}%
\AgdaSymbol{→}\AgdaSpace{}%
\AgdaFunction{is-embedding}\AgdaSpace{}%
\AgdaSymbol{(}\AgdaFunction{≅-map}\AgdaSpace{}%
\AgdaBound{ϕ}\AgdaSymbol{)}\<%
\\
%
\>[1]\AgdaFunction{≅-map-is-embedding}\AgdaSpace{}%
\AgdaBound{ϕ}\AgdaSpace{}%
\AgdaSymbol{=}\AgdaSpace{}%
\AgdaFunction{equivs-are-embeddings}\AgdaSpace{}%
\AgdaSymbol{(}\AgdaFunction{≅-map}\AgdaSpace{}%
\AgdaBound{ϕ}\AgdaSymbol{)}\AgdaSpace{}%
\AgdaSymbol{(}\AgdaFunction{≅-map-is-equiv}\AgdaSpace{}%
\AgdaBound{ϕ}\AgdaSymbol{)}\<%
\end{code}

\subsubsection{Isomorphism is an equivalence relation}\label{isomorphism-is-an-equivalence-relation}

\begin{code}%
\>[0]\AgdaFunction{REFL-≅}\AgdaSpace{}%
\AgdaFunction{ID≅}\AgdaSpace{}%
\AgdaSymbol{:}\AgdaSpace{}%
\AgdaSymbol{\{}\AgdaBound{𝓤}\AgdaSpace{}%
\AgdaSymbol{:}\AgdaSpace{}%
\AgdaFunction{Universe}\AgdaSymbol{\}}\AgdaSpace{}%
\AgdaSymbol{(}\AgdaBound{𝑨}\AgdaSpace{}%
\AgdaSymbol{:}\AgdaSpace{}%
\AgdaFunction{Algebra}\AgdaSpace{}%
\AgdaBound{𝓤}\AgdaSpace{}%
\AgdaBound{𝑆}\AgdaSymbol{)}\AgdaSpace{}%
\AgdaSymbol{→}\AgdaSpace{}%
\AgdaBound{𝑨}\AgdaSpace{}%
\AgdaOperator{\AgdaFunction{≅}}\AgdaSpace{}%
\AgdaBound{𝑨}\<%
\\
\>[0]\AgdaFunction{ID≅}\AgdaSpace{}%
\AgdaBound{𝑨}\AgdaSpace{}%
\AgdaSymbol{=}\AgdaSpace{}%
\AgdaFunction{𝒾𝒹}\AgdaSpace{}%
\AgdaBound{𝑨}\AgdaSpace{}%
\AgdaOperator{\AgdaInductiveConstructor{,}}\AgdaSpace{}%
\AgdaFunction{𝒾𝒹}\AgdaSpace{}%
\AgdaBound{𝑨}\AgdaSpace{}%
\AgdaOperator{\AgdaInductiveConstructor{,}}\AgdaSpace{}%
\AgdaSymbol{(λ}\AgdaSpace{}%
\AgdaBound{a}\AgdaSpace{}%
\AgdaSymbol{→}\AgdaSpace{}%
\AgdaInductiveConstructor{𝓇ℯ𝒻𝓁}\AgdaSymbol{)}\AgdaSpace{}%
\AgdaOperator{\AgdaInductiveConstructor{,}}\AgdaSpace{}%
\AgdaSymbol{(λ}\AgdaSpace{}%
\AgdaBound{a}\AgdaSpace{}%
\AgdaSymbol{→}\AgdaSpace{}%
\AgdaInductiveConstructor{𝓇ℯ𝒻𝓁}\AgdaSymbol{)}\<%
\\
\>[0]\AgdaFunction{REFL-≅}\AgdaSpace{}%
\AgdaSymbol{=}\AgdaSpace{}%
\AgdaFunction{ID≅}\<%
\\
%
\\[\AgdaEmptyExtraSkip]%
\>[0]\AgdaFunction{refl-≅}\AgdaSpace{}%
\AgdaFunction{id≅}\AgdaSpace{}%
\AgdaSymbol{:}\AgdaSpace{}%
\AgdaSymbol{\{}\AgdaBound{𝓤}\AgdaSpace{}%
\AgdaSymbol{:}\AgdaSpace{}%
\AgdaFunction{Universe}\AgdaSymbol{\}}\AgdaSpace{}%
\AgdaSymbol{\{}\AgdaBound{𝑨}\AgdaSpace{}%
\AgdaSymbol{:}\AgdaSpace{}%
\AgdaFunction{Algebra}\AgdaSpace{}%
\AgdaBound{𝓤}\AgdaSpace{}%
\AgdaBound{𝑆}\AgdaSymbol{\}}\AgdaSpace{}%
\AgdaSymbol{→}\AgdaSpace{}%
\AgdaBound{𝑨}\AgdaSpace{}%
\AgdaOperator{\AgdaFunction{≅}}\AgdaSpace{}%
\AgdaBound{𝑨}\<%
\\
\>[0]\AgdaFunction{id≅}\AgdaSpace{}%
\AgdaSymbol{\{}\AgdaBound{𝓤}\AgdaSymbol{\}\{}\AgdaBound{𝑨}\AgdaSymbol{\}}\AgdaSpace{}%
\AgdaSymbol{=}\AgdaSpace{}%
\AgdaFunction{ID≅}\AgdaSpace{}%
\AgdaSymbol{\{}\AgdaBound{𝓤}\AgdaSymbol{\}}\AgdaSpace{}%
\AgdaBound{𝑨}\<%
\\
\>[0]\AgdaFunction{refl-≅}\AgdaSpace{}%
\AgdaSymbol{=}\AgdaSpace{}%
\AgdaFunction{id≅}\<%
\\
%
\\[\AgdaEmptyExtraSkip]%
\>[0]\AgdaFunction{sym-≅}\AgdaSpace{}%
\AgdaSymbol{:}\AgdaSpace{}%
\AgdaSymbol{\{}\AgdaBound{𝓠}\AgdaSpace{}%
\AgdaBound{𝓤}\AgdaSpace{}%
\AgdaSymbol{:}\AgdaSpace{}%
\AgdaFunction{Universe}\AgdaSymbol{\}\{}\AgdaBound{𝑨}\AgdaSpace{}%
\AgdaSymbol{:}\AgdaSpace{}%
\AgdaFunction{Algebra}\AgdaSpace{}%
\AgdaBound{𝓠}\AgdaSpace{}%
\AgdaBound{𝑆}\AgdaSymbol{\}\{}\AgdaBound{𝑩}\AgdaSpace{}%
\AgdaSymbol{:}\AgdaSpace{}%
\AgdaFunction{Algebra}\AgdaSpace{}%
\AgdaBound{𝓤}\AgdaSpace{}%
\AgdaBound{𝑆}\AgdaSymbol{\}}\<%
\\
\>[0][@{}l@{\AgdaIndent{0}}]%
\>[1]\AgdaSymbol{→}%
\>[8]\AgdaBound{𝑨}\AgdaSpace{}%
\AgdaOperator{\AgdaFunction{≅}}\AgdaSpace{}%
\AgdaBound{𝑩}\AgdaSpace{}%
\AgdaSymbol{→}\AgdaSpace{}%
\AgdaBound{𝑩}\AgdaSpace{}%
\AgdaOperator{\AgdaFunction{≅}}\AgdaSpace{}%
\AgdaBound{𝑨}\<%
\\
\>[0]\AgdaFunction{sym-≅}\AgdaSpace{}%
\AgdaBound{h}\AgdaSpace{}%
\AgdaSymbol{=}\AgdaSpace{}%
\AgdaFunction{fst}\AgdaSpace{}%
\AgdaOperator{\AgdaFunction{∥}}\AgdaSpace{}%
\AgdaBound{h}\AgdaSpace{}%
\AgdaOperator{\AgdaFunction{∥}}\AgdaSpace{}%
\AgdaOperator{\AgdaInductiveConstructor{,}}\AgdaSpace{}%
\AgdaFunction{fst}\AgdaSpace{}%
\AgdaBound{h}\AgdaSpace{}%
\AgdaOperator{\AgdaInductiveConstructor{,}}\AgdaSpace{}%
\AgdaOperator{\AgdaFunction{∥}}\AgdaSpace{}%
\AgdaFunction{snd}\AgdaSpace{}%
\AgdaOperator{\AgdaFunction{∥}}\AgdaSpace{}%
\AgdaBound{h}\AgdaSpace{}%
\AgdaOperator{\AgdaFunction{∥}}\AgdaSpace{}%
\AgdaOperator{\AgdaFunction{∥}}\AgdaSpace{}%
\AgdaOperator{\AgdaInductiveConstructor{,}}\AgdaSpace{}%
\AgdaOperator{\AgdaFunction{∣}}\AgdaSpace{}%
\AgdaFunction{snd}\AgdaSpace{}%
\AgdaOperator{\AgdaFunction{∥}}\AgdaSpace{}%
\AgdaBound{h}\AgdaSpace{}%
\AgdaOperator{\AgdaFunction{∥}}\AgdaSpace{}%
\AgdaOperator{\AgdaFunction{∣}}\<%
\\
%
\\[\AgdaEmptyExtraSkip]%
\>[0]\AgdaFunction{trans-≅}\AgdaSpace{}%
\AgdaSymbol{:}%
\>[378I]\AgdaSymbol{\{}\AgdaBound{𝓠}\AgdaSpace{}%
\AgdaBound{𝓤}\AgdaSpace{}%
\AgdaBound{𝓦}\AgdaSpace{}%
\AgdaSymbol{:}\AgdaSpace{}%
\AgdaFunction{Universe}\AgdaSymbol{\}}\<%
\\
\>[.][@{}l@{}]\<[378I]%
\>[10]\AgdaSymbol{(}\AgdaBound{𝑨}\AgdaSpace{}%
\AgdaSymbol{:}\AgdaSpace{}%
\AgdaFunction{Algebra}\AgdaSpace{}%
\AgdaBound{𝓠}\AgdaSpace{}%
\AgdaBound{𝑆}\AgdaSymbol{)(}\AgdaBound{𝑩}\AgdaSpace{}%
\AgdaSymbol{:}\AgdaSpace{}%
\AgdaFunction{Algebra}\AgdaSpace{}%
\AgdaBound{𝓤}\AgdaSpace{}%
\AgdaBound{𝑆}\AgdaSymbol{)(}\AgdaBound{𝑪}\AgdaSpace{}%
\AgdaSymbol{:}\AgdaSpace{}%
\AgdaFunction{Algebra}\AgdaSpace{}%
\AgdaBound{𝓦}\AgdaSpace{}%
\AgdaBound{𝑆}\AgdaSymbol{)}\<%
\\
\>[0][@{}l@{\AgdaIndent{0}}]%
\>[1]\AgdaSymbol{→}%
\>[10]\AgdaBound{𝑨}\AgdaSpace{}%
\AgdaOperator{\AgdaFunction{≅}}\AgdaSpace{}%
\AgdaBound{𝑩}\AgdaSpace{}%
\AgdaSymbol{→}\AgdaSpace{}%
\AgdaBound{𝑩}\AgdaSpace{}%
\AgdaOperator{\AgdaFunction{≅}}\AgdaSpace{}%
\AgdaBound{𝑪}\<%
\\
%
\>[10]\AgdaComment{-------------}\<%
\\
%
\>[1]\AgdaSymbol{→}%
\>[10]\AgdaBound{𝑨}\AgdaSpace{}%
\AgdaOperator{\AgdaFunction{≅}}\AgdaSpace{}%
\AgdaBound{𝑪}\<%
\\
%
\\[\AgdaEmptyExtraSkip]%
\>[0]\AgdaFunction{trans-≅}\AgdaSpace{}%
\AgdaBound{𝑨}\AgdaSpace{}%
\AgdaBound{𝑩}\AgdaSpace{}%
\AgdaBound{𝑪}\AgdaSpace{}%
\AgdaBound{ab}\AgdaSpace{}%
\AgdaBound{bc}\AgdaSpace{}%
\AgdaSymbol{=}\AgdaSpace{}%
\AgdaFunction{f}\AgdaSpace{}%
\AgdaOperator{\AgdaInductiveConstructor{,}}\AgdaSpace{}%
\AgdaFunction{g}\AgdaSpace{}%
\AgdaOperator{\AgdaInductiveConstructor{,}}\AgdaSpace{}%
\AgdaFunction{α}\AgdaSpace{}%
\AgdaOperator{\AgdaInductiveConstructor{,}}\AgdaSpace{}%
\AgdaFunction{β}\<%
\\
\>[0][@{}l@{\AgdaIndent{0}}]%
\>[1]\AgdaKeyword{where}\<%
\\
%
\>[1]\AgdaFunction{f1}\AgdaSpace{}%
\AgdaSymbol{:}\AgdaSpace{}%
\AgdaFunction{hom}\AgdaSpace{}%
\AgdaBound{𝑨}\AgdaSpace{}%
\AgdaBound{𝑩}\<%
\\
%
\>[1]\AgdaFunction{f1}\AgdaSpace{}%
\AgdaSymbol{=}\AgdaSpace{}%
\AgdaOperator{\AgdaFunction{∣}}\AgdaSpace{}%
\AgdaBound{ab}\AgdaSpace{}%
\AgdaOperator{\AgdaFunction{∣}}\<%
\\
%
\>[1]\AgdaFunction{f2}\AgdaSpace{}%
\AgdaSymbol{:}\AgdaSpace{}%
\AgdaFunction{hom}\AgdaSpace{}%
\AgdaBound{𝑩}\AgdaSpace{}%
\AgdaBound{𝑪}\<%
\\
%
\>[1]\AgdaFunction{f2}\AgdaSpace{}%
\AgdaSymbol{=}\AgdaSpace{}%
\AgdaOperator{\AgdaFunction{∣}}\AgdaSpace{}%
\AgdaBound{bc}\AgdaSpace{}%
\AgdaOperator{\AgdaFunction{∣}}\<%
\\
%
\>[1]\AgdaFunction{f}\AgdaSpace{}%
\AgdaSymbol{:}\AgdaSpace{}%
\AgdaFunction{hom}\AgdaSpace{}%
\AgdaBound{𝑨}\AgdaSpace{}%
\AgdaBound{𝑪}\<%
\\
%
\>[1]\AgdaFunction{f}\AgdaSpace{}%
\AgdaSymbol{=}\AgdaSpace{}%
\AgdaFunction{HCompClosed}\AgdaSpace{}%
\AgdaBound{𝑨}\AgdaSpace{}%
\AgdaBound{𝑩}\AgdaSpace{}%
\AgdaBound{𝑪}\AgdaSpace{}%
\AgdaFunction{f1}\AgdaSpace{}%
\AgdaFunction{f2}\<%
\\
%
\\[\AgdaEmptyExtraSkip]%
%
\>[1]\AgdaFunction{g1}\AgdaSpace{}%
\AgdaSymbol{:}\AgdaSpace{}%
\AgdaFunction{hom}\AgdaSpace{}%
\AgdaBound{𝑪}\AgdaSpace{}%
\AgdaBound{𝑩}\<%
\\
%
\>[1]\AgdaFunction{g1}\AgdaSpace{}%
\AgdaSymbol{=}\AgdaSpace{}%
\AgdaFunction{fst}\AgdaSpace{}%
\AgdaOperator{\AgdaFunction{∥}}\AgdaSpace{}%
\AgdaBound{bc}\AgdaSpace{}%
\AgdaOperator{\AgdaFunction{∥}}\<%
\\
%
\>[1]\AgdaFunction{g2}\AgdaSpace{}%
\AgdaSymbol{:}\AgdaSpace{}%
\AgdaFunction{hom}\AgdaSpace{}%
\AgdaBound{𝑩}\AgdaSpace{}%
\AgdaBound{𝑨}\<%
\\
%
\>[1]\AgdaFunction{g2}\AgdaSpace{}%
\AgdaSymbol{=}\AgdaSpace{}%
\AgdaFunction{fst}\AgdaSpace{}%
\AgdaOperator{\AgdaFunction{∥}}\AgdaSpace{}%
\AgdaBound{ab}\AgdaSpace{}%
\AgdaOperator{\AgdaFunction{∥}}\<%
\\
%
\>[1]\AgdaFunction{g}\AgdaSpace{}%
\AgdaSymbol{:}\AgdaSpace{}%
\AgdaFunction{hom}\AgdaSpace{}%
\AgdaBound{𝑪}\AgdaSpace{}%
\AgdaBound{𝑨}\<%
\\
%
\>[1]\AgdaFunction{g}\AgdaSpace{}%
\AgdaSymbol{=}\AgdaSpace{}%
\AgdaFunction{HCompClosed}\AgdaSpace{}%
\AgdaBound{𝑪}\AgdaSpace{}%
\AgdaBound{𝑩}\AgdaSpace{}%
\AgdaBound{𝑨}\AgdaSpace{}%
\AgdaFunction{g1}\AgdaSpace{}%
\AgdaFunction{g2}\<%
\\
%
\\[\AgdaEmptyExtraSkip]%
%
\>[1]\AgdaFunction{f1∼g2}\AgdaSpace{}%
\AgdaSymbol{:}\AgdaSpace{}%
\AgdaOperator{\AgdaFunction{∣}}\AgdaSpace{}%
\AgdaFunction{f1}\AgdaSpace{}%
\AgdaOperator{\AgdaFunction{∣}}\AgdaSpace{}%
\AgdaOperator{\AgdaFunction{∘}}\AgdaSpace{}%
\AgdaOperator{\AgdaFunction{∣}}\AgdaSpace{}%
\AgdaFunction{g2}\AgdaSpace{}%
\AgdaOperator{\AgdaFunction{∣}}\AgdaSpace{}%
\AgdaOperator{\AgdaFunction{∼}}\AgdaSpace{}%
\AgdaOperator{\AgdaFunction{∣}}\AgdaSpace{}%
\AgdaFunction{𝒾𝒹}\AgdaSpace{}%
\AgdaBound{𝑩}\AgdaSpace{}%
\AgdaOperator{\AgdaFunction{∣}}\<%
\\
%
\>[1]\AgdaFunction{f1∼g2}\AgdaSpace{}%
\AgdaSymbol{=}\AgdaSpace{}%
\AgdaOperator{\AgdaFunction{∣}}\AgdaSpace{}%
\AgdaFunction{snd}\AgdaSpace{}%
\AgdaOperator{\AgdaFunction{∥}}\AgdaSpace{}%
\AgdaBound{ab}\AgdaSpace{}%
\AgdaOperator{\AgdaFunction{∥}}\AgdaSpace{}%
\AgdaOperator{\AgdaFunction{∣}}\<%
\\
%
\\[\AgdaEmptyExtraSkip]%
%
\>[1]\AgdaFunction{g2∼f1}\AgdaSpace{}%
\AgdaSymbol{:}\AgdaSpace{}%
\AgdaOperator{\AgdaFunction{∣}}\AgdaSpace{}%
\AgdaFunction{g2}\AgdaSpace{}%
\AgdaOperator{\AgdaFunction{∣}}\AgdaSpace{}%
\AgdaOperator{\AgdaFunction{∘}}\AgdaSpace{}%
\AgdaOperator{\AgdaFunction{∣}}\AgdaSpace{}%
\AgdaFunction{f1}\AgdaSpace{}%
\AgdaOperator{\AgdaFunction{∣}}\AgdaSpace{}%
\AgdaOperator{\AgdaFunction{∼}}\AgdaSpace{}%
\AgdaOperator{\AgdaFunction{∣}}\AgdaSpace{}%
\AgdaFunction{𝒾𝒹}\AgdaSpace{}%
\AgdaBound{𝑨}\AgdaSpace{}%
\AgdaOperator{\AgdaFunction{∣}}\<%
\\
%
\>[1]\AgdaFunction{g2∼f1}\AgdaSpace{}%
\AgdaSymbol{=}\AgdaSpace{}%
\AgdaOperator{\AgdaFunction{∥}}\AgdaSpace{}%
\AgdaFunction{snd}\AgdaSpace{}%
\AgdaOperator{\AgdaFunction{∥}}\AgdaSpace{}%
\AgdaBound{ab}\AgdaSpace{}%
\AgdaOperator{\AgdaFunction{∥}}\AgdaSpace{}%
\AgdaOperator{\AgdaFunction{∥}}\<%
\\
%
\\[\AgdaEmptyExtraSkip]%
%
\>[1]\AgdaFunction{f2∼g1}\AgdaSpace{}%
\AgdaSymbol{:}\AgdaSpace{}%
\AgdaOperator{\AgdaFunction{∣}}\AgdaSpace{}%
\AgdaFunction{f2}\AgdaSpace{}%
\AgdaOperator{\AgdaFunction{∣}}\AgdaSpace{}%
\AgdaOperator{\AgdaFunction{∘}}\AgdaSpace{}%
\AgdaOperator{\AgdaFunction{∣}}\AgdaSpace{}%
\AgdaFunction{g1}\AgdaSpace{}%
\AgdaOperator{\AgdaFunction{∣}}\AgdaSpace{}%
\AgdaOperator{\AgdaFunction{∼}}\AgdaSpace{}%
\AgdaOperator{\AgdaFunction{∣}}\AgdaSpace{}%
\AgdaFunction{𝒾𝒹}\AgdaSpace{}%
\AgdaBound{𝑪}\AgdaSpace{}%
\AgdaOperator{\AgdaFunction{∣}}\<%
\\
%
\>[1]\AgdaFunction{f2∼g1}\AgdaSpace{}%
\AgdaSymbol{=}%
\>[10]\AgdaOperator{\AgdaFunction{∣}}\AgdaSpace{}%
\AgdaFunction{snd}\AgdaSpace{}%
\AgdaOperator{\AgdaFunction{∥}}\AgdaSpace{}%
\AgdaBound{bc}\AgdaSpace{}%
\AgdaOperator{\AgdaFunction{∥}}\AgdaSpace{}%
\AgdaOperator{\AgdaFunction{∣}}\<%
\\
%
\\[\AgdaEmptyExtraSkip]%
%
\>[1]\AgdaFunction{g1∼f2}\AgdaSpace{}%
\AgdaSymbol{:}\AgdaSpace{}%
\AgdaOperator{\AgdaFunction{∣}}\AgdaSpace{}%
\AgdaFunction{g1}\AgdaSpace{}%
\AgdaOperator{\AgdaFunction{∣}}\AgdaSpace{}%
\AgdaOperator{\AgdaFunction{∘}}\AgdaSpace{}%
\AgdaOperator{\AgdaFunction{∣}}\AgdaSpace{}%
\AgdaFunction{f2}\AgdaSpace{}%
\AgdaOperator{\AgdaFunction{∣}}\AgdaSpace{}%
\AgdaOperator{\AgdaFunction{∼}}\AgdaSpace{}%
\AgdaOperator{\AgdaFunction{∣}}\AgdaSpace{}%
\AgdaFunction{𝒾𝒹}\AgdaSpace{}%
\AgdaBound{𝑩}\AgdaSpace{}%
\AgdaOperator{\AgdaFunction{∣}}\<%
\\
%
\>[1]\AgdaFunction{g1∼f2}\AgdaSpace{}%
\AgdaSymbol{=}\AgdaSpace{}%
\AgdaOperator{\AgdaFunction{∥}}\AgdaSpace{}%
\AgdaFunction{snd}\AgdaSpace{}%
\AgdaOperator{\AgdaFunction{∥}}\AgdaSpace{}%
\AgdaBound{bc}\AgdaSpace{}%
\AgdaOperator{\AgdaFunction{∥}}\AgdaSpace{}%
\AgdaOperator{\AgdaFunction{∥}}\<%
\\
%
\\[\AgdaEmptyExtraSkip]%
%
\>[1]\AgdaFunction{α}\AgdaSpace{}%
\AgdaSymbol{:}\AgdaSpace{}%
\AgdaOperator{\AgdaFunction{∣}}\AgdaSpace{}%
\AgdaFunction{f}\AgdaSpace{}%
\AgdaOperator{\AgdaFunction{∣}}\AgdaSpace{}%
\AgdaOperator{\AgdaFunction{∘}}\AgdaSpace{}%
\AgdaOperator{\AgdaFunction{∣}}\AgdaSpace{}%
\AgdaFunction{g}\AgdaSpace{}%
\AgdaOperator{\AgdaFunction{∣}}\AgdaSpace{}%
\AgdaOperator{\AgdaFunction{∼}}\AgdaSpace{}%
\AgdaOperator{\AgdaFunction{∣}}\AgdaSpace{}%
\AgdaFunction{𝒾𝒹}\AgdaSpace{}%
\AgdaBound{𝑪}\AgdaSpace{}%
\AgdaOperator{\AgdaFunction{∣}}\<%
\\
%
\>[1]\AgdaFunction{α}\AgdaSpace{}%
\AgdaBound{x}\AgdaSpace{}%
\AgdaSymbol{=}%
\>[566I]\AgdaSymbol{(}\AgdaOperator{\AgdaFunction{∣}}\AgdaSpace{}%
\AgdaFunction{f}\AgdaSpace{}%
\AgdaOperator{\AgdaFunction{∣}}\AgdaSpace{}%
\AgdaOperator{\AgdaFunction{∘}}\AgdaSpace{}%
\AgdaOperator{\AgdaFunction{∣}}\AgdaSpace{}%
\AgdaFunction{g}\AgdaSpace{}%
\AgdaOperator{\AgdaFunction{∣}}\AgdaSymbol{)}\AgdaSpace{}%
\AgdaBound{x}%
\>[43]\AgdaOperator{\AgdaFunction{≡⟨}}\AgdaSpace{}%
\AgdaInductiveConstructor{𝓇ℯ𝒻𝓁}\AgdaSpace{}%
\AgdaOperator{\AgdaFunction{⟩}}\<%
\\
\>[.][@{}l@{}]\<[566I]%
\>[7]\AgdaOperator{\AgdaFunction{∣}}\AgdaSpace{}%
\AgdaFunction{f2}\AgdaSpace{}%
\AgdaOperator{\AgdaFunction{∣}}\AgdaSymbol{((}\AgdaOperator{\AgdaFunction{∣}}\AgdaSpace{}%
\AgdaFunction{f1}\AgdaSpace{}%
\AgdaOperator{\AgdaFunction{∣}}\AgdaSpace{}%
\AgdaOperator{\AgdaFunction{∘}}\AgdaSpace{}%
\AgdaOperator{\AgdaFunction{∣}}\AgdaSpace{}%
\AgdaFunction{g2}\AgdaSpace{}%
\AgdaOperator{\AgdaFunction{∣}}\AgdaSymbol{)(}\AgdaOperator{\AgdaFunction{∣}}\AgdaSpace{}%
\AgdaFunction{g1}\AgdaSpace{}%
\AgdaOperator{\AgdaFunction{∣}}\AgdaSpace{}%
\AgdaBound{x}\AgdaSymbol{))}\AgdaSpace{}%
\AgdaOperator{\AgdaFunction{≡⟨}}\AgdaSpace{}%
\AgdaFunction{ap}\AgdaSpace{}%
\AgdaOperator{\AgdaFunction{∣}}\AgdaSpace{}%
\AgdaFunction{f2}\AgdaSpace{}%
\AgdaOperator{\AgdaFunction{∣}}\AgdaSymbol{(}\AgdaFunction{f1∼g2}\AgdaSymbol{(}\AgdaOperator{\AgdaFunction{∣}}\AgdaSpace{}%
\AgdaFunction{g1}\AgdaSpace{}%
\AgdaOperator{\AgdaFunction{∣}}\AgdaSpace{}%
\AgdaBound{x}\AgdaSymbol{))}\AgdaOperator{\AgdaFunction{⟩}}\<%
\\
%
\>[7]\AgdaSymbol{(}\AgdaOperator{\AgdaFunction{∣}}\AgdaSpace{}%
\AgdaFunction{f2}\AgdaSpace{}%
\AgdaOperator{\AgdaFunction{∣}}\AgdaSpace{}%
\AgdaOperator{\AgdaFunction{∘}}\AgdaSpace{}%
\AgdaOperator{\AgdaFunction{∣}}\AgdaSpace{}%
\AgdaFunction{g1}\AgdaSpace{}%
\AgdaOperator{\AgdaFunction{∣}}\AgdaSymbol{)}\AgdaSpace{}%
\AgdaBound{x}%
\>[44]\AgdaOperator{\AgdaFunction{≡⟨}}\AgdaSpace{}%
\AgdaFunction{f2∼g1}\AgdaSpace{}%
\AgdaBound{x}\AgdaSpace{}%
\AgdaOperator{\AgdaFunction{⟩}}\<%
\\
%
\>[7]\AgdaOperator{\AgdaFunction{∣}}\AgdaSpace{}%
\AgdaFunction{𝒾𝒹}\AgdaSpace{}%
\AgdaBound{𝑪}\AgdaSpace{}%
\AgdaOperator{\AgdaFunction{∣}}\AgdaSpace{}%
\AgdaBound{x}%
\>[42]\AgdaOperator{\AgdaFunction{∎}}\<%
\\
%
\\[\AgdaEmptyExtraSkip]%
%
\>[1]\AgdaFunction{β}\AgdaSpace{}%
\AgdaSymbol{:}\AgdaSpace{}%
\AgdaOperator{\AgdaFunction{∣}}\AgdaSpace{}%
\AgdaFunction{g}\AgdaSpace{}%
\AgdaOperator{\AgdaFunction{∣}}\AgdaSpace{}%
\AgdaOperator{\AgdaFunction{∘}}\AgdaSpace{}%
\AgdaOperator{\AgdaFunction{∣}}\AgdaSpace{}%
\AgdaFunction{f}\AgdaSpace{}%
\AgdaOperator{\AgdaFunction{∣}}\AgdaSpace{}%
\AgdaOperator{\AgdaFunction{∼}}\AgdaSpace{}%
\AgdaOperator{\AgdaFunction{∣}}\AgdaSpace{}%
\AgdaFunction{𝒾𝒹}\AgdaSpace{}%
\AgdaBound{𝑨}\AgdaSpace{}%
\AgdaOperator{\AgdaFunction{∣}}\<%
\\
%
\>[1]\AgdaFunction{β}\AgdaSpace{}%
\AgdaBound{x}\AgdaSpace{}%
\AgdaSymbol{=}\AgdaSpace{}%
\AgdaSymbol{(}\AgdaFunction{ap}\AgdaSpace{}%
\AgdaOperator{\AgdaFunction{∣}}\AgdaSpace{}%
\AgdaFunction{g2}\AgdaSpace{}%
\AgdaOperator{\AgdaFunction{∣}}\AgdaSpace{}%
\AgdaSymbol{(}\AgdaFunction{g1∼f2}\AgdaSpace{}%
\AgdaSymbol{(}\AgdaOperator{\AgdaFunction{∣}}\AgdaSpace{}%
\AgdaFunction{f1}\AgdaSpace{}%
\AgdaOperator{\AgdaFunction{∣}}\AgdaSpace{}%
\AgdaBound{x}\AgdaSymbol{)))}\AgdaSpace{}%
\AgdaOperator{\AgdaFunction{∙}}\AgdaSpace{}%
\AgdaFunction{g2∼f1}\AgdaSpace{}%
\AgdaBound{x}\<%
\end{code}
\ccpad
To make \af{trans-≅} easier to apply in certain situations, we define a couple of alternatives where the only difference is which arguments are implicit.
\ccpad
\begin{code}%
\>[0]\AgdaKeyword{module}\AgdaSpace{}%
\AgdaModule{\AgdaUnderscore{}}\AgdaSpace{}%
\AgdaSymbol{\{}\AgdaBound{𝓧}\AgdaSpace{}%
\AgdaBound{𝓨}\AgdaSpace{}%
\AgdaBound{𝓩}\AgdaSpace{}%
\AgdaSymbol{:}\AgdaSpace{}%
\AgdaFunction{Universe}\AgdaSymbol{\}}\AgdaSpace{}%
\AgdaKeyword{where}\<%
\\
%
\\[\AgdaEmptyExtraSkip]%
\>[0][@{}l@{\AgdaIndent{0}}]%
\>[1]\AgdaFunction{TRANS-≅}\AgdaSpace{}%
\AgdaSymbol{:}\AgdaSpace{}%
\AgdaSymbol{\{}\AgdaBound{𝑨}\AgdaSpace{}%
\AgdaSymbol{:}\AgdaSpace{}%
\AgdaFunction{Algebra}\AgdaSpace{}%
\AgdaBound{𝓧}\AgdaSpace{}%
\AgdaBound{𝑆}\AgdaSymbol{\}\{}\AgdaBound{𝑩}\AgdaSpace{}%
\AgdaSymbol{:}\AgdaSpace{}%
\AgdaFunction{Algebra}\AgdaSpace{}%
\AgdaBound{𝓨}\AgdaSpace{}%
\AgdaBound{𝑆}\AgdaSymbol{\}\{}\AgdaBound{𝑪}\AgdaSpace{}%
\AgdaSymbol{:}\AgdaSpace{}%
\AgdaFunction{Algebra}\AgdaSpace{}%
\AgdaBound{𝓩}\AgdaSpace{}%
\AgdaBound{𝑆}\AgdaSymbol{\}}\<%
\\
\>[1][@{}l@{\AgdaIndent{0}}]%
\>[2]\AgdaSymbol{→}%
\>[11]\AgdaBound{𝑨}\AgdaSpace{}%
\AgdaOperator{\AgdaFunction{≅}}\AgdaSpace{}%
\AgdaBound{𝑩}\AgdaSpace{}%
\AgdaSymbol{→}\AgdaSpace{}%
\AgdaBound{𝑩}\AgdaSpace{}%
\AgdaOperator{\AgdaFunction{≅}}\AgdaSpace{}%
\AgdaBound{𝑪}\<%
\\
%
\>[11]\AgdaComment{-------------}\<%
\\
%
\>[2]\AgdaSymbol{→}%
\>[11]\AgdaBound{𝑨}\AgdaSpace{}%
\AgdaOperator{\AgdaFunction{≅}}\AgdaSpace{}%
\AgdaBound{𝑪}\<%
\\
%
\\[\AgdaEmptyExtraSkip]%
%
\>[1]\AgdaFunction{TRANS-≅}\AgdaSpace{}%
\AgdaSymbol{\{}\AgdaBound{𝑨}\AgdaSymbol{\}\{}\AgdaBound{𝑩}\AgdaSymbol{\}\{}\AgdaBound{𝑪}\AgdaSymbol{\}}\AgdaSpace{}%
\AgdaSymbol{=}\AgdaSpace{}%
\AgdaFunction{trans-≅}\AgdaSpace{}%
\AgdaBound{𝑨}\AgdaSpace{}%
\AgdaBound{𝑩}\AgdaSpace{}%
\AgdaBound{𝑪}\<%
\\
%
\\[\AgdaEmptyExtraSkip]%
%
\>[1]\AgdaFunction{Trans-≅}\AgdaSpace{}%
\AgdaSymbol{:}\AgdaSpace{}%
\AgdaSymbol{(}\AgdaBound{𝑨}\AgdaSpace{}%
\AgdaSymbol{:}\AgdaSpace{}%
\AgdaFunction{Algebra}\AgdaSpace{}%
\AgdaBound{𝓧}\AgdaSpace{}%
\AgdaBound{𝑆}\AgdaSymbol{)\{}\AgdaBound{𝑩}\AgdaSpace{}%
\AgdaSymbol{:}\AgdaSpace{}%
\AgdaFunction{Algebra}\AgdaSpace{}%
\AgdaBound{𝓨}\AgdaSpace{}%
\AgdaBound{𝑆}\AgdaSymbol{\}(}\AgdaBound{𝑪}\AgdaSpace{}%
\AgdaSymbol{:}\AgdaSpace{}%
\AgdaFunction{Algebra}\AgdaSpace{}%
\AgdaBound{𝓩}\AgdaSpace{}%
\AgdaBound{𝑆}\AgdaSymbol{)}\<%
\\
\>[1][@{}l@{\AgdaIndent{0}}]%
\>[2]\AgdaSymbol{→}%
\>[11]\AgdaBound{𝑨}\AgdaSpace{}%
\AgdaOperator{\AgdaFunction{≅}}\AgdaSpace{}%
\AgdaBound{𝑩}\AgdaSpace{}%
\AgdaSymbol{→}\AgdaSpace{}%
\AgdaBound{𝑩}\AgdaSpace{}%
\AgdaOperator{\AgdaFunction{≅}}\AgdaSpace{}%
\AgdaBound{𝑪}\<%
\\
%
\>[11]\AgdaComment{-------------}\<%
\\
%
\>[2]\AgdaSymbol{→}%
\>[11]\AgdaBound{𝑨}\AgdaSpace{}%
\AgdaOperator{\AgdaFunction{≅}}\AgdaSpace{}%
\AgdaBound{𝑪}\<%
\\
%
\>[1]\AgdaFunction{Trans-≅}\AgdaSpace{}%
\AgdaBound{𝑨}\AgdaSpace{}%
\AgdaSymbol{\{}\AgdaBound{𝑩}\AgdaSymbol{\}}\AgdaSpace{}%
\AgdaBound{𝑪}\AgdaSpace{}%
\AgdaSymbol{=}\AgdaSpace{}%
\AgdaFunction{trans-≅}\AgdaSpace{}%
\AgdaBound{𝑨}\AgdaSpace{}%
\AgdaBound{𝑩}\AgdaSpace{}%
\AgdaBound{𝑪}\<%
\end{code}

\subsubsection{Lift is an algebraic invariant}\label{lift-is-an-algebraic-invariant}

Fortunately, the lift operation preserves isomorphism (i.e., it's an ``algebraic invariant''), which is why it is workable as a solution to the technical problems that arise from the noncumulativity of the universe hierarchy~(\cite[2.1]{DeMeo:2021-1}).
\ccpad
\begin{code}%
\>[0]\AgdaKeyword{open}\AgdaSpace{}%
\AgdaModule{Lift}\<%
\\
%
\\[\AgdaEmptyExtraSkip]%
\>[0]\AgdaFunction{lift-alg-≅}\AgdaSpace{}%
\AgdaSymbol{:}\AgdaSpace{}%
\AgdaSymbol{\{}\AgdaBound{𝓤}\AgdaSpace{}%
\AgdaBound{𝓦}\AgdaSpace{}%
\AgdaSymbol{:}\AgdaSpace{}%
\AgdaFunction{Universe}\AgdaSymbol{\}\{}\AgdaBound{𝑨}\AgdaSpace{}%
\AgdaSymbol{:}\AgdaSpace{}%
\AgdaFunction{Algebra}\AgdaSpace{}%
\AgdaBound{𝓤}\AgdaSpace{}%
\AgdaBound{𝑆}\AgdaSymbol{\}}\AgdaSpace{}%
\AgdaSymbol{→}\AgdaSpace{}%
\AgdaBound{𝑨}\AgdaSpace{}%
\AgdaOperator{\AgdaFunction{≅}}\AgdaSpace{}%
\AgdaSymbol{(}\AgdaFunction{lift-alg}\AgdaSpace{}%
\AgdaBound{𝑨}\AgdaSpace{}%
\AgdaBound{𝓦}\AgdaSymbol{)}\<%
\\
%
\\[\AgdaEmptyExtraSkip]%
\>[0]\AgdaFunction{lift-alg-≅}\AgdaSpace{}%
\AgdaSymbol{=}\AgdaSpace{}%
\AgdaSymbol{(}\AgdaInductiveConstructor{lift}\AgdaSpace{}%
\AgdaOperator{\AgdaInductiveConstructor{,}}\AgdaSpace{}%
\AgdaSymbol{λ}\AgdaSpace{}%
\AgdaBound{\AgdaUnderscore{}}\AgdaSpace{}%
\AgdaBound{\AgdaUnderscore{}}\AgdaSpace{}%
\AgdaSymbol{→}\AgdaSpace{}%
\AgdaInductiveConstructor{𝓇ℯ𝒻𝓁}\AgdaSymbol{)}\AgdaOperator{\AgdaInductiveConstructor{,}}\AgdaSpace{}%
\AgdaSymbol{(}\AgdaField{lower}\AgdaSpace{}%
\AgdaOperator{\AgdaInductiveConstructor{,}}\AgdaSpace{}%
\AgdaSymbol{λ}\AgdaSpace{}%
\AgdaBound{\AgdaUnderscore{}}\AgdaSpace{}%
\AgdaBound{\AgdaUnderscore{}}\AgdaSpace{}%
\AgdaSymbol{→}\AgdaSpace{}%
\AgdaInductiveConstructor{𝓇ℯ𝒻𝓁}\AgdaSymbol{)}\AgdaOperator{\AgdaInductiveConstructor{,}}\AgdaSpace{}%
\AgdaSymbol{(λ}\AgdaSpace{}%
\AgdaBound{\AgdaUnderscore{}}\AgdaSpace{}%
\AgdaSymbol{→}\AgdaSpace{}%
\AgdaInductiveConstructor{𝓇ℯ𝒻𝓁}\AgdaSymbol{)}\AgdaOperator{\AgdaInductiveConstructor{,}}\AgdaSpace{}%
\AgdaSymbol{(λ}\AgdaSpace{}%
\AgdaBound{\AgdaUnderscore{}}\AgdaSpace{}%
\AgdaSymbol{→}\AgdaSpace{}%
\AgdaInductiveConstructor{𝓇ℯ𝒻𝓁}\AgdaSymbol{)}\<%
\\
%
\\[\AgdaEmptyExtraSkip]%
\>[0]\AgdaFunction{lift-alg-hom}\AgdaSpace{}%
\AgdaSymbol{:}%
\>[741I]\AgdaSymbol{\{}\AgdaBound{𝓧}\AgdaSpace{}%
\AgdaSymbol{:}\AgdaSpace{}%
\AgdaFunction{Universe}\AgdaSymbol{\}\{}\AgdaBound{𝓨}\AgdaSpace{}%
\AgdaSymbol{:}\AgdaSpace{}%
\AgdaFunction{Universe}\AgdaSymbol{\}(}\AgdaBound{𝓩}\AgdaSpace{}%
\AgdaSymbol{:}\AgdaSpace{}%
\AgdaFunction{Universe}\AgdaSymbol{)(}\AgdaBound{𝓦}\AgdaSpace{}%
\AgdaSymbol{:}\AgdaSpace{}%
\AgdaFunction{Universe}\AgdaSymbol{)}\<%
\\
\>[.][@{}l@{}]\<[741I]%
\>[15]\AgdaSymbol{\{}\AgdaBound{𝑨}\AgdaSpace{}%
\AgdaSymbol{:}\AgdaSpace{}%
\AgdaFunction{Algebra}\AgdaSpace{}%
\AgdaBound{𝓧}\AgdaSpace{}%
\AgdaBound{𝑆}\AgdaSymbol{\}(}\AgdaBound{𝑩}\AgdaSpace{}%
\AgdaSymbol{:}\AgdaSpace{}%
\AgdaFunction{Algebra}\AgdaSpace{}%
\AgdaBound{𝓨}\AgdaSpace{}%
\AgdaBound{𝑆}\AgdaSymbol{)}\<%
\\
%
\>[15]\AgdaComment{----------------------------------------------}\<%
\\
\>[0][@{}l@{\AgdaIndent{0}}]%
\>[1]\AgdaSymbol{→}%
\>[15]\AgdaFunction{hom}\AgdaSpace{}%
\AgdaBound{𝑨}\AgdaSpace{}%
\AgdaBound{𝑩}%
\>[24]\AgdaSymbol{→}%
\>[27]\AgdaFunction{hom}\AgdaSpace{}%
\AgdaSymbol{(}\AgdaFunction{lift-alg}\AgdaSpace{}%
\AgdaBound{𝑨}\AgdaSpace{}%
\AgdaBound{𝓩}\AgdaSymbol{)}\AgdaSpace{}%
\AgdaSymbol{(}\AgdaFunction{lift-alg}\AgdaSpace{}%
\AgdaBound{𝑩}\AgdaSpace{}%
\AgdaBound{𝓦}\AgdaSymbol{)}\<%
\\
%
\\[\AgdaEmptyExtraSkip]%
\>[0]\AgdaFunction{lift-alg-hom}\AgdaSpace{}%
\AgdaBound{𝓩}\AgdaSpace{}%
\AgdaBound{𝓦}\AgdaSpace{}%
\AgdaSymbol{\{}\AgdaBound{𝑨}\AgdaSymbol{\}}\AgdaSpace{}%
\AgdaBound{𝑩}\AgdaSpace{}%
\AgdaSymbol{(}\AgdaBound{f}\AgdaSpace{}%
\AgdaOperator{\AgdaInductiveConstructor{,}}\AgdaSpace{}%
\AgdaBound{fhom}\AgdaSymbol{)}\AgdaSpace{}%
\AgdaSymbol{=}\AgdaSpace{}%
\AgdaInductiveConstructor{lift}\AgdaSpace{}%
\AgdaOperator{\AgdaFunction{∘}}\AgdaSpace{}%
\AgdaBound{f}\AgdaSpace{}%
\AgdaOperator{\AgdaFunction{∘}}\AgdaSpace{}%
\AgdaField{lower}\AgdaSpace{}%
\AgdaOperator{\AgdaInductiveConstructor{,}}\AgdaSpace{}%
\AgdaFunction{γ}\<%
\\
\>[0][@{}l@{\AgdaIndent{0}}]%
\>[1]\AgdaKeyword{where}\<%
\\
\>[1][@{}l@{\AgdaIndent{0}}]%
\>[2]\AgdaFunction{lh}\AgdaSpace{}%
\AgdaSymbol{:}\AgdaSpace{}%
\AgdaFunction{is-homomorphism}\AgdaSpace{}%
\AgdaSymbol{(}\AgdaFunction{lift-alg}\AgdaSpace{}%
\AgdaBound{𝑨}\AgdaSpace{}%
\AgdaBound{𝓩}\AgdaSymbol{)}\AgdaSpace{}%
\AgdaBound{𝑨}\AgdaSpace{}%
\AgdaField{lower}\<%
\\
%
\>[2]\AgdaFunction{lh}\AgdaSpace{}%
\AgdaSymbol{=}\AgdaSpace{}%
\AgdaSymbol{λ}\AgdaSpace{}%
\AgdaBound{\AgdaUnderscore{}}\AgdaSpace{}%
\AgdaBound{\AgdaUnderscore{}}\AgdaSpace{}%
\AgdaSymbol{→}\AgdaSpace{}%
\AgdaInductiveConstructor{𝓇ℯ𝒻𝓁}\<%
\\
%
\\[\AgdaEmptyExtraSkip]%
%
\>[2]\AgdaFunction{lABh}\AgdaSpace{}%
\AgdaSymbol{:}\AgdaSpace{}%
\AgdaFunction{is-homomorphism}\AgdaSpace{}%
\AgdaSymbol{(}\AgdaFunction{lift-alg}\AgdaSpace{}%
\AgdaBound{𝑨}\AgdaSpace{}%
\AgdaBound{𝓩}\AgdaSymbol{)}\AgdaSpace{}%
\AgdaBound{𝑩}\AgdaSpace{}%
\AgdaSymbol{(}\AgdaBound{f}\AgdaSpace{}%
\AgdaOperator{\AgdaFunction{∘}}\AgdaSpace{}%
\AgdaField{lower}\AgdaSymbol{)}\<%
\\
%
\>[2]\AgdaFunction{lABh}\AgdaSpace{}%
\AgdaSymbol{=}\AgdaSpace{}%
\AgdaFunction{∘-hom}\AgdaSpace{}%
\AgdaSymbol{(}\AgdaFunction{lift-alg}\AgdaSpace{}%
\AgdaBound{𝑨}\AgdaSpace{}%
\AgdaBound{𝓩}\AgdaSymbol{)}\AgdaSpace{}%
\AgdaBound{𝑨}\AgdaSpace{}%
\AgdaBound{𝑩}\AgdaSpace{}%
\AgdaSymbol{\{}\AgdaField{lower}\AgdaSymbol{\}\{}\AgdaBound{f}\AgdaSymbol{\}}\AgdaSpace{}%
\AgdaFunction{lh}\AgdaSpace{}%
\AgdaBound{fhom}\<%
\\
%
\\[\AgdaEmptyExtraSkip]%
%
\>[2]\AgdaFunction{Lh}\AgdaSpace{}%
\AgdaSymbol{:}\AgdaSpace{}%
\AgdaFunction{is-homomorphism}\AgdaSpace{}%
\AgdaBound{𝑩}\AgdaSpace{}%
\AgdaSymbol{(}\AgdaFunction{lift-alg}\AgdaSpace{}%
\AgdaBound{𝑩}\AgdaSpace{}%
\AgdaBound{𝓦}\AgdaSymbol{)}\AgdaSpace{}%
\AgdaInductiveConstructor{lift}\<%
\\
%
\>[2]\AgdaFunction{Lh}\AgdaSpace{}%
\AgdaSymbol{=}\AgdaSpace{}%
\AgdaSymbol{λ}\AgdaSpace{}%
\AgdaBound{\AgdaUnderscore{}}\AgdaSpace{}%
\AgdaBound{\AgdaUnderscore{}}\AgdaSpace{}%
\AgdaSymbol{→}\AgdaSpace{}%
\AgdaInductiveConstructor{𝓇ℯ𝒻𝓁}\<%
\\
%
\\[\AgdaEmptyExtraSkip]%
%
\>[2]\AgdaFunction{γ}\AgdaSpace{}%
\AgdaSymbol{:}\AgdaSpace{}%
\AgdaFunction{is-homomorphism}\AgdaSymbol{(}\AgdaFunction{lift-alg}\AgdaSpace{}%
\AgdaBound{𝑨}\AgdaSpace{}%
\AgdaBound{𝓩}\AgdaSymbol{)(}\AgdaFunction{lift-alg}\AgdaSpace{}%
\AgdaBound{𝑩}\AgdaSpace{}%
\AgdaBound{𝓦}\AgdaSymbol{)}\AgdaSpace{}%
\AgdaSymbol{(}\AgdaInductiveConstructor{lift}\AgdaSpace{}%
\AgdaOperator{\AgdaFunction{∘}}\AgdaSpace{}%
\AgdaSymbol{(}\AgdaBound{f}\AgdaSpace{}%
\AgdaOperator{\AgdaFunction{∘}}\AgdaSpace{}%
\AgdaField{lower}\AgdaSymbol{))}\<%
\\
%
\>[2]\AgdaFunction{γ}\AgdaSpace{}%
\AgdaSymbol{=}\AgdaSpace{}%
\AgdaFunction{∘-hom}\AgdaSpace{}%
\AgdaSymbol{(}\AgdaFunction{lift-alg}\AgdaSpace{}%
\AgdaBound{𝑨}\AgdaSpace{}%
\AgdaBound{𝓩}\AgdaSymbol{)}\AgdaSpace{}%
\AgdaBound{𝑩}\AgdaSpace{}%
\AgdaSymbol{(}\AgdaFunction{lift-alg}\AgdaSpace{}%
\AgdaBound{𝑩}\AgdaSpace{}%
\AgdaBound{𝓦}\AgdaSymbol{)\{}\AgdaBound{f}\AgdaSpace{}%
\AgdaOperator{\AgdaFunction{∘}}\AgdaSpace{}%
\AgdaField{lower}\AgdaSymbol{\}\{}\AgdaInductiveConstructor{lift}\AgdaSymbol{\}}\AgdaSpace{}%
\AgdaFunction{lABh}\AgdaSpace{}%
\AgdaFunction{Lh}\<%
\\
%
\\[\AgdaEmptyExtraSkip]%
%
\\[\AgdaEmptyExtraSkip]%
\>[0]\AgdaFunction{lift-alg-iso}\AgdaSpace{}%
\AgdaSymbol{:}%
\>[851I]\AgdaSymbol{\{}\AgdaBound{𝓧}\AgdaSpace{}%
\AgdaSymbol{:}\AgdaSpace{}%
\AgdaFunction{Universe}\AgdaSymbol{\}\{}\AgdaBound{𝓨}\AgdaSpace{}%
\AgdaSymbol{:}\AgdaSpace{}%
\AgdaFunction{Universe}\AgdaSymbol{\}\{}\AgdaBound{𝓩}\AgdaSpace{}%
\AgdaSymbol{:}\AgdaSpace{}%
\AgdaFunction{Universe}\AgdaSymbol{\}\{}\AgdaBound{𝓦}\AgdaSpace{}%
\AgdaSymbol{:}\AgdaSpace{}%
\AgdaFunction{Universe}\AgdaSymbol{\}}\<%
\\
\>[.][@{}l@{}]\<[851I]%
\>[15]\AgdaSymbol{\{}\AgdaBound{𝑨}\AgdaSpace{}%
\AgdaSymbol{:}\AgdaSpace{}%
\AgdaFunction{Algebra}\AgdaSpace{}%
\AgdaBound{𝓧}\AgdaSpace{}%
\AgdaBound{𝑆}\AgdaSymbol{\}\{}\AgdaBound{𝑩}\AgdaSpace{}%
\AgdaSymbol{:}\AgdaSpace{}%
\AgdaFunction{Algebra}\AgdaSpace{}%
\AgdaBound{𝓨}\AgdaSpace{}%
\AgdaBound{𝑆}\AgdaSymbol{\}}\<%
\\
%
\>[15]\AgdaComment{-----------------------------------------}\<%
\\
\>[0][@{}l@{\AgdaIndent{0}}]%
\>[1]\AgdaSymbol{→}%
\>[15]\AgdaBound{𝑨}\AgdaSpace{}%
\AgdaOperator{\AgdaFunction{≅}}\AgdaSpace{}%
\AgdaBound{𝑩}\AgdaSpace{}%
\AgdaSymbol{→}\AgdaSpace{}%
\AgdaSymbol{(}\AgdaFunction{lift-alg}\AgdaSpace{}%
\AgdaBound{𝑨}\AgdaSpace{}%
\AgdaBound{𝓩}\AgdaSymbol{)}\AgdaSpace{}%
\AgdaOperator{\AgdaFunction{≅}}\AgdaSpace{}%
\AgdaSymbol{(}\AgdaFunction{lift-alg}\AgdaSpace{}%
\AgdaBound{𝑩}\AgdaSpace{}%
\AgdaBound{𝓦}\AgdaSymbol{)}\<%
\\
%
\\[\AgdaEmptyExtraSkip]%
\>[0]\AgdaFunction{lift-alg-iso}\AgdaSpace{}%
\AgdaBound{A≅B}\AgdaSpace{}%
\AgdaSymbol{=}\AgdaSpace{}%
\AgdaFunction{TRANS-≅}\AgdaSpace{}%
\AgdaSymbol{(}\AgdaFunction{TRANS-≅}\AgdaSpace{}%
\AgdaSymbol{(}\AgdaFunction{sym-≅}\AgdaSpace{}%
\AgdaFunction{lift-alg-≅}\AgdaSymbol{)}\AgdaSpace{}%
\AgdaBound{A≅B}\AgdaSymbol{)}\AgdaSpace{}%
\AgdaFunction{lift-alg-≅}\<%
\end{code}

\subsubsection{Lift associativity}\label{lift-associativity}

The lift is also associative, up to isomorphism at least.
\ccpad
\begin{code}%
\>[0]\AgdaFunction{lift-alg-assoc}\AgdaSpace{}%
\AgdaSymbol{:}\AgdaSpace{}%
\AgdaSymbol{\{}\AgdaBound{𝓤}\AgdaSpace{}%
\AgdaBound{𝓦}\AgdaSpace{}%
\AgdaBound{𝓘}\AgdaSpace{}%
\AgdaSymbol{:}\AgdaSpace{}%
\AgdaFunction{Universe}\AgdaSymbol{\}\{}\AgdaBound{𝑨}\AgdaSpace{}%
\AgdaSymbol{:}\AgdaSpace{}%
\AgdaFunction{Algebra}\AgdaSpace{}%
\AgdaBound{𝓤}\AgdaSpace{}%
\AgdaBound{𝑆}\AgdaSymbol{\}}\<%
\\
\>[0][@{}l@{\AgdaIndent{0}}]%
\>[1]\AgdaSymbol{→}%
\>[17]\AgdaFunction{lift-alg}\AgdaSpace{}%
\AgdaBound{𝑨}\AgdaSpace{}%
\AgdaSymbol{(}\AgdaBound{𝓦}\AgdaSpace{}%
\AgdaOperator{\AgdaFunction{⊔}}\AgdaSpace{}%
\AgdaBound{𝓘}\AgdaSymbol{)}\AgdaSpace{}%
\AgdaOperator{\AgdaFunction{≅}}\AgdaSpace{}%
\AgdaSymbol{(}\AgdaFunction{lift-alg}\AgdaSpace{}%
\AgdaSymbol{(}\AgdaFunction{lift-alg}\AgdaSpace{}%
\AgdaBound{𝑨}\AgdaSpace{}%
\AgdaBound{𝓦}\AgdaSymbol{)}\AgdaSpace{}%
\AgdaBound{𝓘}\AgdaSymbol{)}\<%
\\
%
\\[\AgdaEmptyExtraSkip]%
\>[0]\AgdaFunction{lift-alg-assoc}\AgdaSpace{}%
\AgdaSymbol{\{}\AgdaBound{𝓤}\AgdaSymbol{\}}\AgdaSpace{}%
\AgdaSymbol{\{}\AgdaBound{𝓦}\AgdaSymbol{\}}\AgdaSpace{}%
\AgdaSymbol{\{}\AgdaBound{𝓘}\AgdaSymbol{\}}\AgdaSpace{}%
\AgdaSymbol{\{}\AgdaBound{𝑨}\AgdaSymbol{\}}\AgdaSpace{}%
\AgdaSymbol{=}\AgdaSpace{}%
\AgdaFunction{TRANS-≅}\AgdaSpace{}%
\AgdaSymbol{(}\AgdaFunction{TRANS-≅}\AgdaSpace{}%
\AgdaFunction{ζ}\AgdaSpace{}%
\AgdaFunction{lift-alg-≅}\AgdaSymbol{)}\AgdaSpace{}%
\AgdaFunction{lift-alg-≅}\<%
\\
\>[0][@{}l@{\AgdaIndent{0}}]%
\>[1]\AgdaKeyword{where}\<%
\\
\>[1][@{}l@{\AgdaIndent{0}}]%
\>[2]\AgdaFunction{ζ}\AgdaSpace{}%
\AgdaSymbol{:}\AgdaSpace{}%
\AgdaFunction{lift-alg}\AgdaSpace{}%
\AgdaBound{𝑨}\AgdaSpace{}%
\AgdaSymbol{(}\AgdaBound{𝓦}\AgdaSpace{}%
\AgdaOperator{\AgdaFunction{⊔}}\AgdaSpace{}%
\AgdaBound{𝓘}\AgdaSymbol{)}\AgdaSpace{}%
\AgdaOperator{\AgdaFunction{≅}}\AgdaSpace{}%
\AgdaBound{𝑨}\<%
\\
%
\>[2]\AgdaFunction{ζ}\AgdaSpace{}%
\AgdaSymbol{=}\AgdaSpace{}%
\AgdaFunction{sym-≅}\AgdaSpace{}%
\AgdaFunction{lift-alg-≅}\<%
\\
%
\\[\AgdaEmptyExtraSkip]%
\>[0]\AgdaFunction{lift-alg-associative}\AgdaSpace{}%
\AgdaSymbol{:}\AgdaSpace{}%
\AgdaSymbol{\{}\AgdaBound{𝓤}\AgdaSpace{}%
\AgdaBound{𝓦}\AgdaSpace{}%
\AgdaBound{𝓘}\AgdaSpace{}%
\AgdaSymbol{:}\AgdaSpace{}%
\AgdaFunction{Universe}\AgdaSymbol{\}(}\AgdaBound{𝑨}\AgdaSpace{}%
\AgdaSymbol{:}\AgdaSpace{}%
\AgdaFunction{Algebra}\AgdaSpace{}%
\AgdaBound{𝓤}\AgdaSpace{}%
\AgdaBound{𝑆}\AgdaSymbol{)}\<%
\\
\>[0][@{}l@{\AgdaIndent{0}}]%
\>[1]\AgdaSymbol{→}%
\>[13]\AgdaFunction{lift-alg}\AgdaSpace{}%
\AgdaBound{𝑨}\AgdaSpace{}%
\AgdaSymbol{(}\AgdaBound{𝓦}\AgdaSpace{}%
\AgdaOperator{\AgdaFunction{⊔}}\AgdaSpace{}%
\AgdaBound{𝓘}\AgdaSymbol{)}\AgdaSpace{}%
\AgdaOperator{\AgdaFunction{≅}}\AgdaSpace{}%
\AgdaSymbol{(}\AgdaFunction{lift-alg}\AgdaSpace{}%
\AgdaSymbol{(}\AgdaFunction{lift-alg}\AgdaSpace{}%
\AgdaBound{𝑨}\AgdaSpace{}%
\AgdaBound{𝓦}\AgdaSymbol{)}\AgdaSpace{}%
\AgdaBound{𝓘}\AgdaSymbol{)}\<%
\\
%
\\[\AgdaEmptyExtraSkip]%
\>[0]\AgdaFunction{lift-alg-associative}\AgdaSymbol{\{}\AgdaBound{𝓤}\AgdaSymbol{\}\{}\AgdaBound{𝓦}\AgdaSymbol{\}\{}\AgdaBound{𝓘}\AgdaSymbol{\}}\AgdaSpace{}%
\AgdaBound{𝑨}\AgdaSpace{}%
\AgdaSymbol{=}\AgdaSpace{}%
\AgdaFunction{lift-alg-assoc}\AgdaSymbol{\{}\AgdaBound{𝓤}\AgdaSymbol{\}\{}\AgdaBound{𝓦}\AgdaSymbol{\}\{}\AgdaBound{𝓘}\AgdaSymbol{\}\{}\AgdaBound{𝑨}\AgdaSymbol{\}}\<%
\end{code}

\subsubsection{Products preserve isomorphisms}\label{products-preserve-isomorphisms}

\begin{code}%
\>[0]\AgdaFunction{⨅≅}\AgdaSpace{}%
\AgdaSymbol{:}%
\>[951I]\AgdaFunction{global-dfunext}\AgdaSpace{}%
\AgdaSymbol{→}\AgdaSpace{}%
\AgdaSymbol{\{}\AgdaBound{𝓠}\AgdaSpace{}%
\AgdaBound{𝓤}\AgdaSpace{}%
\AgdaBound{𝓘}\AgdaSpace{}%
\AgdaSymbol{:}\AgdaSpace{}%
\AgdaFunction{Universe}\AgdaSymbol{\}}\<%
\\
\>[.][@{}l@{}]\<[951I]%
\>[5]\AgdaSymbol{\{}\AgdaBound{I}\AgdaSpace{}%
\AgdaSymbol{:}\AgdaSpace{}%
\AgdaBound{𝓘}\AgdaSpace{}%
\AgdaOperator{\AgdaFunction{̇}}\AgdaSymbol{\}\{}\AgdaBound{𝒜}\AgdaSpace{}%
\AgdaSymbol{:}\AgdaSpace{}%
\AgdaBound{I}\AgdaSpace{}%
\AgdaSymbol{→}\AgdaSpace{}%
\AgdaFunction{Algebra}\AgdaSpace{}%
\AgdaBound{𝓠}\AgdaSpace{}%
\AgdaBound{𝑆}\AgdaSymbol{\}\{}\AgdaBound{ℬ}\AgdaSpace{}%
\AgdaSymbol{:}\AgdaSpace{}%
\AgdaBound{I}\AgdaSpace{}%
\AgdaSymbol{→}\AgdaSpace{}%
\AgdaFunction{Algebra}\AgdaSpace{}%
\AgdaBound{𝓤}\AgdaSpace{}%
\AgdaBound{𝑆}\AgdaSymbol{\}}\<%
\\
\>[0][@{}l@{\AgdaIndent{0}}]%
\>[1]\AgdaSymbol{→}%
\>[5]\AgdaSymbol{(∀}\AgdaSpace{}%
\AgdaBound{i}\AgdaSpace{}%
\AgdaSymbol{→}\AgdaSpace{}%
\AgdaSymbol{(}\AgdaBound{𝒜}\AgdaSpace{}%
\AgdaBound{i}\AgdaSymbol{)}\AgdaSpace{}%
\AgdaOperator{\AgdaFunction{≅}}\AgdaSpace{}%
\AgdaSymbol{(}\AgdaBound{ℬ}\AgdaSpace{}%
\AgdaBound{i}\AgdaSymbol{))}\<%
\\
%
\>[5]\AgdaComment{----------------------}\<%
\\
%
\>[1]\AgdaSymbol{→}%
\>[5]\AgdaFunction{⨅}\AgdaSpace{}%
\AgdaBound{𝒜}\AgdaSpace{}%
\AgdaOperator{\AgdaFunction{≅}}\AgdaSpace{}%
\AgdaFunction{⨅}\AgdaSpace{}%
\AgdaBound{ℬ}\<%
\\
%
\\[\AgdaEmptyExtraSkip]%
\>[0]\AgdaFunction{⨅≅}\AgdaSpace{}%
\AgdaBound{gfe}\AgdaSpace{}%
\AgdaSymbol{\{}\AgdaBound{𝓠}\AgdaSymbol{\}\{}\AgdaBound{𝓤}\AgdaSymbol{\}\{}\AgdaBound{𝓘}\AgdaSymbol{\}\{}\AgdaBound{I}\AgdaSymbol{\}\{}\AgdaBound{𝒜}\AgdaSymbol{\}\{}\AgdaBound{ℬ}\AgdaSymbol{\}}\AgdaSpace{}%
\AgdaBound{AB}\AgdaSpace{}%
\AgdaSymbol{=}\AgdaSpace{}%
\AgdaFunction{γ}\<%
\\
\>[0][@{}l@{\AgdaIndent{0}}]%
\>[1]\AgdaKeyword{where}\<%
\\
\>[1][@{}l@{\AgdaIndent{0}}]%
\>[2]\AgdaFunction{F}\AgdaSpace{}%
\AgdaSymbol{:}\AgdaSpace{}%
\AgdaSymbol{∀}\AgdaSpace{}%
\AgdaBound{i}\AgdaSpace{}%
\AgdaSymbol{→}\AgdaSpace{}%
\AgdaOperator{\AgdaFunction{∣}}\AgdaSpace{}%
\AgdaBound{𝒜}\AgdaSpace{}%
\AgdaBound{i}\AgdaSpace{}%
\AgdaOperator{\AgdaFunction{∣}}\AgdaSpace{}%
\AgdaSymbol{→}\AgdaSpace{}%
\AgdaOperator{\AgdaFunction{∣}}\AgdaSpace{}%
\AgdaBound{ℬ}\AgdaSpace{}%
\AgdaBound{i}\AgdaSpace{}%
\AgdaOperator{\AgdaFunction{∣}}\<%
\\
%
\>[2]\AgdaFunction{F}\AgdaSpace{}%
\AgdaBound{i}\AgdaSpace{}%
\AgdaSymbol{=}\AgdaSpace{}%
\AgdaOperator{\AgdaFunction{∣}}\AgdaSpace{}%
\AgdaFunction{fst}\AgdaSpace{}%
\AgdaSymbol{(}\AgdaBound{AB}\AgdaSpace{}%
\AgdaBound{i}\AgdaSymbol{)}\AgdaSpace{}%
\AgdaOperator{\AgdaFunction{∣}}\<%
\\
%
\>[2]\AgdaFunction{Fhom}\AgdaSpace{}%
\AgdaSymbol{:}\AgdaSpace{}%
\AgdaSymbol{∀}\AgdaSpace{}%
\AgdaBound{i}\AgdaSpace{}%
\AgdaSymbol{→}\AgdaSpace{}%
\AgdaFunction{is-homomorphism}\AgdaSpace{}%
\AgdaSymbol{(}\AgdaBound{𝒜}\AgdaSpace{}%
\AgdaBound{i}\AgdaSymbol{)}\AgdaSpace{}%
\AgdaSymbol{(}\AgdaBound{ℬ}\AgdaSpace{}%
\AgdaBound{i}\AgdaSymbol{)}\AgdaSpace{}%
\AgdaSymbol{(}\AgdaFunction{F}\AgdaSpace{}%
\AgdaBound{i}\AgdaSymbol{)}\<%
\\
%
\>[2]\AgdaFunction{Fhom}\AgdaSpace{}%
\AgdaBound{i}\AgdaSpace{}%
\AgdaSymbol{=}\AgdaSpace{}%
\AgdaOperator{\AgdaFunction{∥}}\AgdaSpace{}%
\AgdaFunction{fst}\AgdaSpace{}%
\AgdaSymbol{(}\AgdaBound{AB}\AgdaSpace{}%
\AgdaBound{i}\AgdaSymbol{)}\AgdaSpace{}%
\AgdaOperator{\AgdaFunction{∥}}\<%
\\
%
\\[\AgdaEmptyExtraSkip]%
%
\>[2]\AgdaFunction{G}\AgdaSpace{}%
\AgdaSymbol{:}\AgdaSpace{}%
\AgdaSymbol{∀}\AgdaSpace{}%
\AgdaBound{i}\AgdaSpace{}%
\AgdaSymbol{→}\AgdaSpace{}%
\AgdaOperator{\AgdaFunction{∣}}\AgdaSpace{}%
\AgdaBound{ℬ}\AgdaSpace{}%
\AgdaBound{i}\AgdaSpace{}%
\AgdaOperator{\AgdaFunction{∣}}\AgdaSpace{}%
\AgdaSymbol{→}\AgdaSpace{}%
\AgdaOperator{\AgdaFunction{∣}}\AgdaSpace{}%
\AgdaBound{𝒜}\AgdaSpace{}%
\AgdaBound{i}\AgdaSpace{}%
\AgdaOperator{\AgdaFunction{∣}}\<%
\\
%
\>[2]\AgdaFunction{G}\AgdaSpace{}%
\AgdaBound{i}\AgdaSpace{}%
\AgdaSymbol{=}\AgdaSpace{}%
\AgdaFunction{fst}\AgdaSpace{}%
\AgdaOperator{\AgdaFunction{∣}}\AgdaSpace{}%
\AgdaFunction{snd}\AgdaSpace{}%
\AgdaSymbol{(}\AgdaBound{AB}\AgdaSpace{}%
\AgdaBound{i}\AgdaSymbol{)}\AgdaSpace{}%
\AgdaOperator{\AgdaFunction{∣}}\<%
\\
%
\>[2]\AgdaFunction{Ghom}\AgdaSpace{}%
\AgdaSymbol{:}\AgdaSpace{}%
\AgdaSymbol{∀}\AgdaSpace{}%
\AgdaBound{i}\AgdaSpace{}%
\AgdaSymbol{→}\AgdaSpace{}%
\AgdaFunction{is-homomorphism}\AgdaSpace{}%
\AgdaSymbol{(}\AgdaBound{ℬ}\AgdaSpace{}%
\AgdaBound{i}\AgdaSymbol{)}\AgdaSpace{}%
\AgdaSymbol{(}\AgdaBound{𝒜}\AgdaSpace{}%
\AgdaBound{i}\AgdaSymbol{)}\AgdaSpace{}%
\AgdaSymbol{(}\AgdaFunction{G}\AgdaSpace{}%
\AgdaBound{i}\AgdaSymbol{)}\<%
\\
%
\>[2]\AgdaFunction{Ghom}\AgdaSpace{}%
\AgdaBound{i}\AgdaSpace{}%
\AgdaSymbol{=}\AgdaSpace{}%
\AgdaFunction{snd}\AgdaSpace{}%
\AgdaOperator{\AgdaFunction{∣}}\AgdaSpace{}%
\AgdaFunction{snd}\AgdaSpace{}%
\AgdaSymbol{(}\AgdaBound{AB}\AgdaSpace{}%
\AgdaBound{i}\AgdaSymbol{)}\AgdaSpace{}%
\AgdaOperator{\AgdaFunction{∣}}\<%
\\
%
\\[\AgdaEmptyExtraSkip]%
%
\>[2]\AgdaFunction{F∼G}\AgdaSpace{}%
\AgdaSymbol{:}\AgdaSpace{}%
\AgdaSymbol{∀}\AgdaSpace{}%
\AgdaBound{i}\AgdaSpace{}%
\AgdaSymbol{→}\AgdaSpace{}%
\AgdaSymbol{(}\AgdaFunction{F}\AgdaSpace{}%
\AgdaBound{i}\AgdaSymbol{)}\AgdaSpace{}%
\AgdaOperator{\AgdaFunction{∘}}\AgdaSpace{}%
\AgdaSymbol{(}\AgdaFunction{G}\AgdaSpace{}%
\AgdaBound{i}\AgdaSymbol{)}\AgdaSpace{}%
\AgdaOperator{\AgdaFunction{∼}}\AgdaSpace{}%
\AgdaSymbol{(}\AgdaOperator{\AgdaFunction{∣}}\AgdaSpace{}%
\AgdaFunction{𝒾𝒹}\AgdaSpace{}%
\AgdaSymbol{(}\AgdaBound{ℬ}\AgdaSpace{}%
\AgdaBound{i}\AgdaSymbol{)}\AgdaSpace{}%
\AgdaOperator{\AgdaFunction{∣}}\AgdaSymbol{)}\<%
\\
%
\>[2]\AgdaFunction{F∼G}\AgdaSpace{}%
\AgdaBound{i}\AgdaSpace{}%
\AgdaSymbol{=}\AgdaSpace{}%
\AgdaFunction{fst}\AgdaSpace{}%
\AgdaOperator{\AgdaFunction{∥}}\AgdaSpace{}%
\AgdaFunction{snd}\AgdaSpace{}%
\AgdaSymbol{(}\AgdaBound{AB}\AgdaSpace{}%
\AgdaBound{i}\AgdaSymbol{)}\AgdaSpace{}%
\AgdaOperator{\AgdaFunction{∥}}\<%
\\
%
\\[\AgdaEmptyExtraSkip]%
%
\>[2]\AgdaFunction{G∼F}\AgdaSpace{}%
\AgdaSymbol{:}\AgdaSpace{}%
\AgdaSymbol{∀}\AgdaSpace{}%
\AgdaBound{i}\AgdaSpace{}%
\AgdaSymbol{→}\AgdaSpace{}%
\AgdaSymbol{(}\AgdaFunction{G}\AgdaSpace{}%
\AgdaBound{i}\AgdaSymbol{)}\AgdaSpace{}%
\AgdaOperator{\AgdaFunction{∘}}\AgdaSpace{}%
\AgdaSymbol{(}\AgdaFunction{F}\AgdaSpace{}%
\AgdaBound{i}\AgdaSymbol{)}\AgdaSpace{}%
\AgdaOperator{\AgdaFunction{∼}}\AgdaSpace{}%
\AgdaSymbol{(}\AgdaOperator{\AgdaFunction{∣}}\AgdaSpace{}%
\AgdaFunction{𝒾𝒹}\AgdaSpace{}%
\AgdaSymbol{(}\AgdaBound{𝒜}\AgdaSpace{}%
\AgdaBound{i}\AgdaSymbol{)}\AgdaSpace{}%
\AgdaOperator{\AgdaFunction{∣}}\AgdaSymbol{)}\<%
\\
%
\>[2]\AgdaFunction{G∼F}\AgdaSpace{}%
\AgdaBound{i}\AgdaSpace{}%
\AgdaSymbol{=}\AgdaSpace{}%
\AgdaFunction{snd}\AgdaSpace{}%
\AgdaOperator{\AgdaFunction{∥}}\AgdaSpace{}%
\AgdaFunction{snd}\AgdaSpace{}%
\AgdaSymbol{(}\AgdaBound{AB}\AgdaSpace{}%
\AgdaBound{i}\AgdaSymbol{)}\AgdaSpace{}%
\AgdaOperator{\AgdaFunction{∥}}\<%
\\
%
\\[\AgdaEmptyExtraSkip]%
%
\>[2]\AgdaFunction{ϕ}\AgdaSpace{}%
\AgdaSymbol{:}\AgdaSpace{}%
\AgdaOperator{\AgdaFunction{∣}}\AgdaSpace{}%
\AgdaFunction{⨅}\AgdaSpace{}%
\AgdaBound{𝒜}\AgdaSpace{}%
\AgdaOperator{\AgdaFunction{∣}}\AgdaSpace{}%
\AgdaSymbol{→}\AgdaSpace{}%
\AgdaOperator{\AgdaFunction{∣}}\AgdaSpace{}%
\AgdaFunction{⨅}\AgdaSpace{}%
\AgdaBound{ℬ}\AgdaSpace{}%
\AgdaOperator{\AgdaFunction{∣}}\<%
\\
%
\>[2]\AgdaFunction{ϕ}\AgdaSpace{}%
\AgdaBound{a}\AgdaSpace{}%
\AgdaBound{i}\AgdaSpace{}%
\AgdaSymbol{=}\AgdaSpace{}%
\AgdaFunction{F}\AgdaSpace{}%
\AgdaBound{i}\AgdaSpace{}%
\AgdaSymbol{(}\AgdaBound{a}\AgdaSpace{}%
\AgdaBound{i}\AgdaSymbol{)}\<%
\\
%
\\[\AgdaEmptyExtraSkip]%
%
\>[2]\AgdaFunction{ϕhom}\AgdaSpace{}%
\AgdaSymbol{:}\AgdaSpace{}%
\AgdaFunction{is-homomorphism}\AgdaSpace{}%
\AgdaSymbol{(}\AgdaFunction{⨅}\AgdaSpace{}%
\AgdaBound{𝒜}\AgdaSymbol{)}\AgdaSpace{}%
\AgdaSymbol{(}\AgdaFunction{⨅}\AgdaSpace{}%
\AgdaBound{ℬ}\AgdaSymbol{)}\AgdaSpace{}%
\AgdaFunction{ϕ}\<%
\\
%
\>[2]\AgdaFunction{ϕhom}\AgdaSpace{}%
\AgdaBound{𝑓}\AgdaSpace{}%
\AgdaBound{𝒂}\AgdaSpace{}%
\AgdaSymbol{=}\AgdaSpace{}%
\AgdaBound{gfe}\AgdaSpace{}%
\AgdaSymbol{(λ}\AgdaSpace{}%
\AgdaBound{i}\AgdaSpace{}%
\AgdaSymbol{→}\AgdaSpace{}%
\AgdaSymbol{(}\AgdaFunction{Fhom}\AgdaSpace{}%
\AgdaBound{i}\AgdaSymbol{)}\AgdaSpace{}%
\AgdaBound{𝑓}\AgdaSpace{}%
\AgdaSymbol{(λ}\AgdaSpace{}%
\AgdaBound{x}\AgdaSpace{}%
\AgdaSymbol{→}\AgdaSpace{}%
\AgdaBound{𝒂}\AgdaSpace{}%
\AgdaBound{x}\AgdaSpace{}%
\AgdaBound{i}\AgdaSymbol{))}\<%
\\
%
\\[\AgdaEmptyExtraSkip]%
%
\>[2]\AgdaFunction{ψ}\AgdaSpace{}%
\AgdaSymbol{:}\AgdaSpace{}%
\AgdaOperator{\AgdaFunction{∣}}\AgdaSpace{}%
\AgdaFunction{⨅}\AgdaSpace{}%
\AgdaBound{ℬ}\AgdaSpace{}%
\AgdaOperator{\AgdaFunction{∣}}\AgdaSpace{}%
\AgdaSymbol{→}\AgdaSpace{}%
\AgdaOperator{\AgdaFunction{∣}}\AgdaSpace{}%
\AgdaFunction{⨅}\AgdaSpace{}%
\AgdaBound{𝒜}\AgdaSpace{}%
\AgdaOperator{\AgdaFunction{∣}}\<%
\\
%
\>[2]\AgdaFunction{ψ}\AgdaSpace{}%
\AgdaBound{b}\AgdaSpace{}%
\AgdaBound{i}\AgdaSpace{}%
\AgdaSymbol{=}\AgdaSpace{}%
\AgdaOperator{\AgdaFunction{∣}}\AgdaSpace{}%
\AgdaFunction{fst}\AgdaSpace{}%
\AgdaOperator{\AgdaFunction{∥}}\AgdaSpace{}%
\AgdaBound{AB}\AgdaSpace{}%
\AgdaBound{i}\AgdaSpace{}%
\AgdaOperator{\AgdaFunction{∥}}\AgdaSpace{}%
\AgdaOperator{\AgdaFunction{∣}}\AgdaSpace{}%
\AgdaSymbol{(}\AgdaBound{b}\AgdaSpace{}%
\AgdaBound{i}\AgdaSymbol{)}\<%
\\
%
\\[\AgdaEmptyExtraSkip]%
%
\>[2]\AgdaFunction{ψhom}\AgdaSpace{}%
\AgdaSymbol{:}\AgdaSpace{}%
\AgdaFunction{is-homomorphism}\AgdaSpace{}%
\AgdaSymbol{(}\AgdaFunction{⨅}\AgdaSpace{}%
\AgdaBound{ℬ}\AgdaSymbol{)}\AgdaSpace{}%
\AgdaSymbol{(}\AgdaFunction{⨅}\AgdaSpace{}%
\AgdaBound{𝒜}\AgdaSymbol{)}\AgdaSpace{}%
\AgdaFunction{ψ}\<%
\\
%
\>[2]\AgdaFunction{ψhom}\AgdaSpace{}%
\AgdaBound{𝑓}\AgdaSpace{}%
\AgdaBound{𝒃}\AgdaSpace{}%
\AgdaSymbol{=}\AgdaSpace{}%
\AgdaBound{gfe}\AgdaSpace{}%
\AgdaSymbol{(λ}\AgdaSpace{}%
\AgdaBound{i}\AgdaSpace{}%
\AgdaSymbol{→}\AgdaSpace{}%
\AgdaSymbol{(}\AgdaFunction{Ghom}\AgdaSpace{}%
\AgdaBound{i}\AgdaSymbol{)}\AgdaSpace{}%
\AgdaBound{𝑓}\AgdaSpace{}%
\AgdaSymbol{(λ}\AgdaSpace{}%
\AgdaBound{x}\AgdaSpace{}%
\AgdaSymbol{→}\AgdaSpace{}%
\AgdaBound{𝒃}\AgdaSpace{}%
\AgdaBound{x}\AgdaSpace{}%
\AgdaBound{i}\AgdaSymbol{))}\<%
\\
%
\\[\AgdaEmptyExtraSkip]%
%
\>[2]\AgdaFunction{ϕ\textasciitilde{}ψ}\AgdaSpace{}%
\AgdaSymbol{:}\AgdaSpace{}%
\AgdaFunction{ϕ}\AgdaSpace{}%
\AgdaOperator{\AgdaFunction{∘}}\AgdaSpace{}%
\AgdaFunction{ψ}\AgdaSpace{}%
\AgdaOperator{\AgdaFunction{∼}}\AgdaSpace{}%
\AgdaOperator{\AgdaFunction{∣}}\AgdaSpace{}%
\AgdaFunction{𝒾𝒹}\AgdaSpace{}%
\AgdaSymbol{(}\AgdaFunction{⨅}\AgdaSpace{}%
\AgdaBound{ℬ}\AgdaSymbol{)}\AgdaSpace{}%
\AgdaOperator{\AgdaFunction{∣}}\<%
\\
%
\>[2]\AgdaFunction{ϕ\textasciitilde{}ψ}\AgdaSpace{}%
\AgdaBound{𝒃}\AgdaSpace{}%
\AgdaSymbol{=}\AgdaSpace{}%
\AgdaBound{gfe}\AgdaSpace{}%
\AgdaSymbol{λ}\AgdaSpace{}%
\AgdaBound{i}\AgdaSpace{}%
\AgdaSymbol{→}\AgdaSpace{}%
\AgdaFunction{F∼G}\AgdaSpace{}%
\AgdaBound{i}\AgdaSpace{}%
\AgdaSymbol{(}\AgdaBound{𝒃}\AgdaSpace{}%
\AgdaBound{i}\AgdaSymbol{)}\<%
\\
%
\\[\AgdaEmptyExtraSkip]%
%
\>[2]\AgdaFunction{ψ\textasciitilde{}ϕ}\AgdaSpace{}%
\AgdaSymbol{:}\AgdaSpace{}%
\AgdaFunction{ψ}\AgdaSpace{}%
\AgdaOperator{\AgdaFunction{∘}}\AgdaSpace{}%
\AgdaFunction{ϕ}\AgdaSpace{}%
\AgdaOperator{\AgdaFunction{∼}}\AgdaSpace{}%
\AgdaOperator{\AgdaFunction{∣}}\AgdaSpace{}%
\AgdaFunction{𝒾𝒹}\AgdaSpace{}%
\AgdaSymbol{(}\AgdaFunction{⨅}\AgdaSpace{}%
\AgdaBound{𝒜}\AgdaSymbol{)}\AgdaSpace{}%
\AgdaOperator{\AgdaFunction{∣}}\<%
\\
%
\>[2]\AgdaFunction{ψ\textasciitilde{}ϕ}\AgdaSpace{}%
\AgdaBound{𝒂}\AgdaSpace{}%
\AgdaSymbol{=}\AgdaSpace{}%
\AgdaBound{gfe}\AgdaSpace{}%
\AgdaSymbol{λ}\AgdaSpace{}%
\AgdaBound{i}\AgdaSpace{}%
\AgdaSymbol{→}\AgdaSpace{}%
\AgdaFunction{G∼F}\AgdaSpace{}%
\AgdaBound{i}\AgdaSpace{}%
\AgdaSymbol{(}\AgdaBound{𝒂}\AgdaSpace{}%
\AgdaBound{i}\AgdaSymbol{)}\<%
\\
%
\\[\AgdaEmptyExtraSkip]%
%
\>[2]\AgdaFunction{γ}\AgdaSpace{}%
\AgdaSymbol{:}\AgdaSpace{}%
\AgdaFunction{⨅}\AgdaSpace{}%
\AgdaBound{𝒜}\AgdaSpace{}%
\AgdaOperator{\AgdaFunction{≅}}\AgdaSpace{}%
\AgdaFunction{⨅}\AgdaSpace{}%
\AgdaBound{ℬ}\<%
\\
%
\>[2]\AgdaFunction{γ}\AgdaSpace{}%
\AgdaSymbol{=}\AgdaSpace{}%
\AgdaSymbol{(}\AgdaFunction{ϕ}\AgdaSpace{}%
\AgdaOperator{\AgdaInductiveConstructor{,}}\AgdaSpace{}%
\AgdaFunction{ϕhom}\AgdaSymbol{)}\AgdaSpace{}%
\AgdaOperator{\AgdaInductiveConstructor{,}}\AgdaSpace{}%
\AgdaSymbol{((}\AgdaFunction{ψ}\AgdaSpace{}%
\AgdaOperator{\AgdaInductiveConstructor{,}}\AgdaSpace{}%
\AgdaFunction{ψhom}\AgdaSymbol{)}\AgdaSpace{}%
\AgdaOperator{\AgdaInductiveConstructor{,}}\AgdaSpace{}%
\AgdaFunction{ϕ\textasciitilde{}ψ}\AgdaSpace{}%
\AgdaOperator{\AgdaInductiveConstructor{,}}\AgdaSpace{}%
\AgdaFunction{ψ\textasciitilde{}ϕ}\AgdaSymbol{)}\<%
\end{code}
\ccpad
A nearly identical proof goes through for isomorphisms of lifted products.
\ccpad
\begin{code}%
\>[0]\AgdaFunction{lift-alg-⨅≅}\AgdaSpace{}%
\AgdaSymbol{:}%
\>[1257I]\AgdaFunction{global-dfunext}\AgdaSpace{}%
\AgdaSymbol{→}\AgdaSpace{}%
\AgdaSymbol{\{}\AgdaBound{𝓠}\AgdaSpace{}%
\AgdaBound{𝓤}\AgdaSpace{}%
\AgdaBound{𝓘}\AgdaSpace{}%
\AgdaBound{𝓩}\AgdaSpace{}%
\AgdaSymbol{:}\AgdaSpace{}%
\AgdaFunction{Universe}\AgdaSymbol{\}}\<%
\\
\>[.][@{}l@{}]\<[1257I]%
\>[14]\AgdaSymbol{\{}\AgdaBound{I}\AgdaSpace{}%
\AgdaSymbol{:}\AgdaSpace{}%
\AgdaBound{𝓘}\AgdaSpace{}%
\AgdaOperator{\AgdaFunction{̇}}\AgdaSymbol{\}\{}\AgdaBound{𝒜}\AgdaSpace{}%
\AgdaSymbol{:}\AgdaSpace{}%
\AgdaBound{I}\AgdaSpace{}%
\AgdaSymbol{→}\AgdaSpace{}%
\AgdaFunction{Algebra}\AgdaSpace{}%
\AgdaBound{𝓠}\AgdaSpace{}%
\AgdaBound{𝑆}\AgdaSymbol{\}\{}\AgdaBound{ℬ}\AgdaSpace{}%
\AgdaSymbol{:}\AgdaSpace{}%
\AgdaSymbol{(}\AgdaRecord{Lift}\AgdaSymbol{\{}\AgdaBound{𝓩}\AgdaSymbol{\}}\AgdaSpace{}%
\AgdaBound{I}\AgdaSymbol{)}\AgdaSpace{}%
\AgdaSymbol{→}\AgdaSpace{}%
\AgdaFunction{Algebra}\AgdaSpace{}%
\AgdaBound{𝓤}\AgdaSpace{}%
\AgdaBound{𝑆}\AgdaSymbol{\}}\<%
\\
\>[0][@{}l@{\AgdaIndent{0}}]%
\>[1]\AgdaSymbol{→}%
\>[14]\AgdaSymbol{(∀}\AgdaSpace{}%
\AgdaBound{i}\AgdaSpace{}%
\AgdaSymbol{→}\AgdaSpace{}%
\AgdaSymbol{(}\AgdaBound{𝒜}\AgdaSpace{}%
\AgdaBound{i}\AgdaSymbol{)}\AgdaSpace{}%
\AgdaOperator{\AgdaFunction{≅}}\AgdaSpace{}%
\AgdaSymbol{(}\AgdaBound{ℬ}\AgdaSpace{}%
\AgdaSymbol{(}\AgdaInductiveConstructor{lift}\AgdaSpace{}%
\AgdaBound{i}\AgdaSymbol{)))}\<%
\\
%
\>[14]\AgdaComment{----------------------------}\<%
\\
%
\>[1]\AgdaSymbol{→}%
\>[14]\AgdaFunction{lift-alg}\AgdaSpace{}%
\AgdaSymbol{(}\AgdaFunction{⨅}\AgdaSpace{}%
\AgdaBound{𝒜}\AgdaSymbol{)}\AgdaSpace{}%
\AgdaBound{𝓩}\AgdaSpace{}%
\AgdaOperator{\AgdaFunction{≅}}\AgdaSpace{}%
\AgdaFunction{⨅}\AgdaSpace{}%
\AgdaBound{ℬ}\<%
\\
%
\\[\AgdaEmptyExtraSkip]%
\>[0]\AgdaFunction{lift-alg-⨅≅}\AgdaSpace{}%
\AgdaBound{gfe}\AgdaSpace{}%
\AgdaSymbol{\{}\AgdaBound{𝓠}\AgdaSymbol{\}\{}\AgdaBound{𝓤}\AgdaSymbol{\}\{}\AgdaBound{𝓘}\AgdaSymbol{\}\{}\AgdaBound{𝓩}\AgdaSymbol{\}\{}\AgdaBound{I}\AgdaSymbol{\}\{}\AgdaBound{𝒜}\AgdaSymbol{\}\{}\AgdaBound{ℬ}\AgdaSymbol{\}}\AgdaSpace{}%
\AgdaBound{AB}\AgdaSpace{}%
\AgdaSymbol{=}\AgdaSpace{}%
\AgdaFunction{γ}\<%
\\
\>[0][@{}l@{\AgdaIndent{0}}]%
\>[1]\AgdaKeyword{where}\<%
\\
\>[1][@{}l@{\AgdaIndent{0}}]%
\>[2]\AgdaFunction{F}\AgdaSpace{}%
\AgdaSymbol{:}\AgdaSpace{}%
\AgdaSymbol{∀}\AgdaSpace{}%
\AgdaBound{i}\AgdaSpace{}%
\AgdaSymbol{→}\AgdaSpace{}%
\AgdaOperator{\AgdaFunction{∣}}\AgdaSpace{}%
\AgdaBound{𝒜}\AgdaSpace{}%
\AgdaBound{i}\AgdaSpace{}%
\AgdaOperator{\AgdaFunction{∣}}\AgdaSpace{}%
\AgdaSymbol{→}\AgdaSpace{}%
\AgdaOperator{\AgdaFunction{∣}}\AgdaSpace{}%
\AgdaBound{ℬ}\AgdaSpace{}%
\AgdaSymbol{(}\AgdaInductiveConstructor{lift}\AgdaSpace{}%
\AgdaBound{i}\AgdaSymbol{)}\AgdaSpace{}%
\AgdaOperator{\AgdaFunction{∣}}\<%
\\
%
\>[2]\AgdaFunction{F}\AgdaSpace{}%
\AgdaBound{i}\AgdaSpace{}%
\AgdaSymbol{=}\AgdaSpace{}%
\AgdaOperator{\AgdaFunction{∣}}\AgdaSpace{}%
\AgdaFunction{fst}\AgdaSpace{}%
\AgdaSymbol{(}\AgdaBound{AB}\AgdaSpace{}%
\AgdaBound{i}\AgdaSymbol{)}\AgdaSpace{}%
\AgdaOperator{\AgdaFunction{∣}}\<%
\\
%
\>[2]\AgdaFunction{Fhom}\AgdaSpace{}%
\AgdaSymbol{:}\AgdaSpace{}%
\AgdaSymbol{∀}\AgdaSpace{}%
\AgdaBound{i}\AgdaSpace{}%
\AgdaSymbol{→}\AgdaSpace{}%
\AgdaFunction{is-homomorphism}\AgdaSpace{}%
\AgdaSymbol{(}\AgdaBound{𝒜}\AgdaSpace{}%
\AgdaBound{i}\AgdaSymbol{)}\AgdaSpace{}%
\AgdaSymbol{(}\AgdaBound{ℬ}\AgdaSpace{}%
\AgdaSymbol{(}\AgdaInductiveConstructor{lift}\AgdaSpace{}%
\AgdaBound{i}\AgdaSymbol{))}\AgdaSpace{}%
\AgdaSymbol{(}\AgdaFunction{F}\AgdaSpace{}%
\AgdaBound{i}\AgdaSymbol{)}\<%
\\
%
\>[2]\AgdaFunction{Fhom}\AgdaSpace{}%
\AgdaBound{i}\AgdaSpace{}%
\AgdaSymbol{=}\AgdaSpace{}%
\AgdaOperator{\AgdaFunction{∥}}\AgdaSpace{}%
\AgdaFunction{fst}\AgdaSpace{}%
\AgdaSymbol{(}\AgdaBound{AB}\AgdaSpace{}%
\AgdaBound{i}\AgdaSymbol{)}\AgdaSpace{}%
\AgdaOperator{\AgdaFunction{∥}}\<%
\\
%
\\[\AgdaEmptyExtraSkip]%
%
\>[2]\AgdaFunction{G}\AgdaSpace{}%
\AgdaSymbol{:}\AgdaSpace{}%
\AgdaSymbol{∀}\AgdaSpace{}%
\AgdaBound{i}\AgdaSpace{}%
\AgdaSymbol{→}\AgdaSpace{}%
\AgdaOperator{\AgdaFunction{∣}}\AgdaSpace{}%
\AgdaBound{ℬ}\AgdaSpace{}%
\AgdaSymbol{(}\AgdaInductiveConstructor{lift}\AgdaSpace{}%
\AgdaBound{i}\AgdaSymbol{)}\AgdaSpace{}%
\AgdaOperator{\AgdaFunction{∣}}\AgdaSpace{}%
\AgdaSymbol{→}\AgdaSpace{}%
\AgdaOperator{\AgdaFunction{∣}}\AgdaSpace{}%
\AgdaBound{𝒜}\AgdaSpace{}%
\AgdaBound{i}\AgdaSpace{}%
\AgdaOperator{\AgdaFunction{∣}}\<%
\\
%
\>[2]\AgdaFunction{G}\AgdaSpace{}%
\AgdaBound{i}\AgdaSpace{}%
\AgdaSymbol{=}\AgdaSpace{}%
\AgdaFunction{fst}\AgdaSpace{}%
\AgdaOperator{\AgdaFunction{∣}}\AgdaSpace{}%
\AgdaFunction{snd}\AgdaSpace{}%
\AgdaSymbol{(}\AgdaBound{AB}\AgdaSpace{}%
\AgdaBound{i}\AgdaSymbol{)}\AgdaSpace{}%
\AgdaOperator{\AgdaFunction{∣}}\<%
\\
%
\>[2]\AgdaFunction{Ghom}\AgdaSpace{}%
\AgdaSymbol{:}\AgdaSpace{}%
\AgdaSymbol{∀}\AgdaSpace{}%
\AgdaBound{i}\AgdaSpace{}%
\AgdaSymbol{→}\AgdaSpace{}%
\AgdaFunction{is-homomorphism}\AgdaSpace{}%
\AgdaSymbol{(}\AgdaBound{ℬ}\AgdaSpace{}%
\AgdaSymbol{(}\AgdaInductiveConstructor{lift}\AgdaSpace{}%
\AgdaBound{i}\AgdaSymbol{))}\AgdaSpace{}%
\AgdaSymbol{(}\AgdaBound{𝒜}\AgdaSpace{}%
\AgdaBound{i}\AgdaSymbol{)}\AgdaSpace{}%
\AgdaSymbol{(}\AgdaFunction{G}\AgdaSpace{}%
\AgdaBound{i}\AgdaSymbol{)}\<%
\\
%
\>[2]\AgdaFunction{Ghom}\AgdaSpace{}%
\AgdaBound{i}\AgdaSpace{}%
\AgdaSymbol{=}\AgdaSpace{}%
\AgdaFunction{snd}\AgdaSpace{}%
\AgdaOperator{\AgdaFunction{∣}}\AgdaSpace{}%
\AgdaFunction{snd}\AgdaSpace{}%
\AgdaSymbol{(}\AgdaBound{AB}\AgdaSpace{}%
\AgdaBound{i}\AgdaSymbol{)}\AgdaSpace{}%
\AgdaOperator{\AgdaFunction{∣}}\<%
\\
%
\\[\AgdaEmptyExtraSkip]%
%
\>[2]\AgdaFunction{F∼G}\AgdaSpace{}%
\AgdaSymbol{:}\AgdaSpace{}%
\AgdaSymbol{∀}\AgdaSpace{}%
\AgdaBound{i}\AgdaSpace{}%
\AgdaSymbol{→}\AgdaSpace{}%
\AgdaSymbol{(}\AgdaFunction{F}\AgdaSpace{}%
\AgdaBound{i}\AgdaSymbol{)}\AgdaSpace{}%
\AgdaOperator{\AgdaFunction{∘}}\AgdaSpace{}%
\AgdaSymbol{(}\AgdaFunction{G}\AgdaSpace{}%
\AgdaBound{i}\AgdaSymbol{)}\AgdaSpace{}%
\AgdaOperator{\AgdaFunction{∼}}\AgdaSpace{}%
\AgdaSymbol{(}\AgdaOperator{\AgdaFunction{∣}}\AgdaSpace{}%
\AgdaFunction{𝒾𝒹}\AgdaSpace{}%
\AgdaSymbol{(}\AgdaBound{ℬ}\AgdaSpace{}%
\AgdaSymbol{(}\AgdaInductiveConstructor{lift}\AgdaSpace{}%
\AgdaBound{i}\AgdaSymbol{))}\AgdaSpace{}%
\AgdaOperator{\AgdaFunction{∣}}\AgdaSymbol{)}\<%
\\
%
\>[2]\AgdaFunction{F∼G}\AgdaSpace{}%
\AgdaBound{i}\AgdaSpace{}%
\AgdaSymbol{=}\AgdaSpace{}%
\AgdaFunction{fst}\AgdaSpace{}%
\AgdaOperator{\AgdaFunction{∥}}\AgdaSpace{}%
\AgdaFunction{snd}\AgdaSpace{}%
\AgdaSymbol{(}\AgdaBound{AB}\AgdaSpace{}%
\AgdaBound{i}\AgdaSymbol{)}\AgdaSpace{}%
\AgdaOperator{\AgdaFunction{∥}}\<%
\\
%
\\[\AgdaEmptyExtraSkip]%
%
\>[2]\AgdaFunction{G∼F}\AgdaSpace{}%
\AgdaSymbol{:}\AgdaSpace{}%
\AgdaSymbol{∀}\AgdaSpace{}%
\AgdaBound{i}\AgdaSpace{}%
\AgdaSymbol{→}\AgdaSpace{}%
\AgdaSymbol{(}\AgdaFunction{G}\AgdaSpace{}%
\AgdaBound{i}\AgdaSymbol{)}\AgdaSpace{}%
\AgdaOperator{\AgdaFunction{∘}}\AgdaSpace{}%
\AgdaSymbol{(}\AgdaFunction{F}\AgdaSpace{}%
\AgdaBound{i}\AgdaSymbol{)}\AgdaSpace{}%
\AgdaOperator{\AgdaFunction{∼}}\AgdaSpace{}%
\AgdaSymbol{(}\AgdaOperator{\AgdaFunction{∣}}\AgdaSpace{}%
\AgdaFunction{𝒾𝒹}\AgdaSpace{}%
\AgdaSymbol{(}\AgdaBound{𝒜}\AgdaSpace{}%
\AgdaBound{i}\AgdaSymbol{)}\AgdaSpace{}%
\AgdaOperator{\AgdaFunction{∣}}\AgdaSymbol{)}\<%
\\
%
\>[2]\AgdaFunction{G∼F}\AgdaSpace{}%
\AgdaBound{i}\AgdaSpace{}%
\AgdaSymbol{=}\AgdaSpace{}%
\AgdaFunction{snd}\AgdaSpace{}%
\AgdaOperator{\AgdaFunction{∥}}\AgdaSpace{}%
\AgdaFunction{snd}\AgdaSpace{}%
\AgdaSymbol{(}\AgdaBound{AB}\AgdaSpace{}%
\AgdaBound{i}\AgdaSymbol{)}\AgdaSpace{}%
\AgdaOperator{\AgdaFunction{∥}}\<%
\\
%
\\[\AgdaEmptyExtraSkip]%
%
\>[2]\AgdaFunction{ϕ}\AgdaSpace{}%
\AgdaSymbol{:}\AgdaSpace{}%
\AgdaOperator{\AgdaFunction{∣}}\AgdaSpace{}%
\AgdaFunction{⨅}\AgdaSpace{}%
\AgdaBound{𝒜}\AgdaSpace{}%
\AgdaOperator{\AgdaFunction{∣}}\AgdaSpace{}%
\AgdaSymbol{→}\AgdaSpace{}%
\AgdaOperator{\AgdaFunction{∣}}\AgdaSpace{}%
\AgdaFunction{⨅}\AgdaSpace{}%
\AgdaBound{ℬ}\AgdaSpace{}%
\AgdaOperator{\AgdaFunction{∣}}\<%
\\
%
\>[2]\AgdaFunction{ϕ}\AgdaSpace{}%
\AgdaBound{a}\AgdaSpace{}%
\AgdaBound{i}\AgdaSpace{}%
\AgdaSymbol{=}\AgdaSpace{}%
\AgdaFunction{F}\AgdaSpace{}%
\AgdaSymbol{(}\AgdaField{lower}\AgdaSpace{}%
\AgdaBound{i}\AgdaSymbol{)}\AgdaSpace{}%
\AgdaSymbol{(}\AgdaBound{a}\AgdaSpace{}%
\AgdaSymbol{(}\AgdaField{lower}\AgdaSpace{}%
\AgdaBound{i}\AgdaSymbol{))}\<%
\\
%
\\[\AgdaEmptyExtraSkip]%
%
\>[2]\AgdaFunction{ϕhom}\AgdaSpace{}%
\AgdaSymbol{:}\AgdaSpace{}%
\AgdaFunction{is-homomorphism}\AgdaSpace{}%
\AgdaSymbol{(}\AgdaFunction{⨅}\AgdaSpace{}%
\AgdaBound{𝒜}\AgdaSymbol{)}\AgdaSpace{}%
\AgdaSymbol{(}\AgdaFunction{⨅}\AgdaSpace{}%
\AgdaBound{ℬ}\AgdaSymbol{)}\AgdaSpace{}%
\AgdaFunction{ϕ}\<%
\\
%
\>[2]\AgdaFunction{ϕhom}\AgdaSpace{}%
\AgdaBound{𝑓}\AgdaSpace{}%
\AgdaBound{𝒂}\AgdaSpace{}%
\AgdaSymbol{=}\AgdaSpace{}%
\AgdaBound{gfe}\AgdaSpace{}%
\AgdaSymbol{(λ}\AgdaSpace{}%
\AgdaBound{i}\AgdaSpace{}%
\AgdaSymbol{→}\AgdaSpace{}%
\AgdaSymbol{(}\AgdaFunction{Fhom}\AgdaSpace{}%
\AgdaSymbol{(}\AgdaField{lower}\AgdaSpace{}%
\AgdaBound{i}\AgdaSymbol{))}\AgdaSpace{}%
\AgdaBound{𝑓}\AgdaSpace{}%
\AgdaSymbol{(λ}\AgdaSpace{}%
\AgdaBound{x}\AgdaSpace{}%
\AgdaSymbol{→}\AgdaSpace{}%
\AgdaBound{𝒂}\AgdaSpace{}%
\AgdaBound{x}\AgdaSpace{}%
\AgdaSymbol{(}\AgdaField{lower}\AgdaSpace{}%
\AgdaBound{i}\AgdaSymbol{)))}\<%
\\
%
\\[\AgdaEmptyExtraSkip]%
%
\>[2]\AgdaFunction{ψ}\AgdaSpace{}%
\AgdaSymbol{:}\AgdaSpace{}%
\AgdaOperator{\AgdaFunction{∣}}\AgdaSpace{}%
\AgdaFunction{⨅}\AgdaSpace{}%
\AgdaBound{ℬ}\AgdaSpace{}%
\AgdaOperator{\AgdaFunction{∣}}\AgdaSpace{}%
\AgdaSymbol{→}\AgdaSpace{}%
\AgdaOperator{\AgdaFunction{∣}}\AgdaSpace{}%
\AgdaFunction{⨅}\AgdaSpace{}%
\AgdaBound{𝒜}\AgdaSpace{}%
\AgdaOperator{\AgdaFunction{∣}}\<%
\\
%
\>[2]\AgdaFunction{ψ}\AgdaSpace{}%
\AgdaBound{b}\AgdaSpace{}%
\AgdaBound{i}\AgdaSpace{}%
\AgdaSymbol{=}\AgdaSpace{}%
\AgdaOperator{\AgdaFunction{∣}}\AgdaSpace{}%
\AgdaFunction{fst}\AgdaSpace{}%
\AgdaOperator{\AgdaFunction{∥}}\AgdaSpace{}%
\AgdaBound{AB}\AgdaSpace{}%
\AgdaBound{i}\AgdaSpace{}%
\AgdaOperator{\AgdaFunction{∥}}\AgdaSpace{}%
\AgdaOperator{\AgdaFunction{∣}}\AgdaSpace{}%
\AgdaSymbol{(}\AgdaBound{b}\AgdaSpace{}%
\AgdaSymbol{(}\AgdaInductiveConstructor{lift}\AgdaSpace{}%
\AgdaBound{i}\AgdaSymbol{))}\<%
\\
%
\\[\AgdaEmptyExtraSkip]%
%
\>[2]\AgdaFunction{ψhom}\AgdaSpace{}%
\AgdaSymbol{:}\AgdaSpace{}%
\AgdaFunction{is-homomorphism}\AgdaSpace{}%
\AgdaSymbol{(}\AgdaFunction{⨅}\AgdaSpace{}%
\AgdaBound{ℬ}\AgdaSymbol{)}\AgdaSpace{}%
\AgdaSymbol{(}\AgdaFunction{⨅}\AgdaSpace{}%
\AgdaBound{𝒜}\AgdaSymbol{)}\AgdaSpace{}%
\AgdaFunction{ψ}\<%
\\
%
\>[2]\AgdaFunction{ψhom}\AgdaSpace{}%
\AgdaBound{𝑓}\AgdaSpace{}%
\AgdaBound{𝒃}\AgdaSpace{}%
\AgdaSymbol{=}\AgdaSpace{}%
\AgdaBound{gfe}\AgdaSpace{}%
\AgdaSymbol{(λ}\AgdaSpace{}%
\AgdaBound{i}\AgdaSpace{}%
\AgdaSymbol{→}\AgdaSpace{}%
\AgdaSymbol{(}\AgdaFunction{Ghom}\AgdaSpace{}%
\AgdaBound{i}\AgdaSymbol{)}\AgdaSpace{}%
\AgdaBound{𝑓}\AgdaSpace{}%
\AgdaSymbol{(λ}\AgdaSpace{}%
\AgdaBound{x}\AgdaSpace{}%
\AgdaSymbol{→}\AgdaSpace{}%
\AgdaBound{𝒃}\AgdaSpace{}%
\AgdaBound{x}\AgdaSpace{}%
\AgdaSymbol{(}\AgdaInductiveConstructor{lift}\AgdaSpace{}%
\AgdaBound{i}\AgdaSymbol{)))}\<%
\\
%
\\[\AgdaEmptyExtraSkip]%
%
\>[2]\AgdaFunction{ϕ\textasciitilde{}ψ}\AgdaSpace{}%
\AgdaSymbol{:}\AgdaSpace{}%
\AgdaFunction{ϕ}\AgdaSpace{}%
\AgdaOperator{\AgdaFunction{∘}}\AgdaSpace{}%
\AgdaFunction{ψ}\AgdaSpace{}%
\AgdaOperator{\AgdaFunction{∼}}\AgdaSpace{}%
\AgdaOperator{\AgdaFunction{∣}}\AgdaSpace{}%
\AgdaFunction{𝒾𝒹}\AgdaSpace{}%
\AgdaSymbol{(}\AgdaFunction{⨅}\AgdaSpace{}%
\AgdaBound{ℬ}\AgdaSymbol{)}\AgdaSpace{}%
\AgdaOperator{\AgdaFunction{∣}}\<%
\\
%
\>[2]\AgdaFunction{ϕ\textasciitilde{}ψ}\AgdaSpace{}%
\AgdaBound{𝒃}\AgdaSpace{}%
\AgdaSymbol{=}\AgdaSpace{}%
\AgdaBound{gfe}\AgdaSpace{}%
\AgdaSymbol{λ}\AgdaSpace{}%
\AgdaBound{i}\AgdaSpace{}%
\AgdaSymbol{→}\AgdaSpace{}%
\AgdaFunction{F∼G}\AgdaSpace{}%
\AgdaSymbol{(}\AgdaField{lower}\AgdaSpace{}%
\AgdaBound{i}\AgdaSymbol{)}\AgdaSpace{}%
\AgdaSymbol{(}\AgdaBound{𝒃}\AgdaSpace{}%
\AgdaBound{i}\AgdaSymbol{)}\<%
\\
%
\\[\AgdaEmptyExtraSkip]%
%
\>[2]\AgdaFunction{ψ\textasciitilde{}ϕ}\AgdaSpace{}%
\AgdaSymbol{:}\AgdaSpace{}%
\AgdaFunction{ψ}\AgdaSpace{}%
\AgdaOperator{\AgdaFunction{∘}}\AgdaSpace{}%
\AgdaFunction{ϕ}\AgdaSpace{}%
\AgdaOperator{\AgdaFunction{∼}}\AgdaSpace{}%
\AgdaOperator{\AgdaFunction{∣}}\AgdaSpace{}%
\AgdaFunction{𝒾𝒹}\AgdaSpace{}%
\AgdaSymbol{(}\AgdaFunction{⨅}\AgdaSpace{}%
\AgdaBound{𝒜}\AgdaSymbol{)}\AgdaSpace{}%
\AgdaOperator{\AgdaFunction{∣}}\<%
\\
%
\>[2]\AgdaFunction{ψ\textasciitilde{}ϕ}\AgdaSpace{}%
\AgdaBound{𝒂}\AgdaSpace{}%
\AgdaSymbol{=}\AgdaSpace{}%
\AgdaBound{gfe}\AgdaSpace{}%
\AgdaSymbol{λ}\AgdaSpace{}%
\AgdaBound{i}\AgdaSpace{}%
\AgdaSymbol{→}\AgdaSpace{}%
\AgdaFunction{G∼F}\AgdaSpace{}%
\AgdaBound{i}\AgdaSpace{}%
\AgdaSymbol{(}\AgdaBound{𝒂}\AgdaSpace{}%
\AgdaBound{i}\AgdaSymbol{)}\<%
\\
%
\\[\AgdaEmptyExtraSkip]%
%
\>[2]\AgdaFunction{A≅B}\AgdaSpace{}%
\AgdaSymbol{:}\AgdaSpace{}%
\AgdaFunction{⨅}\AgdaSpace{}%
\AgdaBound{𝒜}\AgdaSpace{}%
\AgdaOperator{\AgdaFunction{≅}}\AgdaSpace{}%
\AgdaFunction{⨅}\AgdaSpace{}%
\AgdaBound{ℬ}\<%
\\
%
\>[2]\AgdaFunction{A≅B}\AgdaSpace{}%
\AgdaSymbol{=}\AgdaSpace{}%
\AgdaSymbol{(}\AgdaFunction{ϕ}\AgdaSpace{}%
\AgdaOperator{\AgdaInductiveConstructor{,}}\AgdaSpace{}%
\AgdaFunction{ϕhom}\AgdaSymbol{)}\AgdaSpace{}%
\AgdaOperator{\AgdaInductiveConstructor{,}}\AgdaSpace{}%
\AgdaSymbol{((}\AgdaFunction{ψ}\AgdaSpace{}%
\AgdaOperator{\AgdaInductiveConstructor{,}}\AgdaSpace{}%
\AgdaFunction{ψhom}\AgdaSymbol{)}\AgdaSpace{}%
\AgdaOperator{\AgdaInductiveConstructor{,}}\AgdaSpace{}%
\AgdaFunction{ϕ\textasciitilde{}ψ}\AgdaSpace{}%
\AgdaOperator{\AgdaInductiveConstructor{,}}\AgdaSpace{}%
\AgdaFunction{ψ\textasciitilde{}ϕ}\AgdaSymbol{)}\<%
\\
%
\\[\AgdaEmptyExtraSkip]%
%
\>[2]\AgdaFunction{γ}\AgdaSpace{}%
\AgdaSymbol{:}\AgdaSpace{}%
\AgdaFunction{lift-alg}\AgdaSpace{}%
\AgdaSymbol{(}\AgdaFunction{⨅}\AgdaSpace{}%
\AgdaBound{𝒜}\AgdaSymbol{)}\AgdaSpace{}%
\AgdaBound{𝓩}\AgdaSpace{}%
\AgdaOperator{\AgdaFunction{≅}}\AgdaSpace{}%
\AgdaFunction{⨅}\AgdaSpace{}%
\AgdaBound{ℬ}\<%
\\
%
\>[2]\AgdaFunction{γ}\AgdaSpace{}%
\AgdaSymbol{=}\AgdaSpace{}%
\AgdaFunction{Trans-≅}\AgdaSpace{}%
\AgdaSymbol{(}\AgdaFunction{lift-alg}\AgdaSpace{}%
\AgdaSymbol{(}\AgdaFunction{⨅}\AgdaSpace{}%
\AgdaBound{𝒜}\AgdaSymbol{)}\AgdaSpace{}%
\AgdaBound{𝓩}\AgdaSymbol{)}\AgdaSpace{}%
\AgdaSymbol{(}\AgdaFunction{⨅}\AgdaSpace{}%
\AgdaBound{ℬ}\AgdaSymbol{)}\AgdaSpace{}%
\AgdaSymbol{(}\AgdaFunction{sym-≅}\AgdaSpace{}%
\AgdaFunction{lift-alg-≅}\AgdaSymbol{)}\AgdaSpace{}%
\AgdaFunction{A≅B}\<%
\end{code}

\subsubsection{Embedding tools}\label{embedding-tools}

\begin{code}%
\>[0]\AgdaKeyword{module}\AgdaSpace{}%
\AgdaModule{\AgdaUnderscore{}}\AgdaSpace{}%
\AgdaSymbol{\{}\AgdaBound{𝓘}\AgdaSpace{}%
\AgdaBound{𝓤}\AgdaSpace{}%
\AgdaBound{𝓦}\AgdaSpace{}%
\AgdaSymbol{:}\AgdaSpace{}%
\AgdaFunction{Universe}\AgdaSymbol{\}}\AgdaSpace{}%
\AgdaKeyword{where}\<%
\\
%
\\[\AgdaEmptyExtraSkip]%
\>[0][@{}l@{\AgdaIndent{0}}]%
\>[1]\AgdaFunction{embedding-lift-nat}\AgdaSpace{}%
\AgdaSymbol{:}\AgdaSpace{}%
\AgdaFunction{hfunext}\AgdaSpace{}%
\AgdaBound{𝓘}\AgdaSpace{}%
\AgdaBound{𝓤}\AgdaSpace{}%
\AgdaSymbol{→}\AgdaSpace{}%
\AgdaFunction{hfunext}\AgdaSpace{}%
\AgdaBound{𝓘}\AgdaSpace{}%
\AgdaBound{𝓦}\<%
\\
\>[1][@{}l@{\AgdaIndent{0}}]%
\>[2]\AgdaSymbol{→}%
\>[22]\AgdaSymbol{\{}\AgdaBound{I}\AgdaSpace{}%
\AgdaSymbol{:}\AgdaSpace{}%
\AgdaBound{𝓘}\AgdaSpace{}%
\AgdaOperator{\AgdaFunction{̇}}\AgdaSymbol{\}\{}\AgdaBound{A}\AgdaSpace{}%
\AgdaSymbol{:}\AgdaSpace{}%
\AgdaBound{I}\AgdaSpace{}%
\AgdaSymbol{→}\AgdaSpace{}%
\AgdaBound{𝓤}\AgdaSpace{}%
\AgdaOperator{\AgdaFunction{̇}}\AgdaSymbol{\}\{}\AgdaBound{B}\AgdaSpace{}%
\AgdaSymbol{:}\AgdaSpace{}%
\AgdaBound{I}\AgdaSpace{}%
\AgdaSymbol{→}\AgdaSpace{}%
\AgdaBound{𝓦}\AgdaSpace{}%
\AgdaOperator{\AgdaFunction{̇}}\AgdaSymbol{\}}\<%
\\
%
\>[22]\AgdaSymbol{(}\AgdaBound{h}\AgdaSpace{}%
\AgdaSymbol{:}\AgdaSpace{}%
\AgdaFunction{Nat}\AgdaSpace{}%
\AgdaBound{A}\AgdaSpace{}%
\AgdaBound{B}\AgdaSymbol{)}\AgdaSpace{}%
\AgdaSymbol{→}\AgdaSpace{}%
\AgdaSymbol{(∀}\AgdaSpace{}%
\AgdaBound{i}\AgdaSpace{}%
\AgdaSymbol{→}\AgdaSpace{}%
\AgdaFunction{is-embedding}\AgdaSpace{}%
\AgdaSymbol{(}\AgdaBound{h}\AgdaSpace{}%
\AgdaBound{i}\AgdaSymbol{))}\<%
\\
%
\>[22]\AgdaComment{------------------------------------------}\<%
\\
%
\>[2]\AgdaSymbol{→}%
\>[22]\AgdaFunction{is-embedding}\AgdaSymbol{(}\AgdaFunction{NatΠ}\AgdaSpace{}%
\AgdaBound{h}\AgdaSymbol{)}\<%
\\
%
\\[\AgdaEmptyExtraSkip]%
%
\>[1]\AgdaFunction{embedding-lift-nat}\AgdaSpace{}%
\AgdaBound{hfiu}\AgdaSpace{}%
\AgdaBound{hfiw}\AgdaSpace{}%
\AgdaBound{h}\AgdaSpace{}%
\AgdaBound{hem}\AgdaSpace{}%
\AgdaSymbol{=}\AgdaSpace{}%
\AgdaFunction{NatΠ-is-embedding}\AgdaSpace{}%
\AgdaBound{hfiu}\AgdaSpace{}%
\AgdaBound{hfiw}\AgdaSpace{}%
\AgdaBound{h}\AgdaSpace{}%
\AgdaBound{hem}\<%
\\
%
\\[\AgdaEmptyExtraSkip]%
%
\\[\AgdaEmptyExtraSkip]%
%
\>[1]\AgdaFunction{embedding-lift-nat'}\AgdaSpace{}%
\AgdaSymbol{:}\AgdaSpace{}%
\AgdaFunction{hfunext}\AgdaSpace{}%
\AgdaBound{𝓘}\AgdaSpace{}%
\AgdaBound{𝓤}\AgdaSpace{}%
\AgdaSymbol{→}\AgdaSpace{}%
\AgdaFunction{hfunext}\AgdaSpace{}%
\AgdaBound{𝓘}\AgdaSpace{}%
\AgdaBound{𝓦}\<%
\\
\>[1][@{}l@{\AgdaIndent{0}}]%
\>[2]\AgdaSymbol{→}%
\>[23]\AgdaSymbol{\{}\AgdaBound{I}\AgdaSpace{}%
\AgdaSymbol{:}\AgdaSpace{}%
\AgdaBound{𝓘}\AgdaSpace{}%
\AgdaOperator{\AgdaFunction{̇}}\AgdaSymbol{\}\{}\AgdaBound{𝒜}\AgdaSpace{}%
\AgdaSymbol{:}\AgdaSpace{}%
\AgdaBound{I}\AgdaSpace{}%
\AgdaSymbol{→}\AgdaSpace{}%
\AgdaFunction{Algebra}\AgdaSpace{}%
\AgdaBound{𝓤}\AgdaSpace{}%
\AgdaBound{𝑆}\AgdaSymbol{\}\{}\AgdaBound{ℬ}\AgdaSpace{}%
\AgdaSymbol{:}\AgdaSpace{}%
\AgdaBound{I}\AgdaSpace{}%
\AgdaSymbol{→}\AgdaSpace{}%
\AgdaFunction{Algebra}\AgdaSpace{}%
\AgdaBound{𝓦}\AgdaSpace{}%
\AgdaBound{𝑆}\AgdaSymbol{\}}\<%
\\
%
\>[23]\AgdaSymbol{(}\AgdaBound{h}\AgdaSpace{}%
\AgdaSymbol{:}\AgdaSpace{}%
\AgdaFunction{Nat}\AgdaSymbol{(}\AgdaFunction{fst}\AgdaSpace{}%
\AgdaOperator{\AgdaFunction{∘}}\AgdaSpace{}%
\AgdaBound{𝒜}\AgdaSymbol{)(}\AgdaFunction{fst}\AgdaSpace{}%
\AgdaOperator{\AgdaFunction{∘}}\AgdaSpace{}%
\AgdaBound{ℬ}\AgdaSymbol{))}\AgdaSpace{}%
\AgdaSymbol{→}\AgdaSpace{}%
\AgdaSymbol{(∀}\AgdaSpace{}%
\AgdaBound{i}\AgdaSpace{}%
\AgdaSymbol{→}\AgdaSpace{}%
\AgdaFunction{is-embedding}\AgdaSpace{}%
\AgdaSymbol{(}\AgdaBound{h}\AgdaSpace{}%
\AgdaBound{i}\AgdaSymbol{))}\<%
\\
%
\>[23]\AgdaComment{--------------------------------------------------------}\<%
\\
%
\>[2]\AgdaSymbol{→}%
\>[23]\AgdaFunction{is-embedding}\AgdaSymbol{(}\AgdaFunction{NatΠ}\AgdaSpace{}%
\AgdaBound{h}\AgdaSymbol{)}\<%
\\
%
\\[\AgdaEmptyExtraSkip]%
%
\>[1]\AgdaFunction{embedding-lift-nat'}\AgdaSpace{}%
\AgdaBound{hfiu}\AgdaSpace{}%
\AgdaBound{hfiw}\AgdaSpace{}%
\AgdaBound{h}\AgdaSpace{}%
\AgdaBound{hem}\AgdaSpace{}%
\AgdaSymbol{=}\AgdaSpace{}%
\AgdaFunction{NatΠ-is-embedding}\AgdaSpace{}%
\AgdaBound{hfiu}\AgdaSpace{}%
\AgdaBound{hfiw}\AgdaSpace{}%
\AgdaBound{h}\AgdaSpace{}%
\AgdaBound{hem}\<%
\\
%
\\[\AgdaEmptyExtraSkip]%
%
\\[\AgdaEmptyExtraSkip]%
%
\>[1]\AgdaFunction{embedding-lift}\AgdaSpace{}%
\AgdaSymbol{:}\AgdaSpace{}%
\AgdaFunction{hfunext}\AgdaSpace{}%
\AgdaBound{𝓘}\AgdaSpace{}%
\AgdaBound{𝓤}\AgdaSpace{}%
\AgdaSymbol{→}\AgdaSpace{}%
\AgdaFunction{hfunext}\AgdaSpace{}%
\AgdaBound{𝓘}\AgdaSpace{}%
\AgdaBound{𝓦}\<%
\\
\>[1][@{}l@{\AgdaIndent{0}}]%
\>[2]\AgdaSymbol{→}%
\>[18]\AgdaSymbol{\{}\AgdaBound{I}\AgdaSpace{}%
\AgdaSymbol{:}\AgdaSpace{}%
\AgdaBound{𝓘}\AgdaSpace{}%
\AgdaOperator{\AgdaFunction{̇}}\AgdaSymbol{\}}\AgdaSpace{}%
\AgdaSymbol{→}\AgdaSpace{}%
\AgdaSymbol{\{}\AgdaBound{𝒜}\AgdaSpace{}%
\AgdaSymbol{:}\AgdaSpace{}%
\AgdaBound{I}\AgdaSpace{}%
\AgdaSymbol{→}\AgdaSpace{}%
\AgdaFunction{Algebra}\AgdaSpace{}%
\AgdaBound{𝓤}\AgdaSpace{}%
\AgdaBound{𝑆}\AgdaSymbol{\}\{}\AgdaBound{ℬ}\AgdaSpace{}%
\AgdaSymbol{:}\AgdaSpace{}%
\AgdaBound{I}\AgdaSpace{}%
\AgdaSymbol{→}\AgdaSpace{}%
\AgdaFunction{Algebra}\AgdaSpace{}%
\AgdaBound{𝓦}\AgdaSpace{}%
\AgdaBound{𝑆}\AgdaSymbol{\}}\<%
\\
%
\>[2]\AgdaSymbol{→}%
\>[18]\AgdaSymbol{(}\AgdaBound{h}\AgdaSpace{}%
\AgdaSymbol{:}\AgdaSpace{}%
\AgdaSymbol{∀}\AgdaSpace{}%
\AgdaBound{i}\AgdaSpace{}%
\AgdaSymbol{→}\AgdaSpace{}%
\AgdaOperator{\AgdaFunction{∣}}\AgdaSpace{}%
\AgdaBound{𝒜}\AgdaSpace{}%
\AgdaBound{i}\AgdaSpace{}%
\AgdaOperator{\AgdaFunction{∣}}\AgdaSpace{}%
\AgdaSymbol{→}\AgdaSpace{}%
\AgdaOperator{\AgdaFunction{∣}}\AgdaSpace{}%
\AgdaBound{ℬ}\AgdaSpace{}%
\AgdaBound{i}\AgdaSpace{}%
\AgdaOperator{\AgdaFunction{∣}}\AgdaSymbol{)}\AgdaSpace{}%
\AgdaSymbol{→}\AgdaSpace{}%
\AgdaSymbol{(∀}\AgdaSpace{}%
\AgdaBound{i}\AgdaSpace{}%
\AgdaSymbol{→}\AgdaSpace{}%
\AgdaFunction{is-embedding}\AgdaSpace{}%
\AgdaSymbol{(}\AgdaBound{h}\AgdaSpace{}%
\AgdaBound{i}\AgdaSymbol{))}\<%
\\
%
\>[18]\AgdaComment{----------------------------------------------------------}\<%
\\
%
\>[2]\AgdaSymbol{→}%
\>[18]\AgdaFunction{is-embedding}\AgdaSymbol{(λ}\AgdaSpace{}%
\AgdaSymbol{(}\AgdaBound{x}\AgdaSpace{}%
\AgdaSymbol{:}\AgdaSpace{}%
\AgdaOperator{\AgdaFunction{∣}}\AgdaSpace{}%
\AgdaFunction{⨅}\AgdaSpace{}%
\AgdaBound{𝒜}\AgdaSpace{}%
\AgdaOperator{\AgdaFunction{∣}}\AgdaSymbol{)}\AgdaSpace{}%
\AgdaSymbol{(}\AgdaBound{i}\AgdaSpace{}%
\AgdaSymbol{:}\AgdaSpace{}%
\AgdaBound{I}\AgdaSymbol{)}\AgdaSpace{}%
\AgdaSymbol{→}\AgdaSpace{}%
\AgdaSymbol{(}\AgdaBound{h}\AgdaSpace{}%
\AgdaBound{i}\AgdaSymbol{)(}\AgdaBound{x}\AgdaSpace{}%
\AgdaBound{i}\AgdaSymbol{))}\<%
\\
%
\\[\AgdaEmptyExtraSkip]%
%
\>[1]\AgdaFunction{embedding-lift}\AgdaSpace{}%
\AgdaBound{hfiu}\AgdaSpace{}%
\AgdaBound{hfiw}\AgdaSpace{}%
\AgdaSymbol{\{}\AgdaBound{I}\AgdaSymbol{\}\{}\AgdaBound{𝒜}\AgdaSymbol{\}\{}\AgdaBound{ℬ}\AgdaSymbol{\}}\AgdaSpace{}%
\AgdaBound{h}\AgdaSpace{}%
\AgdaBound{hem}\AgdaSpace{}%
\AgdaSymbol{=}\AgdaSpace{}%
\AgdaFunction{embedding-lift-nat'}\AgdaSpace{}%
\AgdaBound{hfiu}\AgdaSpace{}%
\AgdaBound{hfiw}\AgdaSpace{}%
\AgdaSymbol{\{}\AgdaBound{I}\AgdaSymbol{\}\{}\AgdaBound{𝒜}\AgdaSymbol{\}\{}\AgdaBound{ℬ}\AgdaSymbol{\}}\AgdaSpace{}%
\AgdaBound{h}\AgdaSpace{}%
\AgdaBound{hem}\<%
\\
%
\\[\AgdaEmptyExtraSkip]%
%
\\[\AgdaEmptyExtraSkip]%
\>[0]\AgdaFunction{iso→embedding}\AgdaSpace{}%
\AgdaSymbol{:}\AgdaSpace{}%
\AgdaSymbol{\{}\AgdaBound{𝓤}\AgdaSpace{}%
\AgdaBound{𝓦}\AgdaSpace{}%
\AgdaSymbol{:}\AgdaSpace{}%
\AgdaFunction{Universe}\AgdaSymbol{\}\{}\AgdaBound{𝑨}\AgdaSpace{}%
\AgdaSymbol{:}\AgdaSpace{}%
\AgdaFunction{Algebra}\AgdaSpace{}%
\AgdaBound{𝓤}\AgdaSpace{}%
\AgdaBound{𝑆}\AgdaSymbol{\}\{}\AgdaBound{𝑩}\AgdaSpace{}%
\AgdaSymbol{:}\AgdaSpace{}%
\AgdaFunction{Algebra}\AgdaSpace{}%
\AgdaBound{𝓦}\AgdaSpace{}%
\AgdaBound{𝑆}\AgdaSymbol{\}}\<%
\\
\>[0][@{}l@{\AgdaIndent{0}}]%
\>[1]\AgdaSymbol{→}%
\>[16]\AgdaSymbol{(}\AgdaBound{ϕ}\AgdaSpace{}%
\AgdaSymbol{:}\AgdaSpace{}%
\AgdaBound{𝑨}\AgdaSpace{}%
\AgdaOperator{\AgdaFunction{≅}}\AgdaSpace{}%
\AgdaBound{𝑩}\AgdaSymbol{)}\AgdaSpace{}%
\AgdaSymbol{→}\AgdaSpace{}%
\AgdaFunction{is-embedding}\AgdaSpace{}%
\AgdaSymbol{(}\AgdaFunction{fst}\AgdaSpace{}%
\AgdaOperator{\AgdaFunction{∣}}\AgdaSpace{}%
\AgdaBound{ϕ}\AgdaSpace{}%
\AgdaOperator{\AgdaFunction{∣}}\AgdaSymbol{)}\<%
\\
%
\\[\AgdaEmptyExtraSkip]%
\>[0]\AgdaFunction{iso→embedding}\AgdaSpace{}%
\AgdaBound{ϕ}\AgdaSpace{}%
\AgdaSymbol{=}%
\>[1790I]\AgdaFunction{equivs-are-embeddings}\AgdaSpace{}%
\AgdaSymbol{(}\AgdaFunction{fst}\AgdaSpace{}%
\AgdaOperator{\AgdaFunction{∣}}\AgdaSpace{}%
\AgdaBound{ϕ}\AgdaSpace{}%
\AgdaOperator{\AgdaFunction{∣}}\AgdaSymbol{)}\<%
\\
\>[1790I][@{}l@{\AgdaIndent{0}}]%
\>[19]\AgdaSymbol{(}\AgdaFunction{invertibles-are-equivs}\AgdaSpace{}%
\AgdaSymbol{(}\AgdaFunction{fst}\AgdaSpace{}%
\AgdaOperator{\AgdaFunction{∣}}\AgdaSpace{}%
\AgdaBound{ϕ}\AgdaSpace{}%
\AgdaOperator{\AgdaFunction{∣}}\AgdaSymbol{)}\AgdaSpace{}%
\AgdaFunction{finv}\AgdaSymbol{)}\<%
\\
\>[0][@{}l@{\AgdaIndent{0}}]%
\>[1]\AgdaKeyword{where}\<%
\\
%
\>[1]\AgdaFunction{finv}\AgdaSpace{}%
\AgdaSymbol{:}\AgdaSpace{}%
\AgdaFunction{invertible}\AgdaSpace{}%
\AgdaSymbol{(}\AgdaFunction{fst}\AgdaSpace{}%
\AgdaOperator{\AgdaFunction{∣}}\AgdaSpace{}%
\AgdaBound{ϕ}\AgdaSpace{}%
\AgdaOperator{\AgdaFunction{∣}}\AgdaSymbol{)}\<%
\\
%
\>[1]\AgdaFunction{finv}\AgdaSpace{}%
\AgdaSymbol{=}\AgdaSpace{}%
\AgdaOperator{\AgdaFunction{∣}}\AgdaSpace{}%
\AgdaFunction{fst}\AgdaSpace{}%
\AgdaOperator{\AgdaFunction{∥}}\AgdaSpace{}%
\AgdaBound{ϕ}\AgdaSpace{}%
\AgdaOperator{\AgdaFunction{∥}}\AgdaSpace{}%
\AgdaOperator{\AgdaFunction{∣}}\AgdaSpace{}%
\AgdaOperator{\AgdaInductiveConstructor{,}}\AgdaSpace{}%
\AgdaSymbol{(}\AgdaFunction{snd}\AgdaSpace{}%
\AgdaOperator{\AgdaFunction{∥}}\AgdaSpace{}%
\AgdaFunction{snd}\AgdaSpace{}%
\AgdaBound{ϕ}\AgdaSpace{}%
\AgdaOperator{\AgdaFunction{∥}}\AgdaSpace{}%
\AgdaOperator{\AgdaInductiveConstructor{,}}\AgdaSpace{}%
\AgdaFunction{fst}\AgdaSpace{}%
\AgdaOperator{\AgdaFunction{∥}}\AgdaSpace{}%
\AgdaFunction{snd}\AgdaSpace{}%
\AgdaBound{ϕ}\AgdaSpace{}%
\AgdaOperator{\AgdaFunction{∥}}\AgdaSymbol{)}\<%
\end{code}


\subsection{Homomorphic Images}\label{sec:hom-images}
\firstsentence{Homomorphisms}{HomomorphicImages}
\subsubsection{Images of a single algebra}\label{images-of-a-single-algebra}

We begin with what seems, for our purposes, the most useful way to represent the class of \defn{homomorphic images} of an algebra in dependent type theory.
\ccpad
\begin{code}%
\>[0]\AgdaKeyword{module}\AgdaSpace{}%
\AgdaModule{\AgdaUnderscore{}}\AgdaSpace{}%
\AgdaSymbol{\{}\AgdaBound{𝓤}\AgdaSpace{}%
\AgdaBound{𝓦}\AgdaSpace{}%
\AgdaSymbol{:}\AgdaSpace{}%
\AgdaFunction{Universe}\AgdaSymbol{\}}\AgdaSpace{}%
\AgdaKeyword{where}\<%
\\
%
\\[\AgdaEmptyExtraSkip]%
\>[0][@{}l@{\AgdaIndent{0}}]%
\>[1]\AgdaFunction{HomImage}\AgdaSpace{}%
\AgdaSymbol{:}\AgdaSpace{}%
\AgdaSymbol{\{}\AgdaBound{𝑨}\AgdaSpace{}%
\AgdaSymbol{:}\AgdaSpace{}%
\AgdaFunction{Algebra}\AgdaSpace{}%
\AgdaBound{𝓤}\AgdaSpace{}%
\AgdaBound{𝑆}\AgdaSymbol{\}(}\AgdaBound{𝑩}\AgdaSpace{}%
\AgdaSymbol{:}\AgdaSpace{}%
\AgdaFunction{Algebra}\AgdaSpace{}%
\AgdaBound{𝓦}\AgdaSpace{}%
\AgdaBound{𝑆}\AgdaSymbol{)(}\AgdaBound{ϕ}\AgdaSpace{}%
\AgdaSymbol{:}\AgdaSpace{}%
\AgdaFunction{hom}\AgdaSpace{}%
\AgdaBound{𝑨}\AgdaSpace{}%
\AgdaBound{𝑩}\AgdaSymbol{)}\AgdaSpace{}%
\AgdaSymbol{→}\AgdaSpace{}%
\AgdaOperator{\AgdaFunction{∣}}\AgdaSpace{}%
\AgdaBound{𝑩}\AgdaSpace{}%
\AgdaOperator{\AgdaFunction{∣}}\AgdaSpace{}%
\AgdaSymbol{→}\AgdaSpace{}%
\AgdaBound{𝓤}\AgdaSpace{}%
\AgdaOperator{\AgdaFunction{⊔}}\AgdaSpace{}%
\AgdaBound{𝓦}\AgdaSpace{}%
\AgdaOperator{\AgdaFunction{̇}}\<%
\\
%
\>[1]\AgdaFunction{HomImage}\AgdaSpace{}%
\AgdaBound{𝑩}\AgdaSpace{}%
\AgdaBound{ϕ}\AgdaSpace{}%
\AgdaSymbol{=}\AgdaSpace{}%
\AgdaSymbol{λ}\AgdaSpace{}%
\AgdaBound{b}\AgdaSpace{}%
\AgdaSymbol{→}\AgdaSpace{}%
\AgdaOperator{\AgdaDatatype{Image}}\AgdaSpace{}%
\AgdaOperator{\AgdaFunction{∣}}\AgdaSpace{}%
\AgdaBound{ϕ}\AgdaSpace{}%
\AgdaOperator{\AgdaFunction{∣}}\AgdaSpace{}%
\AgdaOperator{\AgdaDatatype{∋}}\AgdaSpace{}%
\AgdaBound{b}\<%
\\
%
\\[\AgdaEmptyExtraSkip]%
%
\>[1]\AgdaFunction{HomImagesOf}\AgdaSpace{}%
\AgdaSymbol{:}\AgdaSpace{}%
\AgdaFunction{Algebra}\AgdaSpace{}%
\AgdaBound{𝓤}\AgdaSpace{}%
\AgdaBound{𝑆}\AgdaSpace{}%
\AgdaSymbol{→}\AgdaSpace{}%
\AgdaBound{𝓞}\AgdaSpace{}%
\AgdaOperator{\AgdaFunction{⊔}}\AgdaSpace{}%
\AgdaBound{𝓥}\AgdaSpace{}%
\AgdaOperator{\AgdaFunction{⊔}}\AgdaSpace{}%
\AgdaBound{𝓤}\AgdaSpace{}%
\AgdaOperator{\AgdaFunction{⊔}}\AgdaSpace{}%
\AgdaBound{𝓦}\AgdaSpace{}%
\AgdaOperator{\AgdaFunction{⁺}}\AgdaSpace{}%
\AgdaOperator{\AgdaFunction{̇}}\<%
\\
%
\>[1]\AgdaFunction{HomImagesOf}\AgdaSpace{}%
\AgdaBound{𝑨}\AgdaSpace{}%
\AgdaSymbol{=}\AgdaSpace{}%
\AgdaFunction{Σ}\AgdaSpace{}%
\AgdaBound{𝑩}\AgdaSpace{}%
\AgdaFunction{꞉}\AgdaSpace{}%
\AgdaSymbol{(}\AgdaFunction{Algebra}\AgdaSpace{}%
\AgdaBound{𝓦}\AgdaSpace{}%
\AgdaBound{𝑆}\AgdaSymbol{)}\AgdaSpace{}%
\AgdaFunction{,}\AgdaSpace{}%
\AgdaFunction{Σ}\AgdaSpace{}%
\AgdaBound{ϕ}\AgdaSpace{}%
\AgdaFunction{꞉}\AgdaSpace{}%
\AgdaSymbol{(}\AgdaOperator{\AgdaFunction{∣}}\AgdaSpace{}%
\AgdaBound{𝑨}\AgdaSpace{}%
\AgdaOperator{\AgdaFunction{∣}}\AgdaSpace{}%
\AgdaSymbol{→}\AgdaSpace{}%
\AgdaOperator{\AgdaFunction{∣}}\AgdaSpace{}%
\AgdaBound{𝑩}\AgdaSpace{}%
\AgdaOperator{\AgdaFunction{∣}}\AgdaSymbol{)}\AgdaSpace{}%
\AgdaFunction{,}\AgdaSpace{}%
\AgdaFunction{is-homomorphism}\AgdaSpace{}%
\AgdaBound{𝑨}\AgdaSpace{}%
\AgdaBound{𝑩}\AgdaSpace{}%
\AgdaBound{ϕ}\AgdaSpace{}%
\AgdaOperator{\AgdaFunction{×}}\AgdaSpace{}%
\AgdaFunction{Epic}\AgdaSpace{}%
\AgdaBound{ϕ}\<%
\end{code}
\ccpad
These types should be self-explanatory, but just to be sure, let's describe the Sigma type appearing in the second definition. Given an \ab{𝑆}-algebra \ab{𝑨} \as : \af{Algebra} \ab 𝓤 \ab 𝑆, the type
\af{HomImagesOf} \ab 𝑨 denotes the class of algebras \ab{𝑩} \as : \af{Algebra} \ab 𝓦 \ab 𝑆 with a map \ab{φ} \as : \af{∣} \ab 𝑨 \af ∣ \as → \af ∣ \ab 𝑩 \af ∣ such that \ab{φ} is a surjective homomorphism.

Since we take the class of homomorphic images of an algebra to be closed under isomorphism, we consider \ab{𝑩} to be a homomorphic image of \ab{𝑨} if there exists an algebra \ab{𝑪} which is a homomorphic image of \ab{𝑨} and isomorphic to \ab{𝑩}. The following type expresses this.
\ccpad
\begin{code}%
\>[0][@{}l@{\AgdaIndent{1}}]%
\>[1]\AgdaOperator{\AgdaFunction{\AgdaUnderscore{}is-hom-image-of\AgdaUnderscore{}}}\AgdaSpace{}%
\AgdaSymbol{:}\AgdaSpace{}%
\AgdaSymbol{(}\AgdaBound{𝑩}\AgdaSpace{}%
\AgdaSymbol{:}\AgdaSpace{}%
\AgdaFunction{Algebra}\AgdaSpace{}%
\AgdaBound{𝓦}\AgdaSpace{}%
\AgdaBound{𝑆}\AgdaSymbol{)(}\AgdaBound{𝑨}\AgdaSpace{}%
\AgdaSymbol{:}\AgdaSpace{}%
\AgdaFunction{Algebra}\AgdaSpace{}%
\AgdaBound{𝓤}\AgdaSpace{}%
\AgdaBound{𝑆}\AgdaSymbol{)}\AgdaSpace{}%
\AgdaSymbol{→}\AgdaSpace{}%
\AgdaBound{𝓞}\AgdaSpace{}%
\AgdaOperator{\AgdaFunction{⊔}}\AgdaSpace{}%
\AgdaBound{𝓥}\AgdaSpace{}%
\AgdaOperator{\AgdaFunction{⊔}}\AgdaSpace{}%
\AgdaBound{𝓤}\AgdaSpace{}%
\AgdaOperator{\AgdaFunction{⊔}}\AgdaSpace{}%
\AgdaBound{𝓦}\AgdaSpace{}%
\AgdaOperator{\AgdaFunction{⁺}}\AgdaSpace{}%
\AgdaOperator{\AgdaFunction{̇}}\<%
\\
%
\>[1]\AgdaBound{𝑩}\AgdaSpace{}%
\AgdaOperator{\AgdaFunction{is-hom-image-of}}\AgdaSpace{}%
\AgdaBound{𝑨}\AgdaSpace{}%
\AgdaSymbol{=}\AgdaSpace{}%
\AgdaFunction{Σ}\AgdaSpace{}%
\AgdaBound{𝑪ϕ}\AgdaSpace{}%
\AgdaFunction{꞉}\AgdaSpace{}%
\AgdaSymbol{(}\AgdaFunction{HomImagesOf}\AgdaSpace{}%
\AgdaBound{𝑨}\AgdaSymbol{)}\AgdaSpace{}%
\AgdaFunction{,}\AgdaSpace{}%
\AgdaOperator{\AgdaFunction{∣}}\AgdaSpace{}%
\AgdaBound{𝑪ϕ}\AgdaSpace{}%
\AgdaOperator{\AgdaFunction{∣}}\AgdaSpace{}%
\AgdaOperator{\AgdaFunction{≅}}\AgdaSpace{}%
\AgdaBound{𝑩}\<%
\end{code}
\ccpad
\subsubsection{Images of a class of algebras}\label{images-of-a-class-of-algebras}

Given a class \ab{𝒦} of \ab{𝑆}-algebras, we need a type that expresses the assertion that a given algebra is a \emph{homomorphic image} of some algebra in the class, as well as a type that represents all such homomorphic images.
\ccpad
\begin{code}%
\>[0]\AgdaKeyword{module}\AgdaSpace{}%
\AgdaModule{\AgdaUnderscore{}}\AgdaSpace{}%
\AgdaSymbol{\{}\AgdaBound{𝓤}\AgdaSpace{}%
\AgdaSymbol{:}\AgdaSpace{}%
\AgdaFunction{Universe}\AgdaSymbol{\}}\AgdaSpace{}%
\AgdaKeyword{where}\<%
\\
%
\\[\AgdaEmptyExtraSkip]%
\>[0][@{}l@{\AgdaIndent{0}}]%
\>[1]\AgdaOperator{\AgdaFunction{\AgdaUnderscore{}is-hom-image-of-class\AgdaUnderscore{}}}\AgdaSpace{}%
\AgdaSymbol{:}\AgdaSpace{}%
\AgdaFunction{Algebra}\AgdaSpace{}%
\AgdaBound{𝓤}\AgdaSpace{}%
\AgdaBound{𝑆}\AgdaSpace{}%
\AgdaSymbol{→}\AgdaSpace{}%
\AgdaFunction{Pred}\AgdaSpace{}%
\AgdaSymbol{(}\AgdaFunction{Algebra}\AgdaSpace{}%
\AgdaBound{𝓤}\AgdaSpace{}%
\AgdaBound{𝑆}\AgdaSymbol{)(}\AgdaBound{𝓤}\AgdaSpace{}%
\AgdaOperator{\AgdaFunction{⁺}}\AgdaSymbol{)}\AgdaSpace{}%
\AgdaSymbol{→}\AgdaSpace{}%
\AgdaBound{𝓞}\AgdaSpace{}%
\AgdaOperator{\AgdaFunction{⊔}}\AgdaSpace{}%
\AgdaBound{𝓥}\AgdaSpace{}%
\AgdaOperator{\AgdaFunction{⊔}}\AgdaSpace{}%
\AgdaBound{𝓤}\AgdaSpace{}%
\AgdaOperator{\AgdaFunction{⁺}}\AgdaSpace{}%
\AgdaOperator{\AgdaFunction{̇}}\<%
\\
%
\>[1]\AgdaBound{𝑩}\AgdaSpace{}%
\AgdaOperator{\AgdaFunction{is-hom-image-of-class}}\AgdaSpace{}%
\AgdaBound{𝓚}\AgdaSpace{}%
\AgdaSymbol{=}\AgdaSpace{}%
\AgdaFunction{Σ}\AgdaSpace{}%
\AgdaBound{𝑨}\AgdaSpace{}%
\AgdaFunction{꞉}\AgdaSpace{}%
\AgdaSymbol{(}\AgdaFunction{Algebra}\AgdaSpace{}%
\AgdaBound{𝓤}\AgdaSpace{}%
\AgdaBound{𝑆}\AgdaSymbol{)}\AgdaSpace{}%
\AgdaFunction{,}\AgdaSpace{}%
\AgdaSymbol{(}\AgdaBound{𝑨}\AgdaSpace{}%
\AgdaOperator{\AgdaFunction{∈}}\AgdaSpace{}%
\AgdaBound{𝓚}\AgdaSymbol{)}\AgdaSpace{}%
\AgdaOperator{\AgdaFunction{×}}\AgdaSpace{}%
\AgdaSymbol{(}\AgdaBound{𝑩}\AgdaSpace{}%
\AgdaOperator{\AgdaFunction{is-hom-image-of}}\AgdaSpace{}%
\AgdaBound{𝑨}\AgdaSymbol{)}\<%
\\
%
\\[\AgdaEmptyExtraSkip]%
%
\>[1]\AgdaFunction{HomImagesOfClass}\AgdaSpace{}%
\AgdaSymbol{:}\AgdaSpace{}%
\AgdaFunction{Pred}\AgdaSpace{}%
\AgdaSymbol{(}\AgdaFunction{Algebra}\AgdaSpace{}%
\AgdaBound{𝓤}\AgdaSpace{}%
\AgdaBound{𝑆}\AgdaSymbol{)}\AgdaSpace{}%
\AgdaSymbol{(}\AgdaBound{𝓤}\AgdaSpace{}%
\AgdaOperator{\AgdaFunction{⁺}}\AgdaSymbol{)}\AgdaSpace{}%
\AgdaSymbol{→}\AgdaSpace{}%
\AgdaBound{𝓞}\AgdaSpace{}%
\AgdaOperator{\AgdaFunction{⊔}}\AgdaSpace{}%
\AgdaBound{𝓥}\AgdaSpace{}%
\AgdaOperator{\AgdaFunction{⊔}}\AgdaSpace{}%
\AgdaBound{𝓤}\AgdaSpace{}%
\AgdaOperator{\AgdaFunction{⁺}}\AgdaSpace{}%
\AgdaOperator{\AgdaFunction{̇}}\<%
\\
%
\>[1]\AgdaFunction{HomImagesOfClass}\AgdaSpace{}%
\AgdaBound{𝓚}\AgdaSpace{}%
\AgdaSymbol{=}\AgdaSpace{}%
\AgdaFunction{Σ}\AgdaSpace{}%
\AgdaBound{𝑩}\AgdaSpace{}%
\AgdaFunction{꞉}\AgdaSpace{}%
\AgdaSymbol{(}\AgdaFunction{Algebra}\AgdaSpace{}%
\AgdaBound{𝓤}\AgdaSpace{}%
\AgdaBound{𝑆}\AgdaSymbol{)}\AgdaSpace{}%
\AgdaFunction{,}\AgdaSpace{}%
\AgdaSymbol{(}\AgdaBound{𝑩}\AgdaSpace{}%
\AgdaOperator{\AgdaFunction{is-hom-image-of-class}}\AgdaSpace{}%
\AgdaBound{𝓚}\AgdaSymbol{)}\<%
\end{code}
\ccpad
\subsubsection{Lifting tools}\label{lifting-tools}

Here are some tools that have been useful (e.g., in the road to the proof of Birkhoff's HSP theorem).

The first states and proves the simple fact that the lift of an epimorphism is an epimorphism.
\ccpad
\begin{code}%
\>[0]\AgdaKeyword{open}\AgdaSpace{}%
\AgdaModule{Lift}\<%
\\
%
\\[\AgdaEmptyExtraSkip]%
\>[0]\AgdaKeyword{module}\AgdaSpace{}%
\AgdaModule{\AgdaUnderscore{}}\AgdaSpace{}%
\AgdaSymbol{\{}\AgdaBound{𝓧}\AgdaSpace{}%
\AgdaBound{𝓨}\AgdaSpace{}%
\AgdaSymbol{:}\AgdaSpace{}%
\AgdaFunction{Universe}\AgdaSymbol{\}}\AgdaSpace{}%
\AgdaKeyword{where}\<%
\\
%
\\[\AgdaEmptyExtraSkip]%
\>[0][@{}l@{\AgdaIndent{0}}]%
\>[1]\AgdaFunction{lift-of-alg-epic-is-epic}\AgdaSpace{}%
\AgdaSymbol{:}%
\>[220I]\AgdaSymbol{(}\AgdaBound{𝓩}\AgdaSpace{}%
\AgdaSymbol{:}\AgdaSpace{}%
\AgdaFunction{Universe}\AgdaSymbol{)\{}\AgdaBound{𝓦}\AgdaSpace{}%
\AgdaSymbol{:}\AgdaSpace{}%
\AgdaFunction{Universe}\AgdaSymbol{\}}\<%
\\
\>[.][@{}l@{}]\<[220I]%
\>[28]\AgdaSymbol{\{}\AgdaBound{𝑨}\AgdaSpace{}%
\AgdaSymbol{:}\AgdaSpace{}%
\AgdaFunction{Algebra}\AgdaSpace{}%
\AgdaBound{𝓧}\AgdaSpace{}%
\AgdaBound{𝑆}\AgdaSymbol{\}(}\AgdaBound{𝑩}\AgdaSpace{}%
\AgdaSymbol{:}\AgdaSpace{}%
\AgdaFunction{Algebra}\AgdaSpace{}%
\AgdaBound{𝓨}\AgdaSpace{}%
\AgdaBound{𝑆}\AgdaSymbol{)(}\AgdaBound{h}\AgdaSpace{}%
\AgdaSymbol{:}\AgdaSpace{}%
\AgdaFunction{hom}\AgdaSpace{}%
\AgdaBound{𝑨}\AgdaSpace{}%
\AgdaBound{𝑩}\AgdaSymbol{)}\<%
\\
%
\>[28]\AgdaComment{-----------------------------------------------}\<%
\\
\>[1][@{}l@{\AgdaIndent{0}}]%
\>[2]\AgdaSymbol{→}%
\>[28]\AgdaFunction{Epic}\AgdaSpace{}%
\AgdaOperator{\AgdaFunction{∣}}\AgdaSpace{}%
\AgdaBound{h}\AgdaSpace{}%
\AgdaOperator{\AgdaFunction{∣}}%
\>[40]\AgdaSymbol{→}%
\>[43]\AgdaFunction{Epic}\AgdaSpace{}%
\AgdaOperator{\AgdaFunction{∣}}\AgdaSpace{}%
\AgdaFunction{lift-alg-hom}\AgdaSpace{}%
\AgdaBound{𝓩}\AgdaSpace{}%
\AgdaBound{𝓦}\AgdaSpace{}%
\AgdaBound{𝑩}\AgdaSpace{}%
\AgdaBound{h}\AgdaSpace{}%
\AgdaOperator{\AgdaFunction{∣}}\<%
\\
%
\\[\AgdaEmptyExtraSkip]%
%
\>[1]\AgdaFunction{lift-of-alg-epic-is-epic}\AgdaSpace{}%
\AgdaBound{𝓩}\AgdaSpace{}%
\AgdaSymbol{\{}\AgdaBound{𝓦}\AgdaSymbol{\}}\AgdaSpace{}%
\AgdaSymbol{\{}\AgdaBound{𝑨}\AgdaSymbol{\}}\AgdaSpace{}%
\AgdaBound{𝑩}\AgdaSpace{}%
\AgdaBound{h}\AgdaSpace{}%
\AgdaBound{hepi}\AgdaSpace{}%
\AgdaBound{y}\AgdaSpace{}%
\AgdaSymbol{=}\AgdaSpace{}%
\AgdaInductiveConstructor{eq}\AgdaSpace{}%
\AgdaBound{y}\AgdaSpace{}%
\AgdaSymbol{(}\AgdaInductiveConstructor{lift}\AgdaSpace{}%
\AgdaFunction{a}\AgdaSymbol{)}\AgdaSpace{}%
\AgdaFunction{η}\<%
\\
\>[1][@{}l@{\AgdaIndent{0}}]%
\>[2]\AgdaKeyword{where}\<%
\\
%
\>[2]\AgdaFunction{lh}\AgdaSpace{}%
\AgdaSymbol{:}\AgdaSpace{}%
\AgdaFunction{hom}\AgdaSpace{}%
\AgdaSymbol{(}\AgdaFunction{lift-alg}\AgdaSpace{}%
\AgdaBound{𝑨}\AgdaSpace{}%
\AgdaBound{𝓩}\AgdaSymbol{)}\AgdaSpace{}%
\AgdaSymbol{(}\AgdaFunction{lift-alg}\AgdaSpace{}%
\AgdaBound{𝑩}\AgdaSpace{}%
\AgdaBound{𝓦}\AgdaSymbol{)}\<%
\\
%
\>[2]\AgdaFunction{lh}\AgdaSpace{}%
\AgdaSymbol{=}\AgdaSpace{}%
\AgdaFunction{lift-alg-hom}\AgdaSpace{}%
\AgdaBound{𝓩}\AgdaSpace{}%
\AgdaBound{𝓦}\AgdaSpace{}%
\AgdaBound{𝑩}\AgdaSpace{}%
\AgdaBound{h}\<%
\\
%
\\[\AgdaEmptyExtraSkip]%
%
\>[2]\AgdaFunction{ζ}\AgdaSpace{}%
\AgdaSymbol{:}\AgdaSpace{}%
\AgdaOperator{\AgdaDatatype{Image}}\AgdaSpace{}%
\AgdaOperator{\AgdaFunction{∣}}\AgdaSpace{}%
\AgdaBound{h}\AgdaSpace{}%
\AgdaOperator{\AgdaFunction{∣}}\AgdaSpace{}%
\AgdaOperator{\AgdaDatatype{∋}}\AgdaSpace{}%
\AgdaSymbol{(}\AgdaField{lower}\AgdaSpace{}%
\AgdaBound{y}\AgdaSymbol{)}\<%
\\
%
\>[2]\AgdaFunction{ζ}\AgdaSpace{}%
\AgdaSymbol{=}\AgdaSpace{}%
\AgdaBound{hepi}\AgdaSpace{}%
\AgdaSymbol{(}\AgdaField{lower}\AgdaSpace{}%
\AgdaBound{y}\AgdaSymbol{)}\<%
\\
%
\\[\AgdaEmptyExtraSkip]%
%
\>[2]\AgdaFunction{a}\AgdaSpace{}%
\AgdaSymbol{:}\AgdaSpace{}%
\AgdaOperator{\AgdaFunction{∣}}\AgdaSpace{}%
\AgdaBound{𝑨}\AgdaSpace{}%
\AgdaOperator{\AgdaFunction{∣}}\<%
\\
%
\>[2]\AgdaFunction{a}\AgdaSpace{}%
\AgdaSymbol{=}\AgdaSpace{}%
\AgdaFunction{Inv}\AgdaSpace{}%
\AgdaOperator{\AgdaFunction{∣}}\AgdaSpace{}%
\AgdaBound{h}\AgdaSpace{}%
\AgdaOperator{\AgdaFunction{∣}}\AgdaSpace{}%
\AgdaFunction{ζ}\<%
\\
%
\\[\AgdaEmptyExtraSkip]%
%
\>[2]\AgdaFunction{β}\AgdaSpace{}%
\AgdaSymbol{:}\AgdaSpace{}%
\AgdaInductiveConstructor{lift}\AgdaSpace{}%
\AgdaSymbol{(}\AgdaOperator{\AgdaFunction{∣}}\AgdaSpace{}%
\AgdaBound{h}\AgdaSpace{}%
\AgdaOperator{\AgdaFunction{∣}}\AgdaSpace{}%
\AgdaFunction{a}\AgdaSymbol{)}\AgdaSpace{}%
\AgdaOperator{\AgdaDatatype{≡}}\AgdaSpace{}%
\AgdaSymbol{(}\AgdaInductiveConstructor{lift}\AgdaSpace{}%
\AgdaOperator{\AgdaFunction{∘}}\AgdaSpace{}%
\AgdaOperator{\AgdaFunction{∣}}\AgdaSpace{}%
\AgdaBound{h}\AgdaSpace{}%
\AgdaOperator{\AgdaFunction{∣}}\AgdaSpace{}%
\AgdaOperator{\AgdaFunction{∘}}\AgdaSpace{}%
\AgdaField{lower}\AgdaSymbol{\{}\AgdaBound{𝓦}\AgdaSymbol{\})}\AgdaSpace{}%
\AgdaSymbol{(}\AgdaInductiveConstructor{lift}\AgdaSpace{}%
\AgdaFunction{a}\AgdaSymbol{)}\<%
\\
%
\>[2]\AgdaFunction{β}\AgdaSpace{}%
\AgdaSymbol{=}\AgdaSpace{}%
\AgdaFunction{ap}\AgdaSpace{}%
\AgdaSymbol{(λ}\AgdaSpace{}%
\AgdaBound{-}\AgdaSpace{}%
\AgdaSymbol{→}\AgdaSpace{}%
\AgdaInductiveConstructor{lift}\AgdaSpace{}%
\AgdaSymbol{(}\AgdaOperator{\AgdaFunction{∣}}\AgdaSpace{}%
\AgdaBound{h}\AgdaSpace{}%
\AgdaOperator{\AgdaFunction{∣}}\AgdaSpace{}%
\AgdaSymbol{(}\AgdaSpace{}%
\AgdaBound{-}\AgdaSpace{}%
\AgdaFunction{a}\AgdaSymbol{)))}\AgdaSpace{}%
\AgdaSymbol{(}\AgdaFunction{lower∼lift}\AgdaSpace{}%
\AgdaSymbol{\{}\AgdaBound{𝓦}\AgdaSymbol{\}}\AgdaSpace{}%
\AgdaSymbol{)}\<%
\\
%
\\[\AgdaEmptyExtraSkip]%
%
\>[2]\AgdaFunction{η}\AgdaSpace{}%
\AgdaSymbol{:}\AgdaSpace{}%
\AgdaBound{y}\AgdaSpace{}%
\AgdaOperator{\AgdaDatatype{≡}}\AgdaSpace{}%
\AgdaOperator{\AgdaFunction{∣}}\AgdaSpace{}%
\AgdaFunction{lh}\AgdaSpace{}%
\AgdaOperator{\AgdaFunction{∣}}\AgdaSpace{}%
\AgdaSymbol{(}\AgdaInductiveConstructor{lift}\AgdaSpace{}%
\AgdaFunction{a}\AgdaSymbol{)}\<%
\\
%
\>[2]\AgdaFunction{η}\AgdaSpace{}%
\AgdaSymbol{=}%
\>[336I]\AgdaBound{y}%
\>[22]\AgdaOperator{\AgdaFunction{≡⟨}}\AgdaSpace{}%
\AgdaSymbol{(}\AgdaFunction{extfun}\AgdaSpace{}%
\AgdaFunction{lift∼lower}\AgdaSymbol{)}\AgdaSpace{}%
\AgdaBound{y}\AgdaSpace{}%
\AgdaOperator{\AgdaFunction{⟩}}\<%
\\
\>[.][@{}l@{}]\<[336I]%
\>[6]\AgdaInductiveConstructor{lift}\AgdaSpace{}%
\AgdaSymbol{(}\AgdaField{lower}\AgdaSpace{}%
\AgdaBound{y}\AgdaSymbol{)}%
\>[22]\AgdaOperator{\AgdaFunction{≡⟨}}\AgdaSpace{}%
\AgdaFunction{ap}\AgdaSpace{}%
\AgdaInductiveConstructor{lift}\AgdaSpace{}%
\AgdaSymbol{(}\AgdaFunction{InvIsInv}\AgdaSpace{}%
\AgdaOperator{\AgdaFunction{∣}}\AgdaSpace{}%
\AgdaBound{h}\AgdaSpace{}%
\AgdaOperator{\AgdaFunction{∣}}\AgdaSpace{}%
\AgdaFunction{ζ}\AgdaSymbol{)}\AgdaOperator{\AgdaFunction{⁻¹}}\AgdaSpace{}%
\AgdaOperator{\AgdaFunction{⟩}}\<%
\\
%
\>[6]\AgdaInductiveConstructor{lift}\AgdaSpace{}%
\AgdaSymbol{(}\AgdaOperator{\AgdaFunction{∣}}\AgdaSpace{}%
\AgdaBound{h}\AgdaSpace{}%
\AgdaOperator{\AgdaFunction{∣}}\AgdaSpace{}%
\AgdaFunction{a}\AgdaSymbol{)}%
\>[22]\AgdaOperator{\AgdaFunction{≡⟨}}\AgdaSpace{}%
\AgdaFunction{β}\AgdaSpace{}%
\AgdaOperator{\AgdaFunction{⟩}}\<%
\\
%
\>[6]\AgdaOperator{\AgdaFunction{∣}}\AgdaSpace{}%
\AgdaFunction{lh}\AgdaSpace{}%
\AgdaOperator{\AgdaFunction{∣}}\AgdaSpace{}%
\AgdaSymbol{(}\AgdaInductiveConstructor{lift}\AgdaSpace{}%
\AgdaFunction{a}\AgdaSymbol{)}\AgdaSpace{}%
\AgdaOperator{\AgdaFunction{∎}}\<%
\\
%
\\[\AgdaEmptyExtraSkip]%
%
\\[\AgdaEmptyExtraSkip]%
%
\>[1]\AgdaFunction{lift-alg-hom-image}\AgdaSpace{}%
\AgdaSymbol{:}%
\>[363I]\AgdaSymbol{\{}\AgdaBound{𝓩}\AgdaSpace{}%
\AgdaBound{𝓦}\AgdaSpace{}%
\AgdaSymbol{:}\AgdaSpace{}%
\AgdaFunction{Universe}\AgdaSymbol{\}}\<%
\\
\>[.][@{}l@{}]\<[363I]%
\>[22]\AgdaSymbol{\{}\AgdaBound{𝑨}\AgdaSpace{}%
\AgdaSymbol{:}\AgdaSpace{}%
\AgdaFunction{Algebra}\AgdaSpace{}%
\AgdaBound{𝓧}\AgdaSpace{}%
\AgdaBound{𝑆}\AgdaSymbol{\}\{}\AgdaBound{𝑩}\AgdaSpace{}%
\AgdaSymbol{:}\AgdaSpace{}%
\AgdaFunction{Algebra}\AgdaSpace{}%
\AgdaBound{𝓨}\AgdaSpace{}%
\AgdaBound{𝑆}\AgdaSymbol{\}}\<%
\\
\>[1][@{}l@{\AgdaIndent{0}}]%
\>[2]\AgdaSymbol{→}%
\>[22]\AgdaBound{𝑩}\AgdaSpace{}%
\AgdaOperator{\AgdaFunction{is-hom-image-of}}\AgdaSpace{}%
\AgdaBound{𝑨}\<%
\\
%
\>[22]\AgdaComment{-----------------------------------------------}\<%
\\
%
\>[2]\AgdaSymbol{→}%
\>[22]\AgdaSymbol{(}\AgdaFunction{lift-alg}\AgdaSpace{}%
\AgdaBound{𝑩}\AgdaSpace{}%
\AgdaBound{𝓦}\AgdaSymbol{)}\AgdaSpace{}%
\AgdaOperator{\AgdaFunction{is-hom-image-of}}\AgdaSpace{}%
\AgdaSymbol{(}\AgdaFunction{lift-alg}\AgdaSpace{}%
\AgdaBound{𝑨}\AgdaSpace{}%
\AgdaBound{𝓩}\AgdaSymbol{)}\<%
\\
%
\\[\AgdaEmptyExtraSkip]%
%
\>[1]\AgdaFunction{lift-alg-hom-image}\AgdaSpace{}%
\AgdaSymbol{\{}\AgdaBound{𝓩}\AgdaSymbol{\}\{}\AgdaBound{𝓦}\AgdaSymbol{\}\{}\AgdaBound{𝑨}\AgdaSymbol{\}\{}\AgdaBound{𝑩}\AgdaSymbol{\}}\AgdaSpace{}%
\AgdaSymbol{((}\AgdaBound{𝑪}\AgdaSpace{}%
\AgdaOperator{\AgdaInductiveConstructor{,}}\AgdaSpace{}%
\AgdaBound{ϕ}\AgdaSpace{}%
\AgdaOperator{\AgdaInductiveConstructor{,}}\AgdaSpace{}%
\AgdaBound{ϕhom}\AgdaSpace{}%
\AgdaOperator{\AgdaInductiveConstructor{,}}\AgdaSpace{}%
\AgdaBound{ϕepic}\AgdaSymbol{)}\AgdaSpace{}%
\AgdaOperator{\AgdaInductiveConstructor{,}}\AgdaSpace{}%
\AgdaBound{C≅B}\AgdaSymbol{)}\AgdaSpace{}%
\AgdaSymbol{=}\<%
\\
\>[1][@{}l@{\AgdaIndent{0}}]%
\>[2]\AgdaSymbol{(}\AgdaFunction{lift-alg}\AgdaSpace{}%
\AgdaBound{𝑪}\AgdaSpace{}%
\AgdaBound{𝓦}\AgdaSpace{}%
\AgdaOperator{\AgdaInductiveConstructor{,}}\AgdaSpace{}%
\AgdaOperator{\AgdaFunction{∣}}\AgdaSpace{}%
\AgdaFunction{lϕ}\AgdaSpace{}%
\AgdaOperator{\AgdaFunction{∣}}\AgdaSpace{}%
\AgdaOperator{\AgdaInductiveConstructor{,}}\AgdaSpace{}%
\AgdaOperator{\AgdaFunction{∥}}\AgdaSpace{}%
\AgdaFunction{lϕ}\AgdaSpace{}%
\AgdaOperator{\AgdaFunction{∥}}\AgdaSpace{}%
\AgdaOperator{\AgdaInductiveConstructor{,}}\AgdaSpace{}%
\AgdaFunction{lϕepic}\AgdaSymbol{)}\AgdaSpace{}%
\AgdaOperator{\AgdaInductiveConstructor{,}}\AgdaSpace{}%
\AgdaFunction{lift-alg-iso}\AgdaSpace{}%
\AgdaBound{C≅B}\<%
\\
\>[2][@{}l@{\AgdaIndent{0}}]%
\>[3]\AgdaKeyword{where}\<%
\\
%
\>[3]\AgdaFunction{lϕ}\AgdaSpace{}%
\AgdaSymbol{:}\AgdaSpace{}%
\AgdaFunction{hom}\AgdaSpace{}%
\AgdaSymbol{(}\AgdaFunction{lift-alg}\AgdaSpace{}%
\AgdaBound{𝑨}\AgdaSpace{}%
\AgdaBound{𝓩}\AgdaSymbol{)}\AgdaSpace{}%
\AgdaSymbol{(}\AgdaFunction{lift-alg}\AgdaSpace{}%
\AgdaBound{𝑪}\AgdaSpace{}%
\AgdaBound{𝓦}\AgdaSymbol{)}\<%
\\
%
\>[3]\AgdaFunction{lϕ}\AgdaSpace{}%
\AgdaSymbol{=}\AgdaSpace{}%
\AgdaSymbol{(}\AgdaFunction{lift-alg-hom}\AgdaSpace{}%
\AgdaBound{𝓩}\AgdaSpace{}%
\AgdaBound{𝓦}\AgdaSpace{}%
\AgdaBound{𝑪}\AgdaSymbol{)}\AgdaSpace{}%
\AgdaSymbol{(}\AgdaBound{ϕ}\AgdaSpace{}%
\AgdaOperator{\AgdaInductiveConstructor{,}}\AgdaSpace{}%
\AgdaBound{ϕhom}\AgdaSymbol{)}\<%
\\
%
\\[\AgdaEmptyExtraSkip]%
%
\>[3]\AgdaFunction{lϕepic}\AgdaSpace{}%
\AgdaSymbol{:}\AgdaSpace{}%
\AgdaFunction{Epic}\AgdaSpace{}%
\AgdaOperator{\AgdaFunction{∣}}\AgdaSpace{}%
\AgdaFunction{lϕ}\AgdaSpace{}%
\AgdaOperator{\AgdaFunction{∣}}\<%
\\
%
\>[3]\AgdaFunction{lϕepic}\AgdaSpace{}%
\AgdaSymbol{=}\AgdaSpace{}%
\AgdaFunction{lift-of-alg-epic-is-epic}\AgdaSpace{}%
\AgdaBound{𝓩}\AgdaSpace{}%
\AgdaBound{𝑪}\AgdaSpace{}%
\AgdaSymbol{(}\AgdaBound{ϕ}\AgdaSpace{}%
\AgdaOperator{\AgdaInductiveConstructor{,}}\AgdaSpace{}%
\AgdaBound{ϕhom}\AgdaSymbol{)}\AgdaSpace{}%
\AgdaBound{ϕepic}\<%
\end{code}


%%% Local Variables:
%%% TeX-master: "ualib-part2-20210304.tex"
%%% End:

