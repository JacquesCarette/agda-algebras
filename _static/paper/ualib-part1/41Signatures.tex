% -*- TeX-master: "ualib-part1.tex" -*-
%%% Local Variables: 
%%% mode: latex
%%% TeX-engine: 'xetex
%%% End:
% \subsubsection{Operation type}\label{operation-type}
We begin by defining the type of \defn{operations} as follows.
\ccpad
\begin{code}%
% \>[0]\AgdaKeyword{module}\AgdaSpace{}%
% \AgdaModule{\AgdaUnderscore{}}\AgdaSpace{}%
% \AgdaSymbol{\{}\AgdaBound{𝓤}\AgdaSpace{}%
% \AgdaSymbol{:}\AgdaSpace{}%
% \AgdaPostulate{Universe}\AgdaSymbol{\}}\AgdaSpace{}%
% \AgdaKeyword{where}\<%
% \\
% %
% \\[\AgdaEmptyExtraSkip]%
\>[0][@{}l@{\AgdaIndent{0}}]%
% \>[1]\AgdaComment{--The type of operations}\<%
% \\
% %
\>[1]\AgdaFunction{Op}\AgdaSpace{}%
\AgdaSymbol{:}\AgdaSpace{}%
\AgdaGeneralizable{𝓥}\AgdaSpace{}%
\AgdaOperator{\AgdaFunction{̇}}\AgdaSpace{}%
\AgdaSymbol{→}\AgdaSpace{}%
\AgdaBound{𝓤}\AgdaSpace{}%
\AgdaOperator{\AgdaFunction{̇}}\AgdaSpace{}%
\AgdaSymbol{→}\AgdaSpace{}%
\AgdaBound{𝓤}\AgdaSpace{}%
\AgdaOperator{\AgdaPrimitive{⊔}}\AgdaSpace{}%
\AgdaGeneralizable{𝓥}\AgdaSpace{}%
\AgdaOperator{\AgdaFunction{̇}}\<%
\\
%
\>[1]\AgdaFunction{Op}\AgdaSpace{}%
\AgdaBound{I}\AgdaSpace{}%
\AgdaBound{A}\AgdaSpace{}%
\AgdaSymbol{=}\AgdaSpace{}%
\AgdaSymbol{(}\AgdaBound{I}\AgdaSpace{}%
\AgdaSymbol{→}\AgdaSpace{}%
\AgdaBound{A}\AgdaSymbol{)}\AgdaSpace{}%
\AgdaSymbol{→}\AgdaSpace{}%
\AgdaBound{A}\<%
\end{code}
\ccpad
The type \af{Op} encodes the arity of an operation as an arbitrary type \abt{I}{𝓥}, which gives us a very general way to represent an operation as a function type with domain \ab I \as → \ab A (the type of ``tuples'') and codomain \ab{A}. For example, the \ab{I}-\defn{ary projection operations} on \ab{A} can be codified as inhabitants of the type \af{Op} \ab I \af A in the following way.
\ccpad
\begin{code}%
\>[0]\AgdaFunction{π}\AgdaSpace{}%
\AgdaSymbol{:}\AgdaSpace{}%
\AgdaSymbol{\{}\AgdaBound{I}\AgdaSpace{}%
\AgdaSymbol{:}\AgdaSpace{}%
\AgdaGeneralizable{𝓥}\AgdaSpace{}%
\AgdaOperator{\AgdaFunction{̇}}%
\AgdaSymbol{\}}\AgdaSpace{}%
\AgdaSymbol{\{}\AgdaBound{A}\AgdaSpace{}%
\AgdaSymbol{:}\AgdaSpace{}%
\AgdaGeneralizable{𝓤}\AgdaSpace{}%
\AgdaOperator{\AgdaFunction{̇}}\AgdaSpace{}%
\AgdaSymbol{\}}\AgdaSpace{}%
\AgdaSymbol{→}\AgdaSpace{}%
\AgdaBound{I}\AgdaSpace{}%
\AgdaSymbol{→}\AgdaSpace{}%
\AgdaFunction{Op}\AgdaSpace{}%
\AgdaBound{I}\AgdaSpace{}%
\AgdaBound{A}\<%
\\
\>[0]\AgdaFunction{π}\AgdaSpace{}%
\AgdaBound{i}\AgdaSpace{}%
\AgdaBound{x}\AgdaSpace{}%
\AgdaSymbol{=}\AgdaSpace{}%
\AgdaBound{x}\AgdaSpace{}%
\AgdaBound{i}\<%
\end{code}

% \subsubsection{Signature type}\label{signature-type}

We define the signature of an algebraic structure in Agda like this.
\ccpad
\begin{code}%
\>[0]\AgdaFunction{Signature}\AgdaSpace{}%
\AgdaSymbol{:}\AgdaSpace{}%
\AgdaSymbol{(}\AgdaBound{𝓞}\AgdaSpace{}%
\AgdaBound{𝓥}\AgdaSpace{}%
\AgdaSymbol{:}\AgdaSpace{}%
\AgdaPostulate{Universe}\AgdaSymbol{)}\AgdaSpace{}%
\AgdaSymbol{→}\AgdaSpace{}%
\AgdaSymbol{(}\AgdaBound{𝓞}\AgdaSpace{}%
\AgdaOperator{\AgdaPrimitive{⊔}}\AgdaSpace{}%
\AgdaBound{𝓥}\AgdaSymbol{)}\AgdaSpace{}%
\AgdaOperator{\AgdaPrimitive{⁺}}\AgdaSpace{}%
\AgdaOperator{\AgdaFunction{̇}}\<%
\\
\>[0]\AgdaFunction{Signature}\AgdaSpace{}%
\AgdaBound{𝓞}\AgdaSpace{}%
\AgdaBound{𝓥}\AgdaSpace{}%
\AgdaSymbol{=}\AgdaSpace{}%
\AgdaFunction{Σ}\AgdaSpace{}%
\AgdaBound{F}\AgdaSpace{}%
\AgdaFunction{꞉}\AgdaSpace{}%
\AgdaBound{𝓞}\AgdaSpace{}%
\AgdaOperator{\AgdaFunction{̇}}\AgdaSpace{}%
\AgdaFunction{,}\AgdaSpace{}%
\AgdaSymbol{(}\AgdaBound{F}\AgdaSpace{}%
\AgdaSymbol{→}\AgdaSpace{}%
\AgdaBound{𝓥}\AgdaSpace{}%
\AgdaOperator{\AgdaFunction{̇}}\AgdaSymbol{)}\<%
\end{code}
\ccpad
As mentioned in \S\ref{sec:continuous-relations}, in the \ualib the symbol 𝓞 always denotes the universe of \emph{operation symbol} types, while 𝓥 is always the universe of \emph{arity} types.

In \S\ref{sec:dependent-types} we defined special syntax for the first and second projections---namely, \af ∣\au\af ∣ and \af{∥}\au\af{∥}, respectively. Consequently, if \ab 𝑆 \as : \af{Signature} \ab 𝓞 \ab 𝓥 is a signature, then \af ∣~\ab 𝑆~\af ∣ denotes the set of operation symbols, and \af{∥}~\ab 𝑆~\af{∥} denotes the arity function. If \ab 𝑓 \as : \af ∣~\ab 𝑆~\af ∣ is an operation symbol in the signature \ab 𝑆, then \af{∥}~\ab 𝑆~\af{∥} \ab 𝑓 is the arity of \ab 𝑓.

% For reference, we recall the definition of the Sigma type, \af{Σ}, which is

% \begin{Shaded}
% \begin{Highlighting}[]
% \NormalTok{-Σ }\OtherTok{:} \OtherTok{\{}\NormalTok{𝓤 𝓥 }\OtherTok{:}\NormalTok{ Universe}\OtherTok{\}(}\NormalTok{X }\OtherTok{:}\NormalTok{ 𝓤 ̇}\OtherTok{)(}\NormalTok{Y }\OtherTok{:}\NormalTok{ X }\OtherTok{→}\NormalTok{ 𝓥 ̇}\OtherTok{)} \OtherTok{→}\NormalTok{ 𝓤 ⊔ 𝓥 ̇}
% \NormalTok{-Σ X Y }\OtherTok{=}\NormalTok{ Σ Y}
% \end{Highlighting}
% \end{Shaded}

\paragraph{Example}\label{example}

Here is how we could define the signature for \defn{monoids} as an inhabitant of the type \af{Signature} \ab 𝓞 \ab 𝓥.
\ccpad
\begin{code}%
\>[1]\AgdaKeyword{data}\AgdaSpace{}%
\AgdaDatatype{monoid-op}\AgdaSpace{}%
\AgdaSymbol{:}\AgdaSpace{}%
\AgdaBound{𝓞}\AgdaSpace{}%
\AgdaOperator{\AgdaFunction{̇}}\AgdaSpace{}%
\AgdaKeyword{where}\<%
\\
\>[1][@{}l@{\AgdaIndent{0}}]%
\>[2]\AgdaInductiveConstructor{e}\AgdaSpace{}%
\AgdaSymbol{:}\AgdaSpace{}%
\AgdaDatatype{monoid-op}\<%
\\
%
\>[2]\AgdaInductiveConstructor{·}\AgdaSpace{}%
\AgdaSymbol{:}\AgdaSpace{}%
\AgdaDatatype{monoid-op}\<%
\\
%
\\[\AgdaEmptyExtraSkip]%
%
\>[1]\AgdaFunction{monoid-sig}\AgdaSpace{}%
\AgdaSymbol{:}\AgdaSpace{}%
\AgdaFunction{Signature}\AgdaSpace{}%
\AgdaBound{𝓞}\AgdaSpace{}%
\AgdaPrimitive{𝓤₀}\<%
\\
%
\>[1]\AgdaFunction{monoid-sig}\AgdaSpace{}%
\AgdaSymbol{=}\AgdaSpace{}%
\AgdaDatatype{monoid-op}\AgdaSpace{}%
\AgdaOperator{\AgdaInductiveConstructor{,}}\AgdaSpace{}%
\AgdaSymbol{λ}\AgdaSpace{}%
\AgdaSymbol{\{}\AgdaSpace{}%
\AgdaInductiveConstructor{e}\AgdaSpace{}%
\AgdaSymbol{→}\AgdaSpace{}%
\AgdaFunction{𝟘}\AgdaSymbol{;}\AgdaSpace{}%
\AgdaInductiveConstructor{·}\AgdaSpace{}%
\AgdaSymbol{→}\AgdaSpace{}%
\AgdaFunction{𝟚}\AgdaSpace{}%
\AgdaSymbol{\}}\<%
\end{code}
\ccpad
As expected, the signature for a monoid consists of two operation symbols, \aic{e} and \aic{·}, and a function
\as λ \as{\{} \aic e \as → \af 𝟘 \as ; \aic · \as → \af 𝟚 \as{\}} which maps \aic{e} to the empty type \af 𝟘 (since \aic{e} is the nullary identity) and maps \aic{·} to the two element type \af 𝟚 (since \aic{·} is binary).\footnote{The types \af 𝟘 and \af 𝟚 are defined in the \AgdaModule{MGS-MLTT} module of \typtop.}
