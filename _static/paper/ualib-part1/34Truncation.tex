% -*- TeX-master: "ualib-part1.tex" -*-
%%% Local Variables: 
%%% mode: latex
%%% TeX-engine: 'xetex
%%% End:
Here we discuss \defn{truncation} and \defn{h-sets} (which we just call \defn{sets}). We first give a brief discussion of standard notions of \emph{truncation} from a viewpoint that seems useful for formalizing mathematics in Agda.\footnote{Readers wishing to learn more about truncation may wish to consult~\cite[\S34]{MHE},~\cite{Brunerie:2012}, or~\cite[\S7.1]{HoTT}.}

\subsubsection{Background and motivation}\label{truncation}
This brief introduction to \emph{truncation} is intended for novices. Those already familiar with the concept might wish to skip to the next subsection.

In general, we may have multiple inhabitants of a given type, hence (via Curry-Howard) multiple proofs of a given proposition. For instance, suppose we have a type \ab{X} and an identity relation \aod{\_≡₀\_} on \ab{X} so that, given two inhabitants of \ab{X}, say, \ab{a} \ab b \as : \ab X, we can form the type \ab{a} \aod{≡₀} \ab b. Suppose \ab{p} and \ab{q} inhabit the type \ab{a} \aod{≡₀} \ab b; that is, \ab{p} and \ab{q} are proofs of \ab{a} \aod{≡₀} \ab b, in which case we write \ab{p} \ab q \as : \ab a \aod{≡₀} \ab b. We might then wonder whether the two proofs \ab{p} and \ab{q} are equivalent.

We are asking about an identity type on the identity type \aod{≡₀}, and whether there is some inhabitant, say, \ab{r} of this type; i.e., whether there is a proof \ab{r} \as : \ab p \aod{≡₁} \ab q that the proofs of \ab{a} \aod{≡₀} \ab{b} are the same. If such a proof exists for all \ab{p} \ab q \as : \ab{a} \aod{≡₀} \ab b, then the proof of \ab a \aod{≡₀} \ab b is unique; as a property of the types \ab{X} and \ad{≡₀}, this is sometimes called \defn{uniqueness of identity proofs}.

Now, perhaps we have two proofs, say, \ab{r} \ab s \as : \ab p \aod{≡₁} \ab q that the proofs \ab{p} and \ab{q} are equivalent. Then of course we wonder whether \ab{r} \aod{≡₂} \ab s has a proof!  But at some level we may decide that the potential to distinguish two proofs of an identity in a meaningful way (so-called \emph{proof-relevance}) is not useful or desirable. At that point, say, at level \ab{k}, we would be naturally inclined to assume that there is at most one proof of any identity of the form \ab{p} \aod{≡ₖ} \ab q. This is called \href{https://www.cs.bham.ac.uk/~mhe/HoTT-UF-in-Agda-Lecture-Notes/HoTT-UF-Agda.html\#truncation}{truncation} (at level \ab{k}).

\subsubsection{Sets}\label{sec:sets}
In \href{https://homotopytypetheory.org}{homotopy type theory}, a type \ab{X} with an identity relation \ad{≡₀} is called a \defn{set} (or \defn{0-groupoid}) if for every pair \ab{x} \abt{y}{X} there is at most one proof of \ab{x} \aod{≡₀} \ab y. In other words, the type \ab{X}, along with it's equality type \ad{≡ₓ}, form a \emph{set} if for all \ab{x} \abt{y}{X} there is at most one proof of \ab{x} \aod{≡₀} \ab y.

This notion is formalized in \typtop using the type \af{is-set} which is defined using the \af{is-subsingleton} type (\S\ref{sec:inverse-image-invers}) as follows.
\ccpad
\begin{code}%
\>[1]\AgdaFunction{is-set}\AgdaSpace{}%
\AgdaSymbol{:}\AgdaSpace{}%
\AgdaBound{𝓤}\AgdaSpace{}%
\AgdaOperator{\AgdaFunction{̇}}\AgdaSpace{}%
\AgdaSymbol{→}\AgdaSpace{}%
\AgdaBound{𝓤}\AgdaSpace{}%
\AgdaOperator{\AgdaFunction{̇}}\<%
\\
%
\>[1]\AgdaFunction{is-set}\AgdaSpace{}%
\AgdaBound{A}\AgdaSpace{}%
\AgdaSymbol{=}\AgdaSpace{}%
\AgdaSymbol{(}\AgdaBound{x}\AgdaSpace{}%
\AgdaBound{y}\AgdaSpace{}%
\AgdaSymbol{:}\AgdaSpace{}%
\AgdaBound{A}\AgdaSymbol{)}\AgdaSpace{}%
\AgdaSymbol{→}\AgdaSpace{}%
\AgdaFunction{is-subsingleton}\AgdaSpace{}%
\AgdaSymbol{(}\AgdaBound{x}\AgdaSpace{}%
\AgdaOperator{\AgdaDatatype{≡}}\AgdaSpace{}%
\AgdaBound{y}\AgdaSymbol{)}\<%
\end{code}
\ccpad
Thus, the pair (\ab{X} , \ad{≡₀}) forms a set iff it satisfies \as{∀} \ab x \abt{y}{X} \as → \af{is-subsingleton} (\ab x \aod{≡₀} \ab y).

We will also need the function
\href{https://www.cs.bham.ac.uk/~mhe/HoTT-UF-in-Agda-Lecture-Notes/HoTT-UF-Agda.html\#sigmaequality}{to-Σ-≡},
which is part of Escardó's characterization of \emph{equality in Sigma types} in~\cite{MHE} and is defined as follows.
\ccpad
\begin{code}%
\>[1]\AgdaFunction{to-Σ-≡}\AgdaSpace{}%
\AgdaSymbol{:}\AgdaSpace{}%
\>[111I]\AgdaSymbol{\{}\AgdaBound{A}\AgdaSpace{}%
\AgdaSymbol{:}\AgdaSpace{}%
\AgdaBound{𝓤}\AgdaSpace{}%
\AgdaOperator{\AgdaFunction{̇}}%
\AgdaSymbol{\}}\AgdaSpace{}%
\AgdaSymbol{\{}\AgdaBound{B}\AgdaSpace{}%
\AgdaSymbol{:}\AgdaSpace{}%
\AgdaBound{A}\AgdaSpace{}%
\AgdaSymbol{→}\AgdaSpace{}%
\AgdaBound{𝓦}\AgdaSpace{}%
\AgdaOperator{\AgdaFunction{̇}}%
\AgdaSymbol{\}}\AgdaSpace{}%
\AgdaSymbol{\{}\AgdaBound{σ}\AgdaSpace{}%
\AgdaBound{τ}\AgdaSpace{}%
\AgdaSymbol{:}\AgdaSpace{}%
\AgdaRecord{Σ}\AgdaSpace{}%
\AgdaBound{B}\AgdaSymbol{\}}\<%
\\
\>[1][@{}l@{\AgdaIndent{0}}]%
\>[2]\AgdaSymbol{→}%
\>[.][@{}l@{}]\<[111I]%
\>[10]\AgdaFunction{Σ}\AgdaSpace{}%
\AgdaBound{p}\AgdaSpace{}%
\AgdaFunction{꞉}\AgdaSpace{}%
\AgdaOperator{\AgdaFunction{∣}}\AgdaSpace{}%
\AgdaBound{σ}\AgdaSpace{}%
\AgdaOperator{\AgdaFunction{∣}}\AgdaSpace{}%
\AgdaOperator{\AgdaDatatype{≡}}\AgdaSpace{}%
\AgdaOperator{\AgdaFunction{∣}}\AgdaSpace{}%
\AgdaBound{τ}\AgdaSpace{}%
\AgdaOperator{\AgdaFunction{∣}}\AgdaSpace{}%
\AgdaFunction{,}\AgdaSpace{}%
\AgdaSymbol{(}\AgdaFunction{transport}\AgdaSpace{}%
\AgdaBound{B}\AgdaSpace{}%
\AgdaBound{p}\AgdaSpace{}%
\AgdaOperator{\AgdaFunction{∥}}\AgdaSpace{}%
\AgdaBound{σ}\AgdaSpace{}%
\AgdaOperator{\AgdaFunction{∥}}\AgdaSymbol{)}\AgdaSpace{}%
\AgdaOperator{\AgdaDatatype{≡}}\AgdaSpace{}%
\AgdaOperator{\AgdaFunction{∥}}\AgdaSpace{}%
\AgdaBound{τ}\AgdaSpace{}%
\AgdaOperator{\AgdaFunction{∥}}\<%
\\
% \>[.][@{}l@{}]\<[111I]%
% \>[10]\AgdaComment{--------------------------------------------------------------}\<%
% \\
\>[2]\AgdaSymbol{→}%
\>[10]\AgdaBound{σ}\AgdaSpace{}%
\AgdaOperator{\AgdaDatatype{≡}}\AgdaSpace{}%
\AgdaBound{τ}\<%
\\
%
\\[\AgdaEmptyExtraSkip]%
%
\>[1]\AgdaFunction{to-Σ-≡}\AgdaSpace{}%
\AgdaSymbol{(}\AgdaInductiveConstructor{refl}\AgdaSpace{}%
\AgdaSymbol{\{}\AgdaArgument{x}\AgdaSpace{}%
\AgdaSymbol{=}\AgdaSpace{}%
\AgdaBound{x}\AgdaSymbol{\}}\AgdaSpace{}%
\AgdaOperator{\AgdaInductiveConstructor{,}}\AgdaSpace{}%
\AgdaInductiveConstructor{refl}\AgdaSpace{}%
\AgdaSymbol{\{}\AgdaArgument{x}\AgdaSpace{}%
\AgdaSymbol{=}\AgdaSpace{}%
\AgdaBound{a}\AgdaSymbol{\})}\AgdaSpace{}%
\AgdaSymbol{=}\AgdaSpace{}%
\AgdaInductiveConstructor{refl}\AgdaSpace{}%
\AgdaSymbol{\{}\AgdaArgument{x}\AgdaSpace{}%
\AgdaSymbol{=}\AgdaSpace{}%
\AgdaSymbol{(}\AgdaBound{x}\AgdaSpace{}%
\AgdaOperator{\AgdaInductiveConstructor{,}}\AgdaSpace{}%
\AgdaBound{a}\AgdaSymbol{)\}}\<%
\end{code}
\ccpad
We will use \af{is-embedding}, \af{is-set}, and \af{to-Σ-≡} in the next subsection to prove that a monic function into a set is an embedding.

\subsubsection{Injective functions are set embeddings}\label{injective-functions-are-set-embeddings}
Before moving on to define propositions, we discharge an obligation mentioned but left unfulfilled in the
\href{https://ualib.gitlab.io/Overture.Inverses.html\#embeddings}{embeddings} section of the \ualibhtml{Overture.Inverses} module. Recall, we described and imported the \af{is-embedding} type, and we remarked that an embedding is not simply a monic function. However, if we assume that the codomain is truncated so as to have unique identity proofs, then we can prove that every monic function into that codomain will be an embedding. On the other hand, embeddings are always monic, so we will end up with an equivalence.
% Assume the context contains the following typing judgments: \AgdaSymbol{\{}\AgdaBound{𝓤}\AgdaSpace{}\AgdaBound{𝓦}\AgdaSpace{}%
% \AgdaSymbol{:}\AgdaSpace{}%
% \AgdaPostulate{Universe}\AgdaSymbol{\}\{}\AgdaBound{A}\AgdaSpace{}%
% \AgdaSymbol{:}\AgdaSpace{}%
% \AgdaBound{𝓤}\AgdaSpace{}%
% \AgdaOperator{\AgdaFunction{̇}}\AgdaSymbol{\}\{}\AgdaBound{B}\AgdaSpace{}%
% \AgdaSymbol{:}\AgdaSpace{}%
% \AgdaBound{𝓦}\AgdaSpace{}%
% \AgdaOperator{\AgdaFunction{̇}}\AgdaSymbol{\}}.
\ccpad
\begin{code}%
\>[1]\AgdaFunction{monic-is-embedding|Set}\AgdaSpace{}%
\AgdaSymbol{:}%
\>[143I]\AgdaSymbol{(}\AgdaBound{f}\AgdaSpace{}%
\AgdaSymbol{:}\AgdaSpace{}%
\AgdaBound{A}\AgdaSpace{}%
\AgdaSymbol{→}\AgdaSpace{}%
\AgdaBound{B}\AgdaSymbol{)}\AgdaSpace{}%
\AgdaSymbol{→}\AgdaSpace{}%
\AgdaFunction{is-set}\AgdaSpace{}%
\AgdaBound{B}\AgdaSpace{}%
\AgdaSymbol{→}\AgdaSpace{}%
\AgdaFunction{Monic}\AgdaSpace{}%
\AgdaBound{f}\AgdaSpace{}%
\AgdaSymbol{→}\AgdaSpace{}%
\AgdaFunction{is-embedding}\AgdaSpace{}%
\AgdaBound{f}\<%
\\
%
\\[\AgdaEmptyExtraSkip]%
%
\>[1]\AgdaFunction{monic-is-embedding|Set}\AgdaSpace{}%
\AgdaBound{f}\AgdaSpace{}%
\AgdaBound{Bset}\AgdaSpace{}%
\AgdaBound{fmon}\AgdaSpace{}%
\AgdaBound{b}\AgdaSpace{}%
\AgdaSymbol{(}\AgdaBound{u}\AgdaSpace{}%
\AgdaOperator{\AgdaInductiveConstructor{,}}\AgdaSpace{}%
\AgdaBound{fu≡b}\AgdaSymbol{)}\AgdaSpace{}%
\AgdaSymbol{(}\AgdaBound{v}\AgdaSpace{}%
\AgdaOperator{\AgdaInductiveConstructor{,}}\AgdaSpace{}%
\AgdaBound{fv≡b}\AgdaSymbol{)}\AgdaSpace{}%
\AgdaSymbol{=}\AgdaSpace{}%
\AgdaFunction{γ}\<%
\\
\>[1][@{}l@{\AgdaIndent{0}}]%
\>[2]\AgdaKeyword{where}\<%
\\
%
\>[2]\AgdaFunction{fuv}\AgdaSpace{}%
\AgdaSymbol{:}\AgdaSpace{}%
\AgdaBound{f}\AgdaSpace{}%
\AgdaBound{u}\AgdaSpace{}%
\AgdaOperator{\AgdaDatatype{≡}}\AgdaSpace{}%
\AgdaBound{f}\AgdaSpace{}%
\AgdaBound{v}\<%
\\
%
\>[2]\AgdaFunction{fuv}\AgdaSpace{}%
\AgdaSymbol{=}\AgdaSpace{}%
\AgdaFunction{≡-trans}\AgdaSpace{}%
\AgdaBound{fu≡b}\AgdaSpace{}%
\AgdaSymbol{(}\AgdaBound{fv≡b}\AgdaSpace{}%
\AgdaOperator{\AgdaFunction{⁻¹}}\AgdaSymbol{)}\<%
\\
%
\\[\AgdaEmptyExtraSkip]%
%
\>[2]\AgdaFunction{uv}\AgdaSpace{}%
\AgdaSymbol{:}\AgdaSpace{}%
\AgdaBound{u}\AgdaSpace{}%
\AgdaOperator{\AgdaDatatype{≡}}\AgdaSpace{}%
\AgdaBound{v}\<%
\\
%
\>[2]\AgdaFunction{uv}\AgdaSpace{}%
\AgdaSymbol{=}\AgdaSpace{}%
\AgdaBound{fmon}\AgdaSpace{}%
\AgdaBound{u}\AgdaSpace{}%
\AgdaBound{v}\AgdaSpace{}%
\AgdaFunction{fuv}\<%
\\
%
\\[\AgdaEmptyExtraSkip]%
%
\>[2]\AgdaFunction{arg1}\AgdaSpace{}%
\AgdaSymbol{:}\AgdaSpace{}%
\AgdaFunction{Σ}\AgdaSpace{}%
\AgdaBound{p}\AgdaSpace{}%
\AgdaFunction{꞉}\AgdaSpace{}%
\AgdaSymbol{(}\AgdaBound{u}\AgdaSpace{}%
\AgdaOperator{\AgdaDatatype{≡}}\AgdaSpace{}%
\AgdaBound{v}\AgdaSymbol{)}\AgdaSpace{}%
\AgdaFunction{,}\AgdaSpace{}%
\AgdaSymbol{(}\AgdaFunction{transport}\AgdaSpace{}%
\AgdaSymbol{(λ}\AgdaSpace{}%
\AgdaBound{a}\AgdaSpace{}%
\AgdaSymbol{→}\AgdaSpace{}%
\AgdaBound{f}\AgdaSpace{}%
\AgdaBound{a}\AgdaSpace{}%
\AgdaOperator{\AgdaDatatype{≡}}\AgdaSpace{}%
\AgdaBound{b}\AgdaSymbol{)}\AgdaSpace{}%
\AgdaBound{p}\AgdaSpace{}%
\AgdaBound{fu≡b}\AgdaSymbol{)}\AgdaSpace{}%
\AgdaOperator{\AgdaDatatype{≡}}\AgdaSpace{}%
\AgdaBound{fv≡b}\<%
\\
%
\>[2]\AgdaFunction{arg1}\AgdaSpace{}%
\AgdaSymbol{=}\AgdaSpace{}%
\AgdaFunction{uv}\AgdaSpace{}%
\AgdaOperator{\AgdaInductiveConstructor{,}}\AgdaSpace{}%
\AgdaBound{Bset}\AgdaSpace{}%
\AgdaSymbol{(}\AgdaBound{f}\AgdaSpace{}%
\AgdaBound{v}\AgdaSymbol{)}\AgdaSpace{}%
\AgdaBound{b}\AgdaSpace{}%
\AgdaSymbol{(}\AgdaFunction{transport}\AgdaSpace{}%
\AgdaSymbol{(λ}\AgdaSpace{}%
\AgdaBound{a}\AgdaSpace{}%
\AgdaSymbol{→}\AgdaSpace{}%
\AgdaBound{f}\AgdaSpace{}%
\AgdaBound{a}\AgdaSpace{}%
\AgdaOperator{\AgdaDatatype{≡}}\AgdaSpace{}%
\AgdaBound{b}\AgdaSymbol{)}\AgdaSpace{}%
\AgdaFunction{uv}\AgdaSpace{}%
\AgdaBound{fu≡b}\AgdaSymbol{)}\AgdaSpace{}%
\AgdaBound{fv≡b}\<%
\\
%
\\[\AgdaEmptyExtraSkip]%
%
\>[2]\AgdaFunction{γ}\AgdaSpace{}%
\AgdaSymbol{:}\AgdaSpace{}%
\AgdaBound{u}\AgdaSpace{}%
\AgdaOperator{\AgdaInductiveConstructor{,}}\AgdaSpace{}%
\AgdaBound{fu≡b}\AgdaSpace{}%
\AgdaOperator{\AgdaDatatype{≡}}\AgdaSpace{}%
\AgdaBound{v}\AgdaSpace{}%
\AgdaOperator{\AgdaInductiveConstructor{,}}\AgdaSpace{}%
\AgdaBound{fv≡b}\<%
\\
%
\>[2]\AgdaFunction{γ}\AgdaSpace{}%
\AgdaSymbol{=}\AgdaSpace{}%
\AgdaFunction{to-Σ-≡}\AgdaSpace{}%
\AgdaFunction{arg1}\<%
\end{code}
\ccpad
In stating the previous result, we introduce a new convention to which we will try to adhere. If the antecedent of a theorem includes the assumption that one of the types involved is a set, then we add to the name of the theorem the suffix \af{\textbar{}sets}, which calls to mind the standard notation for the restriction of a function to a subset of its domain.

Embeddings are always monic, so we conclude that when a function's codomain is a set, then that function is an embedding if and only if it is monic.
\ccpad
\begin{code}%
\>[0][@{}l@{\AgdaIndent{1}}]%
\>[1]\AgdaFunction{embedding-iff-monic|Set}\AgdaSpace{}%
\AgdaSymbol{:}%
\>[234I]\AgdaSymbol{(}\AgdaBound{f}\AgdaSpace{}%
\AgdaSymbol{:}\AgdaSpace{}%
\AgdaBound{A}\AgdaSpace{}%
\AgdaSymbol{→}\AgdaSpace{}%
\AgdaBound{B}\AgdaSymbol{)}\AgdaSpace{}%
\AgdaSymbol{→}\AgdaSpace{}%
\AgdaFunction{is-set}\AgdaSpace{}%
\AgdaBound{B}\AgdaSpace{}%
\AgdaSymbol{→}\AgdaSpace{}%
\AgdaFunction{is-embedding}\AgdaSpace{}%
\AgdaBound{f}\AgdaSpace{}%
\AgdaOperator{\AgdaFunction{⇔}}\AgdaSpace{}%
\AgdaFunction{Monic}\AgdaSpace{}%
\AgdaBound{f}\<%
\\
\>[1]\AgdaFunction{embedding-iff-monic|Set}\AgdaSpace{}%
\AgdaBound{f}\AgdaSpace{}%
\AgdaBound{Bset}\AgdaSpace{}%
\AgdaSymbol{=}\AgdaSpace{}%
\AgdaSymbol{(}\AgdaFunction{embedding-is-monic}\AgdaSpace{}%
\AgdaBound{f}\AgdaSymbol{)}\AgdaOperator{\AgdaInductiveConstructor{,}}\AgdaSpace{}%
\AgdaSymbol{(}\AgdaFunction{monic-is-embedding|Set}\AgdaSpace{}%
\AgdaBound{f}\AgdaSpace{}%
\AgdaBound{Bset}\AgdaSymbol{)}\<%
\end{code}

\newcommand\proptype{\href{https://agda.readthedocs.io/en/v2.6.1.3/language/prop.html}{\af{Prop}}\xspace}

\begin{comment}
\paragraph*{The axiom of extensionality} %\label{sec:axiom-ext}
In type theory everything is represented as a type and, as we have just seen, this includes subsets.  Equality of types is a nontrivial matter, and thus so is equality of subsets when represented as unary predicates.  Fortunately, it is straightforward to write down a type that represents what it typically means in informal mathematics to say that two subsets are (extensionally) equal---namely, they contain the same elements. In the \ualib we denote this type by \af{≐} and define it as follows.\footnote{\label{uhints}%
  \textbf{Unicode Hints}. In \agdamode, \texttt{\textbackslash{}.=} ↝ \af{≐}, \texttt{\textbackslash{}u+} ↝ \af{⊎}, \texttt{\textbackslash{}b0} ↝ \af 𝟘, \texttt{\textbackslash{}B0} ↝ \af 𝟎, \texttt{\textbackslash{}b1} ↝ \af 𝟙, \texttt{\textbackslash{}B1} ↝ \af 𝟏.}
\ccpad
\begin{code}
\>[1]\AgdaOperator{\AgdaFunction{\AgdaUnderscore{}≐\AgdaUnderscore{}}}\AgdaSpace{}%
\AgdaSymbol{:}\AgdaSpace{}%
\AgdaFunction{Pred}\AgdaSpace{}%
\AgdaBound{A}\AgdaSpace{}%
\AgdaBound{𝓦}\AgdaSpace{}%
\AgdaSymbol{→}\AgdaSpace{}%
\AgdaFunction{Pred}\AgdaSpace{}%
\AgdaBound{A}\AgdaSpace{}%
\AgdaBound{𝓩}\AgdaSpace{}%
\AgdaSymbol{→}\AgdaSpace{}%
\AgdaBound{𝓤}\AgdaSpace{}%
\AgdaOperator{\AgdaPrimitive{⊔}}\AgdaSpace{}%
\AgdaBound{𝓦}\AgdaSpace{}%
\AgdaOperator{\AgdaPrimitive{⊔}}\AgdaSpace{}%
\AgdaBound{𝓩}\AgdaSpace{}%
\AgdaOperator{\AgdaFunction{̇}}\<%
\\
%
\>[1]\AgdaBound{P}\AgdaSpace{}%
\AgdaOperator{\AgdaFunction{≐}}\AgdaSpace{}%
\AgdaBound{Q}\AgdaSpace{}%
\AgdaSymbol{=}\AgdaSpace{}%
\AgdaSymbol{(}\AgdaBound{P}\AgdaSpace{}%
\AgdaOperator{\AgdaFunction{⊆}}\AgdaSpace{}%
\AgdaBound{Q}\AgdaSymbol{)}\AgdaSpace{}%
\AgdaOperator{\AgdaFunction{×}}\AgdaSpace{}%
\AgdaSymbol{(}\AgdaBound{Q}\AgdaSpace{}%
\AgdaOperator{\AgdaFunction{⊆}}\AgdaSpace{}%
\AgdaBound{P}\AgdaSymbol{)}\<%
\end{code}
\ccpad
A proof of \ab P \aof{≐} Q is a pair (\ab p , \ab q) where \ab p~\as :~\ab P~\aof ⊆~\ab Q and \ab q :~\ab Q~\aof ⊆~\ab P are proofs of the first and second inclusions, respectively. If \ab P and \ab Q are definitionally equal (i.e., \ab P \aod ≡ \ab Q), then both \ab P~\aof ⊆~\ab Q and \ab Q~\aof ⊆~\ab P hold, so \ab P \aof ≐ \ab Q also holds, as we now confirm.
\ccpad
\begin{code}%
\>[0]\AgdaFunction{Pred-≡}\AgdaSpace{}%
\AgdaSymbol{:}\AgdaSpace{}%
% \AgdaSymbol{\{}\AgdaBound{𝓤}\AgdaSpace{}%
% \AgdaBound{𝓦}\AgdaSpace{}%
% \AgdaSymbol{:}\AgdaSpace{}%
% \AgdaPostulate{Universe}\AgdaSymbol{\}\{}\AgdaBound{A}\AgdaSpace{}%
% \AgdaSymbol{:}\AgdaSpace{}%
% \AgdaBound{𝓤}\AgdaSpace{}%
% \AgdaOperator{\AgdaFunction{̇}}
\AgdaSymbol{\{}\AgdaBound{P}\AgdaSpace{}%
\AgdaBound{Q}\AgdaSpace{}%
\AgdaSymbol{:}\AgdaSpace{}%
\AgdaFunction{Pred}\AgdaSpace{}%
\AgdaBound{A}\AgdaSpace{}%
\AgdaBound{𝓦}\AgdaSymbol{\}}\AgdaSpace{}%
\AgdaSymbol{→}\AgdaSpace{}%
\AgdaBound{P}\AgdaSpace{}%
\AgdaOperator{\AgdaDatatype{≡}}\AgdaSpace{}%
\AgdaBound{Q}\AgdaSpace{}%
\AgdaSymbol{→}\AgdaSpace{}%
\AgdaBound{P}\AgdaSpace{}%
\AgdaOperator{\AgdaFunction{≐}}\AgdaSpace{}%
\AgdaBound{Q}\<%
\\
\>[0]\AgdaFunction{Pred-≡}\AgdaSpace{}%
\AgdaInductiveConstructor{refl}\AgdaSpace{}%
\AgdaSymbol{=}\AgdaSpace{}%
\AgdaSymbol{(λ}\AgdaSpace{}%
\AgdaBound{z}\AgdaSpace{}%
\AgdaSymbol{→}\AgdaSpace{}%
\AgdaBound{z}\AgdaSymbol{)}\AgdaSpace{}%
\AgdaOperator{\AgdaInductiveConstructor{,}}\AgdaSpace{}%
\AgdaSymbol{(λ}\AgdaSpace{}%
\AgdaBound{z}\AgdaSpace{}%
\AgdaSymbol{→}\AgdaSpace{}%
\AgdaBound{z}\AgdaSymbol{)}\<%
\end{code}
\ccpad
The converse is not provable in \mltt. However, we can postulate that it holds as an axiom if we wish.  This is called the \defn{axiom of extensionality} and a type that represents it is the following.
\ccpad
\begin{code}%
\>[0]\AgdaFunction{ext-axiom}\AgdaSpace{}%
\AgdaSymbol{:}\AgdaSpace{}%
% \AgdaSymbol{\{}\AgdaBound{𝓤}\AgdaSpace{}%
% \AgdaSymbol{:}\AgdaSpace{}%
% \AgdaPostulate{Universe}\AgdaSymbol{\}}\AgdaSpace{}%
% \AgdaSymbol{→}\AgdaSpace{}%
\AgdaBound{𝓤}\AgdaSpace{}%
\AgdaOperator{\AgdaFunction{̇}}\AgdaSpace{}%
\AgdaSymbol{→}\AgdaSpace{}%
\AgdaSymbol{(}\AgdaBound{𝓦}\AgdaSpace{}%
\AgdaSymbol{:}\AgdaSpace{}%
\AgdaPostulate{Universe}\AgdaSymbol{)}\AgdaSpace{}%
\AgdaSymbol{→}%
\>[53]\AgdaBound{𝓤}\AgdaSpace{}%
\AgdaOperator{\AgdaPrimitive{⊔}}\AgdaSpace{}%
\AgdaBound{𝓦}\AgdaSpace{}%
\AgdaOperator{\AgdaPrimitive{⁺}}\AgdaSpace{}%
\AgdaOperator{\AgdaFunction{̇}}\<%
\\
\>[0]\AgdaFunction{ext-axiom}\AgdaSpace{}%
\AgdaBound{A}\AgdaSpace{}%
\AgdaBound{𝓦}\AgdaSpace{}%
\AgdaSymbol{=}\AgdaSpace{}%
\AgdaSymbol{∀}\AgdaSpace{}%
\AgdaSymbol{(}\AgdaBound{P}\AgdaSpace{}%
\AgdaBound{Q}\AgdaSpace{}%
\AgdaSymbol{:}\AgdaSpace{}%
\AgdaFunction{Pred}\AgdaSpace{}%
\AgdaBound{A}\AgdaSpace{}%
\AgdaBound{𝓦}\AgdaSymbol{)}\AgdaSpace{}%
\AgdaSymbol{→}\AgdaSpace{}%
\AgdaBound{P}\AgdaSpace{}%
\AgdaOperator{\AgdaFunction{≐}}\AgdaSpace{}%
\AgdaBound{Q}\AgdaSpace{}%
\AgdaSymbol{→}\AgdaSpace{}%
\AgdaBound{P}\AgdaSpace{}%
\AgdaOperator{\AgdaDatatype{≡}}\AgdaSpace{}%
\AgdaBound{Q}\<%
\end{code}
\ccpad
Note that \af{ext-axiom} does not itself postulate the axiom of extensionality; it merely says what it is.  If we want to postulate the axiom, we must assume we have an inhabitant (or ``witness'') of the type. We could do this in Agda in a number of ways, but probably the easiest is to simply add the witness as a parameter to a module, like so.\footnote{Agda also has a \ak{postulate} mechanism that we could use, but this would require omitting the \AgdaPragma{--safe} pragma from the \ak{OPTIONS} directive at the start of the module.}
\ccpad
\begin{code}
\>[0]\AgdaKeyword{module}\AgdaSpace{}%
\AgdaModule{ext-axiom-postulated}\AgdaSpace{}%
\AgdaSymbol{\{}\AgdaBound{𝓤}\AgdaSpace{}%
\AgdaBound{𝓦}\AgdaSpace{}%
\AgdaSymbol{:}\AgdaSpace{}%
\AgdaPostulate{Universe}\AgdaSymbol{\}\{}\AgdaBound{A}\AgdaSpace{}%
\AgdaSymbol{:}\AgdaSpace{}%
\AgdaBound{𝓤}\AgdaSpace{}%
\AgdaOperator{\AgdaFunction{̇}}\AgdaSymbol{\}}\AgdaSpace{}%
\AgdaSymbol{\{}\AgdaBound{ea}\AgdaSpace{}%
\AgdaSymbol{:}\AgdaSpace{}%
\AgdaFunction{ext-axiom}\AgdaSpace{}%
\AgdaBound{A}\AgdaSpace{}%
\AgdaBound{𝓦}\AgdaSymbol{\}}\AgdaSpace{}%
\AgdaKeyword{where}\<%
\end{code}
\ccpad
Other notions of extensionality come up often in the \ualib; see, for example, \S\ref{function-extensionality} and \S\ref{sec:trunc-sets-prop}.

\end{comment}



\subsubsection{Predicate extensionality}\label{propositions}

The principle of \defn{proposition extensionality} asserts that logically equivalent propositions are equivalent.  That is, if \ab P and \ab Q are propositions such that \ab P~\af ⊆~\ab Q and \ab Q~\af ⊆~\ab P, then \ab P~\ad ≡~\ab Q. For our purposes, it will suffice to formalize this principle for general predicates, rather than propositions (i.e., truncated predicates). The following definition codifies the extensionality principle we require.
% , as follows.\footnote{\agda now has a type called \proptype, though we have never tried to use it. It likely provides at least some of the functionality we develop here, however, our preference is to assume only a minimal \mltt foundation and build up the types we need ourselves. For details about \proptype, consult the official documentation at \url{https://agda.readthedocs.io/en/v2.6.1.3/language/prop.html}.}
\ccpad
\begin{code}%
\>[0]\AgdaFunction{pred-ext}\AgdaSpace{}%
\AgdaSymbol{:}\AgdaSpace{}%
\AgdaSymbol{(}\AgdaBound{𝓤}\AgdaSpace{}%
\AgdaBound{𝓦}\AgdaSpace{}%
\AgdaSymbol{:}\AgdaSpace{}%
\AgdaPostulate{Universe}\AgdaSymbol{)}\AgdaSpace{}%
\AgdaSymbol{→}\AgdaSpace{}%
\AgdaSymbol{(}\AgdaBound{𝓤}\AgdaSpace{}%
\AgdaOperator{\AgdaPrimitive{⊔}}\AgdaSpace{}%
\AgdaBound{𝓦}\AgdaSymbol{)}\AgdaSpace{}%
\AgdaOperator{\AgdaPrimitive{⁺}}\AgdaSpace{}%
\AgdaOperator{\AgdaFunction{̇}}\<%
\\
\>[0]\AgdaFunction{pred-ext}\AgdaSpace{}%
\AgdaBound{𝓤}\AgdaSpace{}%
\AgdaBound{𝓦}\AgdaSpace{}%
\AgdaSymbol{=}\AgdaSpace{}%
\AgdaSymbol{∀}\AgdaSpace{}%
\AgdaSymbol{\{}\AgdaBound{A}\AgdaSpace{}%
\AgdaSymbol{:}\AgdaSpace{}%
\AgdaBound{𝓤}\AgdaSpace{}%
\AgdaOperator{\AgdaFunction{̇}}\AgdaSymbol{\}\{}\AgdaBound{P}\AgdaSpace{}%
\AgdaBound{Q}\AgdaSpace{}%
\AgdaSymbol{:}\AgdaSpace{}%
\AgdaFunction{Pred}\AgdaSpace{}%
\AgdaBound{A}\AgdaSpace{}%
\AgdaBound{𝓦}\AgdaSpace{}%
\AgdaSymbol{\}}\AgdaSpace{}%
\AgdaSymbol{→}\AgdaSpace{}%
\AgdaBound{P}\AgdaSpace{}%
\AgdaOperator{\AgdaFunction{⊆}}\AgdaSpace{}%
\AgdaBound{Q}\AgdaSpace{}%
\AgdaSymbol{→}\AgdaSpace{}%
\AgdaBound{Q}\AgdaSpace{}%
\AgdaOperator{\AgdaFunction{⊆}}\AgdaSpace{}%
\AgdaBound{P}\AgdaSpace{}%
\AgdaSymbol{→}\AgdaSpace{}%
\AgdaBound{P}\AgdaSpace{}%
\AgdaOperator{\AgdaDatatype{≡}}\AgdaSpace{}%
\AgdaBound{Q}\<%
\end{code}
% \ccpad
% The foregoing easily generalizes to binary relations in general and equivalence
% relations in particular.  If \ab R is a binary relation such that there is at
% most one way to prove that a given pair of elements is \ab R-related, then we call
% \ab R a \defn{binary proposition}. As above, we use \typtop's
% \af{is-subsingleton-valued} type to impose this truncation assumption on a
% binary relation.\footnote{%
% This is another example of \emph{proof-irrelevance}. Indeed, if \ab{R} is a binary proposition and we
% have two proofs of \ab{R} \ab x \ab y, then the proofs are indistinguishable.}
% \ccpad
% \begin{code}%
% \>[0]\AgdaFunction{Pred₂}\AgdaSpace{}%
% \AgdaSymbol{:}\AgdaSpace{}%
% \AgdaSymbol{\{}\AgdaBound{𝓤}\AgdaSpace{}%
% \AgdaSymbol{:}\AgdaSpace{}%
% \AgdaPostulate{Universe}\AgdaSymbol{\}}\AgdaSpace{}%
% \AgdaSymbol{→}\AgdaSpace{}%
% \AgdaBound{𝓤}\AgdaSpace{}%
% \AgdaOperator{\AgdaFunction{̇}}\AgdaSpace{}%
% \AgdaSymbol{→}\AgdaSpace{}%
% \AgdaSymbol{(}\AgdaBound{𝓦}\AgdaSpace{}%
% \AgdaSymbol{:}\AgdaSpace{}%
% \AgdaPostulate{Universe}\AgdaSymbol{)}\AgdaSpace{}%
% \AgdaSymbol{→}\AgdaSpace{}%
% \AgdaBound{𝓤}\AgdaSpace{}%
% \AgdaOperator{\AgdaPrimitive{⊔}}\AgdaSpace{}%
% \AgdaBound{𝓦}\AgdaSpace{}%
% \AgdaOperator{\AgdaPrimitive{⁺}}\AgdaSpace{}%
% \AgdaOperator{\AgdaFunction{̇}}\<%
% \\
% \>[0]\AgdaFunction{Pred₂}\AgdaSpace{}%
% \AgdaBound{A}\AgdaSpace{}%
% \AgdaBound{𝓦}\AgdaSpace{}%
% \AgdaSymbol{=}\AgdaSpace{}%
% \AgdaFunction{Σ}\AgdaSpace{}%
% \AgdaBound{R}\AgdaSpace{}%
% \AgdaFunction{꞉}\AgdaSpace{}%
% \AgdaSymbol{(}\AgdaFunction{Rel}\AgdaSpace{}%
% \AgdaBound{A}\AgdaSpace{}%
% \AgdaBound{𝓦}\AgdaSymbol{)}\AgdaSpace{}%
% \AgdaFunction{,}\AgdaSpace{}%
% \AgdaFunction{is-subsingleton-valued}\AgdaSpace{}%
% \AgdaBound{R}\<%
% \end{code}
% \ccpad
% Recall, \af{is-subsingleton-valued} is defined as\\[-10pt]

% \af{is-subsingleton-valued} \ab R = \as ∀~\ab x~\ab y~\as →~\af{is-subsingleton}~(\ab R~\ab x~\ab y)\\[4pt]
% which is the assertion that for all \ab x~\ab y there is at most one proof that
% \ab x and \ab y are \ab R-related. 
% We call this the \defn{uniqueness-of-membership-proofs} (UMP) property.  The
% types \af{IsContProp} and \af{IsDepProp} defined below generalize this concept
% from binary to arbitrary (continuous and dependent) relations.

% We define a \defn{truncated equivalence} to be an equivalence relation that has unique membership proofs, and we use the following types to represent such relations.
% \ccpad
% \begin{code}%
% \>[0]\<%
% \\
% \>[0]\AgdaKeyword{module}\AgdaSpace{}%
% \AgdaModule{\AgdaUnderscore{}}\AgdaSpace{}%
% \AgdaSymbol{\{}\AgdaBound{𝓤}\AgdaSpace{}%
% \AgdaBound{𝓦}\AgdaSpace{}%
% \AgdaSymbol{:}\AgdaSpace{}%
% \AgdaPostulate{Universe}\AgdaSymbol{\}}\AgdaSpace{}%
% \AgdaKeyword{where}\<%
% \\
% %
% \\[\AgdaEmptyExtraSkip]%
% \>[0][@{}l@{\AgdaIndent{0}}]%
% \>[1]\AgdaKeyword{record}\AgdaSpace{}%
% \AgdaRecord{IsEqv}\AgdaSpace{}%
% \AgdaSymbol{\{}\AgdaBound{A}\AgdaSpace{}%
% \AgdaSymbol{:}\AgdaSpace{}%
% \AgdaBound{𝓤}\AgdaSpace{}%
% \AgdaOperator{\AgdaFunction{̇}}\AgdaSymbol{\}(}\AgdaBound{R}\AgdaSpace{}%
% \AgdaSymbol{:}\AgdaSpace{}%
% \AgdaFunction{Rel}\AgdaSpace{}%
% \AgdaBound{A}\AgdaSpace{}%
% \AgdaBound{𝓦}\AgdaSymbol{)}\AgdaSpace{}%
% \AgdaSymbol{:}\AgdaSpace{}%
% \AgdaBound{𝓤}\AgdaSpace{}%
% \AgdaOperator{\AgdaPrimitive{⊔}}\AgdaSpace{}%
% \AgdaBound{𝓦}\AgdaSpace{}%
% \AgdaOperator{\AgdaFunction{̇}}\AgdaSpace{}%
% \AgdaKeyword{where}\<%
% \\
% \>[1][@{}l@{\AgdaIndent{0}}]%
% \>[2]\AgdaKeyword{field}%
% \>[369I]\AgdaField{is-equivalence}\AgdaSpace{}%
% \AgdaSymbol{:}\AgdaSpace{}%
% \AgdaRecord{IsEquivalence}\AgdaSpace{}%
% \AgdaBound{R}\<%
% \\
% \>[.][@{}l@{}]\<[369I]%
% \>[8]\AgdaField{is-truncated}\AgdaSpace{}%
% \AgdaSymbol{:}\AgdaSpace{}%
% \AgdaFunction{is-subsingleton-valued}\AgdaSpace{}%
% \AgdaBound{R}\<%
% \\
% %
% \\[\AgdaEmptyExtraSkip]%
% %
% \>[1]\AgdaFunction{Eqv}\AgdaSpace{}%
% \AgdaSymbol{:}\AgdaSpace{}%
% \AgdaBound{𝓤}\AgdaSpace{}%
% \AgdaOperator{\AgdaFunction{̇}}\AgdaSpace{}%
% \AgdaSymbol{→}\AgdaSpace{}%
% \AgdaBound{𝓤}\AgdaSpace{}%
% \AgdaOperator{\AgdaPrimitive{⊔}}\AgdaSpace{}%
% \AgdaBound{𝓦}\AgdaSpace{}%
% \AgdaOperator{\AgdaPrimitive{⁺}}\AgdaSpace{}%
% \AgdaOperator{\AgdaFunction{̇}}\<%
% \\
% %
% \>[1]\AgdaFunction{Eqv}\AgdaSpace{}%
% \AgdaBound{A}\AgdaSpace{}%
% \AgdaSymbol{=}\AgdaSpace{}%
% \AgdaFunction{Σ}\AgdaSpace{}%
% \AgdaBound{R}\AgdaSpace{}%
% \AgdaFunction{꞉}\AgdaSpace{}%
% \AgdaFunction{Rel}\AgdaSpace{}%
% \AgdaBound{A}\AgdaSpace{}%
% \AgdaBound{𝓦}\AgdaSpace{}%
% \AgdaFunction{,}\AgdaSpace{}%
% \AgdaRecord{IsEqv}\AgdaSpace{}%
% \AgdaBound{R}\<%
% \end{code}


\subsubsection{Quotient extensionality}\label{quotient-extensionality}

We need an identity type for congruence classes (blocks) over sets so that two different presentations of the same block (e.g., using different representatives) may be identified. This requires two postulates: (1) \emph{predicate
extensionality} (captured above by the \texttt{pred-ext} type) and (2) \emph{block truncation} (or ``quotient class truncation'') which we now define.

Recall, we defined \af{IsBlock}~\ab C in~\S\ref{equivalence-classes-1} to be the type \af Σ \ab u~\af ꞉~\ab A~\af ,~\ab C~\ad ≡~\af [~\ab u~\af ] representing pairs (\ab a , \ab p) such that \ab p is a proof that \ab C is the equivalence class containing \ab a. We will need to assume that for each predicate \ab C~\as :~\af{Pred}~\ab A~\ab 𝓦 there is at most one proof of \af{IsBlock}~\ab C. We call this proof-irrelevance principle ``uniqueness of block identity proofs''. We first define a type that represents this principle and then discuss whether and when this is a reasonable thing to assume.
\ccpad
\begin{code}%
\>[0]\AgdaFunction{blk-uip}\AgdaSpace{}%
\AgdaSymbol{:}\AgdaSpace{}%
\AgdaSymbol{\{}\AgdaBound{𝓦}\AgdaSpace{}%
\AgdaBound{𝓤}\AgdaSpace{}%
\AgdaSymbol{:}\AgdaSpace{}%
\AgdaPostulate{Universe}\AgdaSymbol{\}(}\AgdaBound{A}\AgdaSpace{}%
\AgdaSymbol{:}\AgdaSpace{}%
\AgdaBound{𝓤}\AgdaSpace{}%
\AgdaOperator{\AgdaFunction{̇}}\AgdaSymbol{)(}\AgdaBound{R}\AgdaSpace{}%
\AgdaSymbol{:}\AgdaSpace{}%
\AgdaFunction{Rel}\AgdaSpace{}%
\AgdaBound{A}\AgdaSpace{}%
\AgdaBound{𝓦}\AgdaSpace{}%
\AgdaSymbol{)}\AgdaSpace{}%
\AgdaSymbol{→}\AgdaSpace{}%
\AgdaBound{𝓤}\AgdaSpace{}%
\AgdaOperator{\AgdaPrimitive{⊔}}\AgdaSpace{}%
\AgdaBound{𝓦}\AgdaSpace{}%
\AgdaOperator{\AgdaPrimitive{⁺}}\AgdaSpace{}%
\AgdaOperator{\AgdaFunction{̇}}\<%
\\
\>[0]\AgdaFunction{blk-uip}\AgdaSpace{}%
\AgdaSymbol{\{}\AgdaBound{𝓦}\AgdaSymbol{\}}\AgdaSpace{}%
\AgdaBound{A}\AgdaSpace{}%
\AgdaBound{R}\AgdaSpace{}%
\AgdaSymbol{=}\AgdaSpace{}%
\AgdaSymbol{∀}\AgdaSpace{}%
\AgdaSymbol{(}\AgdaBound{C}\AgdaSpace{}%
\AgdaSymbol{:}\AgdaSpace{}%
\AgdaFunction{Pred}\AgdaSpace{}%
\AgdaBound{A}\AgdaSpace{}%
\AgdaBound{𝓦}\AgdaSymbol{)}\AgdaSpace{}%
\AgdaSymbol{→}\AgdaSpace{}%
\AgdaFunction{is-subsingleton}\AgdaSpace{}%
\AgdaSymbol{(}\AgdaFunction{IsBlock}\AgdaSpace{}%
\AgdaBound{C}\AgdaSpace{}%
\AgdaSymbol{\{}\AgdaBound{R}\AgdaSymbol{\})}\<%
\end{code}
\ccpad
It might seem unreasonable to postulate that there is at most one inhabitant of \af{IsBlock}~\ab C, since it is typically the case that equivalence classes have multiple members, any one of which could serve as a class representative.  However, postulating \af{blk-uip} \ab A \ab R is tantamount to collapsing each \ab R-block to a single point, which is precisely how we should treat the elements of the quotient \ab A~\af /~\ab R.

We now use \af{pred-ext} and \af{blk-uip} to define a type called \af{block-ext\textbar{}uip} which we require for the proof of the First Homomorphism Theorem presented in \ualibhtml{Homomorphisms.Noether}.
\ccpad
\begin{code}%
\>[0]\AgdaKeyword{module}\AgdaSpace{}%
\AgdaModule{\AgdaUnderscore{}}\AgdaSpace{}%
\AgdaSymbol{\{}\AgdaBound{𝓤}\AgdaSpace{}%
\AgdaBound{𝓦}\AgdaSpace{}%
\AgdaSymbol{:}\AgdaSpace{}%
\AgdaPostulate{Universe}\AgdaSymbol{\}\{}\AgdaBound{A}\AgdaSpace{}%
\AgdaSymbol{:}\AgdaSpace{}%
\AgdaBound{𝓤}\AgdaSpace{}%
\AgdaOperator{\AgdaFunction{̇}}\AgdaSymbol{\}\{}\AgdaBound{R}\AgdaSpace{}%
\AgdaSymbol{:}\AgdaSpace{}%
\AgdaFunction{Rel}\AgdaSpace{}%
\AgdaBound{A}\AgdaSpace{}%
\AgdaBound{𝓦}\AgdaSymbol{\}}\AgdaSpace{}%
\AgdaKeyword{where}\<%
\\
%
\\[\AgdaEmptyExtraSkip]%
\>[0][@{}l@{\AgdaIndent{0}}]%
\>[1]\AgdaFunction{block-ext}\AgdaSpace{}%
\AgdaSymbol{:}\AgdaSpace{}%
\AgdaFunction{pred-ext}\AgdaSpace{}%
\AgdaBound{𝓤}\AgdaSpace{}%
\AgdaBound{𝓦}\AgdaSpace{}%
\AgdaSymbol{→}\AgdaSpace{}%
\AgdaRecord{IsEquivalence}\AgdaSpace{}%
\AgdaBound{R}\AgdaSpace{}%
\AgdaSymbol{→}\AgdaSpace{}%
\AgdaSymbol{\{}\AgdaBound{u}\AgdaSpace{}%
\AgdaBound{v}\AgdaSpace{}%
\AgdaSymbol{:}\AgdaSpace{}%
\AgdaBound{A}\AgdaSymbol{\}}\AgdaSpace{}%
\AgdaSymbol{→}\AgdaSpace{}%
\AgdaBound{R}\AgdaSpace{}%
\AgdaBound{u}\AgdaSpace{}%
\AgdaBound{v}\AgdaSpace{}%
\AgdaSymbol{→}\AgdaSpace{}%
\AgdaOperator{\AgdaFunction{[}}\AgdaSpace{}%
\AgdaBound{u}\AgdaSpace{}%
\AgdaOperator{\AgdaFunction{]}}\AgdaSymbol{\{}\AgdaBound{R}\AgdaSymbol{\}}\AgdaSpace{}%
\AgdaOperator{\AgdaDatatype{≡}}\AgdaSpace{}%
\AgdaOperator{\AgdaFunction{[}}\AgdaSpace{}%
\AgdaBound{v}\AgdaSpace{}%
\AgdaOperator{\AgdaFunction{]}}\AgdaSymbol{\{}\AgdaBound{R}\AgdaSymbol{\}}\<%
\\
%
\>[1]\AgdaFunction{block-ext}\AgdaSpace{}%
\AgdaBound{pe}\AgdaSpace{}%
\AgdaBound{Req}\AgdaSpace{}%
\AgdaSymbol{\{}\AgdaBound{u}\AgdaSymbol{\}\{}\AgdaBound{v}\AgdaSymbol{\}}\AgdaSpace{}%
\AgdaBound{Ruv}\AgdaSpace{}%
\AgdaSymbol{=}\AgdaSpace{}%
\AgdaBound{pe}\AgdaSpace{}%
\AgdaSymbol{(}\AgdaFunction{/-subset}\AgdaSpace{}%
\AgdaBound{Req}\AgdaSpace{}%
\AgdaBound{Ruv}\AgdaSymbol{)}\AgdaSpace{}%
\AgdaSymbol{(}\AgdaFunction{/-supset}\AgdaSpace{}%
\AgdaBound{Req}\AgdaSpace{}%
\AgdaBound{Ruv}\AgdaSymbol{)}\<%
\\
%
\\[\AgdaEmptyExtraSkip]%
%
\>[1]\AgdaFunction{to-subtype|uip}\AgdaSpace{}%
\AgdaSymbol{:}\AgdaSpace{}%
\AgdaFunction{blk-uip}\AgdaSpace{}%
\AgdaBound{A}\AgdaSpace{}%
\AgdaBound{R}\AgdaSpace{}%
\AgdaSymbol{→}\AgdaSpace{}%
\AgdaSymbol{\{}\AgdaBound{C}\AgdaSpace{}%
\AgdaBound{D}\AgdaSpace{}%
\AgdaSymbol{:}\AgdaSpace{}%
\AgdaFunction{Pred}\AgdaSpace{}%
\AgdaBound{A}\AgdaSpace{}%
\AgdaBound{𝓦}\AgdaSymbol{\}\{}\AgdaBound{c}\AgdaSpace{}%
\AgdaSymbol{:}\AgdaSpace{}%
\AgdaFunction{IsBlock}\AgdaSpace{}%
\AgdaBound{C}\AgdaSpace{}%
\AgdaSymbol{\{}\AgdaBound{R}\AgdaSymbol{\}\}\{}\AgdaBound{d}\AgdaSpace{}%
\AgdaSymbol{:}\AgdaSpace{}%
\AgdaFunction{IsBlock}\AgdaSpace{}%
\AgdaBound{D}\AgdaSpace{}%
\AgdaSymbol{\{}\AgdaBound{R}\AgdaSymbol{\}\}}\<%
\\
\>[1][@{}l@{\AgdaIndent{0}}]%
\>[2]\AgdaSymbol{→}%
\>[18]\AgdaBound{C}\AgdaSpace{}%
\AgdaOperator{\AgdaDatatype{≡}}\AgdaSpace{}%
\AgdaBound{D}%
\>[25]\AgdaSymbol{→}%
\>[28]\AgdaSymbol{(}\AgdaBound{C}\AgdaSpace{}%
\AgdaOperator{\AgdaInductiveConstructor{,}}\AgdaSpace{}%
\AgdaBound{c}\AgdaSymbol{)}\AgdaSpace{}%
\AgdaOperator{\AgdaDatatype{≡}}\AgdaSpace{}%
\AgdaSymbol{(}\AgdaBound{D}\AgdaSpace{}%
\AgdaOperator{\AgdaInductiveConstructor{,}}\AgdaSpace{}%
\AgdaBound{d}\AgdaSymbol{)}\<%
\\
%
% \\[\AgdaEmptyExtraSkip]%
%
\>[1]\AgdaFunction{to-subtype|uip}\AgdaSpace{}%
\AgdaBound{buip}\AgdaSpace{}%
\AgdaSymbol{\{}\AgdaBound{C}\AgdaSymbol{\}\{}\AgdaBound{D}\AgdaSymbol{\}\{}\AgdaBound{c}\AgdaSymbol{\}\{}\AgdaBound{d}\AgdaSymbol{\}}\AgdaBound{CD}\AgdaSpace{}%
\AgdaSymbol{=}\AgdaSpace{}%
\AgdaFunction{to-Σ-≡}\AgdaSymbol{(}\AgdaBound{CD}\AgdaSpace{}%
\AgdaOperator{\AgdaInductiveConstructor{,}}\AgdaSpace{}%
\AgdaBound{buip}\AgdaSpace{}%
\AgdaBound{D}\AgdaSymbol{(}\AgdaFunction{transport}\AgdaSymbol{(λ}\AgdaSpace{}%
\AgdaBound{B}\AgdaSpace{}%
\AgdaSymbol{→}\AgdaSpace{}%
\AgdaFunction{IsBlock}\AgdaSpace{}%
\AgdaBound{B}\AgdaSymbol{)}\AgdaBound{CD}\AgdaSpace{}%
\AgdaBound{c}\AgdaSymbol{)}\AgdaBound{d}\AgdaSymbol{)}\<%
\\
%
\\[\AgdaEmptyExtraSkip]%
%
% \\[\AgdaEmptyExtraSkip]%
%
\>[1]\AgdaFunction{block-ext|uip}\AgdaSpace{}%
\AgdaSymbol{:}\AgdaSpace{}%
\AgdaFunction{pred-ext}\AgdaSpace{}%
\AgdaBound{𝓤}\AgdaSpace{}%
\AgdaBound{𝓦}\AgdaSpace{}%
\AgdaSymbol{→}\AgdaSpace{}%
\AgdaFunction{blk-uip}\AgdaSpace{}%
\AgdaBound{A}\AgdaSpace{}%
\AgdaBound{R}\AgdaSpace{}%
\AgdaSymbol{→}\AgdaSpace{}%
\AgdaRecord{IsEquivalence}\AgdaSpace{}%
\AgdaBound{R}\<%
\\
\>[1][@{}l@{\AgdaIndent{0}}]%
\>[2]\AgdaSymbol{→}%
\>[17]\AgdaSymbol{\{}\AgdaBound{u}\AgdaSpace{}%
\AgdaBound{v}\AgdaSpace{}%
\AgdaSymbol{:}\AgdaSpace{}%
\AgdaBound{A}\AgdaSymbol{\}}\AgdaSpace{}%
\AgdaSymbol{→}\AgdaSpace{}%
\AgdaBound{R}\AgdaSpace{}%
\AgdaBound{u}\AgdaSpace{}%
\AgdaBound{v}%
\>[36]\AgdaSymbol{→}%
\>[39]\AgdaOperator{\AgdaFunction{⟪}}\AgdaSpace{}%
\AgdaBound{u}\AgdaSpace{}%
\AgdaOperator{\AgdaFunction{⟫}}\AgdaSpace{}%
\AgdaOperator{\AgdaDatatype{≡}}\AgdaSpace{}%
\AgdaOperator{\AgdaFunction{⟪}}\AgdaSpace{}%
\AgdaBound{v}\AgdaSpace{}%
\AgdaOperator{\AgdaFunction{⟫}}\<%
\\
%
% \\[\AgdaEmptyExtraSkip]%
%
\>[1]\AgdaFunction{block-ext|uip}\AgdaSpace{}%
\AgdaBound{pe}\AgdaSpace{}%
\AgdaBound{buip}\AgdaSpace{}%
\AgdaBound{Req}\AgdaSpace{}%
\AgdaBound{Ruv}\AgdaSpace{}%
\AgdaSymbol{=}\AgdaSpace{}%
\AgdaFunction{to-subtype|uip}\AgdaSpace{}%
\AgdaBound{buip}\AgdaSpace{}%
\AgdaSymbol{(}\AgdaFunction{block-ext}\AgdaSpace{}%
\AgdaBound{pe}\AgdaSpace{}%
\AgdaBound{Req}\AgdaSpace{}%
\AgdaBound{Ruv}\AgdaSymbol{)}\<%
\end{code}

\subsubsection{General propositions*\protect\cref{starred}}\label{sec:general-props}
In this final subsection of our presentation of relations in type theory, we offer a few interesting new types to complement those defined in the \ualibhtml{Relations.Continuous} module (\S\ref{sec:continuous-relations}).  (So far no other modules of the \ualib depend on the types defined in this subsection, so the reader may safely skip to Section~\ref{sec:algebra-types}.) % without fear that this will lead to confusion at some point later on.

Earlier (in \S\ref{sec:continuous-relations}) we defined the \af{DepRel} type to represent relations of arbitrary arity over an arbitrary collection of sorts. Here we introduce a new type of \emph{truncated dependent relations}, the inhabitants of which we call \defn{dependent propositions}. (Assume the context includes \AgdaBound{𝓤}\AgdaSpace{}%
\AgdaSymbol{:}\AgdaSpace{}%
\AgdaPostulate{Universe} and \AgdaBound{I}\AgdaSpace{}%
\AgdaSymbol{:}\AgdaSpace{}%
\AgdaGeneralizable{𝓥}\AgdaSpace{}%
\AgdaOperator{\AgdaFunction{̇}}.)
\ccpad
\begin{code}%
\>[0][@{}l@{\AgdaIndent{1}}]%
\>[1]\AgdaFunction{IsDepProp}\AgdaSpace{}%
\AgdaSymbol{:}\AgdaSpace{}%
\AgdaSymbol{\{}\AgdaBound{I}\AgdaSpace{}%
\AgdaSymbol{:}\AgdaSpace{}%
\AgdaBound{𝓥}\AgdaSpace{}%
\AgdaOperator{\AgdaFunction{̇}}\AgdaSymbol{\}\{}\AgdaBound{𝒜}\AgdaSpace{}%
\AgdaSymbol{:}\AgdaSpace{}%
\AgdaBound{I}\AgdaSpace{}%
\AgdaSymbol{→}\AgdaSpace{}%
\AgdaBound{𝓤}\AgdaSpace{}%
\AgdaOperator{\AgdaFunction{̇}}\AgdaSymbol{\}\{}\AgdaBound{𝓦}\AgdaSpace{}%
\AgdaSymbol{:}\AgdaSpace{}%
\AgdaPostulate{Universe}\AgdaSymbol{\}}\AgdaSpace{}%
\AgdaSymbol{→}\AgdaSpace{}%
\AgdaFunction{DepRel}\AgdaSpace{}%
\AgdaBound{I}\AgdaSpace{}%
\AgdaBound{𝒜}\AgdaSpace{}%
\AgdaBound{𝓦}%
\>[66]\AgdaSymbol{→}\AgdaSpace{}%
\AgdaBound{𝓥}\AgdaSpace{}%
\AgdaOperator{\AgdaPrimitive{⊔}}\AgdaSpace{}%
\AgdaBound{𝓤}\AgdaSpace{}%
\AgdaOperator{\AgdaPrimitive{⊔}}\AgdaSpace{}%
\AgdaBound{𝓦}\AgdaSpace{}%
\AgdaOperator{\AgdaFunction{̇}}\<%
\\
%
\>[1]\AgdaFunction{IsDepProp}\AgdaSpace{}%
\AgdaSymbol{\{}\AgdaArgument{I}\AgdaSpace{}%
\AgdaSymbol{=}\AgdaSpace{}%
\AgdaBound{I}\AgdaSymbol{\}}\AgdaSpace{}%
\AgdaSymbol{\{}\AgdaBound{𝒜}\AgdaSymbol{\}}\AgdaSpace{}%
\AgdaBound{P}\AgdaSpace{}%
\AgdaSymbol{=}\AgdaSpace{}%
\AgdaFunction{Π}\AgdaSpace{}%
\AgdaBound{𝑎}\AgdaSpace{}%
\AgdaFunction{꞉}\AgdaSpace{}%
\AgdaFunction{Π}\AgdaSpace{}%
\AgdaBound{𝒜}\AgdaSpace{}%
\AgdaFunction{,}\AgdaSpace{}%
\AgdaFunction{is-subsingleton}\AgdaSpace{}%
\AgdaSymbol{(}\AgdaBound{P}\AgdaSpace{}%
\AgdaBound{𝑎}\AgdaSymbol{)}\<%
\\
%
\\[\AgdaEmptyExtraSkip]%
%
\>[1]\AgdaFunction{DepProp}\AgdaSpace{}%
\AgdaSymbol{:}\AgdaSpace{}%
\AgdaSymbol{(}\AgdaBound{I}\AgdaSpace{}%
\AgdaSymbol{→}\AgdaSpace{}%
\AgdaBound{𝓤}\AgdaSpace{}%
\AgdaOperator{\AgdaFunction{̇}}\AgdaSymbol{)}\AgdaSpace{}%
\AgdaSymbol{→}\AgdaSpace{}%
\AgdaSymbol{(}\AgdaBound{𝓦}\AgdaSpace{}%
\AgdaSymbol{:}\AgdaSpace{}%
\AgdaPostulate{Universe}\AgdaSymbol{)}\AgdaSpace{}%
\AgdaSymbol{→}\AgdaSpace{}%
\AgdaBound{𝓤}\AgdaSpace{}%
\AgdaOperator{\AgdaPrimitive{⊔}}\AgdaSpace{}%
\AgdaBound{𝓥}\AgdaSpace{}%
\AgdaOperator{\AgdaPrimitive{⊔}}\AgdaSpace{}%
\AgdaBound{𝓦}\AgdaSpace{}%
\AgdaOperator{\AgdaPrimitive{⁺}}\AgdaSpace{}%
\AgdaOperator{\AgdaFunction{̇}}\<%
\\
%
\>[1]\AgdaFunction{DepProp}\AgdaSpace{}%
\AgdaBound{𝒜}\AgdaSpace{}%
\AgdaBound{𝓦}\AgdaSpace{}%
\AgdaSymbol{=}\AgdaSpace{}%
\AgdaFunction{Σ}\AgdaSpace{}%
\AgdaBound{P}\AgdaSpace{}%
\AgdaFunction{꞉}\AgdaSpace{}%
\AgdaSymbol{(}\AgdaFunction{DepRel}\AgdaSpace{}%
\AgdaBound{I}\AgdaSpace{}%
\AgdaBound{𝒜}\AgdaSpace{}%
\AgdaBound{𝓦}\AgdaSymbol{)}\AgdaSpace{}%
\AgdaFunction{,}\AgdaSpace{}%
\AgdaFunction{IsDepProp}\AgdaSpace{}%
\AgdaBound{P}\<%
\\
%
\\[\AgdaEmptyExtraSkip]%
%
\>[1]\AgdaFunction{dep-prop-ext}\AgdaSpace{}%
\AgdaSymbol{:}\AgdaSpace{}%
\AgdaSymbol{(}\AgdaBound{I}\AgdaSpace{}%
\AgdaSymbol{→}\AgdaSpace{}%
\AgdaBound{𝓤}\AgdaSpace{}%
\AgdaOperator{\AgdaFunction{̇}}\AgdaSymbol{)}\AgdaSpace{}%
\AgdaSymbol{→}\AgdaSpace{}%
\AgdaSymbol{(}\AgdaBound{𝓦}\AgdaSpace{}%
\AgdaSymbol{:}\AgdaSpace{}%
\AgdaPostulate{Universe}\AgdaSymbol{)}\AgdaSpace{}%
\AgdaSymbol{→}\AgdaSpace{}%
\AgdaBound{𝓤}\AgdaSpace{}%
\AgdaOperator{\AgdaPrimitive{⊔}}\AgdaSpace{}%
\AgdaBound{𝓥}\AgdaSpace{}%
\AgdaOperator{\AgdaPrimitive{⊔}}\AgdaSpace{}%
\AgdaBound{𝓦}\AgdaSpace{}%
\AgdaOperator{\AgdaPrimitive{⁺}}\AgdaSpace{}%
\AgdaOperator{\AgdaFunction{̇}}\<%
\\
%
\>[1]\AgdaFunction{dep-prop-ext}\AgdaSpace{}%
\AgdaBound{𝒜}\AgdaSpace{}%
\AgdaBound{𝓦}\AgdaSpace{}%
\AgdaSymbol{=}\AgdaSpace{}%
\AgdaSymbol{\{}\AgdaBound{P}\AgdaSpace{}%
\AgdaBound{Q}\AgdaSpace{}%
\AgdaSymbol{:}\AgdaSpace{}%
\AgdaFunction{DepProp}\AgdaSpace{}%
\AgdaBound{𝒜}\AgdaSpace{}%
\AgdaBound{𝓦}\AgdaSymbol{\}}\AgdaSpace{}%
\AgdaSymbol{→}\AgdaSpace{}%
\AgdaOperator{\AgdaFunction{∣}}\AgdaSpace{}%
\AgdaBound{P}\AgdaSpace{}%
\AgdaOperator{\AgdaFunction{∣}}\AgdaSpace{}%
\AgdaOperator{\AgdaFunction{⊆}}\AgdaSpace{}%
\AgdaOperator{\AgdaFunction{∣}}\AgdaSpace{}%
\AgdaBound{Q}\AgdaSpace{}%
\AgdaOperator{\AgdaFunction{∣}}\AgdaSpace{}%
\AgdaSymbol{→}\AgdaSpace{}%
\AgdaOperator{\AgdaFunction{∣}}\AgdaSpace{}%
\AgdaBound{Q}\AgdaSpace{}%
\AgdaOperator{\AgdaFunction{∣}}\AgdaSpace{}%
\AgdaOperator{\AgdaFunction{⊆}}\AgdaSpace{}%
\AgdaOperator{\AgdaFunction{∣}}\AgdaSpace{}%
\AgdaBound{P}\AgdaSpace{}%
\AgdaOperator{\AgdaFunction{∣}}\AgdaSpace{}%
\AgdaSymbol{→}\AgdaSpace{}%
\AgdaBound{P}\AgdaSpace{}%
\AgdaOperator{\AgdaDatatype{≡}}\AgdaSpace{}%
\AgdaBound{Q}\<%
\end{code}
\ccpad
To see the point of the types just defined, suppose \af{dep-prop-ext} \ab A \ab 𝓦 holds. Then we can prove that logically equivalent dependent propositions (of type \af{DepProp} \ab A \ab 𝓦) are equivalent. In other words, under the stated hypotheses, we obtain the following  extensionality lemma for dependent propositions (assuming the context includes
% \>[0][@{}l@{\AgdaIndent{1}}]%
% \>[1]\AgdaKeyword{module}\AgdaSpace{}%
% \AgdaModule{\AgdaUnderscore{}}\AgdaSpace{}%
% \AgdaSymbol{(}
\AgdaBound{𝒜}\AgdaSpace{}%
\AgdaSymbol{:}\AgdaSpace{}%
\AgdaBound{I}\AgdaSpace{}%
\AgdaSymbol{→}\AgdaSpace{}%
\AgdaBound{𝓤}\AgdaSpace{}%
\AgdaOperator{\AgdaFunction{̇}}
% \AgdaSymbol{)(}
and \AgdaBound{𝓦}\AgdaSpace{}%
\AgdaSymbol{:}\AgdaSpace{}%
\AgdaPostulate{Universe}).
% \AgdaSymbol{)}\AgdaSpace{}%
% \AgdaKeyword{where}\<%
\ccpad
\begin{code}%
\>[1][@{}l@{\AgdaIndent{0}}]%
\>[2]\AgdaFunction{dep-prop-ext'}\AgdaSpace{}%
\AgdaSymbol{:}\AgdaSpace{}%
\AgdaFunction{dep-prop-ext}\AgdaSpace{}%
\AgdaBound{𝒜}\AgdaSpace{}%
\AgdaBound{𝓦}\AgdaSpace{}%
\AgdaSymbol{→}\AgdaSpace{}%
\AgdaSymbol{\{}\AgdaBound{P}\AgdaSpace{}%
\AgdaBound{Q}\AgdaSpace{}%
\AgdaSymbol{:}\AgdaSpace{}%
\AgdaFunction{DepProp}\AgdaSpace{}%
\AgdaBound{𝒜}\AgdaSpace{}%
\AgdaBound{𝓦}\AgdaSymbol{\}}\AgdaSpace{}%
\AgdaSymbol{→}\AgdaSpace{}%
\AgdaOperator{\AgdaFunction{∣}}\AgdaSpace{}%
\AgdaBound{P}\AgdaSpace{}%
\AgdaOperator{\AgdaFunction{∣}}\AgdaSpace{}%
\AgdaOperator{\AgdaFunction{≐}}\AgdaSpace{}%
\AgdaOperator{\AgdaFunction{∣}}\AgdaSpace{}%
\AgdaBound{Q}\AgdaSpace{}%
\AgdaOperator{\AgdaFunction{∣}}\AgdaSpace{}%
\AgdaSymbol{→}\AgdaSpace{}%
\AgdaBound{P}\AgdaSpace{}%
\AgdaOperator{\AgdaDatatype{≡}}\AgdaSpace{}%
\AgdaBound{Q}\<%
\\
%
\>[2]\AgdaFunction{dep-prop-ext'}\AgdaSpace{}%
\AgdaBound{pe}\AgdaSpace{}%
\AgdaBound{hyp}\AgdaSpace{}%
\AgdaSymbol{=}\AgdaSpace{}%
\AgdaBound{pe}%
\>[29]\AgdaOperator{\AgdaFunction{∣}}\AgdaSpace{}%
\AgdaBound{hyp}\AgdaSpace{}%
\AgdaOperator{\AgdaFunction{∣}}%
\>[38]\AgdaOperator{\AgdaFunction{∥}}\AgdaSpace{}%
\AgdaBound{hyp}\AgdaSpace{}%
\AgdaOperator{\AgdaFunction{∥}}\<%
\end{code}
