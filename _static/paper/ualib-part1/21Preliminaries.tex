% -*- TeX-master: "ualib-part1.tex" -*-
%%% Local Variables: 
%%% mode: latex
%%% TeX-engine: 'xetex
%%% End:
Here we define or import the basic types of \defn{Martin-Löf type theory} (\mltt).  Although this is standard stuff, we take this opportunity to highlight aspects of the \ualib syntax that may differ from that of ``standard Agda.''

\subsubsection*{Logical foundations} % \label{sec:logical-foundations}
% Mainly for the benefit of readers who are not yet proficient in Agda or type theory, we briefly describe the type theoretic foundations of the \ualib. 
The \agdaualib is based on a type theory that is the same or very close to the one on which on which Martín Escardó's \typetopology (\typtop) Agda library is based. We don't discuss \mltt in great detail here because there are already nice and freely available resources covering the theory. (See, for example, the section \href{https://www.cs.bham.ac.uk/~mhe/HoTT-UF-in-Agda-Lecture-Notes/HoTT-UF-Agda.html\#mlttinagda}{A spartan Martin-Löf type theory} of the lecture notes by Escard\'o~\cite{MHE}, the \href{https://ncatlab.org/nlab/show/Martin-L\%C3\%B6f+dependent+type+theory}{ncatlab entry on Martin-Löf dependent type theory}, or the HoTT Book~\cite{HoTT}.)

The objects and assumptions that form the foundation of \mltt are few.  There are the \defn{primitive types} (\ad 𝟘, \ad 𝟙, and \ad ℕ, denoting the empty type, one-element type, and natural numbers), the \defn{type formers} (\ad +, \ad Π, \ad Σ, \ad{Id}, denoting \defn{binary sum}, \defn{product}, \defn{sum}, and the \defn{identity} type). Each of these type formers is defined by a \defn{type forming rule} which specifies how that type is constructed. Lastly, we have an infinite collection of \defn{type universes} (types of types) and \defn{universe variables} to denote them. Following Escardó, we denote universes in the \ualib by upper-case calligraphic letters from the second half of the English alphabet; to be precise, these are \ab 𝓞, \ab 𝓠, \ab 𝓡, …, \ab 𝓧, \ab 𝓨, \ab 𝓩.\footnote{We avoid using \ab 𝓟 as a universe variable because it is used to denote the powerset type in \typtop.}

That's all. There are no further axioms or logical deduction (proof derivation) rules needed for the foundation of \mltt that we take as the starting point of the \agdaualib. The logical semantics come from the \href{https://ncatlab.org/nlab/show/propositions+as+types}{propositions-as-types correspondence}~\cite{nlab:propositions_as_types}: propositions and predicates are represented by types and the inhabitants of these types are the proofs of the propositions and predicates. As such, proofs are constructed using the type forming rules. In other words, the type forming rules \emph{are} the proof derivation rules.

To this foundation, we add certain \textit{extensionality principles} when and were we need them.  These will be developed as we progress.  However, classical axioms such as the \href{https://ncatlab.org/nlab/show/axiom+of+choice}{\textit{Axiom of Choice}} or the \href{https://ncatlab.org/nlab/show/excluded+middle}{\textit{Law of the Excluded Middle}} are not needed and are not assumed anywhere in the library.  In that sense, all theorems and proofs in the \ualib are \href{https://ncatlab.org/nlab/show/constructive+mathematics}{\textit{constructive}} (as defined, e.g., in~\cite{nlab:constructive_mathematics}).

A few specific instances (e.g., the proof of the Noether isomorphism theorems and Birkhoff's HSP theorem) require certain \textit{truncation} assumptions. In such cases, the theory is not \href{https://ncatlab.org/nlab/show/predicative+mathematics}{\textit{predicative}} (as defined, e.g., in~\cite{nlab:predicative_mathematics}). These instances are always clearly identified.




\paragraph*{Specifying logical foundations in Agda}
An Agda program typically begins by setting some options and by importing types from existing Agda libraries.
Options are specified with the \AgdaKeyword{OPTIONS} \emph{pragma} and control the way Agda behaves, for example, by specifying the logical axioms and deduction rules we wish to assume when the program is type-checked to verify its correctness. Every Agda program in the \ualib begins with the following line.
\ccpad
\begin{code}[number=code:options]
\>[0]\AgdaSymbol{\{-\#}\AgdaSpace{}%
\AgdaKeyword{OPTIONS}\AgdaSpace{}%
\AgdaPragma{--without-K}\AgdaSpace{}%
\AgdaPragma{--exact-split}\AgdaSpace{}%
\AgdaPragma{--safe}\AgdaSpace{}%
\AgdaSymbol{\#-\}}\<%
\end{code}
\ccpad
These options control certain foundational assumptions that Agda makes when type-checking the program to verify its correctness.
\begin{enumerate}
\item \AgdaPragma{--without-K} disables \axiomk, which makes Agda compatible with \emph{proof-relevant} type theories; see the discussion of proof-relevance in~\S~\ref{sec:trunc-sets-prop}; see also~\cite{agdaref-axiomk, Cockx:2017};
\item \AgdaPragma{--exact-split} makes Agda accept only definitions that are \emph{judgmental} equalities; see~\cite{agdatools-patternmatching};
\item \AgdaPragma{--safe} ensures that nothing is postulated outright---every non-\mltt axiom has to be an explicit assumption (e.g., an argument to a function or module); see~\cite{agdaref-safeagda, agdatools-patternmatching}.
\end{enumerate}
%% \end{enumerate}
Throughout this paper we take assumptions 1--3 for granted without mentioning them explicitly.


\paragraph*{Agda Modules} %\label{sec:agda-modules}
The \ak{OPTIONS} pragma is usually followed by the start of a module.  For example, the \ualibhtml{Overture.Preliminaries} module begins with the following line.
\ccpad
\begin{code}%
\>[0]\AgdaKeyword{module}\AgdaSpace{}%
\AgdaModule{Overture.Preliminaries}\AgdaSpace{}%
\AgdaKeyword{where}\<%
\end{code}
\ccpad
Sometimes we want to declare parameters that will be assumed throughout the module.  For instance, when working with algebras, we often assume they come from a particular fixed signature, and this signature is something we could fix as a parameter at the start of a module. Thus, we might start an \defn{anonymous submodule} of the main module with a line like\footnote{The \af{Signature} type will be defined in Section~\ref{sec:oper-sign}.}
\ccpad
\begin{code}%
\>[0]\AgdaKeyword{module}\AgdaSpace{}%
\AgdaUnderscore\AgdaSpace{}%
\AgdaSymbol{\{}\AgdaBound{𝑆}\AgdaSpace{}%
\AgdaSymbol{:}\AgdaSpace{}%
\AgdaFunction{Signature}\AgdaSpace{}%
\AgdaBound{𝓞}\AgdaSpace{}%
\AgdaBound{𝓥}\AgdaSymbol{\}}\AgdaSpace{}%
\AgdaKeyword{where}\<%
\end{code}
\ccpad
Such a module is called \emph{anonymous} because an underscore appears in place of a module name. Agda determines where a submodule ends by indentation.  This can take some getting used to, but after a short time it will feel very natural. The main module of a file must have the same name as the file, without the \texttt{.agda} or \texttt{.lagda} file extension.  The code inside the main module is not indented. Submodules are declared inside the main module and code inside these submodules must be indented to a fixed column.  As long as the code is indented, Agda considers it part of the submodule.  A submodule is exited as soon as a nonindented line of code appears.



\paragraph*{Universes in Agda} %\label{universes}

For the very small amount of background we require about the notion of \emph{type universe} (or \emph{level}), we refer the reader to the brief \href{https://agda.readthedocs.io/en/v2.6.1.3/language/universe-levels.html}{section
on universe-levels} in the \href{https://agda.readthedocs.io/en/v2.6.1.3/language/universe-levels.html}{Agda
documentation}.\footnote{See \url{https://agda.readthedocs.io/en/v2.6.1.3/language/universe-levels.html}.}

Throughout the \agdaualib we use many of the nice tools that Martín Escardó has developed and made available in \typtop library of Agda code for the \emph{Univalent Foundations} of mathematics.\footnote{%
Escardó has written an outstanding set of notes called
\href{https://www.cs.bham.ac.uk/~mhe/HoTT-UF-in-Agda-Lecture-Notes/index.html}{Introduction to Univalent Foundations of Mathematics with Agda}, which we highly recommend to anyone looking for more details than we provide
here about \mltt and Univalent Foundations/HoTT in Agda.~\cite{MHE}.}
The first of these is the \am{Universes} module which we import as follows.
\ccpad
\begin{code}%
\>[0]\AgdaKeyword{open}\AgdaSpace{}%
\AgdaKeyword{import}\AgdaSpace{}%
\AgdaModule{Universes}\AgdaSpace{}%
\AgdaKeyword{public}\<%
\end{code}
\ccpad
Since we use the \ak{public} directive, the \am{Universes} module will be available to all modules that import
the present module (\ualibhtml{Overture.Preliminaries}). This module declares symbols used to denote universes. As mentioned, we adopt Escardó's convention of denoting universes by capital calligraphic letters, and most of the ones we use are already declared in \am{Universes}; those that are not are declared as follows.
\ccpad
\begin{code}%
\>[0]\AgdaKeyword{variable}\AgdaSpace{}%
\AgdaGeneralizable{𝓞}\AgdaSpace{}%
\AgdaGeneralizable{𝓧}\AgdaSpace{}%
\AgdaGeneralizable{𝓨}\AgdaSpace{}%
\AgdaGeneralizable{𝓩}\AgdaSpace{}%
\AgdaSymbol{:}\AgdaSpace{}%
\AgdaPostulate{Universe}\<%
\end{code}
\scpad

The \am{Universes} module also provides alternative syntax for the primitive operations on universes that Agda supports. Specifically, the \af ̇ operator maps a universe level \ab 𝓤 to the type \ad{Set}~\ab 𝓤, and the latter has type \apt{Set}~(\apr{lsuc}~\ab 𝓤). The Agda level \apr{lzero} is renamed \apr{𝓤₀}, so \apr{𝓤₀}\af ̇ is an alias for \apr{Set}~\apr{lzero}. Thus, \ab 𝓤\af ̇ is simply an alias for \apr{Set}~\ab 𝓤, and we have \apr{Set}~\ab 𝓤~\as :~\apr{Set}~(\apr{lsuc}~\ab 𝓤). Finally, \apr{Set}~(\apr{lsuc}~\apr{lzero}) is equivalent to \apr{Set}~\apr{𝓤₀}\af ⁺, which we (and Escardó) denote by \apr{𝓤₀}\af ⁺\af ̇.

% The following dictionary translates between standard Agda syntax and Type Topology/UALib notation.

% \begin{Shaded}
% \begin{Highlighting}[]
% \NormalTok{Agda              Type Topology/UALib}
% \NormalTok{====              ===================}
% \NormalTok{Level             Universe}
% \NormalTok{lzero             𝓤₀}
% \NormalTok{𝓤 }\OtherTok{:}\NormalTok{ Level         𝓤 }\OtherTok{:}\NormalTok{ Universe}
% \DataTypeTok{Set}\NormalTok{ lzero         𝓤₀ ̇}
% \DataTypeTok{Set}\NormalTok{ 𝓤             𝓤 ̇}
% \NormalTok{lsuc lzero        𝓤₀ ⁺}
% \NormalTok{lsuc 𝓤            𝓤 ⁺}
% \DataTypeTok{Set} \OtherTok{(}\NormalTok{lsuc lzero}\OtherTok{)}\NormalTok{  𝓤₀ ⁺ ̇}
% \DataTypeTok{Set} \OtherTok{(}\NormalTok{lsuc 𝓤}\OtherTok{)}\NormalTok{      𝓤 ⁺ ̇}
% \NormalTok{Setω              𝓤ω}
% \end{Highlighting}
% \end{Shaded}

To justify the introduction of this somewhat nonstandard notation for universe levels, Escardó points out that the Agda library uses \apr{Level} for universes (so what we write as \ab 𝓤\af ̇ is written \apr{Set}~\ab 𝓤 in standard Agda), but in univalent mathematics the types in \ab 𝓤\af ̇ need not be sets, so the standard Agda notation can be
a bit confusing, especially to newcomers.

There will be many occasions calling for a type living in a universe at the level that is the least upper bound of two universe levels, say, \ab 𝓤 and \ab 𝓥. The universe level \ab 𝓤~\apr ⊔~\ab 𝓥\af denotes this least upper bound. Here \apr ⊔ is an Agda primitive designed for precisely this purpose.


\subsubsection*{Dependent types} %\label{sec:dependent-types}

\newcommand\FstUnder{\AgdaOperator{\AgdaFunction{∣\AgdaUnderscore{}∣}}\xspace}
\newcommand\SndUnder{\AgdaOperator{\AgdaFunction{∥\AgdaUnderscore{}∥}}\xspace}
% Convenient notations for the first and second projections out of a product are \FstUnder and \SndUnder, respectively. However, to improve readability or to avoid notation clashes with other modules, we sometimes use more standard alternatives, such as \AgdaFunction{pr₁} and \AgdaFunction{pr₂}, or \AgdaFunction{fst} and \AgdaFunction{snd}, or some combination of these. The definitions are standard so we omit them.\footnote{For details see~\cite{DeMeo:2021} or \ualiburl{Overture.Preliminaries}.}








\paragraph*{Sigma types (dependent pairs)} %\label{ssec:dependent-pairs}

Given universes \ab 𝓤 and \ab 𝓥, a type \ab{A} \as : \ab 𝓤\af ̇, and a type family \ab{B}~\as :~\ab A~\as →~\ab 𝓥\af ̇, the \defn{Sigma type} (or \defn{dependent pair type}, or \defn{dependent product type}) is denoted by \ar{Σ}~\abt{x}{A} ,~\ab B~\ab x and generalizes the Cartesian product \ab A~\af ×~\ab B by allowing the type \ab{B x} of the second argument of the ordered pair (\ab x \af , \ab y) to depend on the value \ab{x} of the first. That is, an inhabitant of the type \ar{Σ}~\abt{x}{A}~,~\ab B~\ab x is a pair (\ab{x}~\af ,~\ab y) such that \abt{x}{A} and \abt{y}{B x}.

The dependent product type is defined in \typtop in a standard way. For pedagogical purposes we repeat the definition here.\footnote{In the \ualib we put such redundant definitions inside ``hidden'' modules so that they doesn't conflict with the original definitions which we import and use. It may seem odd to define something in a hidden module only to import and use an alternative definition, but we do this in order to exhibit all of the types on which the \ualib depends while ensuring that this cannot be misinterpreted as a claim to originality.}
\ccpad
\begin{code}%
\>[1]\AgdaKeyword{record}\AgdaSpace{}%
\AgdaRecord{Σ}\AgdaSpace{}%
\AgdaSymbol{\{}\AgdaBound{𝓤}\AgdaSpace{}%
\AgdaBound{𝓥}\AgdaSymbol{\}}\AgdaSpace{}%
\AgdaSymbol{\{}\AgdaBound{A}\AgdaSpace{}%
\AgdaSymbol{:}\AgdaSpace{}%
\AgdaBound{𝓤}\AgdaSpace{}%
\AgdaOperator{\AgdaFunction{̇}}\AgdaSpace{}%
\AgdaSymbol{\}}\AgdaSpace{}%
\AgdaSymbol{(}\AgdaBound{B}\AgdaSpace{}%
\AgdaSymbol{:}\AgdaSpace{}%
\AgdaBound{A}\AgdaSpace{}%
\AgdaSymbol{→}\AgdaSpace{}%
\AgdaBound{𝓥}\AgdaSpace{}%
\AgdaOperator{\AgdaFunction{̇}}\AgdaSpace{}%
\AgdaSymbol{)}\AgdaSpace{}%
\AgdaSymbol{:}\AgdaSpace{}%
\AgdaBound{𝓤}\AgdaSpace{}%
\AgdaOperator{\AgdaPrimitive{⊔}}\AgdaSpace{}%
\AgdaBound{𝓥}\AgdaSpace{}%
\AgdaOperator{\AgdaFunction{̇}}%
\>[53]\AgdaKeyword{where}\<%
\\
\>[1][@{}l@{\AgdaIndent{0}}]%
\>[2]\AgdaKeyword{constructor}\AgdaSpace{}%
\AgdaOperator{\AgdaInductiveConstructor{\AgdaUnderscore{},\AgdaUnderscore{}}}\<%
\\
%
\>[2]\AgdaKeyword{field}\<%
\\
\>[2][@{}l@{\AgdaIndent{0}}]%
\>[3]\AgdaField{pr₁}\AgdaSpace{}%
\AgdaSymbol{:}\AgdaSpace{}%
\AgdaBound{A}\<%
\\
%
\>[3]\AgdaField{pr₂}\AgdaSpace{}%
\AgdaSymbol{:}\AgdaSpace{}%
\AgdaBound{B}\AgdaSpace{}%
\AgdaField{pr₁}\<%
\end{code}
\ccpad
Agda's default syntax for this type is \AgdaRecord{Σ}\sP{3}\AgdaSymbol{λ}(\ab x\sP{3}꞉\sP{3}\ab A)\sP{3}\as →\sP{3}\ab B, but we prefer the notation \AgdaRecord{Σ}~\abt{x}{A}~,~\ab B, which is closer to the syntax in the preceding paragraph, and will be familiar to readers of the HoTT book~\cite{HoTT}, for example. Fortunately, \typtop makes the preferred notation available with the following type definition and \ak{syntax} declaration (see~\cite[Σ types]{MHE}).\footnote{\label{fncolon}\textbf{Attention!} The symbol \af ꞉ that appears in the special syntax defined here for the \AgdaFunction{Σ} type, and below for the \af{Π} type, is not the ordinary colon; rather, it is the symbol obtained by typing \texttt{\textbackslash{}:4} in \agdatwomode.} 
\ccpad
\begin{code}%
\>[1]\AgdaFunction{-Σ}\AgdaSpace{}%
\AgdaSymbol{:}\AgdaSpace{}%
\AgdaSymbol{\{}\AgdaBound{𝓤}\AgdaSpace{}%
\AgdaBound{𝓥}\AgdaSpace{}%
\AgdaSymbol{:}\AgdaSpace{}%
\AgdaPostulate{Universe}\AgdaSymbol{\}}\AgdaSpace{}%
\AgdaSymbol{(}\AgdaBound{A}\AgdaSpace{}%
\AgdaSymbol{:}\AgdaSpace{}%
\AgdaBound{𝓤}\AgdaSpace{}%
\AgdaOperator{\AgdaFunction{̇}}\AgdaSpace{}%
\AgdaSymbol{)}\AgdaSpace{}%
\AgdaSymbol{(}\AgdaBound{B}\AgdaSpace{}%
\AgdaSymbol{:}\AgdaSpace{}%
\AgdaBound{A}\AgdaSpace{}%
\AgdaSymbol{→}\AgdaSpace{}%
\AgdaBound{𝓥}\AgdaSpace{}%
\AgdaOperator{\AgdaFunction{̇}}\AgdaSpace{}%
\AgdaSymbol{)}\AgdaSpace{}%
\AgdaSymbol{→}\AgdaSpace{}%
\AgdaBound{𝓤}\AgdaSpace{}%
\AgdaOperator{\AgdaPrimitive{⊔}}\AgdaSpace{}%
\AgdaBound{𝓥}\AgdaSpace{}%
\AgdaOperator{\AgdaFunction{̇}}\<%
\\
%
\>[1]\AgdaFunction{-Σ}\AgdaSpace{}%
\AgdaBound{A}\AgdaSpace{}%
\AgdaBound{B}\AgdaSpace{}%
\AgdaSymbol{=}\AgdaSpace{}%
\AgdaRecord{Σ}\AgdaSpace{}%
\AgdaBound{B}\<%
\\
%
\\[\AgdaEmptyExtraSkip]%
%
\>[1]\AgdaKeyword{syntax}\AgdaSpace{}%
\AgdaFunction{-Σ}\AgdaSpace{}%
\AgdaBound{A}\AgdaSpace{}%
\AgdaSymbol{(λ}\AgdaSpace{}%
\AgdaBound{x}\AgdaSpace{}%
\AgdaSymbol{→}\AgdaSpace{}%
\AgdaBound{B}\AgdaSymbol{)}\AgdaSpace{}%
\AgdaSymbol{=}\AgdaSpace{}%
\AgdaFunction{Σ}\AgdaSpace{}%
\AgdaBound{x}\AgdaSpace{}%
\AgdaFunction{꞉}\AgdaSpace{}%
\AgdaBound{A}\AgdaSpace{}%
\AgdaFunction{,}\AgdaSpace{}%
\AgdaBound{B}\<%
\end{code}
\scpad
% \textbf{WARNING!} The symbol \af ꞉ is not the same as \as : despite how similar they may look. The correct colon in the expression \ar{Σ}~\ab x~\af ꞉~\ab A~\af ,~\ab y, above is obtained by typing \texttt{\textbackslash{}:4} in \agdamode.

A special case of the Sigma type is the one in which the type \ab{B} doesn't depend on \ab{A}. This is the usual Cartesian product, defined in Agda as follows.
\ccpad
\begin{code}%
\>[1]\AgdaOperator{\AgdaFunction{\AgdaUnderscore{}×\AgdaUnderscore{}}}\AgdaSpace{}%
\AgdaSymbol{:}\AgdaSpace{}%
\AgdaGeneralizable{𝓤}\AgdaSpace{}%
\AgdaOperator{\AgdaFunction{̇}}\AgdaSpace{}%
\AgdaSymbol{→}\AgdaSpace{}%
\AgdaGeneralizable{𝓥}\AgdaSpace{}%
\AgdaOperator{\AgdaFunction{̇}}\AgdaSpace{}%
\AgdaSymbol{→}\AgdaSpace{}%
\AgdaGeneralizable{𝓤}\AgdaSpace{}%
\AgdaOperator{\AgdaPrimitive{⊔}}\AgdaSpace{}%
\AgdaGeneralizable{𝓥}\AgdaSpace{}%
\AgdaOperator{\AgdaFunction{̇}}\<%
\\
%
\>[1]\AgdaBound{A}\AgdaSpace{}%
\AgdaOperator{\AgdaFunction{×}}\AgdaSpace{}%
\AgdaBound{B}\AgdaSpace{}%
\AgdaSymbol{=}\AgdaSpace{}%
\AgdaFunction{Σ}\AgdaSpace{}%
\AgdaBound{x}\AgdaSpace{}%
\AgdaFunction{꞉}\AgdaSpace{}%
\AgdaBound{A}\AgdaSpace{}%
\AgdaFunction{,}\AgdaSpace{}%
\AgdaBound{B}\<%
\end{code}





\paragraph*{Pi types (dependent functions)} %\label{ssec:dependent-functions}
Given universes \ab 𝓤 and \ab 𝓥, a type \ab{A} \as : \ab 𝓤\af ̇, and a type family \ab{B}~\as :~\ab A~\as →~\ab 𝓥\af ̇, the \defn{Pi type} (or \defn{dependent function type}) is denoted by \ar Π~\abt{x}{A} ,~\ab B~\ab x and generalizes the function type \ab A~\as →~\ab B by letting the type \ab B~\ab x of the codomain depend on the value \ab x of the domain type. The dependent function type is defined in \typtop in a standard way.  For the reader's benefit, however, we repeat the definition here.  % (In the \ualib this definition is included in a named or ``hidden'' module.)
\ccpad
\begin{code}%
\>[1]\AgdaFunction{Π}\AgdaSpace{}%
\AgdaSymbol{:}\AgdaSpace{}%
\AgdaSymbol{\{}\AgdaBound{A}\AgdaSpace{}%
\AgdaSymbol{:}\AgdaSpace{}%
\AgdaBound{𝓤}\AgdaSpace{}%
\AgdaOperator{\AgdaFunction{̇}}\AgdaSpace{}%
\AgdaSymbol{\}}\AgdaSpace{}%
\AgdaSymbol{(}\AgdaBound{A}\AgdaSpace{}%
\AgdaSymbol{:}\AgdaSpace{}%
\AgdaBound{A}\AgdaSpace{}%
\AgdaSymbol{→}\AgdaSpace{}%
\AgdaBound{𝓦}\AgdaSpace{}%
\AgdaOperator{\AgdaFunction{̇}}\AgdaSpace{}%
\AgdaSymbol{)}\AgdaSpace{}%
\AgdaSymbol{→}\AgdaSpace{}%
\AgdaBound{𝓤}\AgdaSpace{}%
\AgdaOperator{\AgdaPrimitive{⊔}}\AgdaSpace{}%
\AgdaBound{𝓦}\AgdaSpace{}%
\AgdaOperator{\AgdaFunction{̇}}\<%
\\
%
\>[1]\AgdaFunction{Π}\AgdaSpace{}%
\AgdaSymbol{\{}\AgdaBound{A}\AgdaSymbol{\}}\AgdaSpace{}%
\AgdaBound{A}\AgdaSpace{}%
\AgdaSymbol{=}\AgdaSpace{}%
\AgdaSymbol{(}\AgdaBound{x}\AgdaSpace{}%
\AgdaSymbol{:}\AgdaSpace{}%
\AgdaBound{A}\AgdaSymbol{)}\AgdaSpace{}%
\AgdaSymbol{→}\AgdaSpace{}%
\AgdaBound{A}\AgdaSpace{}%
\AgdaBound{x}\<%
\end{code}
\ccpad
To make the syntax for \af{Π} conform to the standard notation for Pi types, \escardo uses the same trick as the one used above for Sigma types.\cref{fncolon}
\ccpad
\begin{code}
\>[1]\AgdaFunction{-Π}\AgdaSpace{}%
\AgdaSymbol{:}\AgdaSpace{}%
\AgdaSymbol{(}\AgdaBound{A}\AgdaSpace{}%
\AgdaSymbol{:}\AgdaSpace{}%
\AgdaBound{𝓤}\AgdaSpace{}%
\AgdaOperator{\AgdaFunction{̇}}\AgdaSpace{}%
\AgdaSymbol{)(}\AgdaBound{B}\AgdaSpace{}%
\AgdaSymbol{:}\AgdaSpace{}%
\AgdaBound{A}\AgdaSpace{}%
\AgdaSymbol{→}\AgdaSpace{}%
\AgdaBound{𝓦}\AgdaSpace{}%
\AgdaOperator{\AgdaFunction{̇}}\AgdaSpace{}%
\AgdaSymbol{)}\AgdaSpace{}%
\AgdaSymbol{→}\AgdaSpace{}%
\AgdaBound{𝓤}\AgdaSpace{}%
\AgdaOperator{\AgdaPrimitive{⊔}}\AgdaSpace{}%
\AgdaBound{𝓦}\AgdaSpace{}%
\AgdaOperator{\AgdaFunction{̇}}\<%
\\
%
\>[1]\AgdaFunction{-Π}\AgdaSpace{}%
\AgdaBound{A}\AgdaSpace{}%
\AgdaBound{B}\AgdaSpace{}%
\AgdaSymbol{=}\AgdaSpace{}%
\AgdaFunction{Π}\AgdaSpace{}%
\AgdaBound{B}\<%
\\
%
\\[\AgdaEmptyExtraSkip]%
\>[1]\AgdaKeyword{syntax}\AgdaSpace{}%
\AgdaFunction{-Π}\AgdaSpace{}%
\AgdaBound{A}\AgdaSpace{}%
\AgdaSymbol{(λ}\AgdaSpace{}%
\AgdaBound{x}\AgdaSpace{}%
\AgdaSymbol{→}\AgdaSpace{}%
\AgdaBound{b}\AgdaSymbol{)}\AgdaSpace{}%
\AgdaSymbol{=}\AgdaSpace{}%
\AgdaFunction{Π}\AgdaSpace{}%
\AgdaBound{x}\AgdaSpace{}%
\AgdaFunction{꞉}\AgdaSpace{}%
\AgdaBound{A}\AgdaSpace{}%
\AgdaFunction{,}\AgdaSpace{}%
\AgdaBound{b}\<%
\end{code}
\ccpad
Once we have studied the types (defined in \typtop and repeated here for convenience and illustration purposes), the original definitions are imported like so.
\ccpad
\begin{code}%
\>[0]\AgdaKeyword{open}\AgdaSpace{}%
\AgdaKeyword{import}\AgdaSpace{}%
\AgdaModule{Sigma-Type}\AgdaSpace{}%
\AgdaKeyword{public}\<%
\\
\>[0]\AgdaKeyword{open}\AgdaSpace{}%
\AgdaKeyword{import}\AgdaSpace{}%
\AgdaModule{MGS-MLTT}\AgdaSpace{}%
\AgdaKeyword{using}\AgdaSpace{}%
\AgdaSymbol{(}\AgdaFunction{pr₁}\AgdaSymbol{;}\AgdaSpace{}%
\AgdaFunction{pr₂}\AgdaSymbol{;}\AgdaSpace{}%
\AgdaOperator{\AgdaFunction{\AgdaUnderscore{}×\AgdaUnderscore{}}}\AgdaSymbol{;}\AgdaSpace{}%
\AgdaFunction{-Σ}\AgdaSymbol{;}\AgdaSpace{}%
\AgdaFunction{Π}\AgdaSymbol{;}\AgdaSpace{}%
\AgdaFunction{-Π}\AgdaSymbol{)}\AgdaSpace{}
\AgdaKeyword{public}\<%
\end{code}
%\ccpad
%We use the  directive so that the types are available to all modules that import the present module.
% the \ualibhtml{Overture.Preliminaries} module.

\paragraph*{Projection notation}

The definition of \ar Σ (and thus \af ×) includes the fields  \af{pr₁} and \af{pr₂} representing the first and second projections out of the product.  Sometimes we prefer to denote these projections by \af{∣\_∣} and \af{∥\_∥}, respectively. However, for emphasis or readability we alternate between these and the following standard notations: \af{pr₁} and \af{fst} for the first projection, \af{pr₂} and \af{snd} for the second.  We define these alternative notations for projections as follows.
% \footnote{We have made a concerted effort to avoid duplicating types that are already provided elsewhere, such as the \typetopology library.  We occasionally repeat the definitions of such types for pedagogical purposes, but we always import and work with the original definitions in order to make the sources known and to credit the original authors.}
\ccpad
\begin{code}%
\>[0]\AgdaKeyword{module}\AgdaSpace{}%
\AgdaModule{\AgdaUnderscore{}}\AgdaSpace{}%
\AgdaSymbol{\{}\AgdaBound{𝓤}\AgdaSpace{}%
\AgdaSymbol{:}\AgdaSpace{}%
\AgdaPostulate{Universe}\AgdaSymbol{\}\{}\AgdaBound{A}\AgdaSpace{}%
\AgdaSymbol{:}\AgdaSpace{}%
\AgdaBound{𝓤}\AgdaSpace{}%
\AgdaOperator{\AgdaFunction{̇}}\AgdaSpace{}%
\AgdaSymbol{\}\{}\AgdaBound{B}\AgdaSpace{}%
\AgdaSymbol{:}\AgdaSpace{}%
\AgdaBound{A}\AgdaSpace{}%
\AgdaSymbol{→}\AgdaSpace{}%
\AgdaGeneralizable{𝓥}\AgdaSpace{}%
\AgdaOperator{\AgdaFunction{̇}}\AgdaSymbol{\}}\AgdaSpace{}%
\AgdaKeyword{where}\<%
\\
%
\\[\AgdaEmptyExtraSkip]%
\>[0][@{}l@{\AgdaIndent{0}}]%
\>[1]\AgdaOperator{\AgdaFunction{∣\AgdaUnderscore{}∣}}\AgdaSpace{}%
\AgdaFunction{fst}\AgdaSpace{}%
\AgdaSymbol{:}\AgdaSpace{}%
\AgdaRecord{Σ}\AgdaSpace{}%
\AgdaBound{B}\AgdaSpace{}%
\AgdaSymbol{→}\AgdaSpace{}%
\AgdaBound{A}\<%
\\
%
\>[1]\AgdaOperator{\AgdaFunction{∣}}\AgdaSpace{}%
\AgdaBound{x}\AgdaSpace{}%
\AgdaOperator{\AgdaInductiveConstructor{,}}\AgdaSpace{}%
\AgdaBound{y}\AgdaSpace{}%
\AgdaOperator{\AgdaFunction{∣}}\AgdaSpace{}%
\AgdaSymbol{=}\AgdaSpace{}%
\AgdaBound{x}\<%
\\
%
\>[1]\AgdaFunction{fst}\AgdaSpace{}%
\AgdaSymbol{(}\AgdaBound{x}\AgdaSpace{}%
\AgdaOperator{\AgdaInductiveConstructor{,}}\AgdaSpace{}%
\AgdaBound{y}\AgdaSymbol{)}\AgdaSpace{}%
\AgdaSymbol{=}\AgdaSpace{}%
\AgdaBound{x}\<%
\\
%
\\[\AgdaEmptyExtraSkip]%
%
\>[1]\AgdaOperator{\AgdaFunction{∥\AgdaUnderscore{}∥}}\AgdaSpace{}%
\AgdaFunction{snd}\AgdaSpace{}%
\AgdaSymbol{:}\AgdaSpace{}%
\AgdaSymbol{(}\AgdaBound{z}\AgdaSpace{}%
\AgdaSymbol{:}\AgdaSpace{}%
\AgdaRecord{Σ}\AgdaSpace{}%
\AgdaBound{B}\AgdaSymbol{)}\AgdaSpace{}%
\AgdaSymbol{→}\AgdaSpace{}%
\AgdaBound{B}\AgdaSpace{}%
\AgdaSymbol{(}\AgdaFunction{pr₁}\AgdaSpace{}%
\AgdaBound{z}\AgdaSymbol{)}\<%
\\
%
\>[1]\AgdaOperator{\AgdaFunction{∥}}\AgdaSpace{}%
\AgdaBound{x}\AgdaSpace{}%
\AgdaOperator{\AgdaInductiveConstructor{,}}\AgdaSpace{}%
\AgdaBound{y}\AgdaSpace{}%
\AgdaOperator{\AgdaFunction{∥}}\AgdaSpace{}%
\AgdaSymbol{=}\AgdaSpace{}%
\AgdaBound{y}\<%
\\
%
\>[1]\AgdaFunction{snd}\AgdaSpace{}%
\AgdaSymbol{(}\AgdaBound{x}\AgdaSpace{}%
\AgdaOperator{\AgdaInductiveConstructor{,}}\AgdaSpace{}%
\AgdaBound{y}\AgdaSymbol{)}\AgdaSpace{}%
\AgdaSymbol{=}\AgdaSpace{}%
\AgdaBound{y}\<%
\end{code}
\ccpad
\textbf{Remarks}.
\begin{itemize}
\item We place these definitions (of \AgdaOperator{\AgdaFunction{∣\AgdaUnderscore{}∣}}, \AgdaFunction{fst}, \AgdaOperator{\AgdaFunction{∥\AgdaUnderscore{}∥}} and \AgdaFunction{snd}) inside an \defn{anonymous module}, which is a module that begins with the \ak{module} keyword followed by an underscore character (instead of a module name). The purpose is to move some of the postulated typing judgments---the ``parameters'' of the module (e.g., \ab 𝓤 \as : \AgdaPostulate{Universe})---out of the way so they don't obfuscate the definitions inside the module.  In library documentation, such as the present paper, we often omit such module directives. In contrast, the collection of html pages at \ualibdotorg, which is the most current and comprehensive documentation of the \ualib, omits nothing.
\item As the four definitions above make clear, multiple inhabitants of a single type (e.g., \AgdaOperator{\AgdaFunction{∣\AgdaUnderscore{}∣}} and \AgdaFunction{fst}) may be declared on the same line.
\end{itemize}
