% -*- TeX-master: "ualib-part1.tex" -*-
%%% Local Variables: 
%%% mode: latex
%%% TeX-engine: 'xetex
%%% End:
\paragraph*{Unary relations}

In set theory, given two sets \ab A and \ab P, we say that \ab P is a \defn{subset} of \ab A, and we write \ab P \af ⊆ \ab A, just in case \as ∀~\ab x~(\ab x~\af ∈~\ab P~\as →~\ab x~\af ∈~\ab A). We need a mechanism for representing this notion in Agda. A typical approach is to use a \emph{unary predicate} type which we will denote by \af{Pred} and define as follows.  Given two universes \ab 𝓤 \ab 𝓦 and a type \ab A~\as :~\ab 𝓤\af ̇, the type \af{Pred}~\ab A~\ab 𝓦 represents \emph{properties} that inhabitants of \ab A may or may not satisfy.  We write \ab P~\as :~\af{Pred}~\ab A~\ab 𝓤 to represent the % semantic concept of the 
collection of inhabitants of \ab A that satisfy (or belong to) \ab P. Here is the definition.\footnote{\label{relunary}cf.~\texttt{Relation/Unary.agda} in the \agdastdlib.}
\ccpad
\begin{code}%
\>[0]\AgdaFunction{Pred}\AgdaSpace{}%
\AgdaSymbol{:}\AgdaSpace{}%
% \AgdaSymbol{\{}\AgdaBound{𝓤}\AgdaSpace{}%
% \AgdaSymbol{:}\AgdaSpace{}%
% \AgdaPostulate{Universe}\AgdaSymbol{\}}\AgdaSpace{}%
% \AgdaSymbol{→}\AgdaSpace{}%
\AgdaBound{𝓤}\AgdaSpace{}%
\AgdaOperator{\AgdaFunction{̇}}\AgdaSpace{}%
\AgdaSymbol{→}\AgdaSpace{}%
\AgdaSymbol{(}\AgdaBound{𝓦}\AgdaSpace{}%
\AgdaSymbol{:}\AgdaSpace{}%
\AgdaPostulate{Universe}\AgdaSymbol{)}\AgdaSpace{}%
\AgdaSymbol{→}\AgdaSpace{}%
\AgdaBound{𝓤}\AgdaSpace{}%
\AgdaOperator{\AgdaPrimitive{⊔}}\AgdaSpace{}%
\AgdaBound{𝓦}\AgdaSpace{}%
\AgdaOperator{\AgdaPrimitive{⁺}}\AgdaSpace{}%
\AgdaOperator{\AgdaFunction{̇}}\<%
\\
\>[0]\AgdaFunction{Pred}\AgdaSpace{}%
\AgdaBound{A}\AgdaSpace{}%
\AgdaBound{𝓦}\AgdaSpace{}%
\AgdaSymbol{=}\AgdaSpace{}%
\AgdaBound{A}\AgdaSpace{}%
\AgdaSymbol{→}\AgdaSpace{}%
\AgdaBound{𝓦}\AgdaSpace{}%
\AgdaOperator{\AgdaFunction{̇}}\<%
\end{code}
\ccpad
This is a general unary predicate type, but by taking the codomain to be \af{Bool} = \{0, 1\}, we obtain the usual interpretation of set membership; that is,
% \footnote{The type \af 𝟚 is defined in the \AgdaModule{MGS-MLTT} module of \typtop.}  
for \ab P~\as :~\af{Pred}~\ab A~\af{Bool} and for each \abt{x}{A}, we would interpret \ab P~\ab x~\ad ≡ 0 to mean \ab x ∉ \ab A, and \ab P~\ab x~\ad ≡ 1 to mean \ab x ∈ \ab A.


% To reiterate, given a type \ab A \as : \ab 𝓤\af ̇, we think of \af{Pred} \ab A \ab 𝓦 as the type of a property that inhabitants of \ab A may or may not satisfy. If \ab P \as : \af{Pred} \ab A \ab 𝓦, then we view \ab P as a collection of inhabitants of type \ab A that ``satisfy property \ab P,'' or that ``belong to the subset \ab P of \ab A.''
% Later we consider predicates over the class of algebras in a given signature. In the \ualibhtml{Algebras} module we will define the type \af{Algebra} \ab 𝓤 \ab 𝑆 of \ab 𝑆-algebras with domain type \ab 𝓤\af ̇, and the type \af{Pred} (\af{Algebra} \ab 𝓤 \ab 𝑆) \ab 𝓦 will represent classes of \ab 𝑆-algebras with certain properties.





%\subsubsection{Membership and inclusion relations}\label{membership-and-inclusion-relations}
% Like the \agdastdlib, t
The \ualib includes types that represent the \emph{element inclusion} and \emph{subset inclusion} relations from set theory. For example, given a predicate \af P, we may represent that  ``\ab x belongs to \af P'' or that ``\ab x has property \af{P},'' by writing either \ab x \af ∈ \af P or \af P \ab x.  The definition of \af ∈ is standard, as is the definition of \af{⊆} for the \defn{subset} relation; nonetheless, here they are.\cref{relunary}
\ccpad
\begin{code}%
\>[0]\AgdaOperator{\AgdaFunction{\AgdaUnderscore{}∈\AgdaUnderscore{}}}\AgdaSpace{}%
\AgdaSymbol{:}\AgdaSpace{}%
\AgdaBound{A}\AgdaSpace{}%
\AgdaSymbol{→}\AgdaSpace{}%
\AgdaFunction{Pred}\AgdaSpace{}%
\AgdaBound{A}\AgdaSpace{}%
\AgdaBound{𝓦}\AgdaSpace{}%
\AgdaSymbol{→}\AgdaSpace{}%
\AgdaBound{𝓦}\AgdaSpace{}%
\AgdaOperator{\AgdaFunction{̇}}\<%
\\
\>[0]\AgdaBound{x}\AgdaSpace{}%
\AgdaOperator{\AgdaFunction{∈}}\AgdaSpace{}%
\AgdaBound{P}\AgdaSpace{}%
\AgdaSymbol{=}\AgdaSpace{}%
\AgdaBound{P}\AgdaSpace{}%
\AgdaBound{x}\<%
\end{code}
\scpad
\begin{code}%
\>[1]\AgdaOperator{\AgdaFunction{\AgdaUnderscore{}⊆\AgdaUnderscore{}}}\AgdaSpace{}%
\AgdaSymbol{:}\AgdaSpace{}%
\AgdaFunction{Pred}\AgdaSpace{}%
\AgdaBound{A}\AgdaSpace{}%
\AgdaBound{𝓦}\AgdaSpace{}%
\AgdaSymbol{→}\AgdaSpace{}%
\AgdaFunction{Pred}\AgdaSpace{}%
\AgdaBound{A}\AgdaSpace{}%
\AgdaBound{𝓩}\AgdaSpace{}%
\AgdaSymbol{→}\AgdaSpace{}%
\AgdaBound{𝓤}\AgdaSpace{}%
\AgdaOperator{\AgdaPrimitive{⊔}}\AgdaSpace{}%
\AgdaBound{𝓦}\AgdaSpace{}%
\AgdaOperator{\AgdaPrimitive{⊔}}\AgdaSpace{}%
\AgdaBound{𝓩}\AgdaSpace{}%
\AgdaOperator{\AgdaFunction{̇}}\<%
\\
%
\>[1]\AgdaBound{P}\AgdaSpace{}%
\AgdaOperator{\AgdaFunction{⊆}}\AgdaSpace{}%
\AgdaBound{Q}\AgdaSpace{}%
\AgdaSymbol{=}\AgdaSpace{}%
\AgdaSymbol{∀}\AgdaSpace{}%
\AgdaSymbol{\{}\AgdaBound{x}\AgdaSymbol{\}}\AgdaSpace{}%
\AgdaSymbol{→}\AgdaSpace{}%
\AgdaBound{x}\AgdaSpace{}%
\AgdaOperator{\AgdaFunction{∈}}\AgdaSpace{}%
\AgdaBound{P}\AgdaSpace{}%
\AgdaSymbol{→}\AgdaSpace{}%
\AgdaBound{x}\AgdaSpace{}%
\AgdaOperator{\AgdaFunction{∈}}\AgdaSpace{}%
\AgdaBound{Q}\<%
\end{code}


\paragraph*{Predicates toolbox}
Here is a small collection of tools that will come in handy later. The first is an inductive type that represents \defn{disjoint union}.\footnote{\label{uhints}%
  \textbf{Unicode Hints}. In \agdamode, \texttt{\textbackslash{}u+} ↝ \af{⊎}, \texttt{\textbackslash{}b0} ↝ \af 𝟘, \texttt{\textbackslash{}B0} ↝ \af 𝟎.}
\ccpad % \texttt{\textbackslash{}.=} ↝ \af{≐}, , \texttt{\textbackslash{}b1} ↝ \af 𝟙, \texttt{\textbackslash{}B1} ↝ \af 𝟏
\begin{code}
\>[0]\AgdaKeyword{data}\AgdaSpace{}%
\AgdaOperator{\AgdaDatatype{\AgdaUnderscore{}⊎\AgdaUnderscore{}}}\AgdaSpace{}%
% \AgdaSymbol{\{}\AgdaBound{𝓤}\AgdaSpace{}%
% \AgdaBound{𝓦}\AgdaSpace{}%
% \AgdaSymbol{:}\AgdaSpace{}%
% \AgdaPostulate{Universe}\AgdaSymbol{\}}
\AgdaSymbol{(}\AgdaBound{A}\AgdaSpace{}%
\AgdaSymbol{:}\AgdaSpace{}%
\AgdaBound{𝓤}\AgdaSpace{}%
\AgdaOperator{\AgdaFunction{̇}}\AgdaSymbol{)}\AgdaSpace{}%
\AgdaSymbol{(}\AgdaBound{B}\AgdaSpace{}%
\AgdaSymbol{:}\AgdaSpace{}%
\AgdaBound{𝓦}\AgdaSpace{}%
\AgdaOperator{\AgdaFunction{̇}}\AgdaSymbol{)}\AgdaSpace{}%
\AgdaSymbol{:}\AgdaSpace{}%
\AgdaBound{𝓤}\AgdaSpace{}%
\AgdaOperator{\AgdaPrimitive{⊔}}\AgdaSpace{}%
\AgdaBound{𝓦}\AgdaSpace{}%
\AgdaOperator{\AgdaFunction{̇}}\AgdaSpace{}%
\AgdaKeyword{where}\<%
\\
\>[0][@{}l@{\AgdaIndent{0}}]%
\>[1]\AgdaInductiveConstructor{inj₁}\AgdaSpace{}%
\AgdaSymbol{:}\AgdaSpace{}%
\AgdaSymbol{(}\AgdaBound{x}\AgdaSpace{}%
\AgdaSymbol{:}\AgdaSpace{}%
\AgdaBound{A}\AgdaSymbol{)}\AgdaSpace{}%
\AgdaSymbol{→}\AgdaSpace{}%
\AgdaBound{A}\AgdaSpace{}%
\AgdaOperator{\AgdaDatatype{⊎}}\AgdaSpace{}%
\AgdaBound{B}\<%
\\
%
\>[1]\AgdaInductiveConstructor{inj₂}\AgdaSpace{}%
\AgdaSymbol{:}\AgdaSpace{}%
\AgdaSymbol{(}\AgdaBound{y}\AgdaSpace{}%
\AgdaSymbol{:}\AgdaSpace{}%
\AgdaBound{B}\AgdaSymbol{)}\AgdaSpace{}%
\AgdaSymbol{→}\AgdaSpace{}%
\AgdaBound{A}\AgdaSpace{}%
\AgdaOperator{\AgdaDatatype{⊎}}\AgdaSpace{}%
\AgdaBound{B}\<%
\end{code}
\ccpad
And this can be used to define a type representing \defn{union}, as follows.
\ccpad
\begin{code}
\>[0]\AgdaOperator{\AgdaFunction{\AgdaUnderscore{}∪\AgdaUnderscore{}}}\AgdaSpace{}%
\AgdaSymbol{:}\AgdaSpace{}%
% \AgdaSymbol{\{}\AgdaBound{𝓤}\AgdaSpace{}%
% \AgdaBound{𝓦}\AgdaSpace{}%
% \AgdaBound{𝓩}\AgdaSpace{}%
% \AgdaSymbol{:}\AgdaSpace{}%
% \AgdaPostulate{Universe}\AgdaSymbol{\}\{}\AgdaBound{A}\AgdaSpace{}%
% \AgdaSymbol{:}\AgdaSpace{}%
% \AgdaBound{𝓤}\AgdaSpace{}%
% \AgdaOperator{\AgdaFunction{̇}}\AgdaSymbol{\}}\AgdaSpace{}%
% \AgdaSymbol{→}\AgdaSpace{}%
\AgdaFunction{Pred}\AgdaSpace{}%
\AgdaBound{A}\AgdaSpace{}%
\AgdaBound{𝓦}\AgdaSpace{}%
\AgdaSymbol{→}\AgdaSpace{}%
\AgdaFunction{Pred}\AgdaSpace{}%
\AgdaBound{A}\AgdaSpace{}%
\AgdaBound{𝓩}\AgdaSpace{}%
\AgdaSymbol{→}\AgdaSpace{}%
\AgdaFunction{Pred}\AgdaSpace{}%
\AgdaBound{A}\AgdaSpace{}%
\AgdaSymbol{(}\AgdaBound{𝓦}\AgdaSpace{}%
\AgdaOperator{\AgdaPrimitive{⊔}}\AgdaSpace{}%
\AgdaBound{𝓩}\AgdaSymbol{)}\<%
\\
\>[0]\AgdaBound{P}\AgdaSpace{}%
\AgdaOperator{\AgdaFunction{∪}}\AgdaSpace{}%
\AgdaBound{Q}\AgdaSpace{}%
\AgdaSymbol{=}\AgdaSpace{}%
\AgdaSymbol{λ}\AgdaSpace{}%
\AgdaBound{x}\AgdaSpace{}%
\AgdaSymbol{→}\AgdaSpace{}%
\AgdaBound{x}\AgdaSpace{}%
\AgdaOperator{\AgdaFunction{∈}}\AgdaSpace{}%
\AgdaBound{P}\AgdaSpace{}%
\AgdaOperator{\AgdaDatatype{⊎}}\AgdaSpace{}%
\AgdaBound{x}\AgdaSpace{}%
\AgdaOperator{\AgdaFunction{∈}}\AgdaSpace{}%
\AgdaBound{Q}\<%
\end{code}
\ccpad
Next we define convenient notation for asserting that the image of a function (the first argument) is contained in a predicate (the second argument).
\ccpad
\begin{code}
\>[1]\AgdaOperator{\AgdaFunction{Im\AgdaUnderscore{}⊆\AgdaUnderscore{}}}\AgdaSpace{}%
\AgdaSymbol{:}\AgdaSpace{}%
\AgdaSymbol{(}\AgdaBound{A}\AgdaSpace{}%
\AgdaSymbol{→}\AgdaSpace{}%
\AgdaBound{B}\AgdaSymbol{)}\AgdaSpace{}%
\AgdaSymbol{→}\AgdaSpace{}%
\AgdaFunction{Pred}\AgdaSpace{}%
\AgdaBound{B}\AgdaSpace{}%
\AgdaBound{𝓩}\AgdaSpace{}%
\AgdaSymbol{→}\AgdaSpace{}%
\AgdaBound{𝓤}\AgdaSpace{}%
\AgdaOperator{\AgdaPrimitive{⊔}}\AgdaSpace{}%
\AgdaBound{𝓩}\AgdaSpace{}%
\AgdaOperator{\AgdaFunction{̇}}\<%
\\
%
\>[1]\AgdaOperator{\AgdaFunction{Im}}\AgdaSpace{}%
\AgdaBound{f}\AgdaSpace{}%
\AgdaOperator{\AgdaFunction{⊆}}\AgdaSpace{}%
\AgdaBound{S}\AgdaSpace{}%
\AgdaSymbol{=}\AgdaSpace{}%
\AgdaSymbol{∀}\AgdaSpace{}%
\AgdaBound{x}\AgdaSpace{}%
\AgdaSymbol{→}\AgdaSpace{}%
\AgdaBound{f}\AgdaSpace{}%
\AgdaBound{x}\AgdaSpace{}%
\AgdaOperator{\AgdaFunction{∈}}\AgdaSpace{}%
\AgdaBound{S}\<%
\end{code}
\ccpad
The \defn{empty set} is naturally represented by the \defn{empty type}, \af 𝟘, and the latter is defined in the \am{Empty-Type} module of \typtop.\cref{uhints}$^, $\footnote{%
The empty type is an inductive type with no constructors; that is, \AgdaKeyword{data}\AgdaSpace{}%
\AgdaDatatype{𝟘}\AgdaSpace{}%
\AgdaSymbol{\{}\AgdaBound{𝓤}\AgdaSymbol{\}}\AgdaSpace{}%
\AgdaSymbol{:}\AgdaSpace{}%
\AgdaBound{𝓤}\AgdaSpace{}%
\AgdaOperator{\AgdaFunction{̇}}\AgdaSpace{}%
\AgdaKeyword{where}\AgdaSpace{}%
\AgdaComment{-- (empty body)}.}
\ccpad
\begin{code}
% \>[0]\AgdaKeyword{open}\AgdaSpace{}%
% \AgdaKeyword{import}\AgdaSpace{}%
% \AgdaModule{Empty-Type}\AgdaSpace{}%
% \AgdaKeyword{using}\AgdaSpace{}%
% \AgdaSymbol{(}\AgdaDatatype{𝟘}\AgdaSymbol{)}\<%
% \\
% %
% \\[\AgdaEmptyExtraSkip]%
\>[0]\AgdaFunction{∅}\AgdaSpace{}%
\AgdaSymbol{:}\AgdaSpace{}%
% \AgdaSymbol{\{}\AgdaBound{𝓤}\AgdaSpace{}%
% \AgdaSymbol{:}\AgdaSpace{}%
% \AgdaPostulate{Universe}\AgdaSymbol{\}\{}\AgdaBound{A}\AgdaSpace{}%
% \AgdaSymbol{:}\AgdaSpace{}%
% \AgdaBound{𝓤}\AgdaSpace{}%
% \AgdaOperator{\AgdaFunction{̇}}\AgdaSymbol{\}}\AgdaSpace{}%
% \AgdaSymbol{→}\AgdaSpace{}%
\AgdaFunction{Pred}\AgdaSpace{}%
\AgdaBound{A}\AgdaSpace{}%
\AgdaPrimitive{𝓤₀}\<%
\\
\>[0]\AgdaFunction{∅}\AgdaSpace{}%
\AgdaSymbol{\AgdaUnderscore{}}\AgdaSpace{}%
\AgdaSymbol{=}\AgdaSpace{}%
\AgdaDatatype{𝟘}\<%
\end{code}
\ccpad
We close our little predicates toolbox with a natural way to encode \defn{singletons}.
\ccpad
\begin{code}
\>[0]\AgdaOperator{\AgdaFunction{{\AgdaUnderscore{}}}}\AgdaSpace{}%
\AgdaSymbol{:}\AgdaSpace{}%
% \AgdaSymbol{\{}\AgdaBound{𝓤}\AgdaSpace{}%
% \AgdaSymbol{:}\AgdaSpace{}%
% \AgdaPostulate{Universe}\AgdaSymbol{\}\{}\AgdaBound{A}\AgdaSpace{}%
% \AgdaSymbol{:}\AgdaSpace{}%
% \AgdaBound{𝓤}\AgdaSpace{}%
% \AgdaOperator{\AgdaFunction{̇}}\AgdaSymbol{\}}\AgdaSpace{}%
% \AgdaSymbol{→}\AgdaSpace{}%
\AgdaBound{A}\AgdaSpace{}%
\AgdaSymbol{→}\AgdaSpace{}%
\AgdaFunction{Pred}\AgdaSpace{}%
\AgdaBound{A}\AgdaSpace{}%
\AgdaSymbol{\AgdaUnderscore{}}\<%
\\
\>[0]\AgdaOperator{\AgdaFunction{{}}\AgdaSpace{}%
\AgdaBound{x}\AgdaSpace{}%
\AgdaOperator{\AgdaFunction{}}}\AgdaSpace{}%
\AgdaSymbol{=}\AgdaSpace{}%
\AgdaBound{x}\AgdaSpace{}%
\AgdaOperator{\AgdaDatatype{≡\AgdaUnderscore{}}}\<%
\end{code}





\paragraph*{Binary Relations} %\label{sec:binary-relations}
In set theory, a binary relation on a set \ab{A} is simply a subset of the Cartesian product \ab A × \ab A. As such, we could model such a relation as a (unary) predicate over the product type \ab A \af × \ab A, or as an inhabitant of the function type \ab A \as → \ab A \as → \ab 𝓦\af ̇ (for some universe \ab 𝓦).  Note, however, this is not the same as a unary predicate over the function type \ab A \as → \ab A since the latter has type  (\ab A~\as →~\ab A)~\as →~\ab 𝓦\af ̇, while a binary relation should have type \ab A~\as →~(\ab A~\as →~\ab 𝓦\af ̇). 
% \footnote{Note that a binary relation from \ab A to \ab B is not simply a unary predicate over the type \ab A~\as →~\ab B.  The binary relation has type \ab A~\as →~(\ab B~\as →~\ab 𝓩\af ̇) whereas a unary predicate over \ab A~\as →~\ab B has type (\ab A~\as →~\ab B)~\as →~\ab 𝓩\af ̇.}

A generalization of the notion of binary relation is a \emph{relation from} \ab{A} \emph{to} \ab{B}, which we define first and treat binary relations on a single \ab{A} as a special case.
\ccpad
\begin{code}
\>[1]\AgdaFunction{REL}\AgdaSpace{}%
\AgdaSymbol{:}\AgdaSpace{}%
\AgdaBound{𝓤}%
\AgdaOperator{\AgdaFunction{̇}}\AgdaSpace{}%
\AgdaSymbol{→}\AgdaSpace{}%
\AgdaBound{𝓦}%
\AgdaOperator{\AgdaFunction{̇}}\AgdaSpace{}%
\AgdaSymbol{→}\AgdaSpace{}%
\AgdaSymbol{(}\AgdaBound{𝓩}\AgdaSpace{}%
\AgdaSymbol{:}\AgdaSpace{}%
\AgdaPostulate{Universe}\AgdaSymbol{)}\AgdaSpace{}%
\AgdaSymbol{→}\AgdaSpace{}%
\AgdaBound{𝓤}\AgdaSpace{}%
\AgdaOperator{\AgdaPrimitive{⊔}}\AgdaSpace{}%
\AgdaBound{𝓦}\AgdaSpace{}%
\AgdaOperator{\AgdaPrimitive{⊔}}\AgdaSpace{}%
\AgdaBound{𝓩}\AgdaSpace{}%
\AgdaOperator{\AgdaPrimitive{⁺}}\AgdaSpace{}%
\AgdaOperator{\AgdaFunction{̇}}\<%
\\
%
\>[1]\AgdaFunction{REL}\AgdaSpace{}%
\AgdaBound{A}\AgdaSpace{}%
\AgdaBound{B}\AgdaSpace{}%
\AgdaBound{𝓩}\AgdaSpace{}%
\AgdaSymbol{=}\AgdaSpace{}%
\AgdaBound{A}\AgdaSpace{}%
\AgdaSymbol{→}\AgdaSpace{}%
\AgdaBound{B}\AgdaSpace{}%
\AgdaSymbol{→}\AgdaSpace{}%
\AgdaBound{𝓩}%
\AgdaOperator{\AgdaFunction{̇}}\<%
\\
%
\\[\AgdaEmptyExtraSkip]%
%
\>[1]\AgdaFunction{Rel}\AgdaSpace{}%
\AgdaSymbol{:}\AgdaSpace{}%
\AgdaBound{𝓤}%
\AgdaOperator{\AgdaFunction{̇}}\AgdaSpace{}%
\AgdaSymbol{→}\AgdaSpace{}%
\AgdaSymbol{(}\AgdaBound{𝓩}\AgdaSpace{}%
\AgdaSymbol{:}\AgdaSpace{}%
\AgdaPostulate{Universe}\AgdaSymbol{)}\AgdaSpace{}%
\AgdaSymbol{→}\AgdaSpace{}%
\AgdaBound{𝓤}\AgdaSpace{}%
\AgdaOperator{\AgdaPrimitive{⊔}}\AgdaSpace{}%
\AgdaBound{𝓩}\AgdaSpace{}%
\AgdaOperator{\AgdaPrimitive{⁺}}%
\AgdaOperator{\AgdaFunction{̇}}\<%
\\
%
\>[1]\AgdaFunction{Rel}\AgdaSpace{}%
\AgdaBound{A}\AgdaSpace{}%
\AgdaBound{𝓩}\AgdaSpace{}%
\AgdaSymbol{=}\AgdaSpace{}%
\AgdaFunction{REL}\AgdaSpace{}%
\AgdaBound{A}\AgdaSpace{}%
\AgdaBound{A}\AgdaSpace{}%
\AgdaBound{𝓩}\<%
\end{code}

\paragraph*{The kernel of a function}
The \defn{kernel} of a function \ab f \as : \ab A \AgdaSymbol{→} \ab B is defined informally by \{(\ab x , \ab y) ∈ \ab A × \ab A : \ab f \ab x = \ab f \ab y\}. This can be represented in type theory in a number of ways, each of which may be useful in a particular context. For example, we could define the kernel to be an inhabitant of a (binary) relation type, a (unary) predicate type, a (curried) Sigma type, or an (uncurried) Sigma type. Since the first two alternatives are the ones we use thoughout the \ualib, we present them here.
\ccpad
\begin{code}%
\>[1]\AgdaFunction{ker}\AgdaSpace{}%
\AgdaSymbol{:}\AgdaSpace{}%
\AgdaSymbol{(}\AgdaBound{A}\AgdaSpace{}%
\AgdaSymbol{→}\AgdaSpace{}%
\AgdaBound{B}\AgdaSymbol{)}\AgdaSpace{}%
\AgdaSymbol{→}\AgdaSpace{}%
\AgdaFunction{Rel}\AgdaSpace{}%
\AgdaBound{A}\AgdaSpace{}%
\AgdaBound{𝓦}\<%
\\
%
\>[1]\AgdaFunction{ker}\AgdaSpace{}%
\AgdaBound{g}\AgdaSpace{}%
\AgdaBound{x}\AgdaSpace{}%
\AgdaBound{y}\AgdaSpace{}%
\AgdaSymbol{=}\AgdaSpace{}%
\AgdaBound{g}\AgdaSpace{}%
\AgdaBound{x}\AgdaSpace{}%
\AgdaOperator{\AgdaDatatype{≡}}\AgdaSpace{}%
\AgdaBound{g}\AgdaSpace{}%
\AgdaBound{y}\<%
\\
%
\\[\AgdaEmptyExtraSkip]%
%
\>[1]\AgdaFunction{kernel}\AgdaSpace{}%
\AgdaSymbol{:}\AgdaSpace{}%
\AgdaSymbol{(}\AgdaBound{A}\AgdaSpace{}%
\AgdaSymbol{→}\AgdaSpace{}%
\AgdaBound{B}\AgdaSymbol{)}\AgdaSpace{}%
\AgdaSymbol{→}\AgdaSpace{}%
\AgdaFunction{Pred}\AgdaSpace{}%
\AgdaSymbol{(}\AgdaBound{A}\AgdaSpace{}%
\AgdaOperator{\AgdaFunction{×}}\AgdaSpace{}%
\AgdaBound{A}\AgdaSymbol{)}\AgdaSpace{}%
\AgdaBound{𝓦}\<%
\\
%
\>[1]\AgdaFunction{kernel}\AgdaSpace{}%
\AgdaBound{g}\AgdaSpace{}%
\AgdaSymbol{(}\AgdaBound{x}\AgdaSpace{}%
\AgdaOperator{\AgdaInductiveConstructor{,}}\AgdaSpace{}%
\AgdaBound{y}\AgdaSymbol{)}\AgdaSpace{}%
\AgdaSymbol{=}\AgdaSpace{}%
\AgdaBound{g}\AgdaSpace{}%
\AgdaBound{x}\AgdaSpace{}%
\AgdaOperator{\AgdaDatatype{≡}}\AgdaSpace{}%
\AgdaBound{g}\AgdaSpace{}%
\AgdaBound{y}\<%
% \\
% %
% \\[\AgdaEmptyExtraSkip]%
% %
% \>[1]\AgdaFunction{ker-sigma}\AgdaSpace{}%
% \AgdaSymbol{:}\AgdaSpace{}%
% \AgdaSymbol{(}\AgdaBound{A}\AgdaSpace{}%
% \AgdaSymbol{→}\AgdaSpace{}%
% \AgdaBound{B}\AgdaSymbol{)}\AgdaSpace{}%
% \AgdaSymbol{→}\AgdaSpace{}%
% \AgdaBound{𝓤}\AgdaSpace{}%
% \AgdaOperator{\AgdaPrimitive{⊔}}\AgdaSpace{}%
% \AgdaBound{𝓦}\AgdaSpace{}%
% \AgdaOperator{\AgdaFunction{̇}}\<%
% \\
% %
% \>[1]\AgdaFunction{ker-sigma}\AgdaSpace{}%
% \AgdaBound{g}\AgdaSpace{}%
% \AgdaSymbol{=}\AgdaSpace{}%
% \AgdaFunction{Σ}\AgdaSpace{}%
% \AgdaBound{x}\AgdaSpace{}%
% \AgdaFunction{꞉}\AgdaSpace{}%
% \AgdaBound{A}\AgdaSpace{}%
% \AgdaFunction{,}\AgdaSpace{}%
% \AgdaFunction{Σ}\AgdaSpace{}%
% \AgdaBound{y}\AgdaSpace{}%
% \AgdaFunction{꞉}\AgdaSpace{}%
% \AgdaBound{A}\AgdaSpace{}%
% \AgdaFunction{,}\AgdaSpace{}%
% \AgdaBound{g}\AgdaSpace{}%
% \AgdaBound{x}\AgdaSpace{}%
% \AgdaOperator{\AgdaDatatype{≡}}\AgdaSpace{}%
% \AgdaBound{g}\AgdaSpace{}%
% \AgdaBound{y}\<%
% \\
% %
% \\[\AgdaEmptyExtraSkip]%
% %
% \>[1]\AgdaFunction{ker-sigma'}\AgdaSpace{}%
% \AgdaSymbol{:}\AgdaSpace{}%
% \AgdaSymbol{(}\AgdaBound{A}\AgdaSpace{}%
% \AgdaSymbol{→}\AgdaSpace{}%
% \AgdaBound{B}\AgdaSymbol{)}\AgdaSpace{}%
% \AgdaSymbol{→}\AgdaSpace{}%
% \AgdaBound{𝓤}\AgdaSpace{}%
% \AgdaOperator{\AgdaPrimitive{⊔}}\AgdaSpace{}%
% \AgdaBound{𝓦}\AgdaSpace{}%
% \AgdaOperator{\AgdaFunction{̇}}\<%
% \\
% %
% \>[1]\AgdaFunction{ker-sigma'}\AgdaSpace{}%
% \AgdaBound{g}\AgdaSpace{}%
% \AgdaSymbol{=}\AgdaSpace{}%
% \AgdaFunction{Σ}\AgdaSpace{}%
% \AgdaBound{(x}\AgdaSpace{}%
% \AgdaBound{,}\AgdaSpace{}%
% \AgdaBound{y)}\AgdaSpace{}%
% \AgdaFunction{꞉}\AgdaSpace{}%
% \AgdaSymbol{(}\AgdaBound{A}\AgdaSpace{}%
% \AgdaOperator{\AgdaFunction{×}}\AgdaSpace{}%
% \AgdaBound{A}\AgdaSymbol{)}\AgdaSpace{}%
% \AgdaFunction{,}\AgdaSpace{}%
% \AgdaBound{g}\AgdaSpace{}%
% \AgdaBound{x}\AgdaSpace{}%
% \AgdaOperator{\AgdaDatatype{≡}}\AgdaSpace{}%
% \AgdaBound{g}\AgdaSpace{}%
% \AgdaBound{y}\<%
\end{code}
\ccpad
Similarly, the \defn{identity relation} (which is equivalent to the kernel of an injective function) can be represented by a number of different types. Here we only show the representation that we use later to construct the zero congruence. The notation we use here is close to that of conventional algebra notation, where $0_A$ is used to denote the identity relation $\{(x, y) \in A \times A : x = y\}$.\cref{uhints}
\ccpad
\begin{code}%
\>[1]\AgdaFunction{𝟎}\AgdaSpace{}%
\AgdaSymbol{:}\AgdaSpace{}%
\AgdaFunction{Rel}\AgdaSpace{}%
\AgdaBound{A}\AgdaSpace{}%
\AgdaBound{𝓤}\<%
\\
%
\>[1]\AgdaFunction{𝟎}\AgdaSpace{}%
\AgdaBound{x}\AgdaSpace{}%
\AgdaBound{y}\AgdaSpace{}%
\AgdaSymbol{=}\AgdaSpace{}%
\AgdaBound{x}\AgdaSpace{}%
\AgdaOperator{\AgdaDatatype{≡}}\AgdaSpace{}%
\AgdaBound{y}\<%
% \\
% %
% \\[\AgdaEmptyExtraSkip]%
% %
% \>[1]\AgdaFunction{𝟎-pred}\AgdaSpace{}%
% \AgdaSymbol{:}\AgdaSpace{}%
% \AgdaFunction{Pred}\AgdaSpace{}%
% \AgdaSymbol{(}\AgdaBound{A}\AgdaSpace{}%
% \AgdaOperator{\AgdaFunction{×}}\AgdaSpace{}%
% \AgdaBound{A}\AgdaSymbol{)}\AgdaSpace{}%
% \AgdaBound{𝓤}\<%
% \\
% %
% \>[1]\AgdaFunction{𝟎-pred}\AgdaSpace{}%
% \AgdaSymbol{(}\AgdaBound{x}\AgdaSpace{}%
% \AgdaOperator{\AgdaInductiveConstructor{,}}\AgdaSpace{}%
% \AgdaBound{y}\AgdaSymbol{)}\AgdaSpace{}%
% \AgdaSymbol{=}\AgdaSpace{}%
% \AgdaBound{x}\AgdaSpace{}%
% \AgdaOperator{\AgdaDatatype{≡}}\AgdaSpace{}%
% \AgdaBound{y}\<%
% \\
% %
% \\[\AgdaEmptyExtraSkip]%
% %
% \>[1]\AgdaFunction{𝟎-sigma}\AgdaSpace{}%
% \AgdaSymbol{:}\AgdaSpace{}%
% \AgdaBound{𝓤}\AgdaSpace{}%
% \AgdaOperator{\AgdaFunction{̇}}\<%
% \\
% %
% \>[1]\AgdaFunction{𝟎-sigma}\AgdaSpace{}%
% \AgdaSymbol{=}\AgdaSpace{}%
% \AgdaFunction{Σ}\AgdaSpace{}%
% \AgdaBound{x}\AgdaSpace{}%
% \AgdaFunction{꞉}\AgdaSpace{}%
% \AgdaBound{A}\AgdaSpace{}%
% \AgdaFunction{,}\AgdaSpace{}%
% \AgdaFunction{Σ}\AgdaSpace{}%
% \AgdaBound{y}\AgdaSpace{}%
% \AgdaFunction{꞉}\AgdaSpace{}%
% \AgdaBound{A}\AgdaSpace{}%
% \AgdaFunction{,}\AgdaSpace{}%
% \AgdaBound{x}\AgdaSpace{}%
% \AgdaOperator{\AgdaDatatype{≡}}\AgdaSpace{}%
% \AgdaBound{y}\<%
% \\
% %
% \\[\AgdaEmptyExtraSkip]%
% %
% \>[1]\AgdaFunction{𝟎-sigma'}\AgdaSpace{}%
% \AgdaSymbol{:}\AgdaSpace{}%
% \AgdaBound{𝓤}\AgdaSpace{}%
% \AgdaOperator{\AgdaFunction{̇}}\<%
% \\
% %
% \>[1]\AgdaFunction{𝟎-sigma'}\AgdaSpace{}%
% \AgdaSymbol{=}\AgdaSpace{}%
% \AgdaFunction{Σ}\AgdaSpace{}%
% \AgdaBound{(x}\AgdaSpace{}%
% \AgdaBound{,}\AgdaSpace{}%
% \AgdaBound{y)}\AgdaSpace{}%
% \AgdaFunction{꞉}\AgdaSpace{}%
% \AgdaSymbol{(}\AgdaBound{A}\AgdaSpace{}%
% \AgdaOperator{\AgdaFunction{×}}\AgdaSpace{}%
% \AgdaBound{A}\AgdaSymbol{)}\AgdaSpace{}%
% \AgdaFunction{,}\AgdaSpace{}%
% \AgdaBound{x}\AgdaSpace{}%
% \AgdaOperator{\AgdaDatatype{≡}}\AgdaSpace{}%
% \AgdaBound{y}\<%
\end{code}
% \ccpad
% Finally, the \defn{total relation} over \ab A, which in set theory is the full Cartesian product \ab A~\af ×~\ab A, can be represented using the one-element type from the \am{Unit-Type} module of \typtop as follows.\cref{uhints}$^, $\footnote{%
% The one-element type is an inductive type with a single constructor, denoted \AgdaInductiveConstructor{⋆}, as follows: \AgdaKeyword{data}\AgdaSpace{}%
% \AgdaDatatype{𝟙}\AgdaSpace{}%
% \AgdaSymbol{\{}\AgdaBound{𝓤}\AgdaSymbol{\}}\AgdaSpace{}%
% \AgdaSymbol{:}\AgdaSpace{}%
% \AgdaBound{𝓤}\AgdaSpace{}%
% \AgdaOperator{\AgdaFunction{̇}}\AgdaSpace{}%
% \AgdaKeyword{where}\AgdaSpace{}\AgdaInductiveConstructor{⋆}\AgdaSpace{}%
% \AgdaSymbol{:}\AgdaSpace{}%
% \AgdaDatatype{𝟙}.}
% \ccpad
% \begin{code}%
% \>[1]\AgdaFunction{𝟏}\AgdaSpace{}%
% \AgdaSymbol{:}\AgdaSpace{}%
% \AgdaFunction{Rel}\AgdaSpace{}%
% \AgdaBound{A}\AgdaSpace{}%
% \AgdaPrimitive{𝓤₀}\<%
% \\
% %
% \>[1]\AgdaFunction{𝟏}\AgdaSpace{}%
% \AgdaBound{a}\AgdaSpace{}%
% \AgdaBound{b}\AgdaSpace{}%
% \AgdaSymbol{=}\AgdaSpace{}%
% \AgdaFunction{𝟙}\<%
% \end{code}


\paragraph*{The implication relation\protect\footnotemark}
\footnotetext{The definitions here are from the \agdastdlib, translated into \typtop/\ualib notation.}

The following types represent \defn{implication} for binary relations.
\ccpad
\begin{code}%
\>[0]\AgdaOperator{\AgdaFunction{\AgdaUnderscore{}on\AgdaUnderscore{}}}\AgdaSpace{}%
\AgdaSymbol{:}\AgdaSpace{}%
\AgdaSymbol{(}\AgdaBound{B}\AgdaSpace{}%
\AgdaSymbol{→}\AgdaSpace{}%
\AgdaBound{B}\AgdaSpace{}%
\AgdaSymbol{→}\AgdaSpace{}%
\AgdaBound{C}\AgdaSymbol{)}\AgdaSpace{}%
\AgdaSymbol{→}\AgdaSpace{}%
\AgdaSymbol{(}\AgdaBound{A}\AgdaSpace{}%
\AgdaSymbol{→}\AgdaSpace{}%
\AgdaBound{B}\AgdaSymbol{)}\AgdaSpace{}%
\AgdaSymbol{→}\AgdaSpace{}%
\AgdaSymbol{(}\AgdaBound{A}\AgdaSpace{}%
\AgdaSymbol{→}\AgdaSpace{}%
\AgdaBound{A}\AgdaSpace{}%
\AgdaSymbol{→}\AgdaSpace{}%
\AgdaBound{C}\AgdaSymbol{)}\<%
\\
\>[0]\AgdaBound{R}\AgdaSpace{}%
\AgdaOperator{\AgdaFunction{on}}\AgdaSpace{}%
\AgdaBound{g}\AgdaSpace{}%
\AgdaSymbol{=}\AgdaSpace{}%
\AgdaSymbol{λ}\AgdaSpace{}%
\AgdaBound{x}\AgdaSpace{}%
\AgdaBound{y}\AgdaSpace{}%
\AgdaSymbol{→}\AgdaSpace{}%
\AgdaBound{R}\AgdaSpace{}%
\AgdaSymbol{(}\AgdaBound{g}\AgdaSpace{}%
\AgdaBound{x}\AgdaSymbol{)}\AgdaSpace{}%
\AgdaSymbol{(}\AgdaBound{g}\AgdaSpace{}%
\AgdaBound{y}\AgdaSymbol{)}\<%
\end{code}
\scpad
\begin{code}
\>[1]\AgdaOperator{\AgdaFunction{\AgdaUnderscore{}⇒\AgdaUnderscore{}}}\AgdaSpace{}%
\AgdaSymbol{:}\AgdaSpace{}%
\AgdaFunction{REL}\AgdaSpace{}%
\AgdaBound{A}\AgdaSpace{}%
\AgdaBound{B}\AgdaSpace{}%
\AgdaBound{𝓧}\AgdaSpace{}%
\AgdaSymbol{→}\AgdaSpace{}%
\AgdaFunction{REL}\AgdaSpace{}%
\AgdaBound{A}\AgdaSpace{}%
\AgdaBound{B}\AgdaSpace{}%
\AgdaBound{𝓨}\AgdaSpace{}%
\AgdaSymbol{→}\AgdaSpace{}%
\AgdaBound{𝓤}\AgdaSpace{}%
\AgdaOperator{\AgdaPrimitive{⊔}}\AgdaSpace{}%
\AgdaBound{𝓦}\AgdaSpace{}%
\AgdaOperator{\AgdaPrimitive{⊔}}\AgdaSpace{}%
\AgdaBound{𝓧}\AgdaSpace{}%
\AgdaOperator{\AgdaPrimitive{⊔}}\AgdaSpace{}%
\AgdaBound{𝓨}\AgdaSpace{}%
\AgdaOperator{\AgdaFunction{̇}}\<%
\\
%
\>[1]\AgdaBound{P}\AgdaSpace{}%
\AgdaOperator{\AgdaFunction{⇒}}\AgdaSpace{}%
\AgdaBound{Q}\AgdaSpace{}%
\AgdaSymbol{=}\AgdaSpace{}%
\AgdaSymbol{∀}\AgdaSpace{}%
\AgdaSymbol{\{}\AgdaBound{i}\AgdaSpace{}%
\AgdaBound{j}\AgdaSymbol{\}}\AgdaSpace{}%
\AgdaSymbol{→}\AgdaSpace{}%
\AgdaBound{P}\AgdaSpace{}%
\AgdaBound{i}\AgdaSpace{}%
\AgdaBound{j}\AgdaSpace{}%
\AgdaSymbol{→}\AgdaSpace{}%
\AgdaBound{Q}\AgdaSpace{}%
\AgdaBound{i}\AgdaSpace{}%
\AgdaBound{j}\<%
\end{code}
\ccpad
These combine to give a nice, general implication operation.
\ccpad
\begin{code}%
\>[1]\AgdaOperator{\AgdaFunction{\AgdaUnderscore{}=[\AgdaUnderscore{}]⇒\AgdaUnderscore{}}}\AgdaSpace{}%
\AgdaSymbol{:}\AgdaSpace{}%
\AgdaFunction{Rel}\AgdaSpace{}%
\AgdaBound{A}\AgdaSpace{}%
\AgdaBound{𝓧}\AgdaSpace{}%
\AgdaSymbol{→}\AgdaSpace{}%
\AgdaSymbol{(}\AgdaBound{A}\AgdaSpace{}%
\AgdaSymbol{→}\AgdaSpace{}%
\AgdaBound{B}\AgdaSymbol{)}\AgdaSpace{}%
\AgdaSymbol{→}\AgdaSpace{}%
\AgdaFunction{Rel}\AgdaSpace{}%
\AgdaBound{B}\AgdaSpace{}%
\AgdaBound{𝓨}\AgdaSpace{}%
\AgdaSymbol{→}\AgdaSpace{}%
\AgdaBound{𝓤}\AgdaSpace{}%
\AgdaOperator{\AgdaPrimitive{⊔}}\AgdaSpace{}%
\AgdaBound{𝓧}\AgdaSpace{}%
\AgdaOperator{\AgdaPrimitive{⊔}}\AgdaSpace{}%
\AgdaBound{𝓨}\AgdaSpace{}%
\AgdaOperator{\AgdaFunction{̇}}\<%
\\
%
\>[1]\AgdaBound{P}\AgdaSpace{}%
\AgdaOperator{\AgdaFunction{=[}}\AgdaSpace{}%
\AgdaBound{g}\AgdaSpace{}%
\AgdaOperator{\AgdaFunction{]⇒}}\AgdaSpace{}%
\AgdaBound{Q}\AgdaSpace{}%
\AgdaSymbol{=}\AgdaSpace{}%
\AgdaBound{P}\AgdaSpace{}%
\AgdaOperator{\AgdaFunction{⇒}}\AgdaSpace{}%
\AgdaSymbol{(}\AgdaBound{Q}\AgdaSpace{}%
\AgdaOperator{\AgdaFunction{on}}\AgdaSpace{}%
\AgdaBound{g}\AgdaSymbol{)}\<%
\end{code}

\paragraph*{Operation type and compatibility}%\label{compatibility-of-functions-and-binary-relations}

\textbf{Notation}. For consistency and readability, throughout the \ualib we reserve two universe variables for special purposes.  The first of these is \ab 𝓞 which shall be reserved for types that represent \defn{operation symbols} (see \ualibhtml{Algebras.Signatures}). The second is \ab 𝓥 which we reserve for types representing \defn{arities} of relations or operations.

Below we will define types that are useful for asserting and proving facts about \defn{compatibility} of operations and relations, but first we need a general type with which to represent operations.  Here is the definition, which we justify below.
\ccpad
\begin{code}
% \>[0]\AgdaComment{--The type of operations}\<%
% \\
\>[0]\AgdaFunction{Op}\AgdaSpace{}%
\AgdaSymbol{:}\AgdaSpace{}%
\AgdaGeneralizable{𝓥}\AgdaSpace{}%
\AgdaOperator{\AgdaFunction{̇}}\AgdaSpace{}%
\AgdaSymbol{→}\AgdaSpace{}%
\AgdaGeneralizable{𝓤}\AgdaSpace{}%
\AgdaOperator{\AgdaFunction{̇}}\AgdaSpace{}%
\AgdaSymbol{→}\AgdaSpace{}%
\AgdaGeneralizable{𝓤}\AgdaSpace{}%
\AgdaOperator{\AgdaPrimitive{⊔}}\AgdaSpace{}%
\AgdaGeneralizable{𝓥}\AgdaSpace{}%
\AgdaOperator{\AgdaFunction{̇}}\<%
\\
\>[0]\AgdaFunction{Op}\AgdaSpace{}%
\AgdaBound{I}\AgdaSpace{}%
\AgdaBound{A}\AgdaSpace{}%
\AgdaSymbol{=}\AgdaSpace{}%
\AgdaSymbol{(}\AgdaBound{I}\AgdaSpace{}%
\AgdaSymbol{→}\AgdaSpace{}%
\AgdaBound{A}\AgdaSymbol{)}\AgdaSpace{}%
\AgdaSymbol{→}\AgdaSpace{}%
\AgdaBound{A}\<%
\end{code}
\ccpad
The definition of \af{Op} codifies the arity of an operation as an arbitrary type \ab I~\as :~\ab 𝓥\af ̇, which gives us a very general way to represent an operation as a function type with domain \ab I~\as →~\ab A (the type of ``\ab I-tuples'') and codomain \ab A. For example, the \ab I-\defn{ary projection operations} on \ab A are represented as inhabitants of the type \af{Op}~\ab I~\ab A as follows.
\ccpad
\begin{code}
\>[0]\AgdaFunction{π}\AgdaSpace{}%
\AgdaSymbol{:}\AgdaSpace{}%
\AgdaSymbol{\{}\AgdaBound{I}\AgdaSpace{}%
\AgdaSymbol{:}\AgdaSpace{}%
\AgdaGeneralizable{𝓥}\AgdaSpace{}%
\AgdaOperator{\AgdaFunction{̇}}\AgdaSpace{}%
\AgdaSymbol{\}}\AgdaSpace{}%
\AgdaSymbol{\{}\AgdaBound{A}\AgdaSpace{}%
\AgdaSymbol{:}\AgdaSpace{}%
\AgdaGeneralizable{𝓤}\AgdaSpace{}%
\AgdaOperator{\AgdaFunction{̇}}\AgdaSpace{}%
\AgdaSymbol{\}}\AgdaSpace{}%
\AgdaSymbol{→}\AgdaSpace{}%
\AgdaBound{I}\AgdaSpace{}%
\AgdaSymbol{→}\AgdaSpace{}%
\AgdaFunction{Op}\AgdaSpace{}%
\AgdaBound{I}\AgdaSpace{}%
\AgdaBound{A}\<%
\\
\>[0]\AgdaFunction{π}\AgdaSpace{}%
\AgdaBound{i}\AgdaSpace{}%
\AgdaBound{x}\AgdaSpace{}%
\AgdaSymbol{=}\AgdaSpace{}%
\AgdaBound{x}\AgdaSpace{}%
\AgdaBound{i}\<%
\end{code}
\scpad
% Before discussing general and dependent relations, we pause to define some types that are useful for asserting and proving facts about \emph{compatibility} of functions with binary relations. 

Let us review the informal definition of compatibility. Suppose \ab{A} and \ab{I} are types and fix \ab{𝑓}~\as :~\AgdaFunction{Op}~\AgdaBound{I}~\AgdaBound{A} and \ab{R}~\as :~\af{Rel}~\ab A~\ab 𝓦 (an \ab{I}-ary operation and a binary relation on \ab{A}, respectively). We say that \ab{𝑓} and \ab{R} are \emph{compatible} and we write\footnote{The symbol \af{\textbar{}:} denoting compatibility comes from Cliff Bergman's universal algebra textbook~\cite{Bergman:2012}.} \ab{𝑓}~\af{\textbar{}:}~\ab R just in case \as{∀}~\ab u~\ab v~\as :~\ab I~\as →~\ab A,\\[-8pt]

\ad{Π}~\ab i~\as ꞉ \ab I \af , \ab R (\ab u \ab i) (\ab v \ab i) ~ \as{→} ~ \ab{R} (\ab f \ab u) (\ab f \ab v).\\[4pt]
To implement this in Agda, we first define a function \af{eval-rel} which ``lifts'' a binary relation to the corresponding \ab{I}-ary relation, and we use this to define the function \AgdaFunction{|:} representing \defn{compatibility of an \ab{I}-ary operation with a binary relation}.
\ccpad
\begin{code}%
\>[0]\AgdaFunction{eval-rel}\AgdaSpace{}%
\AgdaSymbol{:}\AgdaSpace{}%
\AgdaSymbol{\{}\AgdaBound{A}\AgdaSpace{}%
\AgdaSymbol{:}\AgdaSpace{}%
\AgdaGeneralizable{𝓤}\AgdaSpace{}%
\AgdaOperator{\AgdaFunction{̇}}\AgdaSymbol{\}\{}\AgdaBound{I}\AgdaSpace{}%
\AgdaSymbol{:}\AgdaSpace{}%
\AgdaGeneralizable{𝓥}\AgdaSpace{}%
\AgdaOperator{\AgdaFunction{̇}}\AgdaSymbol{\}}\AgdaSpace{}%
\AgdaSymbol{→}\AgdaSpace{}%
\AgdaFunction{Rel}\AgdaSpace{}%
\AgdaBound{A}\AgdaSpace{}%
\AgdaGeneralizable{𝓦}\AgdaSpace{}%
\AgdaSymbol{→}\AgdaSpace{}%
\AgdaFunction{Rel}\AgdaSpace{}%
\AgdaSymbol{(}\AgdaBound{I}\AgdaSpace{}%
\AgdaSymbol{→}\AgdaSpace{}%
\AgdaBound{A}\AgdaSymbol{)(}\AgdaGeneralizable{𝓥}\AgdaSpace{}%
\AgdaOperator{\AgdaPrimitive{⊔}}\AgdaSpace{}%
\AgdaGeneralizable{𝓦}\AgdaSymbol{)}\<%
\\
\>[0]\AgdaFunction{eval-rel}\AgdaSpace{}%
\AgdaBound{R}\AgdaSpace{}%
\AgdaBound{u}\AgdaSpace{}%
\AgdaBound{v}\AgdaSpace{}%
\AgdaSymbol{=}\AgdaSpace{}%
\AgdaFunction{Π}\AgdaSpace{}%
\AgdaBound{i}\AgdaSpace{}%
\AgdaFunction{꞉}\AgdaSpace{}%
\AgdaSymbol{\AgdaUnderscore{}}\AgdaSpace{}%
\AgdaFunction{,}\AgdaSpace{}%
\AgdaBound{R}\AgdaSpace{}%
\AgdaSymbol{(}\AgdaBound{u}\AgdaSpace{}%
\AgdaBound{i}\AgdaSymbol{)}\AgdaSpace{}%
\AgdaSymbol{(}\AgdaBound{v}\AgdaSpace{}%
\AgdaBound{i}\AgdaSymbol{)}\<%
\\
%
\\[\AgdaEmptyExtraSkip]%
\>[0]\AgdaOperator{\AgdaFunction{\AgdaUnderscore{}|:\AgdaUnderscore{}}}\AgdaSpace{}%
\AgdaSymbol{:}\AgdaSpace{}%
\AgdaSymbol{\{}\AgdaBound{A}\AgdaSpace{}%
\AgdaSymbol{:}\AgdaSpace{}%
\AgdaGeneralizable{𝓤}\AgdaSpace{}%
\AgdaOperator{\AgdaFunction{̇}}\AgdaSymbol{\}\{}\AgdaBound{I}\AgdaSpace{}%
\AgdaSymbol{:}\AgdaSpace{}%
\AgdaGeneralizable{𝓥}\AgdaSpace{}%
\AgdaOperator{\AgdaFunction{̇}}\AgdaSymbol{\}}\AgdaSpace{}%
\AgdaSymbol{→}\AgdaSpace{}%
\AgdaFunction{Op}\AgdaSpace{}%
\AgdaBound{I}\AgdaSpace{}%
\AgdaBound{A}\AgdaSpace{}%
\AgdaSymbol{→}\AgdaSpace{}%
\AgdaFunction{Rel}\AgdaSpace{}%
\AgdaBound{A}\AgdaSpace{}%
\AgdaGeneralizable{𝓦}\AgdaSpace{}%
\AgdaSymbol{→}\AgdaSpace{}%
\AgdaGeneralizable{𝓤}\AgdaSpace{}%
\AgdaOperator{\AgdaPrimitive{⊔}}\AgdaSpace{}%
\AgdaGeneralizable{𝓥}\AgdaSpace{}%
\AgdaOperator{\AgdaPrimitive{⊔}}\AgdaSpace{}%
\AgdaGeneralizable{𝓦}\AgdaSpace{}%
\AgdaOperator{\AgdaFunction{̇}}\<%
\\
\>[0]\AgdaBound{f}\AgdaSpace{}%
\AgdaOperator{\AgdaFunction{|:}}\AgdaSpace{}%
\AgdaBound{R}%
\>[8]\AgdaSymbol{=}\AgdaSpace{}%
\AgdaSymbol{(}\AgdaFunction{eval-rel}\AgdaSpace{}%
\AgdaBound{R}\AgdaSymbol{)}\AgdaSpace{}%
\AgdaOperator{\AgdaFunction{=[}}\AgdaSpace{}%
\AgdaBound{f}\AgdaSpace{}%
\AgdaOperator{\AgdaFunction{]⇒}}\AgdaSpace{}%
\AgdaBound{R}\<%
\end{code}
\ccpad
% \footnote{Initially we called the first function
% \af{lift-rel} because it ``lifts'' a binary relation on \ab A to a binary relation on tuples of type \ab I~\as →~\ab A. However, we renamed it \af{eval-rel} to avoid confusion with the universe level \AgdaRecord{Lift} type defined in the \ualibhtml{Overture.Lifts} module, or with \af{free-lift} (\ualibhtml{Terms.Basic}) which lifts a map defined on generators to a map on the thing being generated. Also, o
% (Here we silently added the type \ab I~\as :~\ab 𝓥\af ̇, representing \emph{relation arity}, to the context; we will have more to say about relation arities in the next section (\S\ref{sec:continuous-relations}).)