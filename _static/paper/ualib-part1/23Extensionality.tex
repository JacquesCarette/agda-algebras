\subsubsection{Background and motivation}\label{background-and-motivation}
This brief introduction to the basics of \emph{function extensionality} is intended for novices. If you're already familiar with the concept, you may want to skip to the next subsection.

What does it mean to say that two functions \(f\ g \colon X → Y\) are equal? Suppose $f$ and \(g\) are defined on \(X = ℤ\) (the integers) as follows: \(f x := x + 2\) and \(g x := ((2 * x) - 8)/2 + 6\). Would you say that \(f\) and \(g\) are equal?  Are they the ``same'' function?  What does that even mean?

If you know a little bit of basic algebra, then you probably can't resist the urge to reduce \(g\) to the form \(x + 2\) and proclaim that \(f\) and \(g\) are, indeed, equal. And you would be right, at least in middle school,
and the discussion would end there. In the science of computing, however, more attention is paid to equality, and with good reason.

We can probably all agree that the functions \(f\) and \(g\) above, while not syntactically equal, do produce the same output when given the same input so it seems fine to think of the functions as the same, for all intents and purposes. But we should ask ourselves, at what point do we notice or care about the difference in the way functions are defined?

What if we had started out this discussion with two functions \(f\) and \(g\) both of which take a list as input and produce as output a correctly sorted version of that list? Are the functions the same? What if \(f\) was defined using the \href{https://en.wikipedia.org/wiki/Merge_sort}{merge sort} algorithm, while \(g\) used \href{https://en.wikipedia.org/wiki/Quicksort}{quick sort}? Probably most of us wouldn't think of \(f\) and \(g\) as the same function in that case.

In the examples above, it is common to say that the two functions \(f\) and \(g\) are \href{https://en.wikipedia.org/wiki/Extensionality}{\defn{extensionally equal}}, since they produce the same (external) output when given the same input, but they are not \href{https://en.wikipedia.org/wiki/Intension}{\defn{intensionally equal}}, since their (internal) definitions differ.

In this subsection, we describe types that manifest this idea of \emph{extensional equality of functions}, or \emph{function extensionality}. (Most of these types are already defined in the \typetopology library, so the \ualib merely imports the definitions from there.)

\subsubsection{Definition of function extensionality}\label{sec:defin-funct-extens}

As alluded to above, a natural notion of function equality, sometimes called \emph{point-wise equality}, is defined as follows: \(f\) and \(g\) are said to be \emph{pointwise equal} provided \(∀ x → f x ≡ g x\). Here is how this notion of equality is expressed as a type in the \typetopology library.
\ccpad
\begin{code}%
\>[1]\AgdaOperator{\AgdaFunction{\AgdaUnderscore{}∼\AgdaUnderscore{}}}\AgdaSpace{}%
\AgdaSymbol{:}\AgdaSpace{}%
\AgdaSymbol{\{}\AgdaBound{𝓤}\AgdaSpace{}%
\AgdaBound{𝓥}\AgdaSpace{}%
\AgdaSymbol{:}\AgdaSpace{}%
\AgdaPostulate{Universe}\AgdaSymbol{\}\{}\AgdaBound{X}\AgdaSpace{}%
\AgdaSymbol{:}\AgdaSpace{}%
\AgdaBound{𝓤}\AgdaSpace{}%
\AgdaOperator{\AgdaFunction{̇}}\AgdaSpace{}%
\AgdaSymbol{\}}\AgdaSpace{}%
\AgdaSymbol{\{}\AgdaBound{A}\AgdaSpace{}%
\AgdaSymbol{:}\AgdaSpace{}%
\AgdaBound{X}\AgdaSpace{}%
\AgdaSymbol{→}\AgdaSpace{}%
\AgdaBound{𝓥}\AgdaSpace{}%
\AgdaOperator{\AgdaFunction{̇}}\AgdaSpace{}%
\AgdaSymbol{\}}\AgdaSpace{}%
\AgdaSymbol{→}\AgdaSpace{}%
\AgdaFunction{Π}\AgdaSpace{}%
\AgdaBound{A}\AgdaSpace{}%
\AgdaSymbol{→}\AgdaSpace{}%
\AgdaFunction{Π}\AgdaSpace{}%
\AgdaBound{A}\AgdaSpace{}%
\AgdaSymbol{→}\AgdaSpace{}%
\AgdaBound{𝓤}\AgdaSpace{}%
\AgdaOperator{\AgdaPrimitive{⊔}}\AgdaSpace{}%
\AgdaBound{𝓥}\AgdaSpace{}%
\AgdaOperator{\AgdaFunction{̇}}\<%
\\
%
\>[1]\AgdaBound{f}\AgdaSpace{}%
\AgdaOperator{\AgdaFunction{∼}}\AgdaSpace{}%
\AgdaBound{g}\AgdaSpace{}%
\AgdaSymbol{=}\AgdaSpace{}%
\AgdaSymbol{∀}\AgdaSpace{}%
\AgdaBound{x}\AgdaSpace{}%
\AgdaSymbol{→}\AgdaSpace{}%
\AgdaBound{f}\AgdaSpace{}%
\AgdaBound{x}\AgdaSpace{}%
\AgdaOperator{\AgdaDatatype{≡}}\AgdaSpace{}%
\AgdaBound{g}\AgdaSpace{}%
\AgdaBound{x}\<%
\end{code}
\ccpad
\defn{Function extensionality} is the assertion that point-wise equal functions are \defn{definitionally equal}; that is, \(\forall\ f\ g\ (f ∼ g → f ≡ g)\). In the \typetopology library, the types that represent this notion are \af{funext} (for nondependent functions) and \af{dfunext} (for dependent functions). They are defined as follows.
\ccpad
\begin{code}%
\>[1]\AgdaFunction{funext}\AgdaSpace{}%
\AgdaSymbol{:}\AgdaSpace{}%
\AgdaSymbol{∀}\AgdaSpace{}%
\AgdaBound{𝓤}\AgdaSpace{}%
\AgdaBound{𝓦}%
\>[17]\AgdaSymbol{→}\AgdaSpace{}%
\as (\AgdaBound{𝓤}\AgdaSpace{}%
\AgdaOperator{\AgdaPrimitive{⊔}}\AgdaSpace{}%
\AgdaBound{𝓦}\as )\AgdaSpace{}%
\AgdaOperator{\AgdaPrimitive{⁺}}\AgdaSpace{}%
\AgdaOperator{\AgdaFunction{̇}}\<%
\\
%
\>[1]\AgdaFunction{funext}\AgdaSpace{}%
\AgdaBound{𝓤}\AgdaSpace{}%
\AgdaBound{𝓦}\AgdaSpace{}%
\AgdaSymbol{=}\AgdaSpace{}%
\AgdaSymbol{\{}\AgdaBound{A}\AgdaSpace{}%
\AgdaSymbol{:}\AgdaSpace{}%
\AgdaBound{𝓤}\AgdaSpace{}%
\AgdaOperator{\AgdaFunction{̇}}\AgdaSpace{}%
\AgdaSymbol{\}}\AgdaSpace{}%
\AgdaSymbol{\{}\AgdaBound{B}\AgdaSpace{}%
\AgdaSymbol{:}\AgdaSpace{}%
\AgdaBound{𝓦}\AgdaSpace{}%
\AgdaOperator{\AgdaFunction{̇}}\AgdaSpace{}%
\AgdaSymbol{\}}\AgdaSpace{}%
\AgdaSymbol{\{}\AgdaBound{f}\AgdaSpace{}%
\AgdaBound{g}\AgdaSpace{}%
\AgdaSymbol{:}\AgdaSpace{}%
\AgdaBound{A}\AgdaSpace{}%
\AgdaSymbol{→}\AgdaSpace{}%
\AgdaBound{B}\AgdaSymbol{\}}\AgdaSpace{}%
\AgdaSymbol{→}\AgdaSpace{}%
\AgdaBound{f}\AgdaSpace{}%
\AgdaOperator{\AgdaFunction{∼}}\AgdaSpace{}%
\AgdaBound{g}\AgdaSpace{}%
\AgdaSymbol{→}\AgdaSpace{}%
\AgdaBound{f}\AgdaSpace{}%
\AgdaOperator{\AgdaDatatype{≡}}\AgdaSpace{}%
\AgdaBound{g}\<%
\end{code}
\scpad
\begin{code}%
\>[1]\AgdaFunction{dfunext}\AgdaSpace{}%
\AgdaSymbol{:}\AgdaSpace{}%
\AgdaSymbol{∀}\AgdaSpace{}%
\AgdaBound{𝓤}\AgdaSpace{}%
\AgdaBound{𝓦}\AgdaSpace{}%
\AgdaSymbol{→}\AgdaSpace{}%
\as (\AgdaBound{𝓤}\AgdaSpace{}%
\AgdaOperator{\AgdaPrimitive{⊔}}\AgdaSpace{}%
\AgdaBound{𝓦}\as )\AgdaSpace{}%
\AgdaOperator{\AgdaPrimitive{⁺}}\AgdaSpace{}%
\AgdaOperator{\AgdaFunction{̇}}\<%
\\
%
\>[1]\AgdaFunction{dfunext}\AgdaSpace{}%
\AgdaBound{𝓤}\AgdaSpace{}%
\AgdaBound{𝓦}\AgdaSpace{}%
\AgdaSymbol{=}\AgdaSpace{}%
\AgdaSymbol{\{}\AgdaBound{A}\AgdaSpace{}%
\AgdaSymbol{:}\AgdaSpace{}%
\AgdaBound{𝓤}\AgdaSpace{}%
\AgdaOperator{\AgdaFunction{̇}}\AgdaSpace{}%
\AgdaSymbol{\}\{}\AgdaBound{B}\AgdaSpace{}%
\AgdaSymbol{:}\AgdaSpace{}%
\AgdaBound{A}\AgdaSpace{}%
\AgdaSymbol{→}\AgdaSpace{}%
\AgdaBound{𝓦}\AgdaSpace{}%
\AgdaOperator{\AgdaFunction{̇}}\AgdaSpace{}%
\AgdaSymbol{\}\{}\AgdaBound{f}\AgdaSpace{}%
\AgdaBound{g}\AgdaSpace{}%
\AgdaSymbol{:}\AgdaSpace{}%
\AgdaSymbol{∀(}\AgdaBound{x}\AgdaSpace{}%
\AgdaSymbol{:}\AgdaSpace{}%
\AgdaBound{A}\AgdaSymbol{)}\AgdaSpace{}%
\AgdaSymbol{→}\AgdaSpace{}%
\AgdaBound{B}\AgdaSpace{}%
\AgdaBound{x}\AgdaSymbol{\}}\AgdaSpace{}%
\AgdaSymbol{→}%
\>[65]\AgdaBound{f}\AgdaSpace{}%
\AgdaOperator{\AgdaFunction{∼}}\AgdaSpace{}%
\AgdaBound{g}%
\>[72]\AgdaSymbol{→}%
\>[75]\AgdaBound{f}\AgdaSpace{}%
\AgdaOperator{\AgdaDatatype{≡}}\AgdaSpace{}%
\AgdaBound{g}\<%
\end{code}
\ccpad
In informal settings, this so-called ``pointwise equality of functions'' is typically what one means when one
asserts that two functions are ``equal.''\footnote{%
If one assumes the \emph{univalence axiom} of Homotopy Type Theory, then the \af{\_∼\_} relation is equivalent to equality of functions. See the section \href{https://www.cs.bham.ac.uk/~mhe/HoTT-UF-in-Agda-Lecture-Notes/HoTT-UF-Agda.html\#funextfromua}{``Function extensionality from univalence''} of Escard\'o's notes~\cite{MHE}.}
However, it is important to keep in mind the following fact: \textit{function extensionality is known to be neither provable nor disprovable in Martin-Löf type theory. It is an independent statement}~(\cite{MHE}).

\subsubsection{Global function extensionality}\label{sec:glob-funct-extens}

An assumption that we adopt throughout much of the current version of the \ualib is a \emph{global function extensionality principle}. This asserts that function extensionality holds at all universe levels. Agda is capable of expressing types representing global principles as the language has a special universe level for such types.
Following Escardó, we denote this universe by 𝓤ω (which is just an alias for Agda's \ad{Setω} universe).\footnote{More details about the \ad{𝓤ω} type are available at
\href{https://agda.readthedocs.io/en/latest/language/universe-levels.html\#expressions-of-kind-set}{agda.readthedocs.io}.}
The types \af{global-funext} and \af{global-dfunext} are defined in the \typetopology library as follows.
\ccpad
\begin{code}%
\>[1]\AgdaFunction{global-funext}\AgdaSpace{}%
\AgdaSymbol{:}\AgdaSpace{}%
\AgdaPrimitive{𝓤ω}\<%
\\
%
\>[1]\AgdaFunction{global-funext}\AgdaSpace{}%
\AgdaSymbol{=}\AgdaSpace{}%
\AgdaSymbol{∀}%
\>[20]\AgdaSymbol{\{}\AgdaBound{𝓤}\AgdaSpace{}%
\AgdaBound{𝓥}\AgdaSymbol{\}}\AgdaSpace{}%
\AgdaSymbol{→}\AgdaSpace{}%
\AgdaFunction{funext}\AgdaSpace{}%
\AgdaBound{𝓤}\AgdaSpace{}%
\AgdaBound{𝓥}\<%
\\
%
\\[\AgdaEmptyExtraSkip]%
%
\>[1]\AgdaFunction{global-dfunext}\AgdaSpace{}%
\AgdaSymbol{:}\AgdaSpace{}%
\AgdaPrimitive{𝓤ω}\<%
\\
%
\>[1]\AgdaFunction{global-dfunext}\AgdaSpace{}%
\AgdaSymbol{=}\AgdaSpace{}%
\AgdaSymbol{∀}\AgdaSpace{}%
\AgdaSymbol{\{}\AgdaBound{𝓤}\AgdaSpace{}%
\AgdaBound{𝓥}\AgdaSymbol{\}}\AgdaSpace{}%
\AgdaSymbol{→}\AgdaSpace{}%
\AgdaFunction{funext}\AgdaSpace{}%
\AgdaBound{𝓤}\AgdaSpace{}%
\AgdaBound{𝓥}\<%
\end{code}
\ccpad
The next two types define the converse of function extensionality.\footnote{In previous versions of the
\ualib this function was called \af{intensionality}, indicating that it represented the concept of \emph{function intensionality}, but we realized this isn't quite right and changed the name to the less controvertial \af{extfun}.  Also, we later realized that a function called \af{happly}, which is nearly identical to \af{extdfun}, was already defined in the \am{MGS-FunExt-from-Univalence} module of the \typetopology library.}
\ccpad
\begin{code}%
\>[1]\AgdaFunction{extfun}\AgdaSpace{}%
\AgdaSymbol{:}\AgdaSpace{}%
\AgdaSymbol{\{}\AgdaBound{A}\AgdaSpace{}%
\AgdaSymbol{:}\AgdaSpace{}%
\AgdaBound{𝓤}\AgdaSpace{}%
\AgdaOperator{\AgdaFunction{̇}}\AgdaSymbol{\}\{}\AgdaBound{B}\AgdaSpace{}%
\AgdaSymbol{:}\AgdaSpace{}%
\AgdaBound{𝓦}\AgdaSpace{}%
\AgdaOperator{\AgdaFunction{̇}}\AgdaSymbol{\}\{}\AgdaBound{f}\AgdaSpace{}%
\AgdaBound{g}\AgdaSpace{}%
\AgdaSymbol{:}\AgdaSpace{}%
\AgdaBound{A}\AgdaSpace{}%
\AgdaSymbol{→}\AgdaSpace{}%
\AgdaBound{B}\AgdaSymbol{\}}\AgdaSpace{}%
\AgdaSymbol{→}\AgdaSpace{}%
\AgdaBound{f}\AgdaSpace{}%
\AgdaOperator{\AgdaDatatype{≡}}\AgdaSpace{}%
\AgdaBound{g}%
\>[51]\AgdaSymbol{→}%
\>[54]\AgdaBound{f}\AgdaSpace{}%
\AgdaOperator{\AgdaFunction{∼}}\AgdaSpace{}%
\AgdaBound{g}\<%
\\
%
\>[1]\AgdaFunction{extfun}\AgdaSpace{}%
\AgdaInductiveConstructor{𝓇ℯ𝒻𝓁}\AgdaSpace{}%
\AgdaSymbol{\AgdaUnderscore{}}\AgdaSpace{}%
\AgdaSymbol{=}\AgdaSpace{}%
\AgdaInductiveConstructor{𝓇ℯ𝒻𝓁}\<%
\end{code}
\scpad
\begin{code}%
\>[1]\AgdaFunction{extdfun}\AgdaSpace{}%
\AgdaSymbol{:}\AgdaSpace{}%
\AgdaSymbol{\{}\AgdaBound{A}\AgdaSpace{}%
\AgdaSymbol{:}\AgdaSpace{}%
\AgdaBound{𝓤}\AgdaSpace{}%
\AgdaOperator{\AgdaFunction{̇}}\AgdaSpace{}%
\AgdaSymbol{\}\{}\AgdaBound{B}\AgdaSpace{}%
\AgdaSymbol{:}\AgdaSpace{}%
\AgdaBound{A}\AgdaSpace{}%
\AgdaSymbol{→}\AgdaSpace{}%
\AgdaBound{𝓦}\AgdaSpace{}%
\AgdaOperator{\AgdaFunction{̇}}\AgdaSpace{}%
\AgdaSymbol{\}(}\AgdaBound{f}\AgdaSpace{}%
\AgdaBound{g}\AgdaSpace{}%
\AgdaSymbol{:}\AgdaSpace{}%
\AgdaFunction{Π}\AgdaSpace{}%
\AgdaBound{B}\AgdaSymbol{)}\AgdaSpace{}%
\AgdaSymbol{→}\AgdaSpace{}%
\AgdaBound{f}\AgdaSpace{}%
\AgdaOperator{\AgdaDatatype{≡}}\AgdaSpace{}%
\AgdaBound{g}\AgdaSpace{}%
\AgdaSymbol{→}\AgdaSpace{}%
\AgdaBound{f}\AgdaSpace{}%
\AgdaOperator{\AgdaFunction{∼}}\AgdaSpace{}%
\AgdaBound{g}\<%
\\
%
\>[1]\AgdaFunction{extdfun}\AgdaSpace{}%
\AgdaSymbol{\AgdaUnderscore{}}\AgdaSpace{}%
\AgdaSymbol{\AgdaUnderscore{}}\AgdaSpace{}%
\AgdaInductiveConstructor{𝓇ℯ𝒻𝓁}\AgdaSpace{}%
\AgdaSymbol{\AgdaUnderscore{}}\AgdaSpace{}%
\AgdaSymbol{=}\AgdaSpace{}%
\AgdaInductiveConstructor{𝓇ℯ𝒻𝓁}\<%
\end{code}
\scpad

Though it may seem obvious to some readers, we wish to emphasize the important conceptual distinction between two different forms of type definitions we have seen so far.  We do so by comparing the definitions of \af{funext} and \af{extfun}.  In the definition of \af{funext}, the codomain is a generic type (namely, (\ab 𝓤 \ap ⊔ \ab 𝓥)\af ⁺\af ̇). In the definition of \af{extfun}, the codomain is an assertion (namely, \ab f \af ∼ \ab g).  Also, the defining equation of \af{funext} is an assertion, while the defining equation of \af{extfun} is a proof.  As such, \af{extfun} is a \defn{proof object}; it proves (inhabits the type that represents) the proposition asserting that definitionally equivalent functions are point-wise equal. In contrast, \af{funext} is a type, and we may or may not wish to assume we have a proof for this type. That is, we could postulate that function extensionality holds and assume we have a witness, say, \ab{fe} \as : \af{funext} \ab 𝓤 \ab 𝓥 (i.e., a proof that point-wise equal functions are equal), but as noted above the existence of such a witness cannot be proved in Martin-Löf type theory.

\subsubsection{Alternative extensionality type}\label{alternative-extensionality-type}

Finally, a useful alternative for expressing dependent function extensionality, which is essentially equivalent to \af{dfunext}, is to assert that \af{extdfun} is actually an \emph{equivalence}. This requires a few definitions from the \am{MGS-Equivalences} module of the \typetopology library, which we now describe.

First, a type is a \defn{singleton} if it has exactly one inhabitant and a \defn{subsingleton} if it has at most one inhabitant. These properties are represented in the \typetopology library by the following type.
\ccpad
\begin{code}%
\>[1]\AgdaFunction{is-center}\AgdaSpace{}%
\AgdaSymbol{:}\AgdaSpace{}%
\AgdaSymbol{(}\AgdaBound{X}\AgdaSpace{}%
\AgdaSymbol{:}\AgdaSpace{}%
\AgdaBound{𝓤}\AgdaSpace{}%
\AgdaOperator{\AgdaFunction{̇}}\AgdaSpace{}%
\AgdaSymbol{)}\AgdaSpace{}%
\AgdaSymbol{→}\AgdaSpace{}%
\AgdaBound{X}\AgdaSpace{}%
\AgdaSymbol{→}\AgdaSpace{}%
\AgdaBound{𝓤}\AgdaSpace{}%
\AgdaOperator{\AgdaFunction{̇}}\<%
\\
%
\>[1]\AgdaFunction{is-center}\AgdaSpace{}%
\AgdaBound{X}\AgdaSpace{}%
\AgdaBound{c}\AgdaSpace{}%
\AgdaSymbol{=}\AgdaSpace{}%
\AgdaSymbol{(}\AgdaBound{x}\AgdaSpace{}%
\AgdaSymbol{:}\AgdaSpace{}%
\AgdaBound{X}\AgdaSymbol{)}\AgdaSpace{}%
\AgdaSymbol{→}\AgdaSpace{}%
\AgdaBound{c}\AgdaSpace{}%
\AgdaOperator{\AgdaDatatype{≡}}\AgdaSpace{}%
\AgdaBound{x}\<%
\\
%
\\[\AgdaEmptyExtraSkip]%
%
\>[1]\AgdaFunction{is-singleton}\AgdaSpace{}%
\AgdaSymbol{:}\AgdaSpace{}%
\AgdaBound{𝓤}\AgdaSpace{}%
\AgdaOperator{\AgdaFunction{̇}}\AgdaSpace{}%
\AgdaSymbol{→}\AgdaSpace{}%
\AgdaBound{𝓤}\AgdaSpace{}%
\AgdaOperator{\AgdaFunction{̇}}\<%
\\
%
\>[1]\AgdaFunction{is-singleton}\AgdaSpace{}%
\AgdaBound{X}\AgdaSpace{}%
\AgdaSymbol{=}\AgdaSpace{}%
\AgdaFunction{Σ}\AgdaSpace{}%
\AgdaBound{c}\AgdaSpace{}%
\AgdaFunction{꞉}\AgdaSpace{}%
\AgdaBound{X}\AgdaSpace{}%
\AgdaFunction{,}\AgdaSpace{}%
\AgdaFunction{is-center}\AgdaSpace{}%
\AgdaBound{X}\AgdaSpace{}%
\AgdaBound{c}\<%
\\
%
\\[\AgdaEmptyExtraSkip]%
%
\>[1]\AgdaFunction{is-subsingleton}\AgdaSpace{}%
\AgdaSymbol{:}\AgdaSpace{}%
\AgdaBound{𝓤}\AgdaSpace{}%
\AgdaOperator{\AgdaFunction{̇}}\AgdaSpace{}%
\AgdaSymbol{→}\AgdaSpace{}%
\AgdaBound{𝓤}\AgdaSpace{}%
\AgdaOperator{\AgdaFunction{̇}}\<%
\\
%
\>[1]\AgdaFunction{is-subsingleton}\AgdaSpace{}%
\AgdaBound{X}\AgdaSpace{}%
\AgdaSymbol{=}\AgdaSpace{}%
\AgdaSymbol{(}\AgdaBound{x}\AgdaSpace{}%
\AgdaBound{y}\AgdaSpace{}%
\AgdaSymbol{:}\AgdaSpace{}%
\AgdaBound{X}\AgdaSymbol{)}\AgdaSpace{}%
\AgdaSymbol{→}\AgdaSpace{}%
\AgdaBound{x}\AgdaSpace{}%
\AgdaOperator{\AgdaDatatype{≡}}\AgdaSpace{}%
\AgdaBound{y}\<%
\end{code}
\ccpad
Next, we consider the type \af{is-equiv} which is used to assert that a function is an equivalence in the sense that we now describe. First we need the concept of a \href{https://ncatlab.org/nlab/show/fiber}{fiber} of a function. In the \typetopology library, \af{fiber} is defined as a Sigma type whose inhabitants represent inverse images of points in the codomain of the given function.
\ccpad
\begin{code}%
\>[1]\AgdaFunction{fiber}\AgdaSpace{}%
\AgdaSymbol{:}\AgdaSpace{}%
\AgdaSymbol{\{}\AgdaBound{X}\AgdaSpace{}%
\AgdaSymbol{:}\AgdaSpace{}%
\AgdaBound{𝓤}\AgdaSpace{}%
\AgdaOperator{\AgdaFunction{̇}}\AgdaSpace{}%
\AgdaSymbol{\}}\AgdaSpace{}%
\AgdaSymbol{\{}\AgdaBound{Y}\AgdaSpace{}%
\AgdaSymbol{:}\AgdaSpace{}%
\AgdaBound{𝓦}\AgdaSpace{}%
\AgdaOperator{\AgdaFunction{̇}}\AgdaSpace{}%
\AgdaSymbol{\}}\AgdaSpace{}%
\AgdaSymbol{(}\AgdaBound{f}\AgdaSpace{}%
\AgdaSymbol{:}\AgdaSpace{}%
\AgdaBound{X}\AgdaSpace{}%
\AgdaSymbol{→}\AgdaSpace{}%
\AgdaBound{Y}\AgdaSymbol{)}\AgdaSpace{}%
\AgdaSymbol{→}\AgdaSpace{}%
\AgdaBound{Y}\AgdaSpace{}%
\AgdaSymbol{→}\AgdaSpace{}%
\AgdaBound{𝓤}\AgdaSpace{}%
\AgdaOperator{\AgdaPrimitive{⊔}}\AgdaSpace{}%
\AgdaBound{𝓦}\AgdaSpace{}%
\AgdaOperator{\AgdaFunction{̇}}\<%
\\
%
\>[1]\AgdaFunction{fiber}\AgdaSpace{}%
\AgdaSymbol{\{}\AgdaBound{X}\AgdaSymbol{\}}\AgdaSpace{}%
\AgdaBound{f}\AgdaSpace{}%
\AgdaBound{y}\AgdaSpace{}%
\AgdaSymbol{=}\AgdaSpace{}%
\AgdaFunction{Σ}\AgdaSpace{}%
\AgdaBound{x}\AgdaSpace{}%
\AgdaFunction{꞉}\AgdaSpace{}%
\AgdaBound{X}\AgdaSpace{}%
\AgdaFunction{,}\AgdaSpace{}%
\AgdaBound{f}\AgdaSpace{}%
\AgdaBound{x}\AgdaSpace{}%
\AgdaOperator{\AgdaDatatype{≡}}\AgdaSpace{}%
\AgdaBound{y}\<%
\end{code}
\ccpad
A function is called an \defn{equivalence} if all of its fibers are singletons.
\ccpad
\begin{code}%
\>[1]\AgdaFunction{is-equiv}\AgdaSpace{}%
\AgdaSymbol{:}\AgdaSpace{}%
\AgdaSymbol{\{}\AgdaBound{X}\AgdaSpace{}%
\AgdaSymbol{:}\AgdaSpace{}%
\AgdaBound{𝓤}\AgdaSpace{}%
\AgdaOperator{\AgdaFunction{̇}}\AgdaSpace{}%
\AgdaSymbol{\}}\AgdaSpace{}%
\AgdaSymbol{\{}\AgdaBound{Y}\AgdaSpace{}%
\AgdaSymbol{:}\AgdaSpace{}%
\AgdaBound{𝓦}\AgdaSpace{}%
\AgdaOperator{\AgdaFunction{̇}}\AgdaSpace{}%
\AgdaSymbol{\}}\AgdaSpace{}%
\AgdaSymbol{→}\AgdaSpace{}%
\AgdaSymbol{(}\AgdaBound{X}\AgdaSpace{}%
\AgdaSymbol{→}\AgdaSpace{}%
\AgdaBound{Y}\AgdaSymbol{)}\AgdaSpace{}%
\AgdaSymbol{→}\AgdaSpace{}%
\AgdaBound{𝓤}\AgdaSpace{}%
\AgdaOperator{\AgdaPrimitive{⊔}}\AgdaSpace{}%
\AgdaBound{𝓦}\AgdaSpace{}%
\AgdaOperator{\AgdaFunction{̇}}\<%
\\
%
\>[1]\AgdaFunction{is-equiv}\AgdaSpace{}%
\AgdaBound{f}\AgdaSpace{}%
\AgdaSymbol{=}\AgdaSpace{}%
\AgdaSymbol{∀}\AgdaSpace{}%
\AgdaBound{y}\AgdaSpace{}%
\AgdaSymbol{→}\AgdaSpace{}%
\AgdaFunction{is-singleton}\AgdaSpace{}%
\AgdaSymbol{(}\AgdaFunction{fiber}\AgdaSpace{}%
\AgdaBound{f}\AgdaSpace{}%
\AgdaBound{y}\AgdaSymbol{)}\<%
\end{code}
\ccpad
Now we are finally ready to define the type \af{hfunext} that gives an alternative means of postulating function
extensionality.
\ccpad
\begin{code}%
\>[1]\AgdaFunction{hfunext}\AgdaSpace{}%
\AgdaSymbol{:}\AgdaSpace{}%
\AgdaSymbol{(}\AgdaBound{𝓤}\AgdaSpace{}%
\AgdaBound{𝓦}\AgdaSpace{}%
\AgdaSymbol{:}\AgdaSpace{}%
\AgdaPostulate{Universe}\AgdaSymbol{)}\AgdaSpace{}%
\AgdaSymbol{→}\AgdaSpace{}%
\AgdaSymbol{(}\AgdaBound{𝓤}\AgdaSpace{}%
\AgdaOperator{\AgdaPrimitive{⊔}}\AgdaSpace{}%
\AgdaBound{𝓦}\AgdaSymbol{)}\AgdaOperator{\AgdaPrimitive{⁺}}\AgdaSpace{}%
\AgdaOperator{\AgdaFunction{̇}}\<%
\\
%
\>[1]\AgdaFunction{hfunext}\AgdaSpace{}%
\AgdaBound{𝓤}\AgdaSpace{}%
\AgdaBound{𝓦}\AgdaSpace{}%
\AgdaSymbol{=}\AgdaSpace{}%
\AgdaSymbol{\{}\AgdaBound{A}\AgdaSpace{}%
\AgdaSymbol{:}\AgdaSpace{}%
\AgdaBound{𝓤}\AgdaSpace{}%
\AgdaOperator{\AgdaFunction{̇}}\AgdaSymbol{\}\{}\AgdaBound{B}\AgdaSpace{}%
\AgdaSymbol{:}\AgdaSpace{}%
\AgdaBound{A}\AgdaSpace{}%
\AgdaSymbol{→}\AgdaSpace{}%
\AgdaBound{𝓦}\AgdaSpace{}%
\AgdaOperator{\AgdaFunction{̇}}\AgdaSymbol{\}}\AgdaSpace{}%
\AgdaSymbol{(}\AgdaBound{f}\AgdaSpace{}%
\AgdaBound{g}\AgdaSpace{}%
\AgdaSymbol{:}\AgdaSpace{}%
\AgdaFunction{Π}\AgdaSpace{}%
\AgdaBound{B}\AgdaSymbol{)}\AgdaSpace{}%
\AgdaSymbol{→}\AgdaSpace{}%
\AgdaFunction{is-equiv}\AgdaSpace{}%
\AgdaSymbol{(}\AgdaFunction{extdfun}\AgdaSpace{}%
\AgdaBound{f}\AgdaSpace{}%
\AgdaBound{g}\AgdaSymbol{)}\<%
\end{code}

%%% Local Variables:
%%% mode: latex
%%% TeX-master: "ualib-part1-20210304.tex"
%%% End:
