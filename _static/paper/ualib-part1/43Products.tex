% -*- TeX-master: "ualib-part1.tex" -*-
%%% Local Variables: 
%%% mode: latex
%%% TeX-engine: 'xetex
%%% End:
We assume a fixed signature \ab{𝑆} \as : \af{Signature} \ab 𝓞 \ab 𝓥 throughout the module by starting with the line
\AgdaKeyword{module}\AgdaSpace{}%
\AgdaModule{Algebras.Products}\AgdaSpace{}%
\AgdaSymbol{\{}\AgdaBound{𝑆}\AgdaSpace{}%
\AgdaSymbol{:}\AgdaSpace{}%
\AgdaFunction{Signature}\AgdaSpace{}%
\AgdaGeneralizable{𝓞}\AgdaSpace{}%
\AgdaGeneralizable{𝓥}\AgdaSymbol{\}}\AgdaSpace{}%
\AgdaKeyword{where}.%
% We begin this module by assuming a signature \ab{𝑆} \as : \af{Signature} \ab 𝓞 \ab 𝓥 which is then present and available throughout the module.  Because of this, in contrast to descriptions of previous modules, we present the first few lines of the \ualibhtml{Algebras.Products} module in full.
% \ccpad
% \begin{code}%
% \>[0]\AgdaSymbol{\{-\#}\AgdaSpace{}%
% \AgdaKeyword{OPTIONS}\AgdaSpace{}%
% \AgdaPragma{--without-K}\AgdaSpace{}%
% \AgdaPragma{--exact-split}\AgdaSpace{}%
% \AgdaPragma{--safe}\AgdaSpace{}%
% \AgdaSymbol{\#-\}}\<%
% \\
% %
% \\[\AgdaEmptyExtraSkip]%
% \>[0]\AgdaKeyword{open}\AgdaSpace{}%
% \AgdaKeyword{import}\AgdaSpace{}%
% \AgdaModule{Algebras.Signatures}\AgdaSpace{}%
% \AgdaKeyword{using}\AgdaSpace{}%
% \AgdaSymbol{(}\AgdaFunction{Signature}\AgdaSymbol{;}\AgdaSpace{}%
% \AgdaGeneralizable{𝓞}\AgdaSymbol{;}\AgdaSpace{}%
% \AgdaGeneralizable{𝓥}\AgdaSymbol{)}\<%
% \\
% %
% % \\[\AgdaEmptyExtraSkip]%
% \>[0]\AgdaKeyword{module}\AgdaSpace{}%
% \AgdaModule{Algebras.Products}\AgdaSpace{}%
% \AgdaSymbol{\{}\AgdaBound{𝑆}\AgdaSpace{}%
% \AgdaSymbol{:}\AgdaSpace{}%
% \AgdaFunction{Signature}\AgdaSpace{}%
% \AgdaGeneralizable{𝓞}\AgdaSpace{}%
% \AgdaGeneralizable{𝓥}\AgdaSymbol{\}}\AgdaSpace{}%
% \AgdaKeyword{where}\<%
% \\
% %
% % \\[\AgdaEmptyExtraSkip]%
% \>[0]\AgdaKeyword{open}\AgdaSpace{}%
% \AgdaKeyword{import}\AgdaSpace{}%
% \AgdaModule{Algebras.Algebras}\AgdaSpace{}%
% \AgdaKeyword{hiding}\AgdaSpace{}%
% \AgdaSymbol{(}\AgdaGeneralizable{𝓞}\AgdaSymbol{;}\AgdaSpace{}%
% \AgdaGeneralizable{𝓥}\AgdaSymbol{)}\AgdaSpace{}%
% \AgdaKeyword{public}\<%
% \end{code}
% \ccpad
% Notice that we must import the \af{Signature} type from \ualibhtml{Algebras.Signatures} first so that we can use it to declare the signature \AgdaBound{𝑆} as a parameter of the \ualibhtml{Algebras.Products} module.

% \>[0]\AgdaKeyword{module}\AgdaSpace{}%
% \AgdaModule{Algebras.Products}\AgdaSpace{}%
% \AgdaSymbol{\{}\AgdaBound{𝑆}\AgdaSpace{}%
% \AgdaSymbol{:}\AgdaSpace{}%
% \AgdaFunction{Signature}\AgdaSpace{}%
% \AgdaGeneralizable{𝓞}\AgdaSpace{}%
% \AgdaGeneralizable{𝓥}\AgdaSymbol{\}}\AgdaSpace{}%
% \AgdaKeyword{where}\<%
% \end{code}

In the \ualib the \defn{product of} \ab 𝑆-\defn{algebras} is represented by the following type.\footnote{Alternative equivalent notation for the domain of the product is \as ∀~\ab i~\as →~\af ∣~\ab 𝒜~\ab i~\af ∣.}
\ccpad
\begin{code}%
% \>[0]\AgdaKeyword{module}\AgdaSpace{}%
% \AgdaModule{\AgdaUnderscore{}}\AgdaSpace{}%
% \AgdaSymbol{\{}\AgdaBound{𝓤}\AgdaSpace{}%
% \AgdaBound{𝓘}\AgdaSpace{}%
% \AgdaSymbol{:}\AgdaSpace{}%
% \AgdaPostulate{Universe}\AgdaSymbol{\}\{}\AgdaBound{I}\AgdaSpace{}%
% \AgdaSymbol{:}\AgdaSpace{}%
% \AgdaBound{𝓘}\AgdaSpace{}%
% \AgdaOperator{\AgdaFunction{̇}}\AgdaSpace{}%
% \AgdaSymbol{\}}\AgdaSpace{}%
% \AgdaKeyword{where}\<%
% \\
% %
% \\[\AgdaEmptyExtraSkip]%
\>[0][@{}l@{\AgdaIndent{0}}]%
\>[1]\AgdaFunction{⨅}\AgdaSpace{}%
\AgdaSymbol{:}\AgdaSpace{}%
\AgdaSymbol{(}\AgdaBound{𝒜}\AgdaSpace{}%
\AgdaSymbol{:}\AgdaSpace{}%
\AgdaBound{I}\AgdaSpace{}%
\AgdaSymbol{→}\AgdaSpace{}%
\AgdaFunction{Algebra}\AgdaSpace{}%
\AgdaBound{𝓤}\AgdaSpace{}%
\AgdaBound{𝑆}\AgdaSpace{}%
\AgdaSymbol{)}\AgdaSpace{}%
\AgdaSymbol{→}\AgdaSpace{}%
\AgdaFunction{Algebra}\AgdaSpace{}%
\AgdaSymbol{(}\AgdaBound{𝓘}\AgdaSpace{}%
\AgdaOperator{\AgdaPrimitive{⊔}}\AgdaSpace{}%
\AgdaBound{𝓤}\AgdaSymbol{)}\AgdaSpace{}%
\AgdaBound{𝑆}\<%
\\
%
\\[\AgdaEmptyExtraSkip]%
%
\>[1]\AgdaFunction{⨅}\AgdaSpace{}%
\AgdaBound{𝒜}\AgdaSpace{}%
\AgdaSymbol{=}%
\>[51I]\AgdaSymbol{(}\AgdaFunction{Π}\AgdaSpace{}%
\AgdaBound{i}\AgdaSpace{}%
\AgdaFunction{꞉}\AgdaSpace{}%
\AgdaBound{I}\AgdaSpace{}%
\AgdaFunction{,}\AgdaSpace{}%
\AgdaOperator{\AgdaFunction{∣}}\AgdaSpace{}%
\AgdaBound{𝒜}\AgdaSpace{}%
\AgdaBound{i}\AgdaSpace{}%
\AgdaOperator{\AgdaFunction{∣}}\AgdaSymbol{)}\AgdaSpace{}%
\AgdaOperator{\AgdaInductiveConstructor{,}}%
\>[43]\AgdaComment{-- domain of the product algebra}\<%
\\
\>[.][@{}l@{}]\<[51I]%
\>[7]\AgdaSymbol{λ}\AgdaSpace{}%
\AgdaBound{𝑓}\AgdaSpace{}%
\AgdaBound{𝑎}\AgdaSpace{}%
\AgdaBound{i}\AgdaSpace{}%
\AgdaSymbol{→}\AgdaSpace{}%
\AgdaSymbol{(}\AgdaBound{𝑓}\AgdaSpace{}%
\AgdaOperator{\AgdaFunction{̂}}\AgdaSpace{}%
\AgdaBound{𝒜}\AgdaSpace{}%
\AgdaBound{i}\AgdaSymbol{)}\AgdaSpace{}%
\AgdaSymbol{λ}\AgdaSpace{}%
\AgdaBound{x}\AgdaSpace{}%
\AgdaSymbol{→}\AgdaSpace{}%
\AgdaBound{𝑎}\AgdaSpace{}%
\AgdaBound{x}\AgdaSpace{}%
\AgdaBound{i}%
\>[43]\AgdaComment{-- basic operations of the product algebra}\<%
\end{code}
\ccpad
The type just defined is the one we use whenever the product of an indexed collection of algebras (of type \af{Algebra}) is required.  However, for the sake of completeness, here is how one could define a type representing the product of algebras inhabiting the record type \AgdaRecord{algebra}.
\ccpad
\begin{code}%
% %
% \\[\AgdaEmptyExtraSkip]%
% \>[0]\AgdaComment{-- product for algebras of record type}\<%
% \\
\>[0][@{}l@{\AgdaIndent{0}}]%
% \>[1]\AgdaKeyword{open}\AgdaSpace{}%
% \AgdaModule{algebra}\<%
% \\
\>[1]\AgdaFunction{⨅'}\AgdaSpace{}%
\AgdaSymbol{:}\AgdaSpace{}%
% \AgdaSymbol{\{}\AgdaBound{𝓤}\AgdaSpace{}%
% \AgdaBound{𝓘}\AgdaSpace{}%
% \AgdaSymbol{:}\AgdaSpace{}%
% \AgdaPostulate{Universe}\AgdaSymbol{\}\{}\AgdaBound{I}\AgdaSpace{}%
% \AgdaSymbol{:}\AgdaSpace{}%
% \AgdaBound{𝓘}\AgdaSpace{}%
% \AgdaOperator{\AgdaFunction{̇}}\AgdaSymbol{\}\AgdaSpace{}%
\AgdaSymbol{(}\AgdaBound{𝒜}\AgdaSpace{}%
\AgdaSymbol{:}\AgdaSpace{}%
\AgdaBound{I}\AgdaSpace{}%
\AgdaSymbol{→}\AgdaSpace{}%
\AgdaRecord{algebra}\AgdaSpace{}%
\AgdaBound{𝓤}\AgdaSpace{}%
\AgdaBound{𝑆}\AgdaSymbol{)}\AgdaSpace{}%
\AgdaSymbol{→}\AgdaSpace{}%
\AgdaRecord{algebra}\AgdaSpace{}%
\AgdaSymbol{(}\AgdaBound{𝓘}\AgdaSpace{}%
\AgdaOperator{\AgdaPrimitive{⊔}}\AgdaSpace{}%
\AgdaBound{𝓤}\AgdaSymbol{)}\AgdaSpace{}%
\AgdaBound{𝑆}\<%
\\
%
% \\[\AgdaEmptyExtraSkip]%
\>[1]\AgdaFunction{⨅'}\AgdaSpace{}%
\AgdaBound{𝒜}\AgdaSpace{}%
\AgdaSymbol{=}\AgdaSpace{}%
\AgdaKeyword{record}%
\>[100I]\AgdaSymbol{\{}\AgdaSpace{}%
\AgdaField{univ}\AgdaSpace{}%
\AgdaSymbol{=}\AgdaSpace{}%
\AgdaSymbol{∀}\AgdaSpace{}%
\AgdaBound{i}\AgdaSpace{}%
\AgdaSymbol{→}\AgdaSpace{}%
\AgdaField{univ}\AgdaSpace{}%
\AgdaSymbol{(}\AgdaBound{𝒜}\AgdaSpace{}%
\AgdaBound{i}\AgdaSymbol{)}\AgdaSpace{}\AgdaSymbol{;}%
\>[55]\AgdaComment{-- domain}\<%
\\
%\AgdaTab{100}%\>[.][@{}l@{}]\<[100I]%
\>[100I][@{}l@{\AgdaIndent{0}}]%
\>[15]\AgdaField{op}\AgdaSpace{}%
\AgdaSymbol{=}\AgdaSpace{}%
\AgdaSymbol{λ}\AgdaSpace{}%
\AgdaBound{𝑓}\AgdaSpace{}%
\AgdaBound{𝑎}\AgdaSpace{}%
\AgdaBound{i}\AgdaSpace{}%
\AgdaSymbol{→}\AgdaSpace{}%
\AgdaSymbol{(}\AgdaField{op}\AgdaSpace{}%
\AgdaSymbol{(}\AgdaBound{𝒜}\AgdaSpace{}%
\AgdaBound{i}\AgdaSymbol{))}\AgdaSpace{}%
\AgdaBound{𝑓}\AgdaSpace{}%
\AgdaSymbol{λ}\AgdaSpace{}%
\AgdaBound{x}\AgdaSpace{}%
\AgdaSymbol{→}\AgdaSpace{}%
\AgdaBound{𝑎}\AgdaSpace{}%
\AgdaBound{x}\AgdaSpace{}%
\AgdaBound{i}\AgdaSpace{}\AgdaSymbol{\}}%
\>[55]\AgdaComment{-- basic operations}\<%
% \\
% \>[10]\AgdaSymbol{\}}\<%
\end{code}
\ccpad
% \textbf{Notation}. Given a signature \ab 𝑆 : \af{Signature} \ab 𝓞 \ab 𝓥 the type \af{Algebra} \ab 𝓤 \ab 𝑆 has universe 𝓞 ⊔ 𝓥 ⊔ 𝓤 ⁺`.  Such types occur so often in the [UALib][] that it is worthwhile to define the following shorthand for their universes.
Before going further, let us agree on another convenient notational convention, which is used in many of the later modules of the \ualib. Given a signature \ab{𝑆}~\as :~\af{Signature}~\ab 𝓞~\ab 𝓥, the type \af{Algebra}~\ab 𝓤~\ab 𝑆 has universe level \ab{𝓞}~\ap ⊔~\ab 𝓥~\ap ⊔~\ab 𝓤~\af ⁺, and the \ab{𝓞} \ap ⊔ \ab 𝓥 part remains fixed since \ab{𝓞} and \ab 𝓥 always denote the universe levels of operation and arity types, respectively. Such levels occur so often in the \ualib that we define the following shorthand for their universes: \AgdaFunction{ov}\AgdaSpace{}%
\AgdaBound{𝓤}\AgdaSpace{}%
\AgdaSymbol{:=}\AgdaSpace{}%
\AgdaBound{𝓞}\AgdaSpace{}%
\AgdaOperator{\AgdaPrimitive{⊔}}\AgdaSpace{}%
\AgdaBound{𝓥}\AgdaSpace{}%
\AgdaOperator{\AgdaPrimitive{⊔}}\AgdaSpace{}%
\AgdaBound{𝓤}\AgdaSpace{}%
\AgdaOperator{\AgdaPrimitive{⁺}}.


\paragraph*{Products of classes of algebras}\label{products-of-classes-of-algebras}
% \context{\AgdaSymbol{\{}\AgdaBound{𝓤}\AgdaSpace{}%
% \AgdaBound{𝓧}\AgdaSpace{}%
% \AgdaSymbol{:}\AgdaSpace{}%
% \AgdaPostulate{Universe}\AgdaSymbol{\}\{}\AgdaBound{X}\AgdaSpace{}%
% \AgdaSymbol{:}\AgdaSpace{}%
% \AgdaBound{𝓧}\AgdaSpace{}%
% \AgdaOperator{\AgdaFunction{̇}}\AgdaSymbol{\}}}

An arbitrary class \ab 𝒦 of algebras is represented as a predicate over the type \AgdaFunction{Algebra}\AgdaSpace{}\AgdaBound{𝓤}\AgdaSpace{}\AgdaBound{𝑆}, for some universe level \AgdaBound{𝓤} and signature \AgdaBound{𝑆}.
That is, \ab 𝒦~\as :~\af{Pred}(\af{Algebra}~\ab 𝓤~\ab 𝑆)~𝓦 for some 𝓦.
Later we will formally state and prove that the product of all subalgebras of algebras in \ab 𝒦  belongs to the class \af{SP}(\ab 𝒦) of subalgebras of products of algebras in \ab 𝒦. That is, \af ⨅ \af S(\ab 𝒦) \af ∈ \af{SP}(\ab 𝒦). This turns out to be a nontrivial exercise.

To begin, we need to define types that represent products over arbitrary (nonindexed) families such as \ab 𝒦 or \af S(\ab 𝒦). Observe that \ad Π~\ab 𝒦 is definitely \emph{not} what we want.  To see why, recall that \af{Pred}(\ab{Algebra}~\ab 𝓤~\ab 𝑆)~\ab 𝓦 is just an alias for the function type \af{Algebra}~\ab 𝓤~\ab 𝑆~\as →~\ab 𝓦\af ̇. We interpret the latter semantically by taking \ab 𝒦~\ab 𝑨 to be the assertion that \ab 𝒦~\ab 𝑨 belongs to \ab 𝒦~\ab 𝑨, denoted \ab 𝑨 ∈ \ab 𝒦. Therefore, by definition, we have\\[-10pt]

% \vskip-5mm
% \begin{align*}
% \ad Π~\ab 𝒦 &= \ad Π~\ab 𝑨~\af ꞉~(\af{Algebra}~\ab 𝓤~\ab 𝑆)~,~\ab 𝒦 𝑨\\
%              &= \ad Π~\ab 𝑨~\af ꞉~(\af{Algebra}~\ab 𝓤~\ab 𝑆)~,~\ab 𝑨~\af ∈~\ab 𝒦.
% \end{align*}
\ad Π~\ab 𝒦 \hskip1mm = \hskip1mm \ad Π~\ab 𝑨~\af ꞉~(\af{Algebra}~\ab 𝓤~\ab 𝑆)~,~\ab 𝒦 𝑨
             \hskip1mm = \hskip1mm\ad Π~\ab 𝑨~\af ꞉~(\af{Algebra}~\ab 𝓤~\ab 𝑆)~,~\ab 𝑨~\af ∈~\ab 𝒦.\\[4pt]
Semantically, this is the assertion that every algebra of type \af{Algebra} \ab 𝓤 \ab 𝑆 belongs to \ab 𝒦, and this bears little resemblance to the product of algebras that we seek.

What we need is a type that serves to index the class \ab 𝒦, and a function \af 𝔄 that maps an index to the inhabitant of \ab 𝒦 at that index. But \ab 𝒦 is a predicate (of type (\af{Algebra}~\ab 𝓤~\ab 𝑆)~\as →~\ab 𝓦\af ̇) and the type \af{Algebra}~\ab 𝓤~\ab 𝑆 seems rather nebulous in that there is no natural indexing class with which to ``enumerate'' all inhabitants of \af{Algebra}~\ab 𝓤~\ab 𝑆 that belong to \ab 𝒦.\footnote{%
If you haven't already seen this before, do yourself a favor and give it some thought; see if the correct type comes to you organically.}


The solution is to essentially take \ab 𝒦 itself to be the index type; at least heuristically that is how one can think of the type \af ℑ that we now define.\footnote{\textbf{Unicode Hints}. Some of our types are denoted with with Gothic (``mathfrak'') symbols. To produce them in \agdamode, type \texttt{\textbackslash{}Mf} followed by a letter. For example, \texttt{\textbackslash{}MfI} ↝ \af ℑ.}
\ccpad
\begin{code}%
% \>[0]\AgdaKeyword{module}\AgdaSpace{}%
% \AgdaModule{class-products}\AgdaSpace{}%
% \AgdaSymbol{\{}\AgdaBound{𝓤}\AgdaSpace{}%
% \AgdaSymbol{:}\AgdaSpace{}%
% \AgdaPostulate{Universe}\AgdaSymbol{\}}\AgdaSpace{}%
% \AgdaSymbol{(}\AgdaBound{𝒦}\AgdaSpace{}%
% \AgdaSymbol{:}\AgdaSpace{}%
% \AgdaFunction{Pred}\AgdaSpace{}%
% \AgdaSymbol{(}\AgdaFunction{Algebra}\AgdaSpace{}%
% \AgdaBound{𝓤}\AgdaSpace{}%
% \AgdaBound{𝑆}\AgdaSymbol{)(}\AgdaFunction{ov}\AgdaSpace{}%
% \AgdaBound{𝓤}\AgdaSymbol{))}\AgdaSpace{}%
% \AgdaKeyword{where}\<%
% \\
% %
% \\[\AgdaEmptyExtraSkip]%
\>[0][@{}l@{\AgdaIndent{0}}]%
\>[1]\AgdaFunction{ℑ}\AgdaSpace{}%
\AgdaSymbol{:}\AgdaSpace{}%
\AgdaFunction{ov}\AgdaSpace{}%
\AgdaBound{𝓤}\AgdaSpace{}%
\AgdaOperator{\AgdaFunction{̇}}\<%
\\
%
\>[1]\AgdaFunction{ℑ}\AgdaSpace{}%
\AgdaSymbol{=}\AgdaSpace{}%
\AgdaFunction{Σ}\AgdaSpace{}%
\AgdaBound{𝑨}\AgdaSpace{}%
\AgdaFunction{꞉}\AgdaSpace{}%
\AgdaSymbol{(}\AgdaFunction{Algebra}\AgdaSpace{}%
\AgdaBound{𝓤}\AgdaSpace{}%
\AgdaBound{𝑆}\AgdaSymbol{)}\AgdaSpace{}%
\AgdaFunction{,}\AgdaSpace{}%
\AgdaSymbol{(}\AgdaBound{𝑨}\AgdaSpace{}%
\AgdaOperator{\AgdaFunction{∈}}\AgdaSpace{}%
\AgdaBound{𝒦}\AgdaSymbol{)}\<%
\end{code}
\ccpad
Taking the product over the index type \af ℑ requires a function that maps an index \abt{i}{ℑ} to the corresponding algebra.  Each \abt{i}{ℑ} denotes a pair, (\ab 𝑨 , \ab p), where \ab 𝑨 is an algebra and \ab p is a proof that \ab 𝑨 belongs to \ab 𝒦, so the function mapping such an index to the corresponding algebra is simply the first projection.
\ccpad
\begin{code}
\>[0][@{}l@{\AgdaIndent{0}}]%
\>[1]\AgdaFunction{𝔄}\AgdaSpace{}%
\AgdaSymbol{:}\AgdaSpace{}%
\AgdaFunction{ℑ}\AgdaSpace{}%
\AgdaSymbol{→}\AgdaSpace{}%
\AgdaFunction{Algebra}\AgdaSpace{}%
\AgdaBound{𝓤}\AgdaSpace{}%
\AgdaBound{𝑆}\<%
\\
%
\>[1]\AgdaFunction{𝔄}\AgdaSpace{}%
\AgdaSymbol{=}\AgdaSpace{}%
\AgdaSymbol{λ}\AgdaSpace{}%
\AgdaSymbol{(}\AgdaBound{i}\AgdaSpace{}%
\AgdaSymbol{:}\AgdaSpace{}%
\AgdaFunction{ℑ}\AgdaSymbol{)}\AgdaSpace{}%
\AgdaSymbol{→}\AgdaSpace{}%
\AgdaOperator{\AgdaFunction{∣}}\AgdaSpace{}%
\AgdaBound{i}\AgdaSpace{}%
\AgdaOperator{\AgdaFunction{∣}}\<%
\end{code}
\ccpad
Finally, we represent the product of all members of the class \ab 𝒦 by the following type.
\ccpad
\begin{code}%
\>[0][@{}l@{\AgdaIndent{0}}]%
\>[1]\AgdaFunction{class-product}\AgdaSpace{}%
\AgdaSymbol{:}\AgdaSpace{}%
\AgdaFunction{Algebra}\AgdaSpace{}%
\AgdaSymbol{(}\AgdaFunction{ov}\AgdaSpace{}%
\AgdaBound{𝓤}\AgdaSymbol{)}\AgdaSpace{}%
\AgdaBound{𝑆}\<%
\\
%
\>[1]\AgdaFunction{class-product}\AgdaSpace{}%
\AgdaSymbol{=}\AgdaSpace{}%
\AgdaFunction{⨅}\AgdaSpace{}%
\AgdaFunction{𝔄}\<%
\end{code}
\ccpad
Observe that the application \af 𝔄 (\ab 𝑨 , \ab p) of \af 𝔄 to the pair (\ab 𝑨 , \ab p) (the result of which is simply the algebra \ab{𝑨}) may be viewed as the projection out of the product \af ⨅~\af 𝔄 and onto the ``(\ab 𝑨 , \ab p)-th component'' of that product.

