% -*- TeX-master: "ualib-part1.tex" -*-
%%% Local Variables: 
%%% mode: latex
%%% TeX-engine: 'xetex
%%% End:
\subsubsection{Agda's universe hierarchy}\label{agdas-universe-hierarchy}

% The hierarchy of universe levels in Agda looks like this:\\[-4pt]

% \ab{𝓤₀} \as : \ab{𝓤₁}\AgdaOperator{\AgdaFunction{̇}} , ~ ~ \ab{𝓤₁} \as : \ab{𝓤₂}\AgdaOperator{\AgdaFunction{̇}} , ~ ~ \ab{𝓤₂} \as : \ab{𝓤₃}\AgdaOperator{\AgdaFunction{̇}} , \ldots{}\\[4pt]
% This means that the type level of \ab{𝓤₀} is \ab{𝓤₁}\AgdaOperator{\AgdaFunction{̇}}, and for each \ab n the type level of \ab{𝓤ₙ} is \ab{𝓤ₙ₊₁}\AgdaOperator{\AgdaFunction{̇}}.
% It is important to note, however, this does \emph{not} imply that \ab{𝓤₀} \as
% : \ab{𝓤₂}\AgdaOperator{\AgdaFunction{̇}} and \ab{𝓤₀} \as :
% \ab{𝓤₃}\AgdaOperator{\AgdaFunction{̇}}, and so on. 
% In As mentioned above (\S~\ref{ssec:agdas-universe-hierarchy}), Agda's universe hierarchy is \defn{noncumulative}.

The hierarchy of universes in Agda is structured as follows: \ab{𝓤}\af ̇ \as : \ab 𝓤 \af ⁺\af ̇,
\ab{𝓤} \af ⁺\af ̇ \as : \ab 𝓤 \af ⁺\af ⁺\af ̇, etc.\footnote{Recall, from the \ualibhtml{Overture.Preliminaries} module~(\S\ref{universes}), the special notation we use to denote Agda's \textit{levels} and \textit{universes}.}
This means that the universe \ab 𝓤\af ̇ has type \ab 𝓤  \af ⁺\af ̇, and 𝓤 \af ⁺\af ̇ has type \ab 𝓤 \af ⁺\af ⁺\af ̇, and so on.  It is important to note, however, this does \emph{not} imply that \ab 𝓤\af ̇ \as : \ab 𝓤 \af ⁺\af ⁺\af ̇. In other words, Agda's universe hierarchy is \emph{noncumulative}. This can be advantageous as it becomes possible to treat universe levels more generally and precisely. On the other hand, a noncumulative hierarchy can sometimes make it seem unduly difficult to convince Agda that a program or proof is correct. To help us overcome this technical issue, there are some general universe lifting and lowering functions, which we describe in the next subsection. In Section~\ref{lifts-of-algebras} we will define a couple of domain-specific analogs of these tools.  Later, in the  modules presented in~\cite{DeMeo:2021-2,DeMeo:2021-3}, we prove various properties that make these effective mechanisms for resolving universe level problems when working with algebra types.


\subsubsection{Lifting and lowering}\label{lifting-and-lowering}
Let us be more concrete about what is at issue here by considering a typical example. Agda frequently encounters errors during the type-checking process and responds by printing an error message. Often the message has the following form.
{\color{red}{\small
\begin{verbatim}
  Birkhoff.lagda:498,20-23
  𝓤 != 𝓞 ⊔ 𝓥 ⊔ (𝓤 ⁺) when checking that... has type...
\end{verbatim}}}
\noindent This error message means that Agda encountered the universe \ab{𝓤} on line 498 (columns 20--23) of the file \af{Birkhoff.lagda}, but was expecting to find the universe \ab{𝓞}~\aop ⊔~\ab 𝓥~\aop ⊔~\ab 𝓤\af ⁺ instead.


The general \AgdaRecord{Lift} record type that we now describe makes these situations easier to deal with. It takes a type inhabiting some universe and embeds it into a higher universe and, apart from syntax and notation, it is equivalent to the \AgdaRecord{Lift} type one finds in the \am{Level} module of the \agdastdlib.
\ccpad
\begin{code}%
\>[0]\AgdaKeyword{record}\AgdaSpace{}%
\AgdaRecord{Lift}\AgdaSpace{}%
\AgdaSymbol{\{}\AgdaBound{𝓦}\AgdaSpace{}%
\AgdaBound{𝓤}\AgdaSpace{}%
\AgdaSymbol{:}\AgdaSpace{}%
\AgdaPostulate{Universe}\AgdaSymbol{\}}\AgdaSpace{}%
\AgdaSymbol{(}\AgdaBound{A}\AgdaSpace{}%
\AgdaSymbol{:}\AgdaSpace{}%
\AgdaBound{𝓤}\AgdaSpace{}%
\AgdaOperator{\AgdaFunction{̇}}\AgdaSymbol{)}\AgdaSpace{}%
\AgdaSymbol{:}\AgdaSpace{}%
\AgdaBound{𝓤}\AgdaSpace{}%
\AgdaOperator{\AgdaPrimitive{⊔}}\AgdaSpace{}%
\AgdaBound{𝓦}\AgdaSpace{}%
\AgdaOperator{\AgdaFunction{̇}}%
\>[50]\AgdaKeyword{where}\<%
\\
\>[0][@{}l@{\AgdaIndent{0}}]%
\>[1]\AgdaKeyword{constructor}\AgdaSpace{}%
\AgdaInductiveConstructor{lift}\<%
\\
%
\>[1]\AgdaKeyword{field}\AgdaSpace{}%
\AgdaField{lower}\AgdaSpace{}%
\AgdaSymbol{:}\AgdaSpace{}%
\AgdaBound{A}\<%
\\
\>[0]\AgdaKeyword{open}\AgdaSpace{}%
\AgdaModule{Lift}\<%
\end{code}
\scpad

The point of having a ramified hierarchy of universes is to avoid Russell's paradox, and this would be subverted if we were to lower the universe of a type that wasn't previously lifted.  However, we can prove that if an application of \afld{lower} is immediately followed by an application of \aic{lift}, then the result is the identity transformation. Similarly, \aic{lift} followed by \afld{lower} is the identity.
\ccpad
\begin{code}%
\>[0]\AgdaFunction{lift∼lower}\AgdaSpace{}%
\AgdaSymbol{:}\AgdaSpace{}%
\AgdaSymbol{\{}\AgdaBound{𝓦}\AgdaSpace{}%
\AgdaBound{𝓤}\AgdaSpace{}%
\AgdaSymbol{:}\AgdaSpace{}%
\AgdaPostulate{Universe}\AgdaSymbol{\}\{}\AgdaBound{A}\AgdaSpace{}%
\AgdaSymbol{:}\AgdaSpace{}%
\AgdaBound{𝓤}\AgdaSpace{}%
\AgdaOperator{\AgdaFunction{̇}}\AgdaSymbol{\}}\AgdaSpace{}%
\AgdaSymbol{→}\AgdaSpace{}%
\AgdaInductiveConstructor{lift}\AgdaSpace{}%
\AgdaOperator{\AgdaFunction{∘}}\AgdaSpace{}%
\AgdaField{lower}\AgdaSpace{}%
\AgdaOperator{\AgdaDatatype{≡}}\AgdaSpace{}%
\AgdaFunction{𝑖𝑑}\AgdaSpace{}%
\AgdaSymbol{(}\AgdaRecord{Lift}\AgdaSymbol{\{}\AgdaBound{𝓦}\AgdaSymbol{\}}\AgdaSpace{}%
\AgdaBound{A}\AgdaSymbol{)}\<%
\\
\>[0]\AgdaFunction{lift∼lower}\AgdaSpace{}%
\AgdaSymbol{=}\AgdaSpace{}%
\AgdaInductiveConstructor{refl}\<%
\\
%
\\[\AgdaEmptyExtraSkip]%
\>[0]\AgdaFunction{lower∼lift}\AgdaSpace{}%
\AgdaSymbol{:}\AgdaSpace{}%
\AgdaSymbol{\{}\AgdaBound{𝓦}\AgdaSpace{}%
\AgdaBound{𝓤}\AgdaSpace{}%
\AgdaSymbol{:}\AgdaSpace{}%
\AgdaPostulate{Universe}\AgdaSymbol{\}\{}\AgdaBound{A}\AgdaSpace{}%
\AgdaSymbol{:}\AgdaSpace{}%
\AgdaBound{𝓤}\AgdaSpace{}%
\AgdaOperator{\AgdaFunction{̇}}\AgdaSymbol{\}}\AgdaSpace{}%
\AgdaSymbol{→}\AgdaSpace{}%
\AgdaField{lower}\AgdaSymbol{\{}\AgdaBound{𝓦}\AgdaSymbol{\}\{}\AgdaBound{𝓤}\AgdaSymbol{\}}\AgdaSpace{}%
\AgdaOperator{\AgdaFunction{∘}}\AgdaSpace{}%
\AgdaInductiveConstructor{lift}\AgdaSpace{}%
\AgdaOperator{\AgdaDatatype{≡}}\AgdaSpace{}%
\AgdaFunction{𝑖𝑑}\AgdaSpace{}%
\AgdaBound{A}\<%
\\
\>[0]\AgdaFunction{lower∼lift}\AgdaSpace{}%
\AgdaSymbol{=}\AgdaSpace{}%
\AgdaInductiveConstructor{refl}\<%
\end{code}
\ccpad
The proofs are trivial. Nonetheless, we'll come across some holes these types can fill.





% Using the \af{Lift} type, we define two functions that lift the domain (respectively, codomain) type of a function to a higher universe.
% \ccpad
% \begin{code}%
% \>[1]\AgdaFunction{lift-dom}\AgdaSpace{}%
% \AgdaSymbol{:}\AgdaSpace{}%
% \AgdaSymbol{\{}\AgdaBound{X}\AgdaSpace{}%
% \AgdaSymbol{:}\AgdaSpace{}%
% \AgdaBound{𝓧}\AgdaSpace{}%
% \AgdaOperator{\AgdaFunction{̇}}\AgdaSymbol{\}\{}\AgdaBound{Y}\AgdaSpace{}%
% \AgdaSymbol{:}\AgdaSpace{}%
% \AgdaBound{𝓨}\AgdaSpace{}%
% \AgdaOperator{\AgdaFunction{̇}}\AgdaSymbol{\}}\AgdaSpace{}%
% \AgdaSymbol{→}\AgdaSpace{}%
% \AgdaSymbol{(}\AgdaBound{X}\AgdaSpace{}%
% \AgdaSymbol{→}\AgdaSpace{}%
% \AgdaBound{Y}\AgdaSymbol{)}\AgdaSpace{}%
% \AgdaSymbol{→}\AgdaSpace{}%
% \AgdaSymbol{(}\AgdaRecord{Lift}\AgdaSymbol{\{}\AgdaBound{𝓦}\AgdaSymbol{\}\{}\AgdaBound{𝓧}\AgdaSymbol{\}}\AgdaSpace{}%
% \AgdaBound{X}\AgdaSpace{}%
% \AgdaSymbol{→}\AgdaSpace{}%
% \AgdaBound{Y}\AgdaSymbol{)}\<%
% \\
% %
% \>[1]\AgdaFunction{lift-dom}\AgdaSpace{}%
% \AgdaBound{f}\AgdaSpace{}%
% \AgdaSymbol{=}\AgdaSpace{}%
% \AgdaSymbol{λ}\AgdaSpace{}%
% \AgdaBound{x}\AgdaSpace{}%
% \AgdaSymbol{→}\AgdaSpace{}%
% \AgdaSymbol{(}\AgdaBound{f}\AgdaSpace{}%
% \AgdaSymbol{(}\AgdaField{lower}\AgdaSpace{}%
% \AgdaBound{x}\AgdaSymbol{))}\<%
% \\
% %
% \\[\AgdaEmptyExtraSkip]%
% %
% \>[1]\AgdaFunction{lift-cod}\AgdaSpace{}%
% \AgdaSymbol{:}\AgdaSpace{}%
% \AgdaSymbol{\{}\AgdaBound{X}\AgdaSpace{}%
% \AgdaSymbol{:}\AgdaSpace{}%
% \AgdaBound{𝓧}\AgdaSpace{}%
% \AgdaOperator{\AgdaFunction{̇}}\AgdaSymbol{\}\{}\AgdaBound{Y}\AgdaSpace{}%
% \AgdaSymbol{:}\AgdaSpace{}%
% \AgdaBound{𝓨}\AgdaSpace{}%
% \AgdaOperator{\AgdaFunction{̇}}\AgdaSymbol{\}}\AgdaSpace{}%
% \AgdaSymbol{→}\AgdaSpace{}%
% \AgdaSymbol{(}\AgdaBound{X}\AgdaSpace{}%
% \AgdaSymbol{→}\AgdaSpace{}%
% \AgdaBound{Y}\AgdaSymbol{)}\AgdaSpace{}%
% \AgdaSymbol{→}\AgdaSpace{}%
% \AgdaSymbol{(}\AgdaBound{X}\AgdaSpace{}%
% \AgdaSymbol{→}\AgdaSpace{}%
% \AgdaRecord{Lift}\AgdaSymbol{\{}\AgdaBound{𝓦}\AgdaSymbol{\}\{}\AgdaBound{𝓨}\AgdaSymbol{\}}\AgdaSpace{}%
% \AgdaBound{Y}\AgdaSymbol{)}\<%
% \\
% %
% \>[1]\AgdaFunction{lift-cod}\AgdaSpace{}%
% \AgdaBound{f}\AgdaSpace{}%
% \AgdaSymbol{=}\AgdaSpace{}%
% \AgdaSymbol{λ}\AgdaSpace{}%
% \AgdaBound{x}\AgdaSpace{}%
% \AgdaSymbol{→}\AgdaSpace{}%
% \AgdaInductiveConstructor{lift}\AgdaSpace{}%
% \AgdaSymbol{(}\AgdaBound{f}\AgdaSpace{}%
% \AgdaBound{x}\AgdaSymbol{)}\<%
% \end{code}
% \ccpad
% Naturally, we can combine these and define a function that lifts both the domain and codomain type.
% \ccpad
% \begin{code}
% \>[1]\AgdaFunction{lift-fun}\AgdaSpace{}%
% \AgdaSymbol{:}\AgdaSpace{}%
% \AgdaSymbol{\{}\AgdaBound{X}\AgdaSpace{}%
% \AgdaSymbol{:}\AgdaSpace{}%
% \AgdaBound{𝓧}\AgdaSpace{}%
% \AgdaOperator{\AgdaFunction{̇}}\AgdaSymbol{\}\{}\AgdaBound{Y}\AgdaSpace{}%
% \AgdaSymbol{:}\AgdaSpace{}%
% \AgdaBound{𝓨}\AgdaSpace{}%
% \AgdaOperator{\AgdaFunction{̇}}\AgdaSymbol{\}}\AgdaSpace{}%
% \AgdaSymbol{→}\AgdaSpace{}%
% \AgdaSymbol{(}\AgdaBound{X}\AgdaSpace{}%
% \AgdaSymbol{→}\AgdaSpace{}%
% \AgdaBound{Y}\AgdaSymbol{)}\AgdaSpace{}%
% \AgdaSymbol{→}\AgdaSpace{}%
% \AgdaSymbol{(}\AgdaRecord{Lift}\AgdaSymbol{\{}\AgdaBound{𝓦}\AgdaSymbol{\}\{}\AgdaBound{𝓧}\AgdaSymbol{\}}\AgdaSpace{}%
% \AgdaBound{X}\AgdaSpace{}%
% \AgdaSymbol{→}\AgdaSpace{}%
% \AgdaRecord{Lift}\AgdaSymbol{\{}\AgdaBound{𝓩}\AgdaSymbol{\}\{}\AgdaBound{𝓨}\AgdaSymbol{\}}\AgdaSpace{}%
% \AgdaBound{Y}\AgdaSymbol{)}\<%
% \\
% %
% \>[1]\AgdaFunction{lift-fun}\AgdaSpace{}%
% \AgdaBound{f}\AgdaSpace{}%
% \AgdaSymbol{=}\AgdaSpace{}%
% \AgdaSymbol{λ}\AgdaSpace{}%
% \AgdaBound{x}\AgdaSpace{}%
% \AgdaSymbol{→}\AgdaSpace{}%
% \AgdaInductiveConstructor{lift}\AgdaSpace{}%
% \AgdaSymbol{(}\AgdaBound{f}\AgdaSpace{}%
% \AgdaSymbol{(}\AgdaField{lower}\AgdaSpace{}%
% \AgdaBound{x}\AgdaSymbol{))}\<%
% \end{code}
% \ccpad
% For example, \af{lift-dom} takes a function \ab f defined on the domain \AgdaBound{X}\AgdaSpace{}\AgdaSymbol{:}\AgdaSpace{}\AgdaBound{𝓧}\AgdaSpace{}\AgdaOperator{\AgdaFunction{̇}}\AgdaSpace{} and returns a function defined on the domain \ar{Lift}\as{\{}\ab 𝓦\as{\}}\as{\{}\ab 𝓧\as{\}} \ab X \as : \ab 𝓧 \aop ⊔ \ab 𝓦\aof ̇.
