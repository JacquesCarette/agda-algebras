% -*- TeX-master: "ualib-part1.tex" -*-
%%% Local Variables: 
%%% mode: latex
%%% TeX-engine: 'xetex
%%% End:
\section{Introduction}\label{sec:introduction}
To support formalization in type theory of research level mathematics in universal algebra and related fields, we present the Agda Universal Algebra Library (\agdaualib), a software library containing formal statements and proofs of the core definitions and results of universal algebra. The \ualib is written in \agda~\cite{Norell:2009}, a programming language and proof assistant based on \MLTT (\mltt) that supports dependent and inductive types.


\subsection{Motivation}\label{sec:motivation}
The seminal idea for the \agdaualib project was the observation that, on the one hand, a number of fundamental constructions in universal algebra can be defined recursively, and theorems about them proved by structural induction, while, on the other hand, inductive and dependent types make possible very precise formal representations of recursively defined objects, which often admit elegant constructive proofs of properties of such objects.  An important feature of such proofs in type theory is that they are total functional programs and, as such, they are computable, composable, and machine-verifiable.
% These observations suggested that there would be much to gain from implementing universal algebra in a language, such as Martin-L\"of type theory, that features dependent and inductive types.

Finally, our own research experience has taught us that a proof assistant and programming language (like Agda), when equipped with specialized libraries and domain-specific tactics to automate the proof idioms of our field, can be an extremely powerful and effective asset. As such we believe that proof assistants and their supporting libraries will eventually become indispensable tools in the working mathematician's toolkit.

\subsection{Attributions and Contributions}\label{sec:contributions}
The mathematical results described in this paper have well known \emph{informal} proofs. Our main contribution is the formalization, mechanization, and verification of the statements and proofs of these results in dependent type theory using Agda.

Unless explicitly stated otherwise, the Agda source code described in this paper is due to the author, with the following caveat: the \ualib depends on the \typetopology library of \MartinEscardo~\cite{MHE}. For convenience, we refer to Escard\'o's library as \typtop throughout the paper. For the sake of completeness and clarity, and to keep the paper mostly self-contained, we repeat some definitions from \typtop, but in each instance we cite the original source.\footnote{In the \ualib, such instances occur only inside hidden modules that are never actually used, followed immediately by a statement that imports the code in question from its original source.}


\subsection{Prior art}\label{sec:prior-art}
There have been a number of efforts to formalize parts of universal algebra in type theory prior to ours, most notably
\begin{itemize}
  \item Capretta~\cite{Capretta:1999} (1999) formalized the basics of universal algebra in the Calculus of Inductive Constructions using the Coq proof assistant;
    \item Spitters and van der Weegen~\cite{Spitters:2011} (2011) formalized the basics of universal algebra and some classical algebraic structures, also in the Calculus of Inductive Constructions using the Coq proof assistant, promoting the use of type classes;
 \item Gunther, et al~\cite{Gunther:2018} (2018) developed what seems to be (prior to the \ualib) the most extensive library of formal universal algebra to date; in particular, this work includes a formalization of some basic equational logic; also (unlike the \ualib) it handles \emph{multisorted} algebraic structures; (like the \ualib) it is based on dependent type theory and the Agda proof assistant.
\end{itemize}
Some other projects aimed at formalizing mathematics generally, and algebra in particular, have developed into very extensive libraries that include definitions, theorems, and proofs about algebraic structures, such as groups, rings, modules, etc.  However, the goals of these efforts seem to be the formalization of special classical algebraic structures, as opposed to the general theory of (universal) algebras. Moreover, the part of universal algebra and equational logic formalized in the \ualib extends beyond the scope of prior efforts. In particular, the library now includes a proof of Birkhoff's variety theorem.  Most other proofs of this theorem that we know of are informal and nonconstructive.\footnote{After completing the formal proof in \agda, we learned about a constructive version of Birkhoff's theorem proved by Carlstr\"om in~\cite{Carlstrom:2008}.  The latter is presented in the informal style of standard mathematical writing, and as far as we know it was never formalized in type theory and type-checked with a proof assistant. Nonetheless, a comparison of Carlstr\"om's proof and the \ualib proof would be interesting.}
 %% We remark briefly on this in~\S\ref{sec:conclusions-and-future-work}.}


\subsection{Organization of the paper}\label{sec:organization}

In this paper we limit ourselves to the presentation of the core foundational modules of the \ualib so that we have space to discuss some of the more interesting type theoretic and foundational issues that arose when developing the library and attempting to represent advanced mathematical notions in type theory and formalize them in Agda.  This is the first in a series of three papers describing the \agdaualib. The second paper (\cite{DeMeo:2021-2}) covers \emph{homomorphisms}, \emph{terms}, and \emph{subalgebras}. The third paper (\cite{DeMeo:2021-3}) covers \emph{free algebras}, \emph{equational classes} of algebras (i.e., \emph{varieties}), and \emph{Birkhoff's HSP theorem}.

This present paper is organized into three parts as follows. The first part is~\S\ref{sec:overture} which introduces the basic concepts of type theory with special emphasis on the way such concepts are formalized in \agda. Specifically, \S\ref{preliminaries} introduces \emph{Sigma types} and Agda's hierarchy of \emph{universes}. The important topics of \emph{equality} and \emph{function extensionality} are discussed in \S\ref{equality} and \S\ref{function-extensionality}; \S\ref{sec:inverse-image-invers} covers inverses and inverse images of functions. In \S\ref{sec:lifts-altern-univ} we describe a technical problem that one frequently encounters when working in a \emph{noncumulative universe hierarchy} and offer some tools for resolving the type-checking errors that arise from this.

The second part is~\S\ref{sec:relat-quot-types} which covers \emph{relation types} and \emph{quotient types}. Specifically,~\S\ref{sec:discrete-relations} defines types that represent \emph{unary} and \emph{binary relations} as well as \emph{function kernels}. These ``discrete relation types,'' are all very standard.  In~\S\ref{sec:continuous-relations} we introduce the (less standard) types that we use to represent \emph{general} and \emph{dependent relations}. We call these ``continuous relations'' because they can have arbitrary arity (general relations) and they can be defined over arbitrary families of types (dependent relations).
In~\S\ref{sec:equiv-relat-quot} we cover standard types for equivalence relations and quotients, and in~\S\ref{sec:trunc-sets-prop} we discuss a family of concepts that are vital to the mechanization of mathematics using type theory; these are the closely related concepts of \emph{truncation}, \emph{sets}, \emph{propositions}, and \emph{proposition extensionality}.

The third part of the paper is~\S\ref{sec:algebra-types} which covers the basic domain-specific types offered by the \ualib. It is here that we finally get to see some types representing algebraic structures.  Specifically, we describe types for \emph{operations} and \emph{signatures}~(\S\ref{sec:oper-sign}), \emph{general algebras}~(\S\ref{sec:algebras}), and \emph{product algebras}~(\S\ref{sec:product-algebras}), including types for representing \emph{products over arbitrary classes of algebraic structures}. Finally, we define types for congruence relations and quotient algebras in~\S\ref{congruences}.

\newcommand\otherparta{\textit{Part 2: homomorphisms, terms, and subalgebras}~\cite{DeMeo:2021-2}.}
\newcommand\otherpartb{\textit{Part 3: free algebras, equational classes, and Birkhoff's theorem}~\cite{DeMeo:2021-3}.}

\subsection{Resources}
We conclude this introduction with some pointers to helpful reference materials.
For the required background in Universal Algebra, we recommend the textbook by Clifford Bergman~\cite{Bergman:2012}.  For the type theory background, we recommend the HoTT Book~\cite{HoTT} and Escard\'o's \ufcourse~\cite{MHE}.

The following are informed the development of the \ualib and are highly recommended.
\begin{itemize}
\item \textit{\ufcourse}, \escardo~\cite{MHE}.
\item \href{http://www.cse.chalmers.se/~peterd/papers/DependentTypesAtWork.pdf}{\it Dependent Types at Work}, Bove and Dybjer~\cite{Bove:2009}.
\item \href{http://www.cse.chalmers.se/~ulfn/papers/afp08/tutorial.pdf}{\it Dependently Typed Programming in Agda}, Norell and Chapman~\cite{Norell:2008}.
\item \href{http://www.sciencedirect.com/science/article/pii/S1571066118300768}{\it Formalization of Universal Algebra in Agda}, Gunther, Gadea, Pagano~\cite{Gunther:2018}.
\item \href{https://plfa.github.io/}{\it Programming Languages Foundations in Agda}, Philip Wadler~\cite{Wadler:2020}.
\end{itemize}

More information about \agdaualib can be obtained from the following official sources.
\begin{itemize}
  \item \href{https://ualib.gitlab.io}{\texttt{ualib.org}} (the web site) documents every line of code in the library.
  \item \href{https://gitlab.com/ualib/ualib.gitlab.io}{\texttt{gitlab.com/ualib/ualib.gitlab.io}} (the source code) \agdaualib is open source.\footnote{License: \href{https://creativecommons.org/licenses/by-sa/4.0/}{Creative Commons Attribution-ShareAlike 4.0 International License.}}
  \item \emph{The Agda UALib}, \otherparta
  \item \emph{The Agda UALib}, \otherpartb
\end{itemize}
The first item links to the official \ualib html documentation which includes complete proofs of every theorem we mention here, and much more, including the Agda modules covered in the first and third installments of this series of papers on the \ualib.

Finally, readers will get much more out of reading the paper if they download the \agdaualib from \url{https://gitlab.com/ualib/ualib.gitlab.io}, install the library, and try it out for themselves.
