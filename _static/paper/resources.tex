For the required background in Universal Algebra, we recommend the textbook by Clifford Bergman~\cite{Bergman:2012}.  For the type theory background, we recommend the HoTT Book~\cite{HoTT} and Escard\'o's \ufcourse.

The following references informed our development of the \ualib, and we recommend these for readers seeking more information about Agda.
\begin{itemize}
\item \textit{\ufcourse}, \escardo~\cite{MHE}
\item \href{http://www.cse.chalmers.se/~peterd/papers/DependentTypesAtWork.pdf}{\it Dependent Types at Work}, Bove and Dybjer~\cite{Bove:2009}
\item \href{http://www.cse.chalmers.se/~ulfn/papers/afp08/tutorial.pdf}{\it Dependently Typed Programming in Agda}, Norell and Chapman~\cite{Norell:2008}
\item \href{http://www.sciencedirect.com/science/article/pii/S1571066118300768}{\it Formalization of Universal Algebra in Agda}, Gunther, Gadea, Pagano~\cite{Gunther:2018}
\item \href{https://plfa.github.io/}{\it Programming Languages Foundations in Agda}~\cite{Wadler:2020}
\end{itemize}

For more information about \agdaualib, the following are the official sources.
\begin{itemize}
  \item \href{https://ualib.gitlab.io}{\texttt{ualib.org}} (the web site) includes every line of code in the library, rendered as html and accompanied by documentation, and
  \item \href{https://gitlab.com/ualib/ualib.gitlab.io}{\texttt{gitlab.com/ualib/ualib.gitlab.io}} (the source code) freely available and licensed under the \href{https://creativecommons.org/licenses/by-sa/4.0/}{Creative Commons Attribution-ShareAlike 4.0 International License}.
  \item \emph{The Agda Universal Algebra Library, Part 2: homomorphisms, terms, and subalgebras}~\cite{DeMeo:2021-2}. 
\item \emph{The Agda Universal Algebra Library, Part 3: free algebras, equational classes, and Birkhoff's theorem}~\cite{DeMeo:2021-3}.
\end{itemize}
The first item links to the official \ualib html documentation which includes complete proofs of every theorem we mention here, and much more, including the Agda modules covered in the first and third installments of this series of papers on the \ualib.

Finally, readers will get much more out of the paper if they download the \ualib from \url{https://gitlab.com/ualib/ualib.gitlab.io/}, install the library, and try it out for themselves.