For the benefit of readers who are not proficient in Agda and/or type theory, we briefly describe some of the most important types and features of Agda used in the \ualib.  Of necessity, some descriptions will be terse, but are usually accompanied by pointers to relevant sections of Appendix~\S\ref{sec:agda-prerequisites} or the \href{https://ualib.gitlab.io}{html documentation}.

We begin by highlighting some of the key parts of the \ualibPreliminaries module of the \ualib. This module imports everything we need from Martin Escardo's \typetopology library~\cite{MHE}, defines some other basic types and proves some of their properties.  We do not cover the entire Preliminaries module here, but call attention to aspects that differ from standard Agda syntax. For more details, see~\cite[\S2]{DeMeo:2021}.

Agda programs typically begin by setting some options and by importing from existing libraries.
Options are specified with the \AgdaKeyword{OPTIONS} \emph{pragma} and control the way Agda behaves by, for example, specifying which logical foundations should be assumed when the program is type-checked to verify its correctness. 
Every Agda program in the \ualib begins with the following pragma, which has the effects described below.
\ccpad
\begin{code}[number=code:options]
\>[0]\AgdaSymbol{\{-\#}\AgdaSpace{}%
\AgdaKeyword{OPTIONS}\AgdaSpace{}%
\AgdaPragma{--without-K}\AgdaSpace{}%
\AgdaPragma{--exact-split}\AgdaSpace{}%
\AgdaPragma{--safe}\AgdaSpace{}%
\AgdaSymbol{\#-\}}\<%
\end{code}
\\[-20pt]
%% \begin{enumerate}
\begin{itemize}
\item \AgdaPragma{without-K} disables \axiomk; see~\cite{agdaref-axiomk};
\item \AgdaPragma{exact-split} makes Agda accept only definitions that are \emph{judgmental} or \emph{definitional} equalities; see~\cite{agdatools-patternmatching};
  %% As \escardo explains, this ``makes sure that pattern matching corresponds to Martin-Löf eliminators;'' for more details
\item \AgdaPragma{safe} ensures that nothing is postulated outright---every non-MLTT axiom has to be an explicit assumption (e.g., an argument to a function or module); see~\cite{agdaref-safeagda} and~\cite{agdatools-patternmatching}.
\end{itemize}
%% \end{enumerate}
Throughout this paper we take assumptions 1--3 for granted without mentioning them explicitly.

\subsection{Agda's universe hierarchy}\label{ssec:agdas-universe-hierarchy}
The \agdaualib adopts the notation of Martin Escardo's \typetopology library~\cite{MHE}. In particular, \emph{universe levels}%
\footnote{See \href{https://agda.readthedocs.io/en/v2.6.1.2/language/universe-levels.html}{agda.readthedocs.io/en/v2.6.1.2/language/universe-levels.html}.}
are denoted by capitalized script letters from the second half of the alphabet, e.g., \ab 𝓤, \ab 𝓥, \ab 𝓦, etc.  Also defined in \typetopology are the operators~\af ̇~and~\af ⁺. These map a universe level \ab 𝓤 to the universe \ab 𝓤 \af ̇ := \Set \ab 𝓤 and the level \ab 𝓤 \af ⁺ \aod := \lsuc \ab 𝓤, respectively.  Thus, \ab 𝓤 \af ̇ is simply an alias for the universe \Set \ab 𝓤, and we have \ab 𝓤 \af ̇ \as : (\ab 𝓤 \af ⁺) \af ̇.
%% Table~\ref{tab:dictionary} translates between standard \agda syntax and \typetopology/\ualib notation.

The hierarchy of universes in Agda is structured as \ab{𝓤} \af ̇ \as : \ab 𝓤 \af ⁺\af ̇, \hskip3mm
\ab{𝓤} \af ⁺\af ̇ \as : \ab 𝓤 \af ⁺\af ⁺\af ̇, etc. This means that the universe \ab 𝓤 \af ̇ has type \ab 𝓤  \af ⁺\af ̇, and 𝓤 \af ⁺\af ̇ has type \ab 𝓤 \af ⁺\af ⁺\af ̇, and so on.  It is important to note, however, this does \emph{not} imply that \ab 𝓤 \af ̇ \as : \ab 𝓤 \af ⁺\af ⁺\af ̇. In other words, Agda's universe hierarchy is \emph{noncummulative}. This makes it possible to treat universe levels more generally and precisely, which is nice. On the other hand, a noncummulative hierarchy can sometimes make for a nonfun proof assistant. Luckily, there are ways to circumvent noncummulativity without introducing logical inconsistencies into the type theory. \S\ref{ssec:tools-for-noncummulative-universes} describes some domain specific tools that we developed for this purpose. (See also~\cite[\S3.3]{DeMeo:2021} for more details).




\subsection{Dependent pairs}\label{ssec:dependent-pairs}
Given universes \ab 𝓤 and \ab 𝓥, a type \ab X \as : \ab 𝓤 \aof ̇, and a type family \ab Y \as : X \as → \ab 𝓥 \aof ̇, the \defn{Sigma type} (or \defn{dependent pair type}) is denoted by \AgdaRecord{Σ}(\ab x \as ꞉ \ab X)\as ,\ab Y(\ab x) and generalizes the Cartesian product \ab X \as × \ab Y by allowing the type \ab Y(\ab x) of the second argument of the ordered pair (\ab x\as , \ab y) to depend on the value \ab x of the first.  That is, \AgdaRecord{Σ}(\ab x \as ꞉ \ab X)\as ,\ab Y(\ab x) is inhabited by pairs (\ab x\as , \ab y) such that \ab x \as : \ab X and \ab y \as : \ab Y(\ab x).

Agda's default syntax for a Sigma type is \AgdaRecord{Σ}\sP{3}\AgdaSymbol{λ}(\ab x\sP{3}꞉\sP{3}\ab X)\sP{3}\as →\sP{3}\ab Y, but we prefer the notation \AgdaRecord{Σ}~\ab x~꞉~\ab X~,~\ab Y, which is closer to the standard syntax described in the preceding paragraph. Fortunately, this preferred notation is available in the \typetopology library (see~\cite[Σ types]{MHE}).\footnote{The symbol \as ꞉ in the expression \AgdaRecord{Σ}\sP{3}\ab x\sP{3}\as ꞉\sP{3}\ab X\sP{3}\AgdaComma\sP{3}\ab Y is not the ordinary colon (:); rather, it is the symbol obtained by typing \texttt{\textbackslash{}:4} in \agdatwomode.} 

\newcommand\FstUnder{\AgdaOperator{\AgdaFunction{∣\AgdaUnderscore{}∣}}\xspace}
\newcommand\SndUnder{\AgdaOperator{\AgdaFunction{∥\AgdaUnderscore{}∥}}\xspace}
Convenient notations for the first and second projections out of a product are \FstUnder and \SndUnder, respectively. However, to improve readability or to avoid notation clashes with other modules, we sometimes use more standard alternatives, such as \AgdaFunction{pr₁} and \AgdaFunction{pr₂}, or \AgdaFunction{fst} and \AgdaFunction{snd}, or some combination of these. The definitions are standard so we omit them (see~\cite{DeMeo:2021} for details).






