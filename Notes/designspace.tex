
\documentclass[a4paper,UKenglish,cleveref, autoref, thm-restate]{lipics-v2021}
%\graphicspath{{./graphics/}}%helpful if your graphic files are in another directory

\bibliographystyle{plainurl}

\title{The design space of generalizing universal algebra}

\titlerunning{Design space of UA}

\author{Jacques Carette}{McMaster University, Canada
  \and \url{http://www.cas.mcmaster.ca/~carette}
}{carette@mcmaster.ca}{https://orcid.org/0000-0001-8993-9804}{Funded in part by NSERC}

\authorrunning{J. Carette}

\Copyright{Jacques Carette}

\ccsdesc[500]{Theory of computation~Type theory}

\keywords{Universal Algebra, Theories}

% \category{} %optional, e.g. invited paper

% \relatedversion{} %optional, e.g. full version hosted on arXiv, HAL, or other respository/website
%\relatedversiondetails[linktext={opt. text shown instead of the URL}, cite=DBLP:books/mk/GrayR93]{Classification (e.g. Full Version, Extended Version, Previous Version}{URL to related version} %linktext and cite are optional

%\supplement{}%optional, e.g. related research data, source code, ... hosted on a repository like zenodo, figshare, GitHub, ...
%\supplementdetails[linktext={opt. text shown instead of the URL}, cite=DBLP:books/mk/GrayR93, subcategory={Description, Subcategory}, swhid={Software Heritage Identifier}]{General Classification (e.g. Software, Dataset, Model, ...)}{URL to related version} %linktext, cite, and subcategory are optional

%\funding{(Optional) general funding statement \dots}%optional, to capture a funding statement, which applies to all authors. Please enter author specific funding statements as fifth argument of the \author macro.

% \acknowledgements{I want to thank \dots}%optional

\nolinenumbers %uncomment to disable line numbering

\hideLIPIcs  %uncomment to remove references to LIPIcs series (logo, DOI, ...), e.g. when preparing a pre-final version to be uploaded to arXiv or another public repository

%Editor-only macros:: begin (do not touch as author)%%%%%%%%%%%%%%%%%%%%%%%%%%%%%%%%%%
\EventEditors{John Q. Open and Joan R. Access}
\EventNoEds{2}
\EventLongTitle{42nd Conference on Very Important Topics (CVIT 2016)}
\EventShortTitle{CVIT 2016}
\EventAcronym{CVIT}
\EventYear{2016}
\EventDate{December 24--27, 2016}
\EventLocation{Little Whinging, United Kingdom}
\EventLogo{}
\SeriesVolume{42}
\ArticleNo{23}
%%%%%%%%%%%%%%%%%%%%%%%%%%%%%%%%%%%%%%%%%%%%%%%%%%%%%%

\newcommand{\nat}{\ensuremath{\mathbb{N}}}
\newcommand{\bool}{\ensuremath{\mathbb{B}}}

\newcommand{\Set}{\ensuremath{\mathsf{Set}}}
\newcommand{\Hom}{\ensuremath{\mathsf{Hom}}}

\begin{document}

\maketitle

\begin{abstract}
Documents the various directions that generalizations of universal algebra
can take.
\end{abstract}

\section{The Design Space}
\label{sec:design-space}

Universal algebra is the study of theories that can be defined in
single-sorted equational logic. Part of it is to look at the syntax of
such theories, and then various constructions that can be performed
uniformly on these.

\subsection{Restrictions I}
Traditionally, the following restrictions are put in place:
\begin{enumerate}
\item \label{1sort} There is a single \emph{sort} (or type). Everything
 is homogeneous in terms of a single carrier ``set''
\item \label{operations} There is a discrete, finite ``set'' of
operation symbols.
\item \label{arity} Each operation symbol has a finite, ordered ``number''
of inputs. In other words, each has an \emph{arity} which can be
identified with the natural numbers \nat.
\item \label{coarity} Each operation symbol has a single ``result''.
In other words, its \emph{coarity} is $1$.
\item \label{equality} There is an assumed notion of equality in the
meta-theory, written as $=$. It is generally only used at the single
assumed sort. No other relations are assumed.
\item \label{set-models} The \emph{models} are assumed to be in 
\Set.
\item \label{UIP} Equalities must simply hold in a model. There is
no content in the proofs that these constraints are satisfied.
This is related to \emph{proof irrelevance} and 
\emph{Uniqueness of Identity Proofs} (UIP).
\item \label{set-equality} It is meaningful to talk about the equality
of functions in \Set.
\item \label{set-quotient} It is meaningful to talk about the quotients
in \Set.
\item \label{totality} The interpretation of operation symbols in \Set
is by \emph{total} functions.
\end{enumerate}

Many of these restrictions can be lifted, and quite a lot of the utility
is retained. In particular, many of the fundamental constructions still work,
and many of the basic results too. The more advanced classification theorems
require re-interpretation, when they are valid at all.

One leading example is being able to make sense of the following (true)
statement: ``A Functor is a homomorphism of categories''. The construction
of what it means to be a homomorphism needs to be lifted to the situation
where $\mathsf{Category}$ as a \emph{theory} can exist. One can then
continue this to say ``A natural transformation is a homomorphism of 
functors'', which is again true, but considerably more difficult to
operationalize.

\subsection{Lifting Restrictions I}

We discuss, to various levels of details, what it means to lift some
some of the above restrictions. The usefulness of doing so is also listed.

\subsubsection{Sorts}
Vector spaces are very naturally $2$-sorted. So are group-actions.

\subsubsection{Operations}
Very little breaks down if one assumes non-finite signatures. Sometimes it
is useful to assume gradedness.

\subsubsection{Arity}
There is no strong reason to assume ordered, finite number of inputs.
Having various ``shapes'' of inputs is quite useful too. 

\subsubsection{Coarity}
Multiple outputs are quite prevalent. Just because they can often be
``coded'' via multiple functions doesn't mean it's a good idea to do so.
Even simple functions like 'factoring a natural number' wants quite
complex coarity to exist.

\subsubsection{Equality}

\subsubsection{Set models}
We know we can have ``algebras'' in all sorts of other places. This makes
the language of universal algebra much more polymorphic, as it can then
talk about structures in all sorts of other interesting places.
Monoid objects are rather interesting in a lot of different categories.

\subsubsection{UIP}
It would be nice to treat the equations as both first-class and meaningful.
So it would be good to not erase the proofs.

\subsubsection{Function Equality}

\subsubsection{Totality}

\subsection{Notes}

There are some items that are not clear where to discuss them. Some
should probably be in a subsequent pair of Restriction/Lift sections,
as they are dependent on seeing new restrictions that arise from lifting
some of the ones above.  In particular:

\begin{itemize}
\item Relations. Classical relations can arise once we go $2$-sorted and
add \bool to the meta-theory. (It can also be added in-theory as well,
but that causes unnecessary bloat.)
\item Relations again. Proof-relevant relations, i.e. relations that are
inhabited by witnesses, require not only $n$-sorted, but a lifting of
having just sorts for coarity. This is thus a bigger move.
\item Indexed sorts. We need these to do (dependently-typed) \Hom for
category theory.
\item Parametrized theories. If one wants to internalize homomorphism, it
is done most elegantly by having parameters. Interestingly, this is a neat
way to cheat $1$-sortedness.
\end{itemize}
%%
%% Bibliography
%%

% \bibliography{mathscheme}

\end{document}
